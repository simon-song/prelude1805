%%%%%%%%%%%%%%%%%%%%%%%%%%%%%%%%%%%%%%%%%%
\documentclass[a4paper,14pt]{memoir}

%\usepackage[utf8]{inputenc}
\usepackage{fontspec}
%\setmainfont[Ligatures=TeX]{Linux Libertine O}
\setmainfont[Ligatures=TeX]{Minion Pro}
%\setmainfont[Mapping=Tex-text]{GFSPorson.ttf} 
%\setmainfont[Mapping=Tex-text]{Gentium} % good 
%\setmainfont[Mapping=Tex-text]{TeXGyrePagella} % not good
%\setmainfont[Mapping=Tex-text]{GFS Porson} % good
%\setmainfont[Mapping=Tex-text]{FreeMono} % works but not very pretty

%\newfontfamily\gentium{Gentium}  % Greek and English
%\newfontfamily\g{GFS Porson}  % Greek only
%\newfontfamily\g{GFSPorson.ttf}
%\newfontfamily\g{Linux Libertine O}
%\newfontfamily\neohellenic{GFS Neohellenic} % Greek only
%\newfontfamily\e{Linux Libertine O}

% \settrimmedsize{11in}{210mm}{*}
% \setlength{\trimtop}{0pt}
% \setlength{\trimedge}{\stockwidth}
% \addtolength{\trimedge}{-\paperwidth}
% \settypeblocksize{8.25in}{33pc}{*}
% %\setulmargins{4cm}{*}{*}
% \setulmargins{4.8cm}{*}{0.5}
% %\setlrmargins{1.25in}{*}{*}
% %\setlrmarginsandblock{1.5in}{*}{*}
% \setlrmarginsandblock{1.0in}{*}{1.5}
% \setmarginnotes{0pt}{0pt}{\onelineskip}  % changed by Pengcheng, no margin
% \setheadfoot{\onelineskip}{2\onelineskip}
% %\setheaderspaces{*}{2\onelineskip}{*}
% \setheaderspaces{*}{\onelineskip}{*} % changed by Pengcheng
% \addtolength{\textwidth}{60pt}     %  increase text width
% \addtolength{\foremargin}{-60pt}
% \checkandfixthelayout

\setheadfoot{\onelineskip}{2\onelineskip}

%%%%   larger line spacing
\usepackage{setspace}
\linespread{1.2}

\usepackage[plainpages=false;pdfpagelabels,bookmarksnumbered]{hyperref}
\usepackage{verse}
\linenumberfrequency{5}

%%%%%%%%%%%%%%%%%%%%%%%%%%%%%%%%%%%%%%%%%%%%%%%%%%%%%%%%%%%%%%%%%%%%
%%  Set \vin which is a short hand of \hspace{\vgap}, default value
%%  of \vgap is 1.5em
%%%%%%%%%%%%%%%%%%%%%%%%%%%%%%%%%%%%%%%%%%%%%%%%%%%%%%%%%%%%%%%%%%%%
\setlength{\vgap}{1em}
\setlength{\vrightskip}{-2em} % distance of line number from right margin

%%%%%%%%%%%%%%%%%%%%%%%%%%%%%%%%%%%%%%%%%%%%%%%%%%%%%%%%%%%%%%%
%% chapter style VZ23 by Vincent Zoonekynd
%% code adapted from Lars Madsen, Various Chapter Styles
%% for the Memoir Class, pp.44-45
%%%%%%%%%%%%%%%%%%%%%%%%%%%%%%%%%%%%%%%%%%%%%%%%%%%%%%%%%%%%%%%
\setlength\midchapskip{10pt}
\makechapterstyle{VZ23} {
  \renewcommand\chapternamenum{}
  \renewcommand\printchaptername{}
  \renewcommand\chapnumfont{\Huge\bfseries\centering}
  \renewcommand\chaptitlefont{\Huge\scshape\centering}
  \renewcommand\afterchapternum{%
    %\par\nobreak\vskip\midchapskip\hrule\vskip\midchapskip}
    \par\nobreak\vskip\midchapskip\vskip\midchapskip}
  \renewcommand\printchapternonum{%
    \vphantom{\chapnumfont \thechapter}
    %\par\nobreak\vskip\midchapskip\hrule\vskip\midchapskip}
    \par\nobreak\vskip\midchapskip\vskip\midchapskip}
}

\chapterstyle{VZ23}

% page style of Oxford Classical Texts 
\makepagestyle{ocd}
  \makeevenhead{ocd}{\thepage}{The Prelude: 1805 Edition}{}
  \makeoddhead{ocd}{}{\rightmark}{\thepage}  % chapter in center
  \makeevenfoot{ocd}{}{}{}
  \makeoddfoot{ocd}{}{}{}

% set \rightmark to chapter name	
\makepsmarks{ocd}{%
  \nouppercaseheads
  \createmark{chapter}{right}{shownumber}{}{. \ }
}

\setsecnumdepth{chapter}

\makepagestyle{app}
  \makeevenhead{app}{\thepage}{The Two-book Prelude (1798--99)}{}
  %\makeoddhead{app}{}{\rightmark}{\thepage}  % chapter in center
  \makeoddhead{app}{}{\leftmark}{\thepage}  % chapter in center
  \makeevenfoot{app}{}{}{}
  \makeoddfoot{app}{}{}{}

% set \rightmark to chapter name	
\makepsmarks{app}{%
  \nouppercaseheads
  %\createmark{section}{right}{shownumber}{}{. \ }
  \createmark{section}{left}{shownumber}{}{. \ }
}

%%%%%%%%%%%%%%%%%% from Peter Wilson, Some Examples of Title Pages, pp.45-46
% \usepackage[T1]{fontenc}
% \usepackage{lmodern}
% \usepackage{url}
% \usepackage[svgnames]{xcolor}
% \ifpdf
% \usepackage{pdfcolmk}
% \fi
% %% check if using xelatex rather than pdflatex
% \ifxetex
% \usepackage{fontspec}
% \fi
% \usepackage{graphicx}
% %%\usepackage{hyperref}
% %% drawing package
% \usepackage{tikz}
% %% for dingbats
% \usepackage{pifont}
% \providecommand{\HUGE}{\Huge}% if not using memoir
% \newlength{\drop}% for my convenience
% %% specify the Webomints family
% % \newcommand*{\wb}[1]{\fontsize{#1}{#2}\usefont{U}{webo}{xl}{n}}
% %% select a (FontSite) font by its font family ID
% \newcommand*{\FSfont}[1]{\fontencoding{T1}\fontfamily{#1}\selectfont}
% %% if you don��t have the FontSite fonts either \renewcommand*{\FSfont}[1]{}
% %% or use your own choice of family.
% %% select a (TeX Font) font by its font family ID
% \newcommand*{\TXfont}[1]{\fontencoding{T1}\fontfamily{#1}\selectfont}
% %% Generic publisher��s logo
% \newcommand*{\plogo}{\fbox{$\mathcal{PL}$}}
% %% some shades
% % \defincolor{Dark}{gray}{0.2}
% % \defincolor{MedDark}{gray}{0.4}
% % \defincolor{Medium}{gray}{0.6}
% % \defincolor{Light}{gray}{0.8}
% %%%% Additional font series macros
% \makeatletter
% %%%% light series
% %% e.g., kernel doc, section s: line 12 or thereabouts
% \DeclareRobustCommand\ltseries
% {\not@math@alphabet\ltseries\relax
% \fontseries\ltdefault\selectfont}
% %% e.g., kernel doc, section t: line 32 or thereabouts
% \newcommand{\ltdefault}{l}
% %% e.g., kernel doc, section v: line 19 or thereabouts
% \DeclareTextFontCommand{\textlt}{\ltseries}
% % heavy(bold) series
% \DeclareRobustCommand\hbseries
% {\not@math@alphabet\hbseries\relax
% \fontseries\hbdefault\selectfont}
% \newcommand{\hbdefault}{hb}
% \DeclareTextFontCommand{\texthb}{\hbseries}
% \makeatother
% %%%%%%%%%%%%%%%%%% end of code from Peter Wilson, Some Examples of Title Pages
% 
% %% adapted from Peter Wilson, Some Examples of Title Pages, p.49
% \newcommand*{\titleRF}{\begingroup % Robert Frost, T&H p 149
% \FSfont{5bp} % FontSite Bergamo (Bembo)
% \drop = 0.2\textheight
% \centering
% \vfill
% {\Huge The Prelude}\\[\baselineskip]
% {\Huge 1805 Edition}\\[\baselineskip]
% {\large by William Wordsworth}\\[0.5\drop]
% {\Large \plogo}\\[0.5\baselineskip]
% {\Large The Publisher}\par
% {\large\scshape year}\par
% \vfill\null
% \endgroup}
% 
% %% adapted from Peter Wilson, Some Examples of Title Pages, p.52
% \newcommand*{\titleP}{\begingroup% The Pyramids, AW p. 81
% \FSfont{5bo} % FontSite Bergamo (Bembo)
% \drop = 0.12\textheight
% \vspace*{\drop}
% \hspace*{0.3\textwidth}
% {\Huge The Prelude}\\[\baselineskip]
% \hspace*{0.3\textwidth}
% {\Huge 1805 Edition}\par
% \vspace*{3\drop}
% {\large By William Wordsworth}
% \vfill
% {\scshape the publisher}
% \vspace*{0.5\drop}
% \endgroup}


%%%%%%%%%%%%%%%%%%%%%%%%
%\usepackage{makeidx}
%\makeindex

%%%%%%%%%%%%%%%%%%%%%%%%%%%%%%%%%%%%%%%%%%%%%%%%%%%%%%
\begin{document}

%%%%%%%%%%%%%%%%%%%%%%
\frontmatter
\pagestyle{empty}
\begin{center}
  \vspace{2.5in} 
  %\HUGE{The Prelude: 1805 Edition} \par \vspace{3in}
  \resizebox{0.75\linewidth}{!}{The Prelude: 1805 Edition} \par \vspace{0.5in}
  \large{with} \par \vspace{0.5in}
  \Large{The Two-Book Prelude (1798--99)} \par \vspace{3in}
  \large{by} \par \vspace{0.5in}
  \Large{William Wordsworth} 
\end{center}

%\titleP
%\titleRF

\cleardoublepage
\pagenumbering{roman}
\pagestyle{headings}

\tableofcontents*

%%%%%%%%%%%%%%%%%%%%
\mainmatter
\pagestyle{ocd}
\chapter*[Book First]{Book First \\ Introduction: Childhood and School-time}
\addcontentsline{toc}{chapter}{Book First Introduction: Childhood and School-time}

\begin{verse} % book first
OH, there is blessing in this gentle breeze,  \\
That blows from the green fields and from the clouds  \\
And from the sky; it beats against my cheek,  \\
And seems half conscious of the joy it gives.  \\
O welcome messenger! O welcome friend!	  \\
A captive greets thee, coming from a house  \\
Of bondage, from yon city's walls set free,  \\
A prison where he hath been long immured.  \\
Now I am free, enfranchised and at large,  \\
May fix my habitation where I will.	  \\
What dwelling shall receive me, in what vale  \\
Shall be my harbour, underneath what grove  \\
Shall I take up my home, and what sweet stream  \\
Shall with its murmurs lull me to my rest?  \\
The earth is all before me---with a heart	  \\
Joyous, nor scared at its own liberty,  \\
I look about, and should the guide I chuse  \\
Be nothing better than a wandering cloud  \\
I cannot miss my way. I breathe again---  \\
Trances of thought and mountings of the mind	  \\
Come fast upon me. It is shaken off,  \\
As by miraculous gift 'tis shaken off,  \\
That burthen of my own unnatural self,  \\
The heavy weight of many a weary day  \\
Not mine, and such as were not made for me.	  \\
Long months of peace---if such bold word accord  \\
With any promises of human life---  \\
Long months of ease and undisturbed delight  \\
Are mine in prospect. Whither shall I turn,  \\
By road or pathway, or through open field,	  \\
Or shall a twig or any floating thing  \\
Upon the river point me out my course?  \\!
Enough that I am free, for months to come  \\
May dedicate myself to chosen tasks,  \\
May quit the tiresome sea and dwell on shore---	  \\
If not a settler on the soil, at least  \\
To drink wild water, and to pluck green herbs,  \\
And gather fruits fresh from their native bough.  \\
Nay more, if I may trust myself, this hour  \\
Hath brought a gift that consecrates my joy;	  \\
For I, methought, while the sweet breath of heaven  \\
Was blowing on my body, felt within  \\
A corresponding mild creative breeze,  \\
A vital breeze which travelled gently on  \\
O'er things which it had made, and is become	  \\
A tempest, a redundant energy,  \\
Vexing its own creation. 'Tis a power  \\
That does not come unrecognised, a storm  \\
Which, breaking up a long-continued frost,  \\
Brings with it vernal promises, the hope	  \\
Of active days, of dignity and thought,  \\
Of prowess in an honorable field,  \\
Pure passions, virtue, knowledge, and delight,  \\
The holy life of music and of verse.  \\!
Thus far, O friend, did I, not used to make	  \\
A present joy the matter of my song,  \\
Pour out that day my soul in measured strains,  \\
Even in the very words which I have here  \\
Recorded. To the open fields I told  \\
A prophesy; poetic numbers came	  \\
Spontaneously, and clothed in priestly robe  \\
My spirit, thus singled out, as it might seem,  \\
For holy services. Great hopes were mine:  \\
My own voice cheared me, and, far more, the mind's  \\
Internal echo of the imperfect sound---	  \\
To both I listened, drawing from them both  \\
A chearful confidence in things to come.  \\!
Whereat, being not unwilling now to give  \\
A respite to this passion, I paced on  \\
Gently, with careless steps, and came erelong	  \\
To a green shady place where down I sate  \\
Beneath a tree, slackening my thoughts by choice  \\
And settling into gentler happiness.  \\
'Twas autumn, and a calm and placid day  \\
With warmth as much as needed from a sun	  \\
Two hours declined towards the west, a day  \\
With silver clouds and sunshine on the grass,  \\
And, in the sheltered grove where I was couched,  \\
A perfect stillness. On the ground I lay  \\
Passing through many thoughts, yet mainly such	  \\
As to myself pertained. I made a choice  \\
Of one sweet vale whither my steps should turn,  \\
And saw, methought, the very house and fields  \\
Present before my eyes; nor did I fail  \\
To add meanwhile assurance of some work	  \\
Of glory there forthwith to be begun---  \\
Perhaps too there performed. Thus long I lay  \\
Cheared by the genial pillow of the earth  \\
Beneath my head, soothed by a sense of touch  \\
From the warm ground, that balanced me, else lost	  \\
Entirely, seeing nought, nought hearing, save  \\
When here and there about the grove of oaks  \\
Where was my bed, an acorn from the trees  \\
Fell audibly, and with a startling sound.  \\!
Thus occupied in mind I lingered here	  \\
Contented, nor rose up until the sun  \\
Had almost touched the horizon; bidding then  \\
A farewell to the city left behind,  \\
Even with the chance equipment of that hour  \\
I journeyed towards the vale which I had chosen.	  \\
It was a splendid evening, and my soul  \\
Did once again make trial of the strength  \\
Restored to her afresh; nor did she want  \\
Eolian visitations---but the harp  \\
Was soon defrauded, and the banded host	  \\
Of harmony dispersed in straggling sounds,  \\
And lastly utter silence. `Be it so,  \\
It is an injury', said I, `to this day  \\
To think of any thing but present joy.'  \\
So, like a peasant, I pursued my road	  \\
Beneath the evening sun, nor had one wish  \\
Again to bend the sabbath of that time  \\
To a servile yoke. What need of many words?---  \\
A pleasant loitering journey, through two days  \\
Continued, brought me to my hermitage.	  \\!
I spare to speak, my friend, of what ensued---  \\
The admiration and the love, the life  \\
In common things, the endless store of things  \\
Rare, or at least so seeming, every day  \\
Found all about me in one neighbourhood,	  \\
The self-congratulations, the complete  \\
Composure, and the happiness entire.  \\
But speedily a longing in me rose  \\
To brace myself to some determined aim,  \\
Reading or thinking, either to lay up	  \\
New stores, or rescue from decay the old  \\
By timely interference. I had hopes  \\
Still higher, that with a frame of outward life  \\
I might endue, might fix in a visible home,  \\
Some portion of those phantoms of conceit,	  \\
That had been floating loose about so long,  \\
And to such beings temperately deal forth  \\
The many feelings that oppressed my heart.  \\
But I have been discouraged: gleams of light  \\
Flash often from the east, then disappear,	  \\
And mock me with a sky that ripens not  \\
Into a steady morning. If my mind,  \\
Remembering the sweet promise of the past,  \\
Would gladly grapple with some noble theme,  \\
Vain is her wish---where'er she turns she finds	  \\
Impediments from day to day renewed.  \\!
And now it would content me to yield up  \\
Those lofty hopes awhile for present gifts  \\
Of humbler industry. But, O dear friend,  \\
The poet, gentle creature as he is,	  \\
Hath like the lover his unruly times---  \\
His fits when he is neither sick nor well,  \\
Though no distress be near him but his own  \\
Unmanageable thoughts. The mind itself,  \\
The meditative mind, best pleased perhaps	  \\
While she as duteous as the mother dove  \\
Sits brooding, lives not always to that end,  \\
But hath less quiet instincts---goadings on  \\
That drive her as in trouble through the groves.  \\
With me is now such passion, which I blame	  \\
No otherwise than as it lasts too long.  \\!
When, as becomes a man who would prepare  \\
For such a glorious work, I through myself  \\
Make rigorous inquisition, the report	  \\
Is often chearing; for I neither seem  \\
To lack that first great gift, the vital soul,  \\
Nor general truths which are themselves a sort  \\
Of elements and agents, under-powers,  \\
Subordinate helpers of the living mind.	  \\
Nor am I naked in external things,  \\
Forms, images, nor numerous other aids  \\
Of less regard, though won perhaps with toil,  \\
And needful to build up a poet's praise.  \\
Time, place, and manners, these I seek, and these	  \\
I find in plenteous store, but nowhere such  \\
As may be singled out with steady choice---  \\
No little band of yet remembered names  \\
Whom I, in perfect confidence, might hope  \\
To summon back from lonesome banishment	  \\
And make them inmates in the hearts of men  \\
Now living, or to live in times to come.  \\
Sometimes, mistaking vainly, as I fear,  \\
Proud spring-tide swellings for a regular sea,  \\
I settle on some British theme, some old	  \\
Romantic tale by Milton left unsung;  \\
More often resting at some gentle place  \\
Within the groves of chivalry I pipe  \\
Among the shepherds, with reposing knights  \\
Sit by a fountain-side and hear their tales.	  \\
Sometimes, more sternly move, I would relate  \\
How vanquished Mithridates northward passed  \\
And, hidden in the cloud of years, became  \\
That Odin, father of a race by whom  \\
Perished the Roman Empire; how the friends	  \\
And followers of Sertorius, out of Spain  \\
Flying, found shelter in the Fortunate Isles,  \\
And left their usages, their arts and laws,  \\
To disappear by a slow gradual death,  \\
To dwindle and to perish one by one,	  \\
Starved in those narrow bounds---but not the soul  \\
Of liberty, which fifteen hundred years  \\
Survived, and, when the European came  \\
With skill and power that could not be withstood,  \\
Did like a pestilence maintain its hold,	  \\
And wasted down by glorious death that race  \\
Of natural heroes. Or I would record  \\
How in tyrannic times, some unknown man,  \\
Unheard of in the chronicles of kings,  \\
Suffered in silence for the love of truth;	  \\
How that one Frenchman, through continued force  \\
Of meditation on the inhuman deeds  \\
Of the first conquerors of the Indian Isles,  \\
Went single in his ministry across  \\
The ocean, not to comfort the oppressed,	  \\
But like a thirsty wind to roam about  \\
Withering the oppressor; how Gustavus found  \\
Help at his need in Dalecarlia's mines;  \\
How Wallace fought for Scotland, left the name  \\
Of Wallace to be found like a wild flower	  \\
All over his dear county, left the deeds  \\
Of Wallace like a family of ghosts  \\
To people the steep rocks and river-banks,  \\
Her natural sanctuaries, with a local soul  \\
Of independence and stern liberty.	  \\
Sometimes it suits me better to shape out  \\
Some tale from my own heart, more near akin  \\
To my own passions and habitual thoughts,  \\
Some variegated story, in the main  \\
Lofty, with interchange of gentler things.	  \\
But deadening admonitions will succeed,  \\
And the whole beauteous fabric seems to lack  \\
Foundation, and withal appears throughout  \\
Shadowy and unsubstantial.  \\
Then, last wish---	  \\
My last and favorite aspiration---then  \\
I yearn towards some philosophic song  \\
Of truth that cherishes our daily life,  \\
With meditations passionate from deep  \\
Recesses in man's heart, immortal verse	  \\
Thoughtfully fitted to the Orphean lyre;  \\
But from this awful burthen I full soon  \\
Take refuge, and beguile myself with trust  \\
That mellower years will bring a riper mind  \\
And clearer insight. Thus from day to day	  \\
I live a mockery of the brotherhood  \\
Of vice and virtue, with no skill to part  \\
Vague longing that is bred by want of power,  \\
From paramount impulse not to be withstood;  \\
A timorous capacity, from prudence;	  \\
From circumspection, infinite delay.  \\
Humility and modest awe themselves  \\
Betray me, serving often for a cloak  \\
To a more subtle selfishness, that now  \\
Doth lock my functions up in blank reserve,	  \\
Now dupes me by an over-anxious eye  \\
That with a false activity beats off  \\
Simplicity and self-presented truth.  \\!
---Ah! better far than this, to stray about  \\
Voluptuously through fields and rural walks	  \\
And ask no record of the hours given up  \\
To vacant musing, unreproved neglect  \\
Of all things, and deliberate holiday.  \\
Far better never to have heard the name  \\
Of zeal and just ambition than to live	  \\
Thus baffled by a mind that every hour  \\
Turns recreant to her task, takes heart again,  \\
Then feels immediately some hollow thought  \\
Hang like an interdict upon her hopes.  \\
This is my lot; for either still I find	  \\
Some imperfection in the chosen theme,  \\
Or see of absolute accomplishment  \\
Much wanting---so much wanting---in myself  \\
That I recoil and droop, and seek repose  \\
In indolence from vain perplexity,	  \\
Unprofitably travelling toward the grave,  \\
Like a false steward who hath much received  \\
And renders nothing back. ---Was it for this  \\
That one, the fairest of all Rivers, lov'd	  \\
To blend his murmurs with my Nurse's song,  \\
And from his alder shades and rocky falls,  \\
And from his fords and shallows, sent a voice  \\
That flow'd along my dreams? For this, didst Thou,  \\
O Derwent! travelling over the green Plains	  \\
Near my 'sweet Birthplace', didst thou, beauteous Stream  \\
Make ceaseless music through the night and day  \\
Which with its steady cadence, tempering  \\
Our human waywardness, compos'd my thoughts  \\
To more than infant softness, giving me,	  \\
Among the fretful dwellings of mankind,  \\
A knowledge, a dim earnest, of the calm  \\
That Nature breathes among the hills and groves.  \\!
When, having left his Mountains, to the Towers  \\
Of Cockermouth that beauteous River came,	  \\
Behind my Father's House he pass'd, close by,  \\
Along the margin of our Terrace Walk.  \\
He was a Playmate whom we dearly lov'd.  \\
Oh! many a time have I, a five years' Child,  \\
A naked Boy, in one delightful Rill,	  \\
A little Mill-race sever'd from his stream,  \\
Made one long bathing of a summer's day,  \\
Bask'd in the sun, and plunged, and bask'd again  \\
Alternate all a summer's day, or cours'd  \\
Over the sandy fields, leaping through groves	  \\
Of yellow grunsel, or when crag and hill,  \\
The woods, and distant Skiddaw's lofty height,  \\
Were bronz'd with a deep radiance, stood alone  \\
Beneath the sky, as if I had been born  \\
On Indian Plains, and from my Mother's hut	  \\
Had run abroad in wantonness, to sport,  \\
A naked Savage, in the thunder shower.  \\!
Fair seed-time had my soul, and I grew up  \\
Foster'd alike by beauty and by fear;  \\
Much favour'd in my birthplace, and no less	  \\
In that beloved Vale to which, erelong,  \\
I was transplanted. Well I call to mind  \\
('Twas at an early age, ere I had seen  \\
Nine summers) when upon the mountain slope  \\
The frost and breath of frosty wind had snapp'd	  \\
The last autumnal crocus, 'twas my joy  \\
To wander half the night among the Cliffs  \\
And the smooth Hollows, where the woodcocks ran  \\
Along the open turf. In thought and wish  \\
That time, my shoulder all with springes hung,	  \\
I was a fell destroyer. On the heights  \\
Scudding away from snare to snare, I plied  \\
My anxious visitation, hurrying on,  \\
Still hurrying, hurrying onward; moon and stars  \\
Were shining o'er my head; I was alone,	  \\
And seem'd to be a trouble to the peace  \\
That was among them. Sometimes it befel  \\
In these night-wanderings, that a strong desire  \\
O'erpower'd my better reason, and the bird  \\
Which was the captive of another's toils	  \\
Became my prey; and, when the deed was done  \\
I heard among the solitary hills  \\
Low breathings coming after me, and sounds  \\
Of undistinguishable motion, steps  \\
Almost as silent as the turf they trod.	  \\!
Nor less in springtime when on southern banks  \\
The shining sun had from his knot of leaves  \\
Decoy'd the primrose flower, and when the Vales  \\
And woods were warm, was I a plunderer then  \\
In the high places, on the lonesome peaks	  \\
Where'er, among the mountains and the winds,  \\
The Mother Bird had built her lodge. Though mean  \\
My object, and inglorious, yet the end  \\
Was not ignoble. Oh! when I have hung  \\
Above the raven's nest, by knots of grass	  \\
And half-inch fissures in the slippery rock  \\
But ill sustain'd, and almost, as it seem'd,  \\
Suspended by the blast which blew amain,  \\
Shouldering the naked crag; Oh! at that time,  \\
While on the perilous ridge I hung alone,	  \\
With what strange utterance did the loud dry wind  \\
Blow through my ears! the sky seem'd not a sky  \\
Of earth, and with what motion mov'd the clouds!  \\!
The mind of Man is fram'd even like the breath  \\
And harmony of music. There is a dark	  \\
Invisible workmanship that reconciles  \\
Discordant elements, and makes them move  \\
In one society. Ah me! that all  \\
The terrors, all the early miseries  \\
Regrets, vexations, lassitudes, that all	  \\
The thoughts and feelings which have been infus'd  \\
Into my mind, should ever have made up  \\
The calm existence that is mine when I  \\
Am worthy of myself! Praise to the end!  \\
Thanks likewise for the means! But I believe	  \\
That Nature, oftentimes, when she would frame  \\
A favor'd Being, from his earliest dawn  \\
Of infancy doth open out the clouds,  \\
As at the touch of lightning, seeking him  \\
With gentlest visitation; not the less,	  \\
Though haply aiming at the self-same end,  \\
Does it delight her sometimes to employ  \\
Severer interventions, ministry  \\
More palpable, and so she dealt with me.  \\!
One evening (surely I was led by her)	  \\
I went alone into a Shepherd's Boat, A  \\
Skiff that to a Willow tree was tied  \\
Within a rocky Cave, its usual home.  \\
'Twas by the shores of Patterdale, a Vale  \\
Wherein I was a Stranger, thither come	  \\
A School-boy Traveller, at the Holidays.  \\
Forth rambled from the Village Inn alone  \\
No sooner had I sight of this small Skiff,  \\
Discover'd thus by unexpected chance,  \\
Than I unloos'd her tether and embark'd.	  \\
The moon was up, the Lake was shining clear  \\
Among the hoary mountains; from the  \\
Shore I push'd, and struck the oars and struck again  \\
In cadence, and my little Boat mov'd on  \\
Even like a Man who walks with stately step	  \\
Though bent on speed. It was an act of stealth  \\
And troubled pleasure; not without the voice  \\
Of mountain-echoes did my Boat move on,  \\
Leaving behind her still on either side  \\
Small circles glittering idly in the moon,	  \\
Until they melted all into one track  \\
Of sparkling light. A rocky Steep uprose  \\
Above the Cavern of the Willow tree  \\
And now, as suited one who proudly row'd  \\
With his best skill, I fix'd a steady view	  \\
Upon the top of that same craggy ridge,  \\
The bound of the horizon, for behind  \\
Was nothing but the stars and the grey sky.  \\
She was an elfin Pinnace; lustily  \\
I dipp'd my oars into the silent Lake,	  \\
And, as I rose upon the stroke, my Boat  \\
Went heaving through the water, like a Swan;  \\
When from behind that craggy Steep, till then  \\
The bound of the horizon, a huge Cliff,  \\
As if with voluntary power instinct,	  \\
Uprear'd its head. I struck, and struck again  \\
And, growing still in stature, the huge  \\
Cliff Rose up between me and the stars, and still,  \\
With measur'd motion, like a living thing,  \\
Strode after me. With trembling hands I turn'd,	  \\
And through the silent water stole my way  \\
Back to the Cavern of the Willow tree.  \\
There, in her mooring-place, I left my Bark,  \\
And, through the meadows homeward went, with grave  \\
And serious thoughts; and after I had seen	  \\
That spectacle, for many days, my brain  \\
Work'd with a dim and undetermin'd sense  \\
Of unknown modes of being; in my thoughts  \\
There was a darkness, call it solitude,  \\
Or blank desertion, no familiar shapes	  \\
Of hourly objects, images of trees,  \\
Of sea or sky, no colours of green fields;  \\
But huge and mighty Forms that do not live  \\
Like living men mov'd slowly through the mind  \\
By day and were the trouble of my dreams.	  \\!
Wisdom and Spirit of the universe!  \\
Thou Soul that art the eternity of thought!  \\
That giv'st to forms and images a breath  \\
And everlasting motion! not in vain,  \\
By day or star-light thus from my first dawn	  \\
Of Childhood didst Thou intertwine for me  \\
The passions that build up our human Soul,  \\
Not with the mean and vulgar works of Man,  \\
But with high objects, with enduring things,  \\
With life and nature, purifying thus	  \\
The elements of feeling and of thought,  \\
And sanctifying, by such discipline,  \\
Both pain and fear, until we recognise  \\
A grandeur in the beatings of the heart.  \\!
Nor was this fellowship vouchsaf'd to me	  \\
With stinted kindness. In November days,  \\
When vapours, rolling down the valleys, made  \\
A lonely scene more lonesome; among woods  \\
At noon, and 'mid the calm of summer nights,  \\
When, by the margin of the trembling Lake,	  \\
Beneath the gloomy hills I homeward went  \\
In solitude, such intercourse was mine;  \\
'Twas mine among the fields both day and night,  \\
And by the waters all the summer long.  \\
---And in the frosty season, when the sun	  \\
Was set, and visible for many a mile  \\
The cottage windows through the twilight blaz'd,  \\
I heeded not the summons:---happy time  \\
It was, indeed, for all of us; to me  \\
It was a time of rapture: clear and loud	  \\
The village clock toll'd six; I wheel'd about,  \\
Proud and exulting, like an untired horse,  \\
That cares not for its home.---All shod with steel,  \\
We hiss'd along the polish'd ice, in games  \\
Confederate, imitative of the chace	  \\
And woodland pleasures, the resounding horn,  \\
The Pack loud bellowing, and the hunted hare.  \\
So through the darkness and the cold we flew,  \\
And not a voice was idle; with the din,  \\
Meanwhile, the precipices rang aloud,	  \\
The leafless trees, and every icy crag  \\
Tinkled like iron, while the distant hills  \\
Into the tumult sent an alien sound  \\
Of melancholy, not unnoticed, while the stars,  \\
Eastward, were sparkling clear, and in the west	  \\
The orange sky of evening died away.  \\!
Not seldom from the uproar I retired  \\
Into a silent bay, or sportively  \\
Glanced sideway, leaving the tumultuous throng,  \\
To cut across the image of a star	  \\
That gleam'd upon the ice: and oftentimes  \\
When we had given our bodies to the wind,  \\
And all the shadowy banks, on either side,  \\
Came sweeping through the darkness, spinning still  \\
The rapid line of motion; then at once	  \\
Have I, reclining back upon my heels,  \\
Stopp'd short, yet still the solitary  \\
Cliffs Wheeled by me, even as if the earth had roll'd  \\
With visible motion her diurnal round;  \\
Behind me did they stretch in solemn train	  \\
Feebler and feebler, and I stood and watch'd  \\
Till all was tranquil as a dreamless sleep.  \\!
Ye Presences of Nature, in the sky  \\
And on the earth! Ye Visions of the hills!  \\
And Souls of lonely places! can I think	  \\
A vulgar hope was yours when Ye employ'd  \\
Such ministry, when Ye through many a year  \\
Haunting me thus among my boyish sports,  \\
On caves and trees, upon the woods and hills,  \\
Impress'd upon all forms the characters	  \\
Of danger or desire, and thus did make  \\
The surface of the universal earth  \\
With triumph, and delight, and hope, and fear,  \\
Work like a sea?  \\
Not uselessly employ'd,	  \\
I might pursue this theme through every change  \\
Of exercise and play, to which the year  \\
Did summon us in its delightful round.  \\
We were a noisy crew, the sun in heaven  \\
Beheld not vales more beautiful than ours,	  \\
Nor saw a race in happiness and joy  \\
More worthy of the ground where they were sown.  \\
I would record with no reluctant voice  \\
The woods of autumn and their hazel bowers  \\
With milk-white clusters hung; the rod and line,	  \\
True symbol of the foolishness of hope,  \\
Which with its strong enchantment led us on  \\
By rocks and pools, shut out from every star  \\
All the green summer, to forlorn cascades  \\
Among the windings of the mountain brooks.	  \\
---Unfading recollections! at this hour  \\
The heart is almost mine with which  \\
I felt From some hill-top, on sunny afternoons  \\
The Kite high up among the fleecy clouds  \\
Pull at its rein, like an impatient Courser,	  \\
Or, from the meadows sent on gusty days,  \\
Beheld her breast the wind, then suddenly  \\
Dash'd headlong; and rejected by the storm.  \\!
Ye lowly Cottages in which we dwelt,  \\
A ministration of your own was yours,	  \\
A sanctity, a safeguard, and a love!  \\
Can I forget you, being as ye were  \\
So beautiful among the pleasant fields  \\
In which ye stood? Or can I here forget  \\
The plain and seemly countenance with which	  \\
Ye dealt out your plain comforts? Yet had ye  \\
Delights and exultations of your own.  \\
Eager and never weary we pursued  \\
Our home amusements by the warm peat-fire  \\
At evening; when with pencil and with slate,	  \\
In square divisions parcell'd out, and all  \\
With crosses and with cyphers scribbled o'er,  \\
We schemed and puzzled, head opposed to head  \\
In strife too humble to be named in Verse.  \\
Or round the naked table, snow-white deal,	  \\
Cherry or maple, sate in close array,  \\
And to the combat, Lu or  \\
Whist, led on thick-ribbed Army; not as in the world  \\
Neglected and ungratefully thrown by  \\
Even for the very service they had wrought,	  \\
But husbanded through many a long campaign.  \\
Uncouth assemblage was it, where no few  \\
Had changed their functions, some, plebeian cards,  \\
Which Fate beyond the promise of their birth  \\
Had glorified, and call'd to represent	  \\
The persons of departed Potentates.  \\
Oh! with what echoes on the Board they fell!  \\
Ironic Diamonds, Clubs, Hearts, Diamonds, Spades,  \\
A congregation piteously akin.  \\
Cheap matter did they give to boyish wit,	  \\
Those sooty knaves, precipitated down  \\
With scoffs and taunts, like Vulcan out of  \\
Heaven, The paramount Ace, a moon in her eclipse,  \\
Queens, gleaming through their splendour's last decay,  \\
And Monarchs, surly at the wrongs sustain'd	  \\
By royal visages. Meanwhile, abroad  \\
The heavy rain was falling, or the frost  \\
Raged bitterly, with keen and silent tooth,  \\
And, interrupting oft the impassion'd game,  \\
From Esthwaite's neighbouring Lake the splitting ice,	  \\
While it sank down towards the water, sent,  \\
Among the meadows and the hills,  \\
Its long And dismal yellings, like the noise of wolves  \\
When they are howling round the Bothnic Main.  \\!
Nor, sedulous as I have been to trace	  \\
How Nature by extrinsic passion first  \\
Peopled my mind with beauteous forms or grand,  \\
And made me love them, may I well forget  \\
How other pleasures have been mine, and joys  \\
Of subtler origin; how I have felt,	  \\
Not seldom, even in that tempestuous time,  \\
Those hallow'd and pure motions of the sense  \\
Which seem, in their simplicity, to own  \\
An intellectual charm, that calm delight  \\
Which, if I err not, surely must belong	  \\
To those first-born affinities that fit  \\
Our new existence to existing things,  \\
And, in our dawn of being, constitute  \\
The bond of union betwixt life and joy.  \\!
Yes, I remember, when the changeful earth,	  \\
And twice five seasons on my mind had stamp'd  \\
The faces of the moving year, even then,  \\
A Child, I held unconscious intercourse  \\
With the eternal Beauty, drinking in  \\
A pure organic pleasure from the lines	  \\
Of curling mist, or from the level plain  \\
Of waters colour'd by the steady clouds.  \\!
The Sands of Westmoreland, the Creeks and Bays  \\
Of Cumbria's rocky limits, they can tell  \\
How when the Sea threw off his evening shade	  \\
And to the Shepherd's huts beneath the crags  \\
Did send sweet notice of the rising moon,  \\
How I have stood, to fancies such as these,  \\
Engrafted in the tenderness of thought,  \\
A stranger, linking with the spectacle	  \\
No conscious memory of a kindred sight,  \\
And bringing with me no peculiar sense  \\
Of quietness or peace, yet I have stood,  \\
Even while mine eye has mov'd o'er three long leagues  \\
Of shining water, gathering, as it seem'd,	  \\
Through every hair-breadth of that field of light,  \\
New pleasure, like a bee among the flowers.  \\!
Thus, often in those fits of vulgar joy  \\
Which, through all seasons, on a child's pursuits  \\
Are prompt attendants, 'mid that giddy bliss	  \\
Which, like a tempest, works along the blood  \\
And is forgotten; even then I felt  \\
Gleams like the flashing of a shield; the earth  \\
And common face of Nature spake to me  \\
Rememberable things; sometimes, 'tis true,	  \\
By chance collisions and quaint accidents  \\
Like those ill-sorted unions, work suppos'd  \\
Of evil-minded fairies, yet not vain  \\
Nor profitless, if haply they impress'd  \\
Collateral objects and appearances,	  \\
Albeit lifeless then, and doom'd to sleep  \\
Until maturer seasons call'd them forth  \\
To impregnate and to elevate the mind.  \\
---And if the vulgar joy by its own weight  \\
Wearied itself out of the memory,	  \\
The scenes which were a witness of that joy  \\
Remained, in their substantial lineaments  \\
Depicted on the brain, and to the eye  \\
Were visible, a daily sight; and thus  \\
By the impressive discipline of fear,	  \\
By pleasure and repeated happiness,  \\
So frequently repeated, and by force  \\
Of obscure feelings representative  \\
Of joys that were forgotten, these same scenes,  \\
So beauteous and majestic in themselves,	  \\
Though yet the day was distant, did at length  \\
Become habitually dear, and all  \\
Their hues and forms were by invisible links  \\
Allied to the affections. I began	  \\
My story early, feeling as I fear,  \\
The weakness of a human love, for days  \\
Disown'd by memory, ere the birth of spring  \\
Planting my snowdrops among winter snows.  \\
Nor will it seem to thee, my Friend! so prompt	  \\
In sympathy, that I have lengthen'd out,  \\
With fond and feeble tongue, a tedious tale.  \\
Meanwhile, my hope has been that I might fetch  \\
Invigorating thoughts from former years,  \\
Might fix the wavering balance of my mind,	  \\
And haply meet reproaches, too, whose power  \\
May spur me on, in manhood now mature,  \\
To honorable toil. Yet should these hopes  \\
Be vain, and thus should neither I be taught  \\
To understand myself, nor thou to know	  \\
With better knowledge how the heart was fram'd  \\
Of him thou lovest, need I dread from thee  \\
Harsh judgments, if I am so loth to quit  \\
Those recollected hours that have the charm  \\
Of visionary things, and lovely forms	  \\
And sweet sensations that throw back our life  \\
And almost make our Infancy itself  \\
A visible scene, on which the sun is shining?  \\!
One end hereby at least hath been attain'd,  \\
My mind hath been revived, and if this mood	  \\
Desert me not, I will forthwith bring down,  \\
Through later years, the story of my life.  \\
The road lies plain before me; 'tis a theme  \\
Single and of determined bounds; and hence  \\
I chuse it rather at this time, than work	  \\
Of ampler or more varied argument.  \\
{[}Where I might be discomfited, (and) lost  \\
And certain hopes are with me that to thee  \\
This Labour will be welcome, honoured friend.{]} \\
\end{verse}  % book first

\chapter*[Book Second]{Book Second \\ Childhood and School-time (Continued)}  
\addcontentsline{toc}{chapter}{Book Second Childhood and School-time (Continued)}  

\begin{verse}
THUS far, O Friend! have we, though leaving much  \\
Unvisited, endeavour'd to retrace  \\
My life through its first years, and measured back  \\
The way I travell'd when I first began  \\
To love the woods and fields; the passion yet	  \\
Was in its birth, sustain'd, as might befal,  \\
By nourishment that came unsought, for still,  \\
From week to week, from month to month, we liv'd  \\
A round of tumult: duly were our games  \\
Prolong'd in summer till the day-light fail'd;	  \\
No chair remain'd before the doors, the bench  \\
And threshold steps were empty; fast asleep  \\
The Labourer, and the old Man who had sate,  \\
A later lingerer, yet the revelry  \\
Continued, and the loud uproar: at last,	  \\
When all the ground was dark, and the huge  \\
clouds Were edged with twinkling stars, to bed we went,  \\
With weary joints, and with a beating mind.  \\
Ah! is there one who ever has been young,  \\
Nor needs a monitory voice to tame	  \\
The pride of virtue, and of intellect?  \\
And is there one, the wisest and the best  \\
Of all mankind, who does not sometimes wish  \\
For things which cannot be, who would not give,  \\
If so he might, to duty and to truth	  \\
The eagerness of infantine desire?  \\
A tranquillizing spirit presses now  \\
On my corporeal frame: so wide appears  \\
The vacancy between me and those days,  \\
Which yet have such self-presence in my mind	  \\
That, sometimes, when I think of them, I seem  \\
Two consciousnesses, conscious of myself  \\
And of some other Being. A grey Stone  \\
Of native rock, left midway in the Square  \\
Of our small market Village, was the home	  \\
And centre of these joys, and when, return'd  \\
After long absence, thither I repair'd,  \\
I found that it was split, and gone to build  \\
A smart Assembly-room that perk'd and flar'd  \\
With wash and rough-cast elbowing the ground	  \\
Which had been ours. But let the fiddle scream,  \\
And be ye happy! yet, my Friends! I know  \\
That more than one of you will think with me  \\
Of those soft starry nights, and that old Dame  \\
From whom the stone was nam'd who there had sate	  \\
And watch'd her Table with its huckster's wares  \\
Assiduous, thro' the length of sixty years.  \\
We ran a boisterous race; the year span round  \\
With giddy motion. But the time approach'd  \\
That brought with it a regular desire	  \\
For calmer pleasures, when the beauteous forms  \\
Of Nature were collaterally attach'd  \\
To every scheme of holiday delight,  \\
And every boyish sport, less grateful else,  \\
And languidly pursued.	  \\
When summer came  \\
It was the pastime of our afternoons  \\
To beat along the plain of Windermere  \\
With rival oars, and the selected bourne  \\
Was now an Island musical with birds	  \\
That sang for ever; now a Sister Isle  \\
Beneath the oaks' umbrageous covert, sown  \\
With lillies of the valley, like a field;  \\
And now a third small Island where remain'd  \\
An old stone Table, and a moulder'd Cave,	  \\
A Hermit's history. In such a race,  \\
So ended, disappointment could be none,  \\
Uneasiness, or pain, or jealousy:  \\
We rested in the shade, all pleas'd alike,  \\
Conquer'd and Conqueror. Thus the pride of strength,	  \\
And the vain-glory of superior skill  \\
Were interfus'd with objects which subdu'd  \\
And temper'd them, and gradually produc'd  \\
A quiet independence of the heart.  \\
And to my Friend, who knows me, I may add,	  \\
Unapprehensive of reproof, that hence  \\
Ensu'd a diffidence and modesty,  \\
And I was taught to feel, perhaps too much,  \\
The self-sufficing power of solitude.  \\
No delicate viands sapp'd our bodily strength;	  \\
More than we wish'd we knew the blessing then  \\
Of vigorous hunger, for our daily meals Were frugal,  \\
Sabine fare! and then, exclude  \\
A little weekly stipend, and we lived  \\
Through three divisions of the quarter'd year	  \\
In pennyless poverty. But now, to School  \\
Return'd, from the half-yearly holidays,  \\
We came with purses more profusely fill'd,  \\
Allowance which abundantly suffic'd  \\
To gratify the palate with repasts	  \\
More costly than the Dame of whom I spake,  \\
That ancient Woman, and her board supplied.  \\
Hence inroads into distant Vales, and long  \\
Excursions far away among the hills,  \\
Hence rustic dinners on the cool green ground,	  \\
Or in the woods, or near a river side,  \\
Or by some shady fountain, while soft airs  \\
Among the leaves were stirring, and the sun  \\
Unfelt, shone sweetly round us in our joy.  \\
Nor is my aim neglected, if I tell	  \\
How twice in the long length of those half-years  \\
We from our funds, perhaps, with bolder hand  \\
Drew largely, anxious for one day, at least,  \\
To feel the motion of the galloping Steed;  \\
And with the good old Inn-keeper, in truth,	  \\
On such occasion sometimes we employ'd  \\
Sly subterfuge; for the intended bound  \\
Of the day's journey was too distant far  \\
For any cautious man, a Structure famed  \\
Beyond its neighbourhood, the antique Walls	  \\
Of that large Abbey which within the vale  \\
Of Nightshade, to St. Mary's honour built,  \\
Stands yet, a mouldering Pile, with fractured  \\
Arch, Belfry, and Images, and living Trees,  \\
A holy Scene! along the smooth green turf	  \\
Our Horses grazed: to more than inland peace  \\
Left by the sea wind passing overhead  \\
(Though wind of roughest temper) trees and towers  \\
May in that Valley oftentimes be seen,  \\
Both silent and both motionless alike;	  \\
Such is the shelter that is there, and such  \\
The safeguard for repose and quietness.  \\
Our steeds remounted, and the summons given,  \\
With whip and spur we by the Chauntry flew  \\
In uncouth race, and left the cross-legg'd Knight,	  \\
And the stone-Abbot, and that single Wren  \\
Which one day sang so sweetly in the Nave  \\
Of the old Church, that, though from recent showers  \\
The earth was comfortless, and, touch'd by faint  \\
Internal breezes, sobbings of the place,	  \\
And respirations, from the roofless walls  \\
The shuddering ivy dripp'd large drops, yet still,  \\
So sweetly 'mid the gloom the invisible  \\
Bird Sang to itself, that there I could have made  \\
My dwelling-place, and liv'd for ever there	  \\
To hear such music. Through the Walls we flew  \\
And down the valley, and a circuit made  \\
In wantonness of heart, through rough and smooth  \\
We scamper'd homeward. Oh! ye Rocks and Streams,  \\
And that still Spirit of the evening air!	  \\
Even in this joyous time I sometimes felt  \\
Your presence, when with slacken'd step we breath'd  \\
Along the sides of the steep hills, or when,  \\
Lighted by gleams of moonlight from the sea,  \\
We beat with thundering hoofs the level sand.	  \\
Upon the Eastern Shore of Windermere,  \\
Above the crescent of a pleasant Bay,  \\
There stood an Inn, no homely-featured Shed,  \\
Brother of the surrounding Cottages,  \\
But 'twas a splendid place, the door beset	  \\
With Chaises, Grooms, and Liveries, and within  \\
Decanters, Glasses, and the blood-red Wine.  \\
In ancient times, or ere the Hall was built  \\
On the large Island, had this Dwelling been  \\
More worthy of a Poet's love, a Hut,	  \\
Proud of its one bright fire, and sycamore shade.  \\
But though the rhymes were gone which once inscribed  \\
The threshold, and large golden characters  \\
On the blue-frosted Signboard had usurp'd  \\
The place of the old Lion, in contempt	  \\
And mockery of the rustic painter's hand,  \\
Yet to this hour the spot to me is dear  \\
With all its foolish pomp. The garden lay  \\
Upon a slope surmounted by the plain  \\
Of a small Bowling-green; beneath us stood	  \\
A grove; with gleams of water through the trees  \\
And over the tree-tops; nor did we want  \\
Refreshment, strawberries and mellow cream.  \\
And there, through half an afternoon, we play'd  \\
On the smooth platform, and the shouts we sent	  \\
Made all the mountains ring. But ere the fall  \\
Of night, when in our pinnace we return'd  \\
Over the dusky Lake, and to the beach  \\
Of some small Island steer'd our course with one,  \\
The Minstrel of our troop, and left him there,	  \\
And row'd off gently, while he blew his flute  \\
Alone upon the rock; Oh! then the calm  \\
And dead still water lay upon my mind  \\
Even with a weight of pleasure, and the sky  \\
Never before so beautiful, sank down	  \\
Into my heart, and held me like a dream.  \\
Thus daily were my sympathies enlarged,  \\
And thus the common range of visible things  \\
Grew dear to me: already I began  \\
To love the sun, a Boy I lov'd the sun,	  \\
Not as I since have lov'd him, as a pledge  \\
And surety of our earthly life, a light  \\
Which while we view we feel we are alive;  \\
But, for this cause, that I had seen him lay  \\
His beauty on the morning hills, had seen	  \\
The western mountain touch his setting orb,  \\
In many a thoughtless hour, when, from excess  \\
Of happiness, my blood appear'd to flow  \\
With its own pleasure, and I breath'd with joy.  \\
And from like feelings, humble though intense,	  \\
To patriotic and domestic love  \\
Analogous, the moon to me was dear;  \\
For I would dream away my purposes,  \\
Standing to look upon her while she hung  \\
Midway between the hills, as if she knew	  \\
No other region; but belong'd to thee,  \\
Yea, appertain'd by a peculiar right  \\
To thee and thy grey huts, my darling Vale!  \\
Those incidental charms which first attach'd  \\
My heart to rural objects, day by day	  \\
Grew weaker, and I hasten on to tell How  \\
Nature, intervenient till this time,  \\
And secondary, now at length was sought  \\
For her own sake. But who shall parcel out  \\
His intellect, by geometric rules,	  \\
Split, like a province, into round and square?  \\
Who knows the individual hour in which  \\
His habits were first sown, even as a seed,  \\
Who that shall point, as with a wand, and say,  \\
'This portion of the river of my mind	  \\
Came from yon fountain?' Thou, my Friend! art one  \\
More deeply read in thy own thoughts; to thee  \\
Science appears but, what in truth she is,  \\
Not as our glory and our absolute boast,  \\
But as a succedaneum, and a prop	  \\
To our infirmity. Thou art no slave  \\
Of that false secondary power, by which,  \\
In weakness, we create distinctions, then  \\
Deem that our puny boundaries are things  \\
Which we perceive, and not which we have made.	  \\
To thee, unblinded by these outward shows,  \\
The unity of all has been reveal'd  \\
And thou wilt doubt with me, less aptly skill'd  \\
Than many are to class the cabinet  \\
Of their sensations, and, in voluble phrase,	  \\
Run through the history and birth of each,  \\
As of a single independent thing.  \\
Hard task to analyse a soul, in which,  \\
Not only general habits and desires,  \\
But each most obvious and particular thought,	  \\
Not in a mystical and idle sense,  \\
But in the words of reason deeply weigh'd,  \\
Hath no beginning.  \\
Bless'd the infant Babe,  \\
(For with my best conjectures I would trace	  \\
The progress of our Being) blest the Babe,  \\
Nurs'd in his Mother's arms, the  \\
Babe who sleeps Upon his Mother's breast,  \\
who, when his soul  \\
Claims manifest kindred with an earthly soul,  \\
Doth gather passion from his Mother's eye!	  \\
Such feelings pass into his torpid life  \\
Like an awakening breeze, and hence his mind  \\
Even [in the first trial of its powers]  \\
Is prompt and watchful, eager to combine  \\
In one appearance, all the elements	  \\
And parts of the same object, else detach'd  \\
And loth to coalesce. Thus, day by day,  \\
Subjected to the discipline of love,  \\
His organs and recipient faculties  \\
Are quicken'd, are more vigorous, his mind spreads,	  \\
Tenacious of the forms which it receives.  \\
In one beloved presence, nay and more,  \\
In that most apprehensive habitude  \\
And those sensations which have been deriv'd  \\
From this beloved Presence, there exists	  \\
A virtue which irradiates and exalts  \\
All objects through all intercourse of sense.  \\
No outcast he, bewilder'd and depress'd;  \\
Along his infant veins are interfus'd  \\
The gravitation and the filial bond	  \\
Of nature, that connect him with the world.  \\
Emphatically such a Being lives,  \\
An inmate of this active universe;  \\
From nature largely he receives; nor so  \\
Is satisfied, but largely gives again,	  \\
For feeling has to him imparted strength,  \\
And powerful in all sentiments of grief,  \\
Of exultation, fear, and joy, his mind,  \\
Even as an agent of the one great mind,  \\
Creates, creator and receiver both,	  \\
Working but in alliance with the works  \\
Which it beholds.�CSuch, verily, is the first  \\
Poetic spirit of our human life;  \\
By uniform control of after years  \\
In most abated or suppress'd, in some,	  \\
Through every change of growth or of decay,  \\
Pre-eminent till death. From early days,  \\
Beginning not long after that first time  \\
In which, a Babe, by intercourse of touch,	  \\
I held mute dialogues with my Mother's heart  \\
I have endeavour'd to display the means  \\
Whereby this infant sensibility,  \\
Great birthright of our Being, was in me  \\
Augmented and sustain'd. Yet is a path	  \\
More difficult before me, and I fear  \\
That in its broken windings we shall need  \\
The chamois' sinews, and the eagle's wing:  \\
For now a trouble came into my mind  \\
From unknown causes. I was left alone,	  \\
Seeking the visible world, nor knowing why.  \\
The props of my affections were remov'd,  \\
And yet the building stood, as if sustain'd  \\
By its own spirit! All that I beheld  \\
Was dear to me, and from this cause it came,	  \\
That now to Nature's finer influxes  \\
My mind lay open, to that more exact  \\
And intimate communion which our hearts  \\
Maintain with the minuter properties  \\
Of objects which already are belov'd,	  \\
And of those only. Many are the joys  \\
Of youth; but oh! what happiness to live  \\
When every hour brings palpable access  \\
Of knowledge, when all knowledge is delight,  \\
And sorrow is not there. The seasons came,	  \\
And every season to my notice brought  \\
A store of transitory qualities  \\
Which, but for this most watchful power of love  \\
Had been neglected, left a register  \\
Of permanent relations, else unknown,	  \\
Hence life, and change, and beauty, solitude  \\
More active, even, than 'best society',  \\
Society made sweet as solitude  \\
By silent inobtrusive sympathies,  \\
And gentle agitations of the mind	  \\
From manifold distinctions, difference  \\
Perceived in things, where to the common eye,  \\
No difference is; and hence, from the same source  \\
Sublimer joy; for I would walk alone,  \\
In storm and tempest, or in starlight nights	  \\
Beneath the quiet Heavens; and, at that time,  \\
Have felt whate'er there is of power in sound  \\
To breathe an elevated mood, by form  \\
Or image unprofaned; and I would stand,  \\
Beneath some rock, listening to sounds that are	  \\
The ghostly language of the ancient earth,  \\
Or make their dim abode in distant winds.  \\
Thence did I drink the visionary power.  \\
I deem not profitless those fleeting moods  \\
Of shadowy exultation: not for this,	  \\
That they are kindred to our purer mind  \\
And intellectual life; but that the soul,  \\
Remembering how she felt, but what she felt  \\
Remembering not, retains an obscure sense  \\
Of possible sublimity, to which,	  \\
With growing faculties she doth aspire,  \\
With faculties still growing, feeling still  \\
That whatsoever point they gain, they still  \\
Have something to pursue. And not alone,	  \\
In grandeur and in tumult, but no less  \\
In tranquil scenes, that universal power  \\
And fitness in the latent qualities  \\
And essences of things, by which the mind  \\
Is mov'd by feelings of delight, to me	  \\
Came strengthen'd with a superadded soul,  \\
A virtue not its own.  \\
My morning walks Were early; oft, before the hours of  \\
School I travell'd round our little Lake, five miles  \\
Of pleasant wandering, happy time! more dear	  \\
For this, that one was by my side, a  \\
Friend Then passionately lov'd; with heart how full  \\
Will he peruse these lines, this page, perhaps  \\
A blank to other men! for many years  \\
Have since flow'd in between us; and our minds,	  \\
Both silent to each other, at this time  \\
We live as if those hours had never been.  \\
Nor seldom did I lift our cottage latch  \\
Far earlier, and before the vernal thrush  \\
Was audible, among the hills I sate	  \\
Alone, upon some jutting eminence  \\
At the first hour of morning, when the Vale  \\
Lay quiet in an utter solitude.  \\
How shall I trace the history, where seek  \\
The origin of what I then have felt?	  \\
Oft in these moments such a holy calm  \\
Did overspread my soul, that I forgot  \\
That I had bodily eyes, and what I saw  \\
Appear'd like something in myself, a dream,  \\
A prospect in my mind.	  \\
'Twere long to tell  \\
What spring and autumn, what the winter snows,  \\
And what the summer shade, what day and night,  \\
The evening and the morning, what my dreams  \\
And what my waking thoughts supplied, to nurse	  \\
That spirit of religious love in which  \\
I walked with Nature. But let this, at least  \\
Be not forgotten, that I still retain'd  \\
My first creative sensibility,  \\
That by the regular action of the world	  \\
My soul was unsubdu'd. A plastic power  \\
Abode with me, a forming hand, at times  \\
Rebellious, acting in a devious mood,  \\
A local spirit of its own, at war  \\
With general tendency, but for the most	  \\
Subservient strictly to the external things  \\
With which it commun'd. An auxiliar light  \\
Came from my mind which on the setting sun  \\
Bestow'd new splendor, the melodious birds,  \\
The gentle breezes, fountains that ran on,	  \\
Murmuring so sweetly in themselves, obey'd  \\
A like dominion; and the midnight storm  \\
Grew darker in the presence of my eye.  \\
Hence by obeisance, my devotion hence,  \\
And hence my transport.	  \\
Nor should this, perchance,  \\
Pass unrecorded, that I still have lov'd  \\
The exercise and produce of a toil  \\
Than analytic industry to me  \\
More pleasing, and whose character I deem	  \\
Is more poetic as resembling more  \\
Creative agency. I mean to speak  \\
Of that interminable building rear'd  \\
By observation of affinities  \\
In objects where no brotherhood exists	  \\
To common minds. My seventeenth year was come  \\
And, whether from this habit, rooted now  \\
So deeply in my mind, or from excess  \\
Of the great social principle of life,  \\
Coercing all things into sympathy,	  \\
To unorganic natures I transferr'd  \\
My own enjoyments, or, the power of truth  \\
Coming in revelation, I convers'd  \\
With things that really are, I, at this time  \\
Saw blessings spread around me like a sea.	  \\
Thus did my days pass on, and now at length  \\
From Nature and her overflowing soul  \\
I had receiv'd so much that all my thoughts  \\
Were steep'd in feeling; I was only then  \\
Contented when with bliss ineffable	  \\
I felt the sentiment of Being spread  \\
O'er all that moves, and all that seemeth still,  \\
O'er all, that, lost beyond the reach of thought  \\
And human knowledge, to the human eye  \\
Invisible, yet liveth to the heart,	  \\
O'er all that leaps, and runs, and shouts, and sings,  \\
Or beats the gladsome air, o'er all that glides  \\
Beneath the wave, yea, in the wave itself  \\
And mighty depth of waters. Wonder not  \\
If such my transports were; for in all things	  \\
I saw one life, and felt that it was joy.  \\
One song they sang, and it was audible,  \\
Most audible then when the fleshly ear,  \\
O'ercome by grosser prelude of that strain,  \\
Forgot its functions, and slept undisturb'd.	  \\
If this be error, and another faith  \\
Find easier access to the pious mind,  \\
Yet were I grossly destitute of all  \\
Those human sentiments which make this earth  \\
So dear, if I should fail, with grateful voice	  \\
To speak of you, Ye Mountains and Ye Lakes,  \\
And sounding Cataracts! Ye Mists and Winds  \\
That dwell among the hills where I was born.  \\
If, in my youth, I have been pure in heart,  \\
If, mingling with the world, I am content	  \\
With my own modest pleasures, and have liv'd,  \\
With God and Nature communing, remov'd  \\
From little enmities and low desires,  \\
The gift is yours; if in these times of fear,  \\
This melancholy waste of hopes o'erthrown,	  \\
If, 'mid indifference and apathy  \\
And wicked exultation, when good men,  \\
On every side fall off we know not how,  \\
To selfishness, disguis'd in gentle names  \\
Of peace, and quiet, and domestic love,	  \\
Yet mingled, not unwillingly, with sneers  \\
On visionary minds; if in this time  \\
Of dereliction and dismay, I yet  \\
Despair not of our nature; but retain  \\
A more than Roman confidence, a faith	  \\
That fails not, in all sorrow my support,  \\
The blessing of my life, the gift is yours,  \\
Ye mountains! thine, O Nature! Thou hast fed  \\
My lofty speculations; and in thee,  \\
For this uneasy heart of ours I find	  \\
A never-failing principle of joy, And purest passion.  \\
Thou, my Friend! wert rear'd  \\
In the great City, 'mid far other scenes;  \\
But we, by different roads at length have gain'd	  \\
The self-same bourne. And for this cause to Thee  \\
I speak, unapprehensive of contempt,  \\
The insinuated scoff of coward tongues,  \\
And all that silent language which so oft  \\
In conversation betwixt man and man	  \\
Blots from the human countenance all trace  \\
Of beauty and of love. For Thou hast sought  \\
The truth in solitude, and Thou art one,  \\
The most intense of Nature's worshippers  \\
In many things my Brother, chiefly here	  \\
In this my deep devotion.  \\
Fare Thee well!  \\
Health, and the quiet of a healthful mind  \\
Attend thee! seeking oft the haunts of men,  \\
And yet more often living with Thyself,	  \\
And for Thyself, so haply shall thy days  \\
Be many, and a blessing to mankind. \\
\end{verse}

%%%%%%%%%%%%%%%%%%%%%%%%%%%%%%%%%%%%%%%%%%%%%%%%%%%%%%%%%%%%%%%%% \\
\chapter*[Book Third]{Book Third \\ Residence at Cambridge}
\addcontentsline{toc}{chapter}{Book Third Residence at Cambridge}

\begin{verse}
IT was a dreary morning when the chaise  \\
Rolled over the flat plains of Huntingdon  \\
And through the open windows first I saw  \\
The long-backed chapel of King's College rear  \\
His pinnacles above the dusky groves.	  \\
Soon afterwards we espied upon the road  \\
A student clothed in gown and tasselled cap;  \\
He passed---nor was I master of my eyes  \\
Till he was left a hundred yards behind.  \\
The place as we approached seemed more and more	  \\
To have an eddy's force, and sucked us in  \\
More eagerly at every step we took.  \\
Onward we drove beneath the castle, down  \\
By Magdalene Bridge we went and crossed the Cam,  \\
And at the Hoop we landed, famous inn.	  \\
My spirit was up, my thoughts were full of hope;  \\
Some friends I had---acquaintances who there  \\
Seemed friends---poor simple schoolboys now hung round  \\
With honour and importance. In a world  \\
Of welcome faces up and down I roved---	  \\
Questions, directions, counsel and advice  \\
Flowed in upon me from all sides.  \\
Fresh day Of pride and pleasure: to myself I seemed  \\
A man of business and expense, and went  \\
From shop to shop about my own affairs,	  \\
To tutors or to tailors as befel,  \\
From street to street with loose and careless heart.  \\
I was the dreamer, they the dream; I roamed  \\
Delighted through the motley spectacle:  \\
Gowns grave or gaudy, doctors, students, streets,	  \\
Lamps, gateways, flocks of churches, courts and towers---  \\
Strange transformation for a mountain youth,  \\
A northern villager. As if by word  \\
Of magic or some fairy's power, at once  \\
Behold me rich in monies and attired	  \\
In splendid clothes, with hose of silk, and hair  \\
Glittering like rimy trees when frost is keen---  \\
My lordly dressing-gown, I pass it by,  \\
With other signs of manhood which supplied  \\
The lack of beard. The weeks went roundly on,	  \\
With invitations, suppers, wine, and fruit,  \\
Smooth housekeeping within, and all without  \\
Liberal and suiting gentleman's array.  \\
The Evangelist St. John my patron was;  \\
Three gloomy courts are his, and in the first	  \\
Was my abiding-place, a nook obscure.  \\
Right underneath, the college kitchens made  \\
A humming sound, less tuneable than bees  \\
But hardly less industrious; with shrill notes  \\
Of sharp command and scolding intermixed.	  \\
Near me was Trinity's loquacious clock  \\
Who never let the quarters, night or day,  \\
Slip by him unproclaimed, and told the hours  \\
Twice over with a male and female voice.  \\
Her pealing organ was my neighbour too;	  \\
And from my bedroom I in moonlight nights  \\
Could see right opposite, a few yards off,  \\
The antechapel, where the statue stood  \\
Of Newton with his prism and silent face.  \\
Of college labours, of the lecturer's room	  \\
All studded round, as thick as chairs could stand,  \\
With loyal students faithful to their books,  \\
Half-and-half idlers, hardy recusants,  \\
And honest dunces; of important days,  \\
Examinations, when the man was weighed	  \\
As in the balance of excessive hopes,  \\
Tremblings withal and commendable fears,  \\
Small jealousies and triumphs good or bad---  \\
I make short mention. Things they were which then  \\
I did not love, nor do I love them now:	  \\
Such glory was but little sought by me,  \\
And little won. But it is right to say  \\
That even so early, from the first crude days  \\
Of settling-time in this my new abode,  \\
Not seldom I had melancholy thoughts	  \\
From personal and family regards,  \\
Wishing to hope without a hope---some fears  \\
About my future worldly maintenance,  \\
And, more than all, a strangeness in my mind,  \\
A feeling that I was not for that hour	  \\
Nor for that place. But wherefore be cast down,  \\
Why should I grieve?---I was a chosen son.  \\
For hither I had come with holy powers  \\
And faculties, whether to work or feel:  \\
To apprehend all passions and all moods	  \\
Which time, and place, and season do impress  \\
Upon the visible universe, and work  \\
Like changes there by force of my own mind.  \\
I was a freeman, in the purest sense  \\
Was free, and to majestic ends was strong---	  \\
I do not speak of learning, moral truth,  \\
Or understanding---'twas enough for me  \\
To know that I was otherwise endowed.  \\
When the first glitter of the show was passed,  \\
And the first dazzle of the taper-light,	  \\
As if with a rebound my mind returned  \\
Into its former self. Oft did I leave  \\
My comrades, and the crowd, buildings and groves,  \\
And walked along the fields, the level fields,  \\
With heaven's blue concave reared above my head.	  \\
And now it was that through such change entire,  \\
And this first absence from those shapes sublime  \\
Wherewith I had been conversant, my mind  \\
Seemed busier in itself than heretofore---  \\
At least I more directly recognised	  \\
My powers and habits. Let me dare to speak  \\
A higher language, say that now I felt  \\
The strength and consolation which were mine.  \\
As if awakened, summoned, rouzed, constrained,  \\
I looked for universal things, perused	  \\
The common countenance of earth and heaven,  \\
And, turning the mind in upon itself,  \\
Pored, watched, expected, listened, spread my thoughts,  \\
And spread them with a wider creeping, felt  \\
Incumbencies more awful, visitings	  \\
Of the upholder, of the tranquil soul,  \\
Which underneath all passion lives secure  \\
A steadfast life. But peace, it is enough  \\
To notice that I was ascending now  \\
To such community with highest truth.	  \\
A track pursuing not untrod before,  \\
From deep analogies by thought supplied,  \\
Or consciousnesses not to be subdued,  \\
To every natural form, rock, fruit or flower,  \\
Even the loose stones that cover the highway,	  \\
I gave a moral life---I saw them feel,  \\
Or linked them to some feeling. The great mass  \\
Lay bedded in a quickening soul, and all  \\
That I beheld respired with inward meaning.  \\
Thus much for the one presence, and the life	  \\
Of the great whole; suffice it here to add  \\
That whatsoe'er of terror, or of love,  \\
Or beauty, Nature's daily face put on  \\
From transitory passion, unto this  \\
I was as wakeful even as waters are	  \\
To the sky's motion, in a kindred sense  \\
Of passion was obedient as a lute  \\
That waits upon the touches of the wind.  \\
So it was with me in my solitude:  \\
So often among multitudes of men.	  \\
Unknown, unthought of, yet I was most rich,  \\
I had a world about me---'twas my own,  \\
I made it; for it only lived to me,  \\
And to the God who looked into my mind.  \\
Such sympathies would sometimes shew themselves	  \\
By outward gestures and by visible looks---  \\
Some called it madness; such indeed it was,  \\
If childlike fruitfulness in passing joy,  \\
If steady moods of thoughtfulness matured  \\
To inspiration, sort with such a name;	  \\
If prophesy be madness, if things viewed  \\
By poets of old time, and higher up  \\
By the first men, earth's first inhabitants,  \\
May in these tutored days no more be seen  \\
With undisordered sight. But leaving this,	  \\
It was no madness, for I had an eye  \\
Which in my strongest workings evermore  \\
Was looking for the shades of difference  \\
As they lie hid in all exterior forms,  \\
Near or remote, minute or vast---an eye	  \\
Which from a stone, a tree, a withered leaf,  \\
To the broad ocean and the azure heavens  \\
Spangled with kindred multitudes of stars,  \\
Could find no surface where its power might sleep,  \\
Which spake perpetual logic to my soul,	  \\
And by an unrelenting agency  \\
Did bind my feelings even as in a chain.  \\
And here, O friend, have I retraced my life  \\
Up to an eminence, and told a tale  \\
Of matters which not falsely I may call	  \\
The glory of my youth. Of genius, power,  \\
Creation, and divinity itself,  \\
I have been speaking, for my theme has been  \\
What passed within me. Not of outward things  \\
Done visibly for other minds---words, signs,	  \\
Symbols or actions---but of my own heart  \\
Have I been speaking, and my youthful mind.  \\
O heavens, how awful is the might of souls,  \\
And what they do within themselves while yet  \\
The yoke of earth is new to them, the world	  \\
Nothing but a wild field where they were sown.  \\
This is in truth heroic argument,  \\
And genuine prowess---which I wished to touch,  \\
With hand however weak---but in the main  \\
It lies far hidden from the reach of words.	  \\
Points have we all of us within our souls  \\
Where all stand single; this I feel, and make  \\
Breathings for incommunicable powers.  \\
Yet each man is a memory to himself,  \\
And, therefore, now that I must quit this theme,	  \\
I am not heartless; for there's not a man  \\
That lives who hath not had his god-like hours,  \\
And knows not what majestic sway we have  \\
As natural beings in the strength of Nature.  \\
Enough, for now into a populous plain	  \\
We must descend. A traveller I am,  \\
And all my tale is of myself---even so---  \\
So be it, if the pure in heart delight  \\
To follow me, and thou, O honoured friend,  \\
Who in my thoughts art ever at my side,	  \\
Uphold as heretofore my fainting steps.  \\
It hath been told already how my sight  \\
Was dazzled by the novel show, and how  \\
Erelong I did into myself return.  \\
So did it seem, and so in truth it was---	  \\
Yet this was but short-lived.  \\
Thereafter came Observance less devout:  \\
I had made a change  \\
In climate, and my nature's outward coat  \\
Changed also, slowly and insensibly.  \\
To the deep quiet and majestic thoughts	  \\
Of loneliness succeeded empty noise  \\
And superficial pastimes, now and then  \\
Forced labour, and more frequently forced hopes,  \\
And, worse than all, a treasonable growth  \\
Of indecisive judgements that impaired	  \\
And shook the mind's simplicity. And yet  \\
This was a gladsome time. Could I behold---  \\
Who less insensible than sodden clay  \\
On a sea-river's bed at ebb of tide  \\
Could have beheld---with undelighted heart	  \\
so many happy youths, so wide and fair  \\
A congregation in its budding-time  \\
Of health, and hope, and beauty, all at once  \\
So many divers samples of the growth  \\
Of life's sweet season, could have seen unmoved	  \\
That miscellaneous garland of wild flowers  \\
Upon the matron temples of a place  \\
So famous through the world? To me at least  \\
It was a goodly prospect; for, through youth,  \\
Though I had been trained up to stand unpropped,	  \\
And independent musings pleased me so  \\
That spells seemed on me when I was alone,  \\
Yet could I only cleave to solitude  \\
In lonesome places---if a throng was near  \\
That way I leaned by nature, for my heart	  \\
Was social and loved idleness and joy.  \\
Not seeking those who might participate  \\
My deeper pleasures---nay, I had not once,  \\
Though not unused to mutter lonesome songs,	  \\
Even with myself divided such delight,  \\
Or looked that way for aught that might be clothed  \\
In human language---easily I passed  \\
From the remembrances of better things,  \\
And slipped into the weekday works of youth,	  \\
Unburthened, unalarmed, and unprofaned.  \\
Caverns there were within my mind which sun  \\
Could never penetrate, yet did there not  \\
Want store of leafy arbours where the light  \\
Might enter in at will. Companionships,	  \\
Friendships, acquaintances, were welcome all;  \\
We sauntered, played, we rioted, we talked  \\
Unprofitable talk at morning hours,  \\
Drifted about along the streets and walks,  \\
Read lazily in lazy books, went forth	  \\
To gallop through the country in blind zeal  \\
Of senseless horsemanship, or on the breast Of  \\
Cam sailed boisterously, and let the stars  \\
Come out, perhaps without one quiet thought.  \\
Such was the tenor of the opening act	  \\
In this new life. Imagination slept,  \\
And yet not utterly: I could not print  \\
Ground where the grass had yielded to the steps  \\
Of generations of illustrious men,  \\
Unmoved; I could not always lightly pass	  \\
Through the same gateways, sleep where they had slept,  \\
Wake where they waked, range that enclosure old,  \\
That garden of great intellects, undisturbed.  \\
Place also by the side of this dark sense  \\
Of nobler feeling, that those spiritual men,	  \\
Even the great Newton's own etherial self,  \\
Seemed humbled in these precincts, thence to be  \\
The more beloved, invested here with tasks  \\
Of life's plain business, as a daily garb---  \\
Dictators at the plough---a change that left	  \\
All genuine admiration unimpaired.  \\
Beside the pleasant mills of Trompington  \\
I laughed with Chaucer; in the hawthorn shade  \\
Heard him, while birds were warbling, tell his tales  \\
Of amorous passion. And that gentle bard	  \\
Chosen by the Muses for their Page of State,  \\
Sweet Spencer, moving through his clouded heaven  \\
With the moon's beauty and the moon's soft pace---  \\
I called him brother, Englishman, and friend.  \\
Yea, our blind poet, who, in his later day	  \\
Stood almost single, uttering odious truth,  \\
Darkness before, and danger's voice behind---  \\
Soul awful, if the earth hath ever lodged  \\
An awful soul--- I seemed to see him here  \\
Familiarly, and in his scholar's dress	  \\
Bounding before me, yet a stripling youth,  \\
A boy, no better, with his rosy cheeks  \\
Angelical, keen eye, courageous look,  \\
And conscious step of purity and pride.  \\
Among the band of my compeers was one,	  \\
My class-fellow at school, whose chance it was  \\
To lodge in the apartments which had been  \\
Time out of mind honored by Milton's name---  \\
The very shell reputed of the abode  \\
Which he had tenanted. O Temperate bard!	  \\
One afternoon, the first time I set foot  \\
In this thy innocent nest and oratory,  \\
Seated with others in a festive ring  \\
Of commonplace convention, I to thee  \\
Poured out libations, to thy memory drank	  \\
Within my private thoughts, till my brain reeled,  \\
Never so clouded by the fumes of wine  \\
Before that hour, or since. Thence, forth I ran  \\
From that assembly, through a length of streets  \\
Ran ostrich-like to reach our chapel door	  \\
In not a desperate or opprobrious time,  \\
Albeit long after the importunate bell  \\
Had stopped, with wearisome Cassandra voice  \\
No longer haunting the dark winter night.  \\
Call back, O friend, a moment to thy mind	  \\
The place itself and fashion of the rites.  \\
Upshouldering in a dislocated lump  \\
With shallow ostentatious carelessness  \\
My surplice, gloried in and yet despised,  \\
I clove in pride through the inferior throng	  \\
Of the plain burghers, who in audience stood  \\
On the last skirts of their permitted ground,  \\
Beneath the pealing organ. Empty thoughts,  \\
I am ashamed of them; and that great bard,  \\
And thou, O friend, who in thy ample mind	  \\
Hast stationed me for reverence and love,  \\
Ye will forgive the weakness of that hour,  \\
In some of its unworthy vanities  \\
Brother of many more. In this mixed sort	  \\
The months passed on, remissly, not given up  \\
To wilful alienation from the right,  \\
Or walks of open scandal, but in vague  \\
And loose indifference, easy likings, aims  \\
Of a low pitch---duty and zeal dismissed,	  \\
Yet Nature, or a happy course of things,  \\
Not doing in their stead the needful work.  \\
The memory languidly revolved, the heart  \\
Reposed in noontide rest, the inner pulse  \\
Of contemplation almost failed to beat.	  \\
Rotted as by a charm, my life became  \\
A floating island, an amphibious thing,  \\
Unsound, of spungy texture, yet withal  \\
Not wanting a fair face of water-weeds  \\
And pleasant flowers. The thirst of living praise,	  \\
A reverence for the glorious dead, the sight  \\
Of those long vistos, catacombs in which  \\
Perennial minds lie visibly entombed,  \\
Have often stirred the heart of youth, and bred  \\
A fervent love of rigorous discipline.	  \\
Alas, such high commotion touched not me;  \\
No look was in these walls to put to shame  \\
My easy spirits, and discountenance  \\
Their light composure---far less to instil  \\
A calm resolve of mind, firmly addressed	  \\
To puissant efforts. Nor was this the blame  \\
Of others, but my own; I should in truth,  \\
As far as doth concern my single self,  \\
Misdeem most widely, lodging it elsewhere.  \\
For I, bred in Nature's lap, was even	  \\
As a spoiled child; and, rambling like the wind  \\
As I had done in daily intercourse  \\
With those delicious rivers, solemn heights,  \\
And mountains, ranging like a fowl of the air,  \\
I was ill-tutored for captivity---	  \\
To quit my pleasure, and from month to month  \\
Take up a station calmly on the perch  \\
Of sedentary peace. Those lovely forms  \\
Had also left less space within my mind,  \\
Which, wrought upon instinctively, had found	  \\
A freshness in those objects of its love,  \\
A winning power beyond all other power.  \\
Not that I slighted books---that were to lack  \\
All sense---but other passions had been mine,  \\
More fervent, making me less prompt perhaps	  \\
To indoor study than was wise or well,  \\
Or suited to my years. Yet I could shape  \\
The image of a place which---soothed and lulled  \\
As I had been, trained up in paradise  \\
Among sweet garlands and delightful sounds,	  \\
Accustomed in my loneliness to walk  \\
With Nature magisterially---yet I  \\
Methinks could shape the image of a place  \\
Which with its aspect should have bent me down  \\
To instantaneous service, should at once	  \\
Have made me pay to science and to arts  \\
And written lore, acknowledged my liege lord,  \\
A homage frankly offered up like that  \\
Which I had paid to Nature. Toil and pains  \\
In this recess which I have bodied forth	  \\
Should spread from heart to heart; and stately groves,  \\
Majestic edifices, should not want  \\
A corresponding dignity within.  \\
The congregating temper which pervades  \\
Our unripe years, not wasted, should be made	  \\
To minister to works of high attempt,  \\
Which the enthusiast would perform with love.  \\
Youth should be awed, possessed, as with a sense  \\
Religious, of what holy joy there is  \\
In knowledge if it be sincerely sought	  \\
For its own sake---in glory, and in praise,  \\
If but by labour won, and to endure.  \\
The passing day should learn to put aside  \\
Her trappings here, should strip them off abashed  \\
Before antiquity and stedfast truth,	  \\
And strong book-mindedness; and over all  \\
Should be a healthy sound simplicity,  \\
A seemly plainness---name it as you will,  \\
Republican or pious. If these thoughts	  \\
Be a gratuitous emblazonry  \\
That does but mock this recreant age, at least  \\
Let Folly and False-seeming (we might say)  \\
Be free to affect whatever formal gait  \\
Of moral or scholastic discipline	  \\
Shall raise them highest in their own esteem;  \\
Let them parade among the schools at will,  \\
But spare the house of God. Was ever known  \\
The witless shepherd who would drive his flock  \\
With serious repetition to a pool	  \\
Of which 'tis plain to sight they never taste?  \\
A weight must surely hang on days begun  \\
And ended with worst mockery. Be wise,  \\
Ye Presidents and Deans, and to your bells  \\
Give seasonable rest, for 'tis a sound	  \\
Hollow as ever vexed the tranquil air,  \\
And your officious doings bring disgrace  \\
On the plain steeples of our English Church,  \\
Whose worship, 'mid remotest village trees,  \\
Suffers for this. Even science too, at hand	  \\
In daily sight of such irreverence,  \\
Is smitten thence with an unnatural taint,  \\
Loses her just authority, falls beneath  \\
Collateral suspicion, else unknown.  \\
This obvious truth did not escape me then,	  \\
Unthinking as I was, and I confess  \\
That---having in my native hills given loose  \\
To a schoolboy's dreaming---I had raised a pile  \\
Upon the basis of the coming time  \\
Which now before me melted fast away,   \\
Which could not live, scarcely had life enough  \\
To mock the builder. Oh, what joy it were  \\
To see a sanctuary for our country's youth  \\
With such a spirit in it as might be  \\
Protection for itself, a virgin grove,   \\
Primaeval in its purity and depth---  \\
Where, though the shades were filled with chearfulness,  \\
Nor indigent of songs warbled from crowds  \\
In under-coverts, yet the countenance  \\
Of the whole place should wear a stamp of awe---   \\
A habitation sober and demure  \\
For ruminating creatures, a domain  \\
For quiet things to wander in, a haunt  \\
In which the heron might delight to feed  \\
By the shy rivers, and the pelican   \\
Upon the cypress-spire in lonely thought  \\
Might sit and sun himself. Alas, alas,  \\
In vain for such solemnity we look;  \\
Our eyes are crossed by butterflies, our ears  \\
Hear chattering popinjays---the inner heart   \\
Is trivial, and the impresses without  \\
Are of a gaudy region. Different sight  \\
Those venerable doctors saw of old   \\
When all who dwelt within these famous walls  \\
Led in abstemiousness a studious life,  \\
When, in forlorn and naked chambers cooped  \\
And crowded, o'er their ponderous books they sate  \\
Like caterpillars eating out their way   \\
In silence, or with keen devouring noise  \\
Not to be tracked or fathered. Princes then  \\
At matins froze, and couched at curfew-time,  \\
Trained up through piety and zeal to prize  \\
Spare diet, patient labour, and plain weeds.	  \\
O seat of Arts, renowned throughout the world,  \\
Far different service in those homely days  \\
The nurslings of the Muses underwent  \\
From their first childhood. In that glorious time  \\
When Learning, like a stranger come from far,	  \\
Sounding through Christian lands her trumpet, rouzed  \\
The peasant and the king; when boys and youths,  \\
The growth of ragged villages and huts,  \\
Forsook their homes and---errant in the quest  \\
Of patron, famous school or friendly nook,	  \\
Where, pensioned, they in shelter might sit down---  \\
From town to town and through wide scattered realms  \\
Journeyed with their huge folios in their hands,  \\
And often, starting from some covert place,  \\
Saluted the chance comer on the road,	  \\
Crying, 'An obolus, a penny give  \\
To a poor scholar'; when illustrious men,  \\
Lovers of truth, by penury constrained,  \\
Bucer, Erasmus, or Melancthon, read  \\
Before the doors or windows of their cells	  \\
By moonshine through mere lack of taper light.  \\
But peace to vain regrets. We see but darkly  \\
Even when we look behind us; and best things  \\
Are not so pure by nature that they needs  \\
Must keep to all---as fondly all believe---	  \\
Their highest promise. If the mariner,  \\
When at reluctant distance he hath passed  \\
Some fair enticing island, did but know  \\
What fate might have been his, could he have brought  \\
His bark to land upon the wished-for spot,	  \\
Good cause full often would he have to bless  \\
The belt of churlish surf that scared him thence,  \\
Or haste of the inexorable wind.  \\
For me, I grieve not; happy is the man  \\
Who only misses what I missed, who falls	  \\
No lower than I fell. I did not love,  \\
As hath been notice heretofore, the guise  \\
Of our scholastic studies---could have wished  \\
The river to have had an ampler range  \\
And freer pace. But this I tax not; far,	  \\
Far more I grieved to see among the band  \\
Of those who in the field of contest stood  \\
As combatants, passions that did to me  \\
Seem low and mean---from ignorance of mine,  \\
In part, and want of just forbearance; yet	  \\
My wiser mind grieves now for what I saw.  \\
Willingly did I part from these, and turn  \\
Out of their track to travel with the shoal  \\
Of more unthinking natures, easy minds  \\
And pillowy, and not wanting love that makes	  \\
The day pass lightly on, when foresight sleeps,  \\
And wisdom and the pledges interchanged  \\
With our own inner being, are forgot.  \\
To books, our daily fare prescribed, I turned  \\
With sickly appetite; and when I went,	  \\
At other times, in quest of my own food,  \\
I chaced not steadily the manly deer,  \\
But laid me down to any casual feast  \\
Of wild wood-honey; or with truant eyes  \\
Unruly, peeped about for vagrant fruit.	  \\
And as for what pertains to human life,  \\
The deeper passions working round me here---  \\
Whether of envy, jealousy, pride, shame,  \\
Ambition, emulation, fear, or hope,  \\
Or those of dissolute pleasure---were by me	  \\
Unshared, and only now and then observed,  \\
So little was their hold upon my being,  \\
As outward things that might administer  \\
To knowledge or instruction. Hushed meanwhile  \\
Was the under-soul, locked up in such a calm,	  \\
That not a leaf of the great nature stirred.  \\
Yet was this deep vacation not given up  \\
To utter waste. Hitherto I had stood  \\
In my own mind remote from human life,  \\
At least from what we commonly so name,	  \\
Even as a shepherd on a promontory,  \\
Who, lacking occupation, looks far forth  \\
Into the endless sea, and rather makes  \\
Than finds what he beholds. And sure it is,  \\
That this first transit from the smooth delights	  \\
And wild outlandish walks of simple youth  \\
To something that resembled an approach  \\
Towards mortal business, to a privileged world  \\
Within a world, a midway residence  \\
With all its intervenient imagery,	  \\
Did better suit my visionary mind---  \\
Far better, than to have been bolted forth,  \\
Thrust out abruptly into fortune's way  \\
Among the conflicts of substantial life---  \\
By a more just gradation did lead on	  \\
To higher things, more naturally matured  \\
For permanent possession, better fruits,  \\
Whether of truth or virtue, to ensue.  \\
In playful zest of fancy did we note---  \\
How could we less?---the manners and the ways	  \\
Of those who in the livery were arrayed  \\
Of good or evil fame, of those with whom  \\
By frame of academic discipline  \\
Perforce we were connected, men whose sway,  \\
And whose authority of office, served	  \\
To set our minds on edge, and did no more.  \\
Nor wanted we rich pastime of this kind---  \\
Found everywhere, but chiefly in the ring  \\
Of the grave elders, men unscoured, grotesque  \\
In character, tricked out like aged trees	  \\
Which through the lapse of their infirmity  \\
Give ready place to any random seed  \\
That chuses to be reared upon their trunks.  \\
Here on my view, confronting as it were  \\
Those shepherd swains whom I had lately left,	  \\
Did flash a different image of old age---  \\
How different---yet both withal alike  \\
A book of rudiments for the unpractised sight,  \\
Objects embossed, and which with sedulous care  \\
Nature holds up before the eye of youth	  \\
In her great school---with further view, perhaps,  \\
To enter early on her tender scheme  \\
Of teaching comprehension with delight  \\
And mingling playful with pathetic thoughts.  \\
The surfaces of artificial life	  \\
And manners finely spun, the delicate race  \\
Of colours, lurking, gleaming up and down  \\
Through that state arras woven with silk and gold:  \\
This wily interchange of snaky hues,  \\
Willingly and unwillingly revealed,	  \\
I had not learned to watch, and at this time  \\
Perhaps, had such been in my daily sight,  \\
I might have been indifferent thereto  \\
As hermits are to tales of distant things.  \\
Hence, for these rarities elaborate	  \\
Having no relish yet, I was content  \\
With the more homely produce rudely piled  \\
In this our coarser warehouse. At this day  \\
I smile in many a mountain solitude  \\
At passages and fragments that remain	  \\
Of that inferior exhibition, played  \\
By wooden images, a theatre  \\
For wake or fair. And oftentimes do flit  \\
Remembrances before me of old men,  \\
Old humourists, who have been long in their graves,	  \\
And, having almost in my mind put off  \\
Their human names, have into phantoms passed  \\
Of texture midway betwixt life and books.  \\
I play the loiterer, 'tis enough to note  \\
That here in dwarf proportions were expressed	  \\
The limbs of the great world---its goings-on  \\
Collaterally pourtrayed as in mock fight,  \\
A tournament of blows, some hardly dealt  \\
Though short of mortal combat---and whate'er  \\
Might of this pageant be supposed to hit	  \\
A simple rustic's notice, this way less,  \\
More that way, was not wasted upon me.  \\
And yet this spectacle may well demand  \\
A more substantial name, no mimic show,  \\
Itself a living part of a live whole,	  \\
A creek of the vast sea. For, all degrees  \\
And shapes of spurious fame and short-lived praise  \\
Here sate in state, and, fed with daily alms,  \\
Retainers won away from solid good.  \\
And here was Labour, his own Bond-slave; Hope	  \\
That never set the pains against the prize;  \\
Idleness, halting with his weary clog;  \\
And poor misguided Shame, and witless Fear,  \\
And simple Pleasure, foraging for Death;  \\
Honour misplaced, and Dignity astray;	  \\
Feuds, factions, flatteries, Enmity and Guile,  \\
Murmuring Submission and bald Government  \\
(The idol weak as the idolator)  \\
And Decency and Custom starving Truth,  \\
And blind Authority beating with his staff	  \\
The child that might have led him;  \\
Emptiness Followed as of good omen, and meek  \\
Worth Left to itself unheard of and unknown.  \\
Of these and other kindred notices  \\
I cannot say what portion is in truth	  \\
The naked recollection of that time,  \\
And what may rather have been called to life  \\
By after-meditation. But delight,  \\
That, in an easy temper lulled asleep,  \\
Is still with innocence its own reward,	  \\
This surely was not wanting. Carelessly  \\
I gazed, roving as through a cabinet  \\
Or wide museum, thronged with fishes, gems,  \\
Birds, crocodiles, shells, where little can be seen,  \\
Well understood, or naturally endeared,	  \\
Yet still does every step bring something forth  \\
That quickens, pleases, stings---and here and there  \\
A casual rarity is singled out  \\
And has its brief perusal, then gives way  \\
To others, all supplanted in their turn.	  \\
Meanwhile, amid this gaudy congress framed  \\
Of things by nature most unneighbourly,  \\
The head turns round, and cannot right itself;  \\
And, though an aching and a barren sense  \\
Of gay confusion still be uppermost,	  \\
With few wise longings and but little love,  \\
Yet something to the memory sticks at last  \\
Whence profit may be drawn in times to come.  \\
Thus in submissive idleness, my friend,  \\
The labouring time of autumn, winter, spring---	  \\
Nine months---rolled pleasingly away, the tenth  \\
Returned me to my native hills again. \\
\end{verse}

%%%%%%%%%%%%%%%%%%%%%%%%%%%%%%%%%%%%%%%%%%%%%%%%%%%%%%%%%% \\
\chapter*[Book Fourth]{Book Fourth \\ Summer Vacation}
\addcontentsline{toc}{chapter}{Book Fourth Summer Vacation}

\begin{verse}
A PLEASANT sight it was when, having clomb  \\
The Heights of Kendal, and that dreary moor  \\
Was crossed, at length as from a rampart's edge  \\
I overlooked the bed of Windermere.  \\
I bounded down the hill, shouting amain	  \\
A lusty summons to the farther shore  \\
For the old ferryman; and when he came  \\
I did not step into the well-known boat  \\
Without a cordial welcome. Thence right forth  \\
I took my way, now drawing towards home,	  \\
To that sweet valley where I had been reared;  \\
'Twas but a short hour's walk ere, veering round,  \\
I saw the snow-white church upon its hill  \\
Sit like a throne`d lady, sending out  \\
A gracious look all over its domain.	  \\
Glad greetings had I, and some tears perhaps,  \\
From my old dame, so motherly and good,  \\
While she perused me with a parent's pride.  \\
The thoughts of gratitude shall fall like dew  \\
Upon thy grave, good creature: while my heart	  \\
Can beat I never will forget thy name.  \\
Heaven's blessing be upon thee where thou liest  \\
After thy innocent and busy stir  \\
In narrow cares, thy little daily growth  \\
Of calm enjoyments, after eighty years,	  \\
And more than eighty, of untroubled life---  \\
Childless, yet by the strangers to they blood  \\
Honoured with little less than filial love.  \\
Great joy was mine to see thee once again,  \\
Thee and thy dwelling, and a throng of things	  \\
About its narrow precincts, all beloved  \\
And many of them seeming yet my own.  \\
Why should I speak of what a thousand hearts  \\
Have felt, and every man alive can guess?  \\
The rooms, the court, the garden were not left	  \\
Long unsaluted, and the spreading pine  \\
And broad stone table underneath its boughs---  \\
Our summer seat in many a festive hour---  \\
And that unruly child of mountain birth,  \\
The froward brook, which, soon as he was boxed	  \\
Within our garden, found himself at once  \\
As if by trick insidious and unkind,  \\
Stripped of his voice, and left to dimple down  \\
Without an effort and without a will  \\
A channel paved by the hand of man.	  \\
I looked at him and smiled, and smiled again,  \\
And in the press of twenty thousand thoughts,  \\
'Ha', quoth I, 'pretty prisoner, are you there!'  \\
---And now, reviewing soberly that hour,  \\
I marvel that a fancy did not flash	  \\
Upon me, and a strong desire, straitway,  \\
At sight of such an emblem that shewed forth  \\
So aptly my late course of even days  \\
And all their smooth enthralment, to pen down  \\
A satire on myself. My aged dame	  \\
Was with me, at my side; she guided me,  \\
I willing, nay---nay, wishing to be led.  \\
The face of every neighbour whom I met  \\
Was as a volume to me; some I hailed  \\
Far off, upon the road, or at their work---	  \\
Unceremonious greetings, interchanged  \\
With half the length of a long field between.  \\
Among my schoolfellows I scattered round  \\
A salutation that was more constrained  \\
Though earnest---doubtless with a little pride,	  \\
But with more shame, for my habiliments,  \\
The transformation and the gay attire.  \\
Delighted did I take my place again  \\
At our domestic table; and, dear friend,  \\
Relating simply as my wish hath been	  \\
A poet's history, can I leave untold  \\
The joy with which I laid me down at night  \\
In my accustomed bed, more welcome now  \\
Perhaps than if it had been more desired,  \\
Or been more often thought of with regret---	  \\
That bed whence I had heard the roaring wind  \\
And clamorous rain, that bed where  \\
I so oft Had lain awake on breezy nights to watch  \\
The moon in splendour couched among the leaves  \\
Of a tall ash that near our cottage stood,	  \\
Had watched her with fixed eyes, while to and fro  \\
In the dark summit of the moving tree  \\
She rocked with every impulse of the wind.  \\
Among the faces which it pleased me well  \\
To see again was one by ancient right	  \\
Our inmate, a rough terrier of the hills,  \\
By birth and call of nature preordained  \\
To hunt the badger and unearth the fox  \\
Among the impervious crags. But having been  \\
From youth our own adopted, he had passed	  \\
Into a gentler service; and when first  \\
The boyish spirit flagged, and day by day  \\
Along my veins I kindled with the stir,  \\
The fermentation and the vernal heat  \\
Of poesy, affecting private shades	  \\
Like a sick lover, then this dog was used  \\
To watch me, an attendant and a friend,  \\
Obsequious to my steps early and late,  \\
Though often of such dilatory walk  \\
Tired, and uneasy at the halts I made.	  \\
A hundred times when in these wanderings  \\
I have been busy with the toil of verse---  \\
Great pains and little progress---and at once  \\
Some fair enchanting image in my mind  \\
Rose up, full-formed like Venus from the sea,	  \\
Have I sprung forth towards him and let loose  \\
My hand upon his back with stormy joy,  \\
Caressing him again and yet again.  \\
And when in the public roads at eventide  \\
I sauntered, like a river murmuring	  \\
And talking to itself, at such a season  \\
It was his custom to jog on before;  \\
But, duly whensoever he had met  \\
A passenger approaching, would he turn  \\
To give me timely notice, and straitway,	  \\
Punctual to such admonishment, I hushed  \\
My voice, composed my gait, and shaped myself  \\
To give and take a greeting that might save  \\
My name from piteous rumours, such as wait  \\
On men suspected to be crazed in brain.	  \\
Those walks, well worthy to be prized and loved---  \\
Regretted, that word too was on my tongue,  \\
But they were richly laden with all good,  \\
And cannot be remembered but with thanks  \\
And gratitude and perfect joy of heart---	  \\
Those walks did now like a returning spring  \\
Come back on me again. When first I made  \\
Once more the circuit of our little lake  \\
If ever happiness hath lodged with man  \\
That day consummate happiness was mine---	  \\
Wide-spreading, steady, calm, contemplative.  \\
The sun was set, or setting, when I left  \\
Our cottage door, and evening soon brought on  \\
A sober hour, not winning or serene,  \\
For cold and raw the air was, and untuned;	  \\
But as a face we love is sweetest then  \\
When sorrow damps it, or, whatever look  \\
It chance to wear, is sweetest if the heart  \\
Have fulness in itself, even so with me  \\
It fared that evening. Gently did my soul	  \\
Put off her veil, and, self-transmuted, stood  \\
Naked as in the presence of her God.  \\
As on I walked, a comfort seemed to touch  \\
A heart that had not been disconsolate,  \\
Strength came where weakness was not known to be,	  \\
At least not felt; and restoration came  \\
Like an intruder knocking at the door  \\
Of unacknowledged weariness. I took  \\
The balance in my hand and weighed myself:  \\
I saw but little, and thereat was pleased;	  \\
Little did I remember, and even this  \\
Still pleased me more---but I had hopes and peace  \\
And swellings of the spirits, was rapt and soothed,  \\
Conversed with promises, had glimmering views  \\
How life pervades the undecaying mind,	  \\
How the immortal soul with godlike power  \\
Informs, creates, and thaws the deepest sleep  \\
That time can lay upon her, how on earth  \\
Man if he do but live within the light  \\
Of high endeavours, daily spreads abroad	  \\
His being with a strength that cannot fail.  \\
Nor was there want of milder thoughts, of love,  \\
Of innocence, and holiday repose,  \\
And more than pastoral quiet in the heart  \\
Of amplest projects, and a peaceful end	  \\
At last, or glorious, by endurance won.  \\
Thus musing, in a wood I sate me down  \\
Alone, continuing there to muse. Meanwhile  \\
The mountain heights were slowly overspread  \\
With darkness, and before a rippling breeze	  \\
The long lake lengthened out its hoary line,  \\
And in the sheltered coppice where I sate,  \\
Around me, from among the hazel leaves---  \\
Now here, now there, stirred by the straggling wind---  \\
Came intermittingly a breath-like sound,	  \\
A respiration short and quick, which oft,  \\
Yea, might I say, again and yet again,  \\
Mistaking for the panting of my dog,  \\
The off-and-on companion of my walk,  \\
I turned my head to look if he were there.	  \\
A freshness also found I at this time  \\
In human life, the life I mean of those  \\
Whose occupations really I loved.  \\
The prospect often touched me with surprize:  \\
Crowded and full, and changed, as seemed to me,	  \\
Even as a garden in the heat of spring  \\
After an eight-days' absence. For---to omit  \\
The things which were the same and yet appeared  \\
So different---amid this solitude,  \\
The little vale where was my chief abode,	  \\
'Twas not indifferent to a youthful mind  \\
To note, perhaps some sheltered seat in which  \\
An old man had been used to sun himself,  \\
Now empty; pale-faced babes whom I had left  \\
In arms, known children of the neighbourhood,	  \\
Now rosy prattlers, tottering up and down;  \\
And growing girls whose beauty, filched away  \\
With all its pleasant promises, was gone  \\
To deck some slighted playmate's homely cheek.  \\
Yes, I had something of another eye,	  \\
And often looking round was moved to smiles  \\
Such as a delicate work of humour breeds.  \\
I read, without design, the opinions, thoughts,  \\
Of those plain-living people, in a sense  \\
Of love and knowledge: with another eye	  \\
I saw the quiet woodman in the woods,  \\
The shepherd on the hills. With new delight,  \\
This chiefly, did I view my grey-haired dame,  \\
Saw her go forth to church, or other work  \\
Of state, equipped in monumental trim---	  \\
Short velvet cloak, her bonnet of the like,  \\
A mantle such as Spanish cavaliers  \\
Wore in old time. Her smooth domestic life---  \\
Affectionate without uneasiness---  \\
Her talk, her business, pleased me; and no less	  \\
Her clear though shallow stream of piety,  \\
That ran on sabbath days a fresher course.  \\
With thoughts unfelt till now I saw her read  \\
Her bible on the Sunday afternoons,  \\
And loved the book when she had dropped asleep	  \\
And made of it a pillow for her head.  \\
Nor less do I remember to have felt  \\
Distinctly manifested at this time,  \\
A dawning, even as of another sense,  \\
A human-heartedness about my love	  \\
For objects hitherto the gladsome air  \\
Of my own private being, and no more---  \\
Which I had loved, even as a blesse`d spirit  \\
Or angel, if he were to dwell on earth,  \\
Might love in individual happiness.	  \\
But now there opened on me other thoughts,  \\
Of change, congratulation and regret,  \\
A new-born feeling. It spread far and wide:  \\
The trees, the mountains shared it, and the brooks,  \\
The stars of heaven, now seen in their old haunts---	  \\
White Sirius glittering o'er the southern crags,  \\
Orion with his belt, and those fair Seven,  \\
Acquaintances of every little child,  \\
And Jupiter, my own beloved star.  \\
Whatever shadings of mortality	  \\
Had fallen upon these objects heretofore  \\
Were different in kind: not tender---strong,  \\
Deep, gloomy were they, and severe, the scatterings  \\
Of childhood, and moreover, had given way  \\
In later youth to beauty and to love	  \\
Enthusiastic, to delight and joy.  \\
As one who hangs down-bending from the side  \\
Of a slow-moving boat upon the breast  \\
Of a still water, solacing himself  \\
With such discoveries as his eye can make	  \\
Beneath him in the bottom of the deeps,  \\
Sees many beauteous sights---weeds, fishes, flowers,  \\
Grots, pebbles, roots of trees---and fancies more,  \\
Yet often is perplexed, and cannot part  \\
The shadow from the substance, rocks and sky,	  \\
Mountains and clouds, from that which is indeed  \\
The region, and the things which there abide  \\
In their true dwelling; now is crossed by gleam  \\
Of his own image, by a sunbeam now,  \\
And motions that are sent he knows not whence,	  \\
Impediments that make his task more sweet;  \\
Such pleasant office have we long pursued  \\
Incumbent o'er the surface of past time---  \\
With like success. Nor have we often looked  \\
On more alluring shows---to me at least---	  \\
More soft, or less ambiguously descried,  \\
Than those which now we have been passing by,  \\
And where we still are lingering. Yet in spite  \\
Of all these new employments of the mind  \\
There was an inner falling off. I loved,	  \\
Loved deeply, all that I had loved before,  \\
More deeply even than ever; but a swarm  \\
Of heady thoughts jostling each other, gawds  \\
And feast and dance and public revelry  \\
And sports and games---less pleasing in themselves	  \\
Than as they were a badge, glossy and fresh,  \\
Of manliness and freedom---these did now  \\
Seduce me from the firm habitual quest  \\
Of feeding pleasures, from that eager zeal,  \\
Those yearnings which had every day been mine,	  \\
A wild, unworldly-minded youth, given up  \\
To Nature and to books, or, at the most,  \\
From time to time by inclination shipped  \\
One among many, in societies  \\
That were, or seemed, as simple as myself.	  \\
But now was come a change---it would demand  \\
Some skill, and longer time than may be spared,  \\
To paint even to myself these vanities,  \\
And how they wrought---but sure it is that now  \\
Contagious air did oft environ me,	  \\
Unknown among these haunts in former days.  \\
The very garments that I wore appeared  \\
To prey upon my strength, and stopped the course  \\
And quiet stream of self-forgetfulness.  \\
Something there was about me that perplexed	  \\
Th' authentic sight of reason, pressed too closely  \\
On that religious dignity of mind  \\
That is the very faculty of truth,  \\
Which wanting---either, from the very first  \\
A function never lighted up, or else	  \\
Extinguished---man, a creature great and good,  \\
Seems but a pageant plaything with vile claws,  \\
And this great frame of breathing elements  \\
A senseless idol. This vague heartless chace	  \\
Of trivial pleasures was a poor exchange  \\
For books and Nature at that early age.  \\
'Tis true, some casual knowledge might be gained  \\
Of character or life; but at that time,  \\
Of manners put to school I took small note,	  \\
And all my deeper passions lay elsewhere---  \\
Far better had it been to exalt the mind  \\
By solitary study, to uphold  \\
Intense desire by thought and quietness.  \\
And yet, in chastisement of these regrets,	  \\
The memory of one particular hour  \\
Doth here rise up against me. In a throng,  \\
A festal company of maids and youths,  \\
Old men and matrons, staid, promiscuous rout,  \\
A medley of all tempers, I had passed	  \\
The night in dancing, gaiety and mirth---  \\
With din of instruments, and shuffling feet,  \\
And glancing forms, and tapers glittering,  \\
And unaimed prattle flying up and down,  \\
Spirits upon the stretch, and here and there	  \\
Slight shocks of young love-liking interspersed  \\
That mounted up like joy into the head,  \\
And tingled through the veins. Ere we retired  \\
The cock had crowed, the sky was bright with day;  \\
Two miles I had to walk along the fields	  \\
Before I reached my home. Magnificent  \\
The morning was, a memorable pomp,  \\
More glorious than I ever had beheld.  \\
The sea was laughing at a distance; all  \\
The solid montains were as bright as clouds,	  \\
Grain-tinctured, drenched in empyrean light;  \\
And in the meadows and the lower grounds  \\
Was all the sweetness of a common dawn---  \\
Dews, vapours, and the melody of birds,  \\
And labourers going forth into the fields.	  \\
Ah, need I say, dear friend, that to the brim  \\
My heart was full? I made no vows, but vows  \\
Were then made for me; bond unknown to me  \\
Was given, that I should be---else sinning greatly---  \\
A dedicated spirit. On I walked	  \\
In blessedness, which even yet remains.  \\
Strange rendezvous my mind was at that time,  \\
A party-coloured shew of grave and gay,  \\
Solid and light, short-sighted and profound,  \\
Of considerate habits and sedate,	  \\
Consorting in one mansion unreproved.  \\
I knew the worth of that which I possessed,  \\
Though slighted and misused. Besides in truth  \\
That summer, swarming as it did with thoughts  \\
Transient and loose, yet wanted not a store	  \\
Of primitive hours, when---by these hindrances  \\
Unthwarted---I experienced in myself  \\
Conformity as just as that of old  \\
To the end and written spirit of God's works,  \\
Whether held forth in Nature or in man.	  \\
From many wanderings that have left behind  \\
Remembrances not lifeless, I will here  \\
Single out one, then pass to other themes.  \\
A favorite pleasure hath it been with me  \\
From time of earliest youth to walk alone	  \\
Along the public way, when, for the night  \\
Deserted, in its silence it assumes  \\
A character of deeper quietness  \\
Than pathless solitudes. At such an hour  \\
Once, ere these summer months were passed away,	  \\
I slowly mounted up a steep ascent  \\
Where the road's wat'ry surface, to the ridge  \\
Of that sharp rising, glittered in the moon  \\
And seemed before my eyes another stream  \\
Creeping with silent lapse to join the brook	  \\
That murmured in the valley. On I went  \\
Tranquil, receiving in my own despite  \\
Amusement, as I slowly passed along,  \\
From such near objects as from time to time  \\
Perforce intruded on the listless sense,	  \\
Quiescent and disposed to sympathy,  \\
With an exhausted mind worn out by toil  \\
And all unworthy of the deeper joy  \\
Which waits on distant prospect---cliff or sea,  \\
The dark blue vault and universe of stars.	  \\
Thus did I steal along that silent road,  \\
My body from the stillness drinking in  \\
A restoration like the calm of sleep,  \\
But sweeter far. Above, before, behind,  \\
Around me, all was peace and solitude;	  \\
I looked not round, nor did the solitude  \\
Speak to my eye, but it was heard and felt,  \\
O happy state! what beauteous pictures now  \\
Rose in harmonious imagery; they rose  \\
As from some distant region of my soul	  \\
And came along like dreams---yet such as left  \\
Obscurely mingled with their passing forms  \\
A consciousness of animal delight,  \\
A self-possession felt in every pause  \\
And every gentle movement of my frame.	  \\
While thus I wandered, step by step led on,  \\
It chanced a sudden turning of the road  \\
Presented to my view an uncouth shape,  \\
So near that, slipping back into the shade  \\
Of a thick hawthorn, I could mark him well,	  \\
Myself unseen. He was of stature tall,  \\
A foot above man's common measure tall,  \\
Stiff in his form, and upright, lank and lean---  \\
A man more meagre, as it seemed to me,  \\
Was never seen abroad by night or day.	  \\
His arms were long, and bare his hands; his mouth  \\
Shewed ghastly in the moonlight; from behind,  \\
A milestone propped him, and his figure seemed  \\
Half sitting, and half standing. I could mark  \\
That he was clad in military garb,	  \\
Though faded yet entire. He was alone,  \\
Had no attendant, neither dog, nor staff,  \\
Nor knapsack; in his very dress appeared  \\
A desolation, a simplicity  \\
That seemed akin to solitude. Long time	  \\
Did I peruse him with a mingled sense  \\
Of fear and sorrow. From his lips meanwhile  \\
There issued murmuring sounds, as if of pain  \\
Or of uneasy thought; yet still his form  \\
Kept the same steadiness, and at his feet	  \\
His shadow lay, and moved not. In a glen  \\
Hard by, a village stood, whose roofs and doors  \\
Were visible among the scattered trees,  \\
Scarce distant from the spot an arrow's flight.  \\
I wished to see him move, but he remained	  \\
Fixed to his place, and still from time to time  \\
Sent forth a murmuring voice of dead complaint,  \\
Groans scarcely audible. Without self-blame  \\
I had not thus prolonged my watch; and now,  \\
Subduing my heart's specious cowardise,	  \\
I left the shady nook where I had stood  \\
And hailed him. Slowly from his resting-place  \\
He rose, and with a lean and wasted arm  \\
In measured gesture lifted to his head  \\
Returned my salutation, then resumed	  \\
His station as before. And when erelong  \\
I asked his history, he in reply  \\
Was neither slow nor eager, but, unmoved,  \\
And with a quiet uncomplaining voice,  \\
A stately air of mild indifference,	  \\
He told in simple words a soldier's tale:  \\
That in the tropic islands he had served,  \\
Whence he had landed scarcely ten days past---  \\
That on his landing he had been dismissed,  \\
And now was travelling to his native home.	  \\
At this I turned and looked towards the village,  \\
But all were gone to rest, the fires all out,  \\
And every silent window to the moon  \\
Shone with a yellow glitter. 'No one there',  \\
Said I, 'is waking; we must measure back	  \\
The way which we have come. Behind yon wood  \\
A labourer dwells, and, take it on my word,  \\
He will not murmur should we break his rest,  \\
And with a ready heart will give you food  \\
And lodging for the night.' At this he stooped,	  \\
And from the ground took up an oaken staff  \\
By me yet unobserved, a traveller's staff  \\
Which I suppose from his slack hand had dropped,  \\
And lain till now neglected in the grass.  \\
Towards the cottage without more delay	  \\
We shaped our course. As it appeared to me  \\
He travelled without pain, and I beheld  \\
With ill-suppressed astonishment his tall  \\
And ghastly figure moving at my side;  \\
Nor while we journeyed thus could I forbear	  \\
To question him of what he had endured  \\
From hardship, battle, or the pestilence.  \\
He all the while was in demeanor calm,  \\
Concise in answer. Solemn and sublime  \\
He might have seemed, but that in all he said	  \\
There was a strange half-absence, and a tone  \\
Of weakness and indifference, as of one  \\
Remembering the importance of his theme  \\
But feeling it no longer. We advanced  \\
Slowly, and ere we to the wood were come	  \\
Discourse had ceased. Together on we passed  \\
In silence through the shades, gloomy and dark;  \\
Then, turning up along an open field,  \\
We gained the cottage. At the door I knocked,  \\
Calling aloud, 'My friend, here is a man	  \\
By sickness overcome. Beneath your roof  \\
This night let him find rest, and give him food  \\
If food he need, for he is faint and tired.'  \\
Assured that now my comrade would repose  \\
In comfort, I entreated that henceforth	  \\
He would not linger in the public ways,  \\
But ask for timely furtherance, and help  \\
Such as his state required. At this reproof,  \\
With the same ghastly mildness in his look,  \\
He said, 'My trust is in the God of Heaven,	  \\
And in the eye of him that passes me.'  \\
The cottage door was speedily unlocked,  \\
And now the soldier touched his hat again  \\
With his lean hand, and in a voice that seemed  \\
To speak with a reviving interest,	  \\
'Till then unfelt, he thanked me; I returned  \\
The blessing of the poor unhappy man,  \\
And so we parted. Back I cast a look,  \\
And lingered near the door a little space,  \\
Then sought with quiet heart my distant home.	 \\
\end{verse}

%%%%%%%%%%%%%%%%%%%%%%%%%%%%%%%%%%%%%%%%%%%%%%%%%%% \\
\chapter*[Book Fifth]{Book Fifth \\ Books}
\addcontentsline{toc}{chapter}{Book Fifth Books}

\begin{verse}
EVEN in the steadiest mood of reason, when  \\
All sorrow for thy transitory pains  \\
Goes out, it grieves me for thy state, O man,  \\
Thou paramount creature, and thy race, while ye  \\
Shall sojourn on this planet, not for woes	  \\
Which thou endur'st---that weight, albeit huge,  \\
I charm away---but for those palms atchieved  \\
Through length of time, by study and hard thought,  \\
The honours of thy high endowments; there  \\
My sadness finds its fuel. Hitherto	  \\
In progress through this verse my mind hath looked  \\
Upon the speaking face of earth and heaven  \\
As her prime teacher, intercourse with man  \\
Established by the Sovereign Intellect,  \\
Who through that bodily image hath diffused	  \\
A soul divine which we participate,  \\
A deathless spirit. Thou also, man, hast wrought,  \\
For commerce of thy nature with itself,  \\
Things worthy of unconquerable life;  \\
And yet we feel---we cannot chuse but feel---	  \\
That these must perish. Tremblings of the heart  \\
It gives, to think that the immortal being  \\
No more shall need such garments; and yet man,  \\
As long as he shall be the child of earth,  \\
Might almost 'weep to have' what he may lose---	  \\
Nor be himself extinguished, but survive  \\
Abject, depressed, forlorn, disconsolate.  \\
A thought is with me sometimes, and I say,  \\
'Should earth by inward throes be wrenched throughout,  \\
Or fire be sent from far to wither all	  \\
Her pleasant habitations, and dry up  \\
Old Ocean in his bed, left singed and bare,  \\
Yet would the living presence still subsist  \\
Victorious; and composure would ensue,  \\
And kindlings like the morning---presage sure,	  \\
Though slow perhaps, of a returning day.'  \\
But all the meditations of mankind,  \\
Yea, all the adamantine holds of truth  \\
By reason built, or passion (which itself  \\
Is highest reason in a soul sublime),	  \\
The consecrated works of bard and sage,  \\
Sensuous or intellectual, wrought by men,  \\
Twin labourers and heirs of the same hopes---  \\
Where would they be? Oh, why hath not the mind  \\
Some element to stamp her image on	  \\
In nature somewhat nearer to her own?  \\
Why, gifted with such powers to send abroad  \\
Her spirit, must it lodge in shrines so frail?  \\
One day, when in the hearing of a friend  \\
I had given utterance to thoughts like these,	  \\
He answered with a smile that in plain truth  \\
'Twas going far to seek disquietude---  \\
But on the front of his reproof confessed  \\
That he at sundry seasons had himself  \\
Yielded to kindred hauntings, and, forthwith,	  \\
Added that once upon a summer's noon  \\
While he was sitting in a rocky cave  \\
By the seaside, perusing as it chanced,  \\
The famous history of the errant knight  \\
Recorded by Cervantes, these same thoughts	  \\
Came to him, and to height unusual rose  \\
While listlessly he sate, and, having closed  \\
The book, had turned his eyes towards the sea.  \\
On poetry and geometric truth  \\
(The knowledge that endures) upon these two,	  \\
And their high privilege of lasting life  \\
Exempt from all internal injury,  \\
He mused---upon these chiefly---and at length,  \\
His senses yielding to the sultry air,  \\
Sleep seized him and he passed into a dream.	  \\
He saw before him an Arabian waste,  \\
A desert, and he fancied that himself  \\
Was sitting there in the wide wilderness  \\
Alone upon the sands. Distress of mind  \\
Was growing in him when, behold, at once	  \\
To his great joy a man was at his side,  \\
Upon a dromedary mounted high.  \\
He seemed an arab of the Bedouin tribes;  \\
A lance he bore, and underneath one arm  \\
A stone, and in the opposite hand a shell	  \\
Of a surpassing brightness. Much rejoiced  \\
The dreaming man that he should have a guide  \\
To lead him through the desert; and he thought,  \\
While questioning himself what this strange freight  \\
Which the newcomer carried through the waste	  \\
Could mean, the arab told him that the stone---  \\
To give it in the language of the dream---  \\
Was Euclid's Elements. 'And this', said he,  \\
'This other', pointing to the shell, 'this book  \\
Is something of more worth.' 'And, at the word,	  \\
The stranger', said my friend continuing,  \\
'Stretched forth the shell towards me, with command  \\
That I should hold it to my ear. I did so  \\
And heard that instant in an unknown tongue,  \\
Which yet I understood, articulate sounds,	  \\
A loud prophetic blast of harmony,  \\
And ode in passion uttered, which foretold  \\
Destruction to the children of the earth  \\
By deluge now at hand. No sooner ceased  \\
The song, but with calm look the arab said	  \\
That all was true, that it was even so  \\
As had been spoken, and that he himself  \\
Was going then to bury those two books---  \\
The one that held acquaintance with the stars,  \\
And wedded man to man by purest bond	  \\
Of nature, undisturbed by space or time;  \\
Th' other that was a god, yea many gods,  \\
Had voices more than all the winds, and was  \\
A joy, a consolation, and a hope.'  \\
My friend continued, 'Strange as it may seem	  \\
I wondered not, although I plainly saw  \\
The one to be a stone, th' other a shell,  \\
Nor doubted once but that they both were books,  \\
Having a perfect faith in all that passed.  \\
A wish was now engendered in my fear	  \\
To cleave unto this man, and I begged leave  \\
To share his errand with him. On he passed  \\
Not heeding me; I followed, and took note  \\
That he looked often backward with wild look,  \\
Grasping his twofold treasure to his side.	  \\
Upon a dromedary, lance in rest,  \\
He rode, I keeping pace with him; and now  \\
I fancied that he was the very knight  \\
Whose tale Cervantes tells, yet not the knight,  \\
But was an arab of the desert too,	  \\
Of these was neither, and was both at once.  \\
His countenance meanwhile grew more disturbed,  \\
And looking backwards when he looked I saw  \\
A glittering light, and asked him whence it came.  \\
"It is", said he, "The waters of the deep	  \\
Gathering upon us." Quickening then his pace  \\
He left me; I called after him aloud;  \\
He heeded not, but with his twofold charge  \\
Beneath his arm---before me full in view---  \\
I saw him riding o'er the desart sands	  \\
With the fleet waters of the drowning world  \\
In chace of him; whereat I waked in terror,  \\
And saw the sea before me, and the book  \\
In which I had been reading at my side.'  \\
Full often, taking from the world of sleep	  \\
This arab phantom which my friend beheld,  \\
This semi-Quixote, I to him have given  \\
A substance, fancied him a living man---  \\
A gentle dweller in the desart, crazed  \\
By love, and feeling, and internal thought	  \\
Protracted among endless solitudes---  \\
Have shaped him, in the oppression of his brain,  \\
Wandering upon this quest and thus equipped.  \\
And I have scarcely pitied him, have felt  \\
A reverence for a being thus employed,	  \\
And thought that in the blind and awful lair  \\
Of such a madness reason did lie couched.  \\
Enow there are on earth to take in charge  \\
Their wives, their children, and their virgin loves,  \\
Or whatsoever else the heart holds dear---	  \\
Enow to think of these---yea, will I say,  \\
In sober contemplation of the approach  \\
Of such great overthrow, made manifest  \\
By certain evidence, that I methinks  \\
Could share that maniac's anxiousness, could go	  \\
Upon like errand. Oftentimes at least  \\
Me hath such deep entrancement half-possessed  \\
When I have held a volume in my hand---  \\
Poor earthly casket of immortal verse---  \\
Shakespeare or Milton, labourers divine.	  \\
Mighty, indeed supreme, must be the power Of living  \\
Nature which could thus so long  \\
Detain me from the best of other thoughts.  \\
Even in the lisping time of infancy  \\
And, later down, in prattling childhood---even	  \\
While I was travelling back among those days---  \\
How could I ever play an ingrate's part?  \\
Once more should I have made those bowers resound,  \\
And intermingled strains of thankfulness  \\
With their own thoughtless melodies. At least	  \\
It might have well beseemed me to repeat  \\
Some simply fashioned tale, to tell again  \\
In slender accents of sweet verse some tale  \\
That did bewitch me then, and soothes me now.  \\
O friend, O poet, brother of my soul,	  \\
Think not that I could ever pass along  \\
Untouched by these remembrances; no, no,  \\
But I was hurried forward by a stream  \\
And could not stop. Yet wherefore should I speak,  \\
Why call upon a few weak words to say	  \\
What is already written in the hearts  \\
Of all that breathe---what in the path of all  \\
Drops daily from the tongue of every child  \\
Wherever man is found? The trickling tear  \\
Upon the cheek of listening infancy	  \\
Tells it, and the insuperable look  \\
That drinks as if it never could be full.  \\
That portion of my story I shall leave  \\
There registered. Whatever else there be  \\
Of power or pleasure, sown or fostered thus---	  \\
Peculiar to myself---let that remain  \\
Where it lies hidden in its endless home  \\
Among the depths of time. And yet it seems  \\
That here, in memory of all books which lay  \\
Their sure foundations in the heart of man,	  \\
Whether by native prose, or numerous verse,  \\
That in the name of all inspire`d souls---  \\
From Homer the great thunderer, from the voice  \\
Which roars along the bed of Jewish song,  \\
And that, more varied and elaborate,	  \\
Those trumpet-tones of harmony that shake  \\
Our shores in England, from those loftiest notes  \\
Down to the low and wren-like warblings, made  \\
For cottagers and spinners at the wheel  \\
And weary travellers when they rest themselves	  \\
By the highways and hedges: ballad-tunes,  \\
Food for the hungry ears of little ones,  \\
And of old men who have survived their joy---  \\
It seemeth in behalf of these, the works,  \\
And of the men who framed them, whether known,	  \\
Or sleeping nameless in their scattered graves,  \\
That I should here assert their rights, attest  \\
Their honours, and should once for all pronounce  \\
Their benediction, speak of them as powers  \\
For ever to be hallowed---only less	  \\
For what we may become, and what we need,  \\
Than Nature's self which is the breath of God.  \\
Rarely and with reluctance would I stoop  \\
To transitory themes, yet I rejoice,  \\
And, by these thoughts admonished, must speak out	  \\
Thanksgivings from my heart that I was reared  \\
Safe from an evil which these days have laid  \\
Upon the children of the land---a pest  \\
That might have dried me up body and soul.  \\
This verse is dedicate to Nature's self	  \\
And things that teach as Nature teaches: then,  \\
Oh, where had been the man, the poet where---  \\
Where had we been we two, belove`d friend,  \\
If we, in lieu of wandering as we did  \\
Through heights and hollows and bye-spots of tales	  \\
Rich with indigenous produce, open ground  \\
Of fancy, happy pastures ranged at will,  \\
Had been attended, followed, watched, and noosed,  \\
Each in his several melancholy walk,  \\
Stringed like a poor man's heifer at its feed,	  \\
Led through the lanes in forlorn servitude;  \\
Or rather like a stalle`d ox shut out  \\
From touch of growing grass, that may not taste  \\
A flower till it have yielded up its sweets  \\
A prelibation to the mower's scythe.	  \\
Behold the parent hen amid her brood,  \\
Though fledged and feathered, and well pleased to part  \\
And straggle from her presence, still a brood,  \\
And she herself from the maternal bond  \\
Still undischarged. Yet doth she little more	  \\
Than move with them in tenderness and love,  \\
A centre of the circle which they make;  \\
And now and then---alike from need of theirs  \\
And call of her own natural appetites---  \\
She scratches, ransacks up the earth for food	  \\
Which they partake at pleasure. Early died  \\
My honoured mother, she who was the heart  \\
And hinge of all our learnings and our loves;  \\
She left us destitute, and as we might  \\
Trooping together. Little suits it me	  \\
To break upon the sabbath of her rest  \\
With any thought that looks at others' blame,  \\
Nor would I praise her but in perfect love;  \\
Hence am I checked, but I will boldly say  \\
In gratitude, and for the sake of truth,	  \\
Unheard by her, that she, not falsely taught,  \\
Fetching her goodness rather from times past  \\
Than shaping novelties from those to come,  \\
Had no presumption, no such jealousy---  \\
Nor did by habit of her thoughts mistrust	  \\
Our nature, but had virtual faith that He  \\
Who fills the mother's breasts with innocent milk  \\
Doth also for our nobler part provide,  \\
Under His great correction and controul,  \\
As innocent instincts, and as innocent food.	  \\
This was her creed, and therefore she was pure  \\
From feverish dread of error and mishap  \\
And evil, overweeningly so called,  \\
Was not puffed up by false unnatural hopes,  \\
Nor selfish with unnecessary cares,	  \\
Nor with impatience from the season asked  \\
More than its timely produce---rather loved  \\
The hours for what they are, than from regards  \\
Glanced on their promises in restless pride.  \\
Such was she: not from faculties more strong	  \\
Than others have, but from the times, perhaps,  \\
And spot in which she lived, and through a grace  \\
Of modest meekness, simple-mindedness,  \\
A heart that found benignity and hope,  \\
Being itself benign.	  \\
My drift hath scarcely  \\
I fear been obvious, for I have recoiled  \\
From showing as it is the monster birth  \\
Engendered by these too industrious times.  \\
Let few words paint it: 'tis a child, no child,	  \\
But a dwarf man; in knowledge, virtue, skill,  \\
In what he is not, and in what he is,  \\
The noontide shadow of a man complete;  \\
A worshipper of worldly seemliness---  \\
Not quarrelsome, for that were far beneath	  \\
His dignity; with gifts he bubbles o'er  \\
As generous as a fountain; selfishness  \\
May not come near him, gluttony or pride;  \\
The wandering beggers propagate his name,  \\
Dumb creatures find him tender as a nun.	  \\
Yet deem him not for this a naked dish  \\
Of goodness merely---he is garnished out.  \\
Arch are his notices, and nice his sense  \\
Of the ridiculous; deceit and guile,  \\
Meanness and falsehood, he detects, can treat	  \\
With apt and graceful laughter; nor is blind  \\
To the broad follies of the licensed world;  \\
Though shrewd, yet innocent himself withal,  \\
And can read lectures upon innocence.  \\
He is fenced round, nay armed, for ought we know,	  \\
In panoply complete; and fear itself,  \\
Natural or supernatural alike,  \\
Unless it leap upon him in a dream, Touches him not.  \\
Briefly, the moral part  \\
Is perfect, and in learning and in books	  \\
He is a prodigy. His discourse moves slow,  \\
Massy and ponderous as a prison door,  \\
Tremendously embossed with terms of art.  \\
Rank growth of propositions overruns  \\
The stripling's brain; the path in which he treads	  \\
Is choked with grammars. Cushion of divine  \\
Was never such a type of thought profound  \\
As is the pillow where he rests his head.  \\
The ensigns of the empire which he holds---  \\
The globe and sceptre of his royalties---	  \\
Are telescopes, and crucibles, and maps.  \\
Ships he can guide across the pathless sea,  \\
And tell you all their cunning; he can read  \\
The inside of the earth, and spell the stars;  \\
He knows the policies of foreign lands,	  \\
Can string you names of districts, cities, towns,  \\
The whole world over, tight as beads of dew  \\
Upon a gossamer thread. He sifts, he weighs,  \\
Takes nothing upon trust. His teachers stare,  \\
The country people pray for God's good grace,	  \\
And tremble at his deep experiments.  \\
All things are put to question: he must live  \\
Knowing that he grows wiser every day,  \\
Or else not live at all, and seeing too  \\
Each little drop of wisdom as it falls	  \\
Into the dimpling cistern of his heart.  \\
Meanwhile old Grandame Earth is grieved to find  \\
The playthings which her love designed for him  \\
Unthought of---in their woodland beds the flowers  \\
Weep, and the river-sides are all forlorn.	  \\
Now this is hollow, 'tis a life of lies  \\
From the beginning, and in lies must end.  \\
Forth bring him to the air of common sense  \\
And, fresh and shewy as it is, the corps  \\
Slips from us into powder. Vanity,	  \\
That is his soul: there lives he, and there moves---  \\
It is the soul of every thing he seeks---  \\
That gone, nothing is left which he can love.  \\
Nay, if a thought of purer birth should rise  \\
To carry him towards a better clime,	  \\
Some busy helper still is on the watch  \\
To drive him back, and pound him like a stray  \\
With the pinfold of his own conceit,  \\
Which is his home, his natural dwelling-place.  \\
Oh, give us once again the wishing-cap	  \\
Of Fortunatus, and the invisible coat  \\
Of Jack the Giant-killer, Robin Hood,  \\
And Sabra in the forest with St. George!  \\
The child whose love is here, at least doth reap  \\
One precious gain---that he forgets himself.	  \\
These mighty workmen of our later age  \\
Who with a broad highway have overbridged T  \\
he froward chaos of futurity,  \\
Tamed to their bidding---they who have the art  \\
To manage books, and things, and make them work	  \\
Gently on infant minds as does the sun  \\
Upon a flower---the tutors of our youth,  \\
The guides, the wardens of our faculties  \\
And stewards of our labour, watchful men  \\
And skilful in the usury of time,	  \\
Sages, who in their prescience would controul  \\
All accidents, and to the very road  \\
Which they have fashioned would confine us down  \\
Like engines---when will they be taught  \\
That in the unreasoning progress of the world	  \\
A wiser spirit is at work for us,  \\
A better eye than theirs, most prodigal  \\
Of blessings, and most studious of our good,  \\
Even in what seem our most unfruitful hours?  \\
There was a boy---ye knew him well, ye cliffs	  \\
And islands of Winander---many a time At evening,  \\
when the stars had just begun  \\
To move along the edges of the hills,  \\
Rising or setting, would he stand alone  \\
Beneath the trees or by the glimmering lake,	  \\
And there, with fingers interwoven, both hands  \\
Pressed closely palm to palm, and to his mouth  \\
Uplifted, he as through an instrument  \\
Blew mimic hootings to the silent owls  \\
That they might answer him. And they would shout	  \\
Across the wat'ry vale, and shout again,  \\
Responsive to his call, with quivering peals  \\
And long halloos, and screams, and echoes loud,  \\
Redoubled and redoubled---concourse wild  \\
Of mirth and jocund din. And when it chanced	  \\
That pauses of deep silence mocked his skill,  \\
Then sometimes in that silence, while he hung  \\
Listening, a gentle shock of mild surprize  \\
Has carried far into his heart the voice  \\
Of mountain torrents; or the visible scene	  \\
Would enter unawares into his mind  \\
With all its solemn imagery, its rocks,  \\
Its woods, and that uncertain heaven, received  \\
Into the bosom of the steady lake.  \\
This boy was taken from his mates, and died	  \\
In childhood ere he was full ten years old.  \\
Fair are the woods, and beauteous is the spot,  \\
The vale where he was born; the churchyard hangs  \\
Upon a slope above the village school,  \\
And there, along that bank, when I have passed	  \\
At evening, I believe that oftentimes  \\
A full half-hour together I have stood  \\
Mute, looking at the grave in which he lies.  \\
Even now methinks I have before my sight  \\
That self-same village church: I see her sit---	  \\
The throne`d lady spoken of erewhile---  \\
On her green hill, forgetful of this boy  \\
Who slumbers at her feet, forgetful too  \\
Of all her silent neighbourhood of graves,  \\
And listening only to the gladsome sounds	  \\
That, from the rural school ascending, play  \\
Beneath her and about her. May she long  \\
Behold a race of young ones like to those  \\
With whom I herded---easily, indeed,  \\
We might have fed upon a fatter soil	  \\
Of Arts and Letters, but be that forgiven---  \\
A race of real children, not too wise,  \\
Too learned, or too good, but wanton, fresh,  \\
And bandied up and down by love and hate;  \\
Fierce, moody, patient, venturous, modest, shy,	  \\
Mad at their sports like withered leaves in winds;  \\
Though doing wrong and suffering, and full oft  \\
Bending beneath our life's mysterious weight  \\
Of pain and fear, yet still in happiness  \\
Not yielding to the happiest upon earth.	  \\
Simplicity in habit, truth in speech,  \\
Be these the daily strengtheners of their minds!  \\
May books and Nature be their early joy,  \\
And knowledge, rightly honored with that name---  \\
Knowledge not purchased with the loss of power!	  \\
Well do I call to mind the very week  \\
When I was first entrusted to the care  \\
Of that sweet valley---when its paths, its shores  \\
And brooks, were like a dream of novelty  \\
To my half-infant thoughts---that very week,	  \\
While I was roving up and down alone  \\
Seeking I knew not what, I chanced to cross  \\
One of those open fields, which, shaped like ears,  \\
Make green peninsulas on Esthwaite's Lake.  \\
Twilight was coming on, yet through the gloom	  \\
I saw distinctly on the opposite shore  \\
A heap of garments, left as I supposed  \\
By one who there was bathing. Long I watched,  \\
But no one owned them; meanwhile the calm lake  \\
Grew dark, with all the shadows on its breast,	  \\
And now and then a fish up-leaping snapped  \\
The breathless stillness. The succeeding day---  \\
Those unclaimed garments telling a plain tale---  \\
Went there a company, and in their boat  \\
Sounded with grappling-irons and long poles:	  \\
At length, the dead man, 'mid that beauteous scene  \\
Of trees and hills and water, bolt upright  \\
Rose with his ghastly face, a spectre shape---  \\
Of terror even. And yet no vulgar fear,  \\
Young as I was, a child not nine years old,	  \\
Possessed me, for my inner eye had seen  \\
Such sights before among the shining streams  \\
Of fairyland, the forests of romance---  \\
Thence came a spirit hallowing what I saw  \\
With decoration and ideal grace,	  \\
A dignity, a smoothness, like the words  \\
Of Grecian art and purest Poesy.  \\
I had a precious treasure at that time,  \\
A little yellow canvass-covered book,  \\
A slender abstract of the Arabian Tales;	  \\
And when I learned, as now I first did learn  \\
From my companions in this new abode,  \\
That this dear prize of mine was but a block  \\
Hewn from a mighty quarry---in a word,  \\
That there were four large volumes, laden all	  \\
With kindred matter---'twas in truth to me  \\
A promise scarcely earthly. Instantly  \\
I made a league, a covenant with a friend  \\
Of my own age, that we should lay aside  \\
The monies we possessed, and hoard up more,	  \\
Till our joint savings had amassed enough  \\
To make this book our own. Through several months  \\
Religiously did we preserve that vow,  \\
And spite of all temptation hoarded up,  \\
And hoarded up; but firmness failed at length,	  \\
Nor were we ever masters of our wish.  \\
And afterwards, when, to my father's house  \\
Returning at the holidays, I found  \\
That golden store of books which I had left  \\
Open to my enjoyment once again,	  \\
What heart was mine! Full often through the course  \\
Of those glad respites in the summertime  \\
When armed with rod and line we went abroad  \\
For a whole day together, I have lain  \\
Down by thy side, O Derwent, murmuring stream,	  \\
On the hot stones and in the glaring sun,  \\
And there have read, devouring as I read,  \\
Defrauding the day's glory---desperate---  \\
Till with a sudden bound of smart reproach  \\
Such as an idler deals with in his shame,	  \\
I to my sport betook myself again.  \\
A gracious spirit o'er this earth presides,  \\
And o'er the heart of man: invisibly  \\
It comes, directing those to works of love	  \\
Who care not, know not, think not, what they do.  \\
The tales that charm away the wakeful night  \\
In Araby---romances, legends penned  \\
For solace by the light of monkish lamps;  \\
Fictions, for ladies of their love, devised	  \\
By youthful squires; adventures endless, spun  \\
By the dismantled warrior in old age  \\
Out of the bowels of those very thoughts  \\
In which his youth did first extravagate---  \\
These spread like day, and something in the shape	  \\
Of these will live till man shall be no more.  \\
Dumb yearnings, hidden appetites, are ours,  \\
And they must have their foot. Our childhood sits,  \\
Our simple childhood, sits upon a throne  \\
That hath more power than all the elements.	  \\
I guess not what this tells of being past,  \\
Nor what it augurs of the life to come,  \\
But so it is, and in that dubious hour,  \\
That twilight when we first begin to see  \\
This dawning earth, to recognise, expect---	  \\
And in the long probation that ensues,  \\
The time of trial ere we learn to live  \\
In reconcilement with our stinted powers,  \\
To endure this state of meagre vassalage,  \\
Unwilling to forego, confess, submit,	  \\
Uneasy and unsettled, yoke-fellows  \\
To custom, mettlesome and not yet tamed  \\
And humbled down---oh, then we feel, we feel,  \\
We know, when we have friends. Ye dreamers, then,  \\
Forgers of lawless tales, we bless you then---	  \\
Impostors, drivellers, dotards, as the ape  \\
Philosophy will call you---then we feel  \\
With what, and how great might ye are in league,  \\
Who make our wish our power, our thought a deed,  \\
An empire, a possession. Ye whom time	  \\
And seasons serve---all faculties---to whom  \\
Earth crouches, th' elements are potter's clay,  \\
Space like a heaven filled up with northern lights,  \\
Here, nowhere, there, and everywhere at once.  \\
It might demand a more impassioned strain	  \\
To tell of later pleasures linked to these,  \\
A tract of the same isthmus which we cross  \\
In progress from our native continent  \\
To earth and human life---I mean to speak  \\
Of that delightful time of growing youth	  \\
When cravings for the marvellous relent,  \\
And we begin to love what we have seen;  \\
And sober truth, experience, sympathy,  \\
Take stronger hold of us; and words themselves  \\
Move us with conscious pleasure.	  \\
I am sad  \\
At thought of raptures now for ever flown,  \\
Even unto tears I sometimes could be sad  \\
To think of, to read over, many a page---  \\
Poems withal of name---which at that time	  \\
Did never fail to entrance me, and are now  \\
Dead in my eyes as is a theatre  \\
Fresh emptied of spectators. Thirteen years,  \\
Or haply less, I might have seen when first  \\
My ears began to open to the charm	  \\
Of words in tuneful order, found them sweet  \\
For their own sakes---a passion and a power---  \\
And phrases pleased me, chosen for delight,  \\
For pomp, or love. Oft in the public roads,  \\
Yet unfrequented, while the morning light	  \\
Was yellowing the hilltops, with that dear friend  \\
(The same whom I have mentioned heretofore)  \\
I went abroad, and for the better part  \\
Of two delightful hours we strolled along  \\
By the still borders of the misty lake	  \\
Repeating favorite verses with one voice,  \\
Or conning more, as happy as the birds  \\
That round us chaunted. Well might we be glad,  \\
Lifted above the ground by airy fancies  \\
More bright than madness or the dreams of wine.	  \\
And though full oft the objects of our love  \\
Were false and in their splendour overwrought,  \\
Yet surely at such time no vulgar power  \\
Was working in us, nothing less in truth  \\
Than that most noble attribute of man---	  \\
Though yet untutored, and inordinate---  \\
That wish for something loftier, more adorned,  \\
Than is the common aspect, daily garb,  \\
Of human life. What wonder then if sounds  \\
Of exultation echoed through the groves---	  \\
For images, and sentiments, and words,  \\
And every thing with which we had to do  \\
In that delicious world of poesy,  \\
Kept holiday, a never-ending show,  \\
With music, incense, festival, and flowers!	  \\
Here must I pause: This only will I add  \\
From heart-experience, and in humblest sense  \\
Of modesty, that he who in his youth  \\
A wanderer among the woods and fields	  \\
With living Nature hath been intimate,  \\
Not only in that raw unpractised time  \\
Is stirred to ecstasy, as others are,  \\
By glittering verse, but he doth furthermore,  \\
In measure only dealt out to himself,	  \\
Receive enduring touches of deep joy  \\
From the great Nature that exists in works  \\
Of mighty poets. Visionary power  \\
Attends upon the motions of the winds  \\
Embodied in the mystery of words;	  \\
There darkness makes abode, and all the host  \\
Of shadowy things do work their changes there  \\
As in a mansion like their proper home.  \\
Even forms and substances are circumfused  \\
By that transparent veil with light divine,	  \\
And through the turnings intricate of verse  \\
Present themselves as objects recognised  \\
In flashes, and with a glory scare their own.  \\
Thus far a scanty record is deduced  \\
Of what I owed to books in early life;	  \\
Their later influence yet remains untold,  \\
But as this work was taking in my thoughts  \\
Proportions that seemed larger than had first  \\
Been meditated, I was indisposed  \\
To any further progress at a time	  \\
When these acknowledgements were left unpaid. \\
\end{verse}

%%%%%%%%%%%%%%%%%%%%%%%%%%%%%%%%%%%%%%%%%%%%%%%%%%%%%%%% \\
\chapter*[Book Sixth]{Book Sixth \\ Cambridge and the Alps}
\addcontentsline{toc}{chapter}{Book Sixth Cambridge and the Alps}

\begin{verse}
THE leaves were yellow when to Furness Fells,  \\
The haunt of shepherds, and to cottage life  \\
I bade adieu, and, one among the flock  \\
Who by that season are convened, like birds  \\
Trooping together at the fowler's lure,	  \\
Went back to Granta's cloisters---not so fond  \\
Or eager, though as gay and undepressed  \\
In spirit, as when I thence had taken flight  \\
A few short months before. I turned my face  \\
Without repining from the mountain pomp	  \\
Of autumn and its beauty (entered in  \\
With calmer lakes and louder streams); and you,  \\
Frank-hearted maids of rocky Cumberland,  \\
You and your not unwelcome days of mirth  \\
I quitted, and your nights of revelry,	  \\
And in my own unlovely cell sate down  \\
In lightsome mood---such privilege has youth,  \\
That cannot take long leave of pleasant thoughts.  \\
We need not linger o'er the ensuing time,  \\
But let me add at once that now, the bonds	  \\
Of indolent and vague society  \\
Relaxing in their hold, I lived henceforth  \\
More to myself, read more, reflected more,  \\
Felt more, and settled daily into habits  \\
More promising. Two winters may be passed	  \\
Without a separate notice; many books  \\
Were read in process of this time---devoured,  \\
Tasted or skimmed, or studiously perused---  \\
Yet with no settled plan. I was detached  \\
Internally from academic cares,	  \\
From every hope of prowess and reward,  \\
And wished to be a lodger in that house  \\
Of letters, and no more---and should have been  \\
Even such, but for some personal concerns  \\
That hung about me in my own despite	  \\
Perpetually, no heavy weight, but still  \\
A baffling and a hindrance, a controul  \\
Which made the thought of planning for myself  \\
A course of independent study seem  \\
An act of disobedience towards them	  \\
Who loved me, proud rebellion and unkind.  \\
This bastard virtue---rather let it have  \\
A name it more deserves, this cowardise---  \\
Gave treacherous sanction to that over-love  \\
Of freedom planted in me from the very first,	  \\
And indolence, by force of which I turned  \\
From regulations even of my own  \\
As from restraints and bonds. And who can tell,  \\
Who knows what thus may have been gained, both then  \\
And at a later season, or preserved---	  \\
What love of Nature, what original strength  \\
Of contemplation, what intuitive truths,  \\
The deepest and the best, and what research  \\
Unbiassed, unbewildered, and unawed?  \\
The poet's soul was with me at that time,	  \\
Sweet meditations, the still overflow  \\
Of happiness and truth. A thousand hopes  \\
Were mine, a thousand tender dreams, of which  \\
No few have since been realized, and some  \\
Do yet remain, hopes for my future life.	  \\
Four years and thirty, told this very week,  \\
Have I been now a sojourner on earth,  \\
And yet the morning gladness is not gone  \\
Which then was in my mind. Those were the days  \\
Which also first encouraged me to trust	  \\
With firmness, hitherto but lightly touched  \\
With such a daring thought, that I might leave  \\
Some monument behind me which pure hearts  \\
Should reverence. The instinctive humbleness,  \\
Uphelp even by the very name and thought	  \\
Of printed books and authorship, began  \\
To melt away; and further, the dread awe  \\
Of mighty names was softened down, and seemed  \\
Approachable, admitting fellowship  \\
Of modest sympathy. Such aspect now,	  \\
Though not familiarly, my mind put on;  \\
I loved and I enjoyed---that was my chief  \\
And ruling business, happy in the strength  \\
And loveliness of imagery and thought.  \\
All winter long, whenever free to take	  \\
My choice, did I at nights frequent our groves  \\
And tributary walks---the last, and oft  \\
The only one, who had been lingering there  \\
Through hours of silence till the porter's bell,  \\
A punctual follower on the stroke of nine,	  \\
Rang with its blunt unceremonious voice,  \\
Inexorable summons. Lofty elms,  \\
Inviting shades of opportune recess,  \\
Did give composure to a neighbourhood  \\
Unpeaceful in itself. A single tree	  \\
There was, no doubt yet standing there, an ash,  \\
With sinuous trunk, boughs exquisitely wreathed:  \\
Up from the ground and almost to the top  \\
The trunk and master branches everywhere  \\
Were green with ivy, and the lightsome twigs	  \\
And outer spray profusely tipped with seeds  \\
That hung in yellow tassels and festoons,  \\
Moving or still---a favorite trimmed out  \\
By Winter for himself, as if in pride,  \\
And with outlandish grace. Oft have I stood	  \\
Foot-bound uplooking at this lovely tree  \\
Beneath a frosty moon. The hemisphere  \\
Of magic fiction, verse of mine perhaps  \\
May never tread, but scarcely Spenser's self  \\
Could have more tranquil visions in his youth,	  \\
More bright appearances could scarcely see  \\
Of human forms and superhuman powers,  \\
Than I beheld standing on winter nights  \\
Alone beneath this fairy work of earth.  \\
'Twould be a waste of labour to detail	  \\
The rambling studies of a truant youth---  \\
Which further may be easily divined,  \\
What, and what kind they were. My inner knowledge  \\
(This barely will I note) was oft in depth  \\
And delicacy like another mind,	  \\
Sequestered from my outward taste in books---  \\
And yet the books which then I loved the most  \\
Are dearest to me now; for, being versed  \\
In living Nature, I had there a guide  \\
Which opened frequently my eyes, else shut,	  \\
A standard which was usefully applied,  \\
Even when unconsciously, to other things  \\
Which less I understood. In general terms,  \\
I was a better judge of thoughts than words,  \\
Misled as to these latter not alone	  \\
By common inexperience of youth,  \\
But by the trade in classic niceties,  \\
Delusion to young scholars incident---  \\
And old ones also---by that overprized  \\
And dangerous craft of picking phrases out	  \\
From languages that want the living voice  \\
To make of them a nature to the heart,  \\
To tell us what is passion,  \\
what is truth, What reason,  \\
what simplicity and sense.  \\
Yet must I not entirely overlook	  \\
The pleasure gathered from the elements  \\
Of geometric science. I had stepped  \\
In these inquiries but a little way,  \\
No farther than the threshold---with regret  \\
Sincere I mention this---but there I found	  \\
Enough to exalt, to chear me and compose.  \\
With Indian awe and wonder, ignorance  \\
Which even was cherished, did I meditate  \\
Upon the alliance of those simple, pure  \\
Proportions and relations, with the frame	  \\
And laws of Nature---how they could become  \\
Herein a leader to the human mind---  \\
And made endeavours frequent to detect  \\
The process by dark guesses of my own.  \\
Yet from this source more frequently I drew	  \\
A pleasure calm and deeper, a still sense  \\
Of permanent and universal sway  \\
And paramount endowment in the mind,  \\
An image not unworthy of the one  \\
Surpassing life, which---out of space and time,	  \\
Nor touched by welterings of passion---is,  \\
And hath the name of, God. Transcendent peace  \\
And silence did await upon these thoughts  \\
That were a frequent comfort to my youth.  \\
And as I have read of one by shipwreck thrown	  \\
With fellow sufferers whom the waves had spared  \\
Upon a region uninhabited,  \\
An island of the deep, who having brought  \\
To land a single volume and no more---  \\
A treatise of geometry---was used,	  \\
Although of food and clothing destitute,  \\
And beyond common wretchedness depressed,  \\
To part from company and take this book,  \\
Then first a self-taught pupil in those truths,  \\
To spots remote and corners of the isle	  \\
By the seaside, and draw his diagrams  \\
With a long stick upon the sand, and thus  \\
Did oft beguile his sorrow, and almost  \\
Forget his feeling: even so---if things  \\
Producing like effect from outward cause	  \\
So different may rightly be compared---  \\
So was it with me then, and so will be  \\
With poets ever. Mighty is the charm  \\
Of those abstractions to a mind beset  \\
With images, and haunted by itself,	  \\
And specially delightful unto me  \\
Was that clear synthesis built up aloft  \\
So gracefully, even then when it appeared  \\
No more than as a plaything, or a toy  \\
Embodied to the sense---not what it is	  \\
In verity, an independent world  \\
Created out of pure intelligence.  \\
Such dispositions then were mine, almost  \\
Through grace of heaven and inborn tenderness.  \\
And not to leave the picture of that time	  \\
Imperfect, with these habits I must rank  \\
A melancholy, from humours of the blood  \\
In part, and partly taken up, that loved  \\
A pensive sky, sad days, and piping winds,  \\
The twilight more than dawn, autumn than spring---	  \\
A treasured and luxurious gloom of choice  \\
And inclination mainly, and the mere  \\
Redundancy of youth's contentedness.  \\
Add unto this a multitude of hours  \\
Pilfered away by what the bard who sang	  \\
Of the enchanter Indolence hath called  \\
'Good-natured lounging', and behold a map  \\
Of my collegiate life: far less intense  \\
Than duty called for, or, without regard  \\
To duty, might have sprung up of itself	  \\
By change of accidents; or even---to speak  \\
Without unkindness---in another place.  \\
In summer among distant nooks I roved---  \\
Dovedale, or Yorkshire dales, or through bye-tracts  \\
Of my own native region---and was blest	  \\
Between those sundry wanderings with a joy  \\
Above all joys, that seemed another morn  \\
Risen on mid-noon: the presence, friend, I mean  \\
Of that sole sister, she who hath been long  \\
Thy treasure also, thy true friend and mine,	  \\
Now after separation desolate  \\
Restored to me---such absence that she seemed  \\
A gift then first bestowed. The gentle banks  \\
Of Emont, hitherto unnamed in song,  \\
And that monastic castle, on a flat,	  \\
Low-standing by the margin of the stream,  \\
A mansion not unvisited of old  \\
By Sidney, where, in sight of our  \\
Helvellyn, Some snatches he might pen for aught we know  \\
Of his Arcadia, by fraternal love	  \\
Inspired---that river and that mouldering dome  \\
Have seen us sit in many a summer hour,  \\
My sister and myself, when, having climbed  \\
In danger through some window's open space,  \\
We looked abroad, or on the turret's head	  \\
Lay listening to the wild-flowers and the grass  \\
As they gave out their whispers to the wind.  \\
Another maid there was, who also breathed  \\
A gladness o'er that season, then to me  \\
By her exulting outside look of youth	  \\
And placid under-countenance first endeared---  \\
That other spirit, Coleridge, who is now  \\
So near to us, that meek confiding heart,  \\
So reverenced by us both. O'er paths and fields  \\
In all that neighbourhood, through narrow lanes	  \\
Of eglantine, and through the shady woods,  \\
And o'er the Border Beacon and the waste  \\
Of naked pools and common crags that lay  \\
Exposed on the bare fell, was scattered love---  \\
A spirit of pleasure, and youth's golden gleam.	  \\
O friend, we had not seen thee at that time,  \\
And yet a power is on me and a strong  \\
Confusion, and I seem to plant thee there.  \\
Far art thou wandered now in search of health,  \\
And milder breezes---melancholy lot---	  \\
But thou art with us, with us in the past,  \\
The present, with us in the times to come.  \\
There is no grief, no sorrow, no despair,  \\
No languor, no dejection, no dismay,  \\
No absence scarcely can there be, for those	  \\
Who love as we do. Speed thee well! divide  \\
Thy pleasure with us; thy returning strength,  \\
Receive it daily as a joy of ours;  \\
Share with us thy fresh spirits, whether gift  \\
Of gales Etesian or of loving thoughts.	  \\
I too have been a wanderer, but, alas,  \\
How different is the fate of different men,  \\
Though twins almost in genius and in mind.  \\
Unknown unto each other, yea, and breathing  \\
As if in different elements, we were framed	  \\
To bend at last to the same discipline,  \\
Predestined, if two beings ever were,  \\
To seek the same delights, and have one health,  \\
One happiness. Throughout this narrative,  \\
Else sooner ended, I have known full well	  \\
For whom I thus record the birth and growth  \\
Of gentleness, simplicity, and truth,  \\
And joyous loves that hallow innocent days  \\
Of peace and self-command. Of rivers, fields,  \\
And groves, I speak to thee, my friend---to thee	  \\
Who, yet a liveried schoolboy in the depths  \\
Of the huge city, on the leaded roof  \\
Of that wide edifice, thy home and school,  \\
Wast used to lie and gaze upon the clouds  \\
Moving in heaven, or haply, tired of this,	  \\
To shut thine eyes and by internal light  \\
See trees, and meadows, and thy native stream  \\
Far distant---thus beheld from year to year  \\
Of thy long exile. Nor could I forget  \\
In this late portion of my argument	  \\
That scarcely had I finally resigned  \\
My rights among those academic bowers  \\
When thou wert thither guided. From the heart  \\
Of London, and from cloisters there, thou cam'st  \\
And didst sit down in temperance and peace,   \\
A rigorous student. What a stormy course  \\
Then followed---oh, it is a pang that calls  \\
For utterance, to think how small a change  \\
Of circumstances might to thee have spared  \\
A world of pain, ripened ten thousand hopes   \\
For ever withered. Through this retrospect  \\
Of my own college life I still have had  \\
Thy after-sojourn in the self-same place  \\
Present before my eyes, have played with times  \\
(I speak of private business of the thought)   \\
And accidents as children do with cards,  \\
Or as a man, who, when his house is built,  \\
A frame locked up in wood and stone, doth still  \\
In impotence of mind by his fireside  \\
Rebuild it to his liking. I have thought   \\
Of thee, thy learning, gorgeous eloquence,  \\
And all the strength and plumage of thy youth,  \\
Thy subtle speculations, toils abstruse  \\
Among the schoolmen, and Platonic forms  \\
Of wild ideal pageantry, shaped out   \\
From things well-matched, or ill, and words for things---  \\
The self-created sustenance of a mind  \\
Debarred from Nature's living images,  \\
Compelled to be a life unto itself,  \\
And unrelentingly possessed by thirst   \\
Of greatness, love, and beauty. Not alone,  \\
Ah, surely not in singleness of heart  \\
Should I have seen the light of evening fade  \\
Upon the silent Cam, if we had met,  \\
Even at that early time: I needs must hope,   \\
Must feel, must trust, that my maturer age  \\
And temperature less willing to be moved,  \\
My calmer habits, and more steady voice,  \\
Would with an influence benign have soothed  \\
Or chased away the airy wretchedness   \\
That battened on thy youth. But thou hast trod,  \\
In watchful meditation thou hast trod,  \\
A march of glory, which doth put to shame  \\
These vain regrets; health suffers in thee, else  \\
Such grief for thee would be the weakest thought   \\
That ever harboured in the breast of man.  \\
A passing word erewhile did lightly touch  \\
On wanderings of my own, and now to these  \\
My poem leads me with an easier mind.  \\
The employments of three winters when I wore   \\
A student's gown have been already told,  \\
Or shadowed forth as far as there is need---  \\
When the third summer brought its liberty  \\
A fellow student and myself, he too  \\
A mountaineer, together sallied forth,   \\
And, staff in hand on foot pursued our way  \\
Towards the distant Alps. An open slight  \\
Of college cares and study was the scheme,  \\
Nor entertained without concern for those  \\
To whom my worldly interests were dear,   \\
But Nature then was sovereign in my heart,  \\
And mighty forms seizing a youthful fancy  \\
Had given a charter to irregular hopes.  \\
In any age, without an impulse sent  \\
From work of nations and their goings-on,   \\
I should have been possessed by like desire;  \\
But 'twas a time when Europe was rejoiced,  \\
France standing on the top of golden hours,  \\
And human nature seeming born again.  \\
Bound, as I said, to the Alps, it was our lot   \\
To land at Calais on the very eve  \\
Of that great federal day; and there we saw,  \\
In a mean city and among a few,  \\
How bright a face is worn when joy of one  \\
Is joy of tens of millions. Southward thence   \\
We took our way, direct through hamlets, towns,  \\
Gaudy with reliques of that festival,  \\
Flowers left to wither on triumphal arcs  \\
And window-garlands. On the public roads---  \\
And once three days successively through paths	  \\
By which our toilsome journey was abridged---  \\
Among sequestered villages we walked  \\
And found benevolence and blessedness  \\
Spread like a fragrance everywhere, like spring  \\
That leaves no corner of the land untouched.	  \\
Where elms for many and many a league in files,  \\
With their thin umbrage, on the stately roads  \\
Of that great kingdom rustled o'er our heads,  \\
For ever near us as we paced along,  \\
'Twas sweet at such a time---with such delights	  \\
On every side, in prime of youthful strength---  \\
To feed a poet's tender melancholy  \\
And fond conceit of sadness, to the noise  \\
And gentle undulation which they made.  \\
Unhoused beneath the evening star we saw	  \\
Dances of liberty, and, in late hours  \\
Of darkness, dances in the open air.  \\
Among the vine-clad hills of Burgundy,  \\
Upon the bosom of the gentle Soane  \\
We glided forward with the flowing stream:	  \\
Swift Rhone, thou wert the wings on which we cut  \\
Between they lofty rocks. Enchanting show  \\
Those woods and farms and orchards did present,  \\
And single cottages and lurking towns---  \\
Reach after reach, procession without end,	  \\
Of deep and stately vales. A lonely pair  \\
Of Englishmen we were, and sailed along  \\
Clustered together with a merry crowd  \\
Of those emancipated, with a host  \\
Of travellers, chiefly delegates returning	  \\
From the great spousals newly solemnized  \\
At their chief city, in the sight of Heaven.  \\
Like bees they swarmed, gaudy and gay as bees;  \\
Some vapoured in the unruliness of joy,  \\
And flourished with their swords as if to fight   \\
The saucy air. In this blithe company  \\
We landed, took with them our evening meal,  \\
Guests welcome almost as the angels were  \\
To Abraham of old. The supper done,  \\
With flowing cups elate and happy thoughts   \\
We rose at signal given, and formed a ring,  \\
And hand in hand danced round and round the board;  \\
All hearts were open, every tongue was loud  \\
With amity and glee. We bore a name  \\
Honoured in France, the name of Englishmen,   \\
And hospitably did they give us hail  \\
As their forerunners in a glorious course;  \\
And round and round the board they danced again.  \\
With this same throng our voyage we pursued  \\
At early dawn; the monastery bells   \\
Made a sweet jingling in our youthful ears---  \\
The rapid river flowing without noise---  \\
And every spire we saw among the rocks  \\
Spake with a sense of peace, at intervals  \\
Touching the heart amid the boisterous crew   \\
With which we were environed. Having parted  \\
From this glad rout, the convent of Chartreuse  \\
Received us two days afterwards, and there  \\
We rested in an awful solitude---  \\
Thence onward to the country of the Swiss.   \\
'Tis not my present purpose to retrace  \\
That variegated journey step by step;  \\
A march it was of military speed,  \\
And earth did change her images and forms   \\
Before us fast as clouds are changed in heaven.  \\
Day after day, up early and down late,  \\
From vale to vale, from hill to hill we went,  \\
From province on to province did we pass,  \\
Keen hunters in a chace of fourteen weeks---   \\
Eager as birds of prey, or as a ship  \\
Upon the stretch when winds are blowing fair.  \\
Sweet coverts did we cross of pastoral life,  \\
Enticing vallies---greeted them, and left  \\
Too soon, while yet the very flash and gleam	  \\
Of salutation were not passed away.  \\
Oh, sorrow for the youth who could have seen  \\
Unchastened, unsubdued, unawed, unraised  \\
To patriarchal dignity of mind  \\
And pure simplicity of wish and will,	  \\
Those sanctified abodes of peaceful man.  \\
My heart leaped up when first I did look down  \\
On that which was first seen of those deep haunts,  \\
A green recess, an aboriginal vale,  \\
Quiet, and lorded over and possessed	  \\
By naked huts, wood-built, and sown like tents  \\
Or Indian cabins over the fresh lawns  \\
And by the river-side.  \\
That day we first  \\
Beheld the summit of Mount Blanc, and grieved	  \\
To have a soulless image on the eye  \\
Which had usurped upon a living thought  \\
That never more could be. The wondrous Vale  \\
Of Chamouny did, on the following dawn,  \\
With its dumb cataracts and streams of ice---	  \\
A motionless array of mighty waves,  \\
Five rivers broad and vast---make rich amends,  \\
And reconciled us to realities.  \\
There small birds warble from the leafy trees,  \\
The eagle soareth in the element,	  \\
There doth the reaper bind the yellow sheaf,  \\
The maiden spread the haycock in the sun,  \\
While Winter like a tame`d lion walks,  \\
Descending from the mountain to make sport  \\
Among the cottages by beds of flowers.	  \\
Whate'er in this wide circuit we beheld  \\
Or heard was fitted to our unripe state  \\
Of intellect and heart. By simple strains  \\
Of feeling, the pure breath of real life,	  \\
We were not left untouched. With such a book  \\
Before our eyes we could not chuse but read  \\
A frequent lesson of sound tenderness,  \\
The universal reason of mankind,  \\
The truth of young and old. Nor, side by side	  \\
Pacing, two brother pilgrims, or alone  \\
Each with his humour, could we fail to abound---  \\
Craft this which hath been hinted at before---  \\
In dreams and fictions pensively composed:  \\
Dejection taken up for pleasure's sake,	  \\
And gilded sympathies, the willow wreath,  \\
Even among those solitudes sublime,  \\
And sober posies of funereal flowers,  \\
Culled from the gardens of the Lady Sorrow,  \\
Did sweeten many a meditative hour.	  \\
Yet still in me, mingling with these delights,  \\
Was something of stern mood, an under-thirst  \\
Of vigor, never utterly asleep.  \\
Far different dejection once was mine---  \\
A deep and genuine sadness then I felt---	  \\
The circumstances I will here relate  \\
Even as they were. Upturning with a band  \\
Of travellers, from the Valais we had clomb  \\
Along the road that leads to Italy;  \\
A length of hours, making of these our guides,	  \\
Did we advance, and, having reached an inn  \\
Among the mountains, we together ate  \\
Our noon's repast, from which the travellers rose  \\
Leaving us at the board. Erelong we followed,  \\
Descending by the beaten road that led	  \\
Right to a rivulet's edge, and there broke off;  \\
The only track now visible was one  \\
Upon the further side, right opposite,  \\
And up a lofty mountain. This we took,  \\
After a little scruple and short pause,	  \\
And climbed with eagerness---though not, at length,  \\
Without surprize and some anxiety  \\
On finding that we did not overtake  \\
Our comrades gone before. By fortunate chance,  \\
While every moment now encreased our doubts,	  \\
A peasant met us, and from him we learned  \\
That to the place which had perplexed us first  \\
We must descend, and there should find the road  \\
Which in the stony channel of the stream  \\
Lay a few steps, and then along its banks---	  \\
And further, that thenceforward all our course  \\
Was downwards with the current of that stream.  \\
Hard of belief, we questioned him again,  \\
And all the answers which the man returned  \\
To our inquiries, in their sense and substance	  \\
Translated by the feelings which we had,  \\
Ended in this---that we had crossed the Alps.  \\
Imagination!---lifting up itself  \\
Before the eye and progress of my song  \\
Like an unfathered vapour, here that power,	  \\
In all the might of its endowments, came  \\
Athwart me. I was lost as in a cloud,  \\
Halted without a struggle to break through,  \\
And now, recovering, to my soul I say  \\
'I recognise thy glory'. In such strength	  \\
Of usurpation, in such visitings  \\
Of awful promise, when the light of sense  \\
Goes out in flashes that have shewn to us  \\
The invisible world, doth greatness make abode,  \\
There harbours whether we be young or old.	  \\
Our destiny, our nature, and our home, Is  \\
with infinitude---and only there;  \\
With hope it is, hope that can never die,  \\
Effort, and expectation, and desire,  \\
And something evermore about to be.	  \\
The mind beneath such banners militant  \\
Thinks not of spoils or trophies, nor of aught  \\
That may attest its prowess, blest in thoughts  \\
That are their own perfection and reward---  \\
Strong in itself, and in the access of joy	  \\
Which hides in like the overflowing Nile.  \\
The dull and heavy slackening which ensued  \\
Upon those tidings by the peasant given  \\
Was soon dislodged; downwards we hurried fast,  \\
And entered with the road which we had missed	  \\
Into a narrow chasm. The brook and road  \\
Were fellow-travellers in this gloomy pass,  \\
And with them did we journey several hours  \\
At a slow step. The immeasurable height  \\
Of woods decaying, never to be decayed,	  \\
The stationary blasts of waterfalls,  \\
And everywhere along the hollow rent  \\
Winds thwarting winds, bewildered and forlorn,  \\
The torrents shooting from the clear blue sky,  \\
The rocks that muttered close upon our ears---	  \\
Black drizzling crags that spake by the wayside  \\
As if a voice were in them---the sick sight  \\
And giddy prospect of the raving stream,  \\
The unfettered clouds and region of the heavens,  \\
Tumult and peace, the darkness and the light,	  \\
Were all like workings of one mind, the features  \\
Of the same face, blossoms upon one tree,  \\
Characters of the great apocalypse,  \\
The types and symbols of eternity,  \\
Of first, and last, and midst, and without end.	  \\
That night our lodging was an alpine house,  \\
An inn, or hospital (as they are named),  \\
Standing in that same valley by itself,  \\
And close upon the confluence of two streams---  \\
A dreary mansion, large beyond all need,	  \\
With high and spacious rooms, deafened and stunned  \\
By noise of waters, making innocent sleep  \\
Lie melancholy among weary bones.  \\
Uprisen betimes, our journey we renewed,  \\
Led by the stream, ere noon-day magnified	  \\
Into a lordly river, broad and deep,  \\
Dimpling along in silent majesty  \\
With mountains for its neighbours, and in view  \\
Of distant mountains and their snowy tops,  \\
And thus proceeding to Locarno's lake,	  \\
Fit resting-place for such a visitant.  \\
Locarno, spreading out in width like heaven,  \\
And Como thou---a treasure by the earth  \\
Kept to itself, a darling bosomed up  \\
In Abyssinian privacy---I spake	  \\
Of thee, thy chestnut woods and garden plots  \\
Of Indian corn tended by dark-eyed maids,  \\
Thy lofty steeps, and pathways roofed with vines  \\
Winding from house to house, from town to town  \\
(Sole link that binds them to each other), walks	  \\
League after league, and cloistral avenues  \\
Where silence is if music be not there:  \\
While yet a youth undisciplined in verse,  \\
Through fond ambition of my heart I told  \\
Your praises, nor can I approach you now	  \\
Ungreeted by a more melodious song,  \\
Where tones of learned art and Nature mixed  \\
May frame enduring language.  \\
Like a breeze Or sunbeam over your domain I passed  \\
In motion without pause; but ye have left	  \\
Your beauty with me, an impassioned sight  \\
Of colours and of forms, whose power is sweet  \\
And gracious, almost, might I dare to say,  \\
As virtue is, or goodness---sweet as love,  \\
Or the remembrance of a noble deed,	  \\
Or gentlest visitations of pure thought  \\
When God, the giver of all joy, is thanked  \\
Religiously in silent blessedness---  \\
Sweet as this last itself, for such it is.  \\
Through those delightful pathways we advanced	  \\
Two days, and still in presence of the lake,  \\
Which winding up among the Alps now changed  \\
Slowly its lovely countenance and put on  \\
A sterner character. The second night,  \\
In eagerness, and by report misled	  \\
Of those Italian clocks that speak the time  \\
In fashion different from ours, we rose  \\
By moonshine, doubting not that day was near,  \\
And that, meanwhile, coasting the water's edge  \\
As hitherto, and with as plain a track	  \\
To be our guide, we might behold the scene  \\
In its most deep repose. We left the town  \\
Of Gravedona with this hope, but soon  \\
Were lost, bewildered among woods immense,  \\
Where, having wandered for a while, we stopped	  \\
And on a rock sate down to wait for day.  \\
An open place it was and overlooked  \\
From high the sullen water underneath,  \\
On which a dull red image of the moon  \\
Lay bedded, changing oftentimes its form	  \\
Like an uneasy snake. Long time we sate,  \\
For scarcely more than one hour of the night---  \\
Such was our error---had been gone when we  \\
Renewed our journey. On the rock we lay  \\
And wished to sleep, but could not for the stings	  \\
Of insects, which with noise like that of noon  \\
Filled all the woods. The cry of unknown birds,  \\
the mountains---more by darkness visible  \\
And their own size, than any outward light---  \\
The breathless wilderness of clouds, the clock	  \\
That told with unintelligible voice  \\
The widely parted hours, the noise of streams,  \\
And sometimes rustling motions nigh at hand  \\
Which did not leave us free from personal fear,  \\
And lastly, the withdrawing moon that set	  \\
Before us while she still was high in heaven---  \\
These were our food, and such a summer night  \\
Did to that pair of golden days succeed,  \\
With now and then a doze and snatch of sleep,  \\
On Como's banks, the same delicious lake.	  \\
But here I must break off, and quit at once,  \\
Though loth, the record of these wanderings,  \\
A theme which may seduce me else beyond  \\
All reasonable bounds. Let this alone  \\
Be mentioned as a parting word, that not	  \\
In hollow exultation, dealing forth  \\
Hyperboles of praise comparative;  \\
Not rich one moment to be poor for ever;  \\
Not prostrate, overborne---as if the mind  \\
Itself were nothing, a mean pensioner	  \\
On outward forms---did we in presence stand  \\
Of that magnificent region. On the front  \\
Of this whole song is written that my heart  \\
Must, in such temple, needs have offered up  \\
A different worship. Finally, whate'er	  \\
I saw, or heard, or felt, was but a stream  \\
That flowed into a kindred stream, a gale  \\
That helped me forwards, did administer  \\
To grandeur and to tenderness---to the one  \\
Directly, but to tender thoughts by means	  \\
Less often instantaneous in effect---  \\
Conducted me to these along a path  \\
Which, in the main, was more circuitous.  \\
Oh most beloved friend, a glorious time,  \\
A happy time that was. Triumphant looks	  \\
Were then the common language of all eyes:  \\
As if awakened from sleep, the nations hailed  \\
Their great expectancy; the fife of war  \\
Was then a spirit-stirring sound indeed,  \\
A blackbird's whistle in a vernal grove.	  \\
We left the Swiss exulting in the fate  \\
Of their neighbours, and, when shortening fast  \\
Our pilgrimage---nor distant far from home---  \\
We crossed the Brabant armies on the fret  \\
For battle in the cause of Liberty.	  \\
A stripling, scarcely of the household then  \\
Of social life, I looked upon these things  \\
As from a distance---heard, and saw, and felt,  \\
Was touched but with no intimate concern---  \\
I seemed to move among them as a bird   \\
Moves through the air, or as a fish pursues  \\
Its business in its proper element.  \\
I needed not that joy, I did not need  \\
Such help: the ever-living universe  \\
And independent spirit of pure youth   \\
Were with me at that season, and delight  \\
Was in all places spread around my steps  \\
As constant as the grass upon the fields. \\
\end{verse}

%%%%%%%%%%%%%%%%%%%%%%%%%%%%%%%%%%%%%%%%%%%%%%%%%%%%%%%% \\
\chapter*[Book Seventh]{Book Seventh \\ Residence in London}
\addcontentsline{toc}{chapter}{Book Seventh Residence in London}

\begin{verse}
FIVE years are vanished since I first poured out,  \\
Saluted by that animating breeze  \\
Which met me issuing from the city's walls,  \\
A glad preamble to this verse. I sang  \\
Aloud in dithyrambic fervour, deep	  \\
But short-lived uproar, like a torrent sent  \\
Out of the bowels of a bursting cloud  \\
Down Scawfell or Blencathara's rugged sides,  \\
A waterspout from heaven. But 'twas not long  \\
Ere the interrupted strain broke forth once more,	  \\
And flowed awhile in strength; then stopped for years---  \\
Not heard again until a little space  \\
Before last primrose-time. Belove`d friend,  \\
The assurances then given unto myself,  \\
Which did beguile me of some heavy thoughts	  \\
At thy departure to a foreign land,  \\
Have failed; for slowly doth this work advance.  \\
Through the whole summer I have been at rest,  \\
Partly from voluntary holiday  \\
And part through outward hindrance. But I heard	  \\
After the hour of sunset yester-even,  \\
Sitting within doors betwixt light and dark,  \\
A voice that stirred me. 'Twas a little band,  \\
A quire of redbreasts gathered somewhere near  \\
My threshold, minstrels from the distant woods	  \\
And dells, sent in by Winter to bespeak  \\
For the old man a welcome, to announce  \\
With preparation artful and benign---  \\
Yea, the most gentle music of the year---  \\
That their rough lord had left the surly north,	  \\
And hath begun his journey. A delight  \\
At this unthought-of-greeting unawares  \\
Smote me, a sweetness of the coming time,  \\
And, listening, I half whispered, 'We will be,  \\
Ye heartsome choristers, ye and I will be	  \\
Brethren, and in the hearing of bleak winds  \\
Will chaunt together.' And, thereafter, walking  \\
By later twilight on the hills I saw  \\
A glow-worm, from beneath a dusky shade  \\
Or canopy of the yet unwithered fern	  \\
Clear shining, like a hermit's taper seen  \\
Through a thick forest. Silence touched me here  \\
No less than sound had done before; the child  \\
Of summer, lingering, shining by itself,  \\
The voiceless worm on the unfrequented hills,	  \\
Seemed sent on the same errand with the quire  \\
Of winter that had warbled at my door,  \\
And the whole year seemed tenderness and love.  \\
The last night's genial feeling overflowed  \\
Upon this morning, and my favorite grove---	  \\
Now tossing its dark boughs in sun and wind---  \\
Spreads through me a commotion like its own,  \\
Something that fits me for the poet's task,  \\
Which we will now resume with chearful hope,  \\
Nor checked by aught of tamer argument	  \\
That lies before us, needful to be told.  \\
Returned from that excursion, soon I bade  \\
Farewell for ever to the private bowers  \\
Of gowned students---quitted these, no more	  \\
To enter them, and pitched my vagrant tent,  \\
A casual dweller and at large, among  \\
The unfenced regions of society.  \\
Yet undetermined to what plan of life  \\
I should adhere, and seeming thence to have	  \\
A little space of intermediate time  \\
Loose and at full command, to London first  \\
I turned, if not in calmness, nevertheless  \\
In no disturbance of excessive hope---  \\
At ease from all ambition personal,	  \\
Frugal as there was need, and though self-willed,  \\
Yet temperate and reserved, and wholly free  \\
From dangerous passions. 'Twas at least two years  \\
Before this season when I first beheld  \\
That mighty place, a transient visitant;	  \\
And now it pleased me my abode to fix  \\
Single in the wide waste. To have a house,  \\
It was enough---what matter for a home?---  \\
That owned me, living chearfully abroad  \\
With fancy on the stir from day to day,	  \\
And all my young affections out of doors.  \\
There was a time when whatso'er is feigned  \\
Of airy palaces and gardens built  \\
By genii of romance, or hath in grave  \\
Authentic history been set forth of Rome,	  \\
Alcairo, Babylon, or Persepolis,  \\
Or given upon report by pilgrim friars  \\
Of golden cities ten months' journey deep  \\
Among Tartarean wilds, fell short, far short,  \\
Of that which I in simpleness believed	  \\
And thought of London---held me by a chain  \\
Less strong of wonder and obscure delight.  \\
I know not that herein I shot beyond  \\
The common mark of childhood, but I well  \\
Remember that among our flock of boys	  \\
Was one, a cripple from the birth, whom chance  \\
Summoned from school to London---fortunate  \\
And envied traveller---and when he returned,  \\
After short absence, and I first set eyes  \\
Upon his person, verily, though strange	  \\
The thing may seem, I was not wholly free  \\
From disappointment to behold the same  \\
Appearance, the same body, not to find  \\
Some change, some beams of glory brought away  \\
From that new region, Much I questioned him,	  \\
And every word he uttered, on my ears  \\
Fell flatter than a cage`d parrot's note,  \\
That answers unexpectedly awry,  \\
And mocks the prompter's listening. Marvellous things  \\
My fancy had shaped forth of sights and shows,	  \\
Processions, equipages, lords and dukes,  \\
The King and the King's palace, and not last  \\
Or least, heaven bless him! the renowned Lord Mayor---  \\
Dreams hardly less intense than those which wrought  \\
A change of purpose in young Whittington	  \\
When he in fiendlessness, a drooping boy,  \\
Sate on a stone and heard the bells speak out  \\
Articulate music. Above all, one thought  \\
Baffled my understanding, how men lived  \\
Even next-door neighbours, as we say, yet still	  \\
Strangers, and knowing not each other's names.  \\
Oh wondrous power of words, how sweet they are  \\
According to the meaning which they bring---  \\
Vauxhall and Ranelagh, I then had heard  \\
Of your green groves and wilderness of lamps,	  \\
Your gorgeous ladies, fairy cataracts,  \\
And pageant fireworks. Nor must we forget  \\
Those other wonders, different in kind  \\
Though scarcely less illustrious in degree,  \\
The river proudly bridged, the giddy top	  \\
And Whispering Gallery of St. Paul's, the tombs  \\
Of Westminster, the Giants of Guildhall,  \\
Bedlam and the two figures at its gates,  \\
Streets without end and churches numberless,  \\
Statues with flowery gardens in vast squares,	  \\
The Monument, and Armoury of the Tower.  \\
These fond imaginations, of themselves,  \\
Had long before given way in season due,  \\
Leaving a throng of others in their stead;  \\
And now I looked upon the real scene,	  \\
Familiarly perused it day by day,  \\
With keen and lively pleasure even there  \\
Where disappointment was the strongest, pleased  \\
Through courteous self-submission, as a tax  \\
Paid to the object by prescriptive right,	  \\
A thing that ought to be. Shall I give way,  \\
Copying the impression of the memory---  \\
Though things remembered idly do half seem  \\
The work of fancy---shall I, as the mood  \\
Inclines me, here describe for pastime's sake,	  \\
Some portion of that motley imagery,  \\
A vivid pleasure of my youth, and now,  \\
Among the lonely places that I love,  \\
A frequent daydream for my riper mind?  \\
And first, the look and aspect of the place---	  \\
The broad highway appearance, as it strikes  \\
On strangers of all ages, the quick dance  \\
Of colours, lights and forms, the Babel din,  \\
The endless stream of men and moving things,  \\
From hour to hour the illimitable walk	  \\
Still among streets, with clouds and sky above,  \\
The wealth, the bustle and the eagerness,  \\
The glittering chariots with their pampered steeds,  \\
Stalls, barrows, porters, midway in the street  \\
The scavenger that begs with hat in hand,	  \\
The labouring hackney-coaches, the rash speed  \\
Of coaches travelling far, whirled on with horn  \\
Loud blowing, and the sturdy drayman's team  \\
Ascending from some alley of the Thames  \\
And striking right across the crowded Strand	  \\
Till the fore-horse veer round with punctual skill;  \\
Here, there, and everywhere, a weary throng,  \\
That comers and the goers face to face---  \\
Face after face---the string of dazzling wares,  \\
Shop after shop, with symbols, blazoned names,	  \\
And all the tradesman's honours overhead:  \\
Here, fronts of houses, like a title-page  \\
With letters huge inscribed from top to toe;  \\
Stationed above the door like guardian saints,  \\
There, allegoric shapes, female or male,	  \\
Or physiognomies of real men,  \\
Land-warriors, kings, or admirals of the sea,  \\
Boyle, Shakespear, Newton, or the attractive head  \\
Of some quack-doctor, famous in his day.  \\
Meanwhile the roar continues, till at length,	  \\
Escaped as from an enemy, we turn  \\
Abruptly into some sequestered nook,  \\
Still as a sheltered place when winds blow loud.  \\
At leisure thence, through tracts of thin resort,  \\
And sights and sounds that come at intervals,	  \\
We take our way---a raree-show is here  \\
With children gathered round, another street  \\
Presents a company of dancing dogs,  \\
Or dromedary with an antic pair  \\
Of monkies on his back, a minstrel-band	  \\
Of Savoyards, single and alone,  \\
An English ballad-singer. Private courts,  \\
Gloomy as coffins, and unsightly lanes  \\
Thrilled by some female vendor's scream---belike  \\
The very shrillest of all London cries---	  \\
May then entangle us awhile,  \\
Conducted through those labyrinths unawares  \\
To privileged regions and inviolate,  \\
Where from their aery lodges studious lawyers  \\
Look out on waters, walks, and gardens green.	  \\
Thence back into the throng, until we reach---  \\
Following the tide that slackens by degrees---  \\
Some half-frequented scene where wider streets  \\
Bring straggling breezes of suburban air.  \\
Here files of ballads dangle from dead walls,	  \\
Advertisements of giant size, from high  \\
Press forward in all colours on the sight---  \\
These, bold in conscious merit---lower down,  \\
That, fronted with a most imposing word,  \\
Is peradventure one in masquerade.	  \\
As on the broadening causeway we advance,  \\
Behold a face turned up towards us, strong  \\
In lineaments, and red with over-toil:  \\
'Tis one perhaps already met elsewhere,  \\
A travelling cripple, by the trunk cut short,	  \\
And stumping with his arms. In sailor's garb  \\
Another lies at length beside a range  \\
Of written characters, with chalk inscribed  \\
Upon the smooth flat stones. The nurse is here,  \\
The bachelor that loves to sun himself,	  \\
The military idler, and the dame  \\
That field-ward takes her walk in decency.  \\
Now homeward through the thickening hubbub, where  \\
See---among less distinguishable shapes---  \\
The Italian, with his frame of images	  \\
Upon his head; with basket at his waist,  \\
The Jew; the stately and slow-moving Turk,  \\
With freight of slippers piled beneath his arm.  \\
Briefly, we find (if tired of random sights,  \\
And haply to that search our thoughts should turn)	  \\
Among the crowd, conspicuous less or more  \\
As we proceed, all specimens of man  \\
Through all the colours which the sun bestows,  \\
And every character of form and face:  \\
The Swede, the Russian; from the genial south,	  \\
The Frenchman and the Spaniard; from remote  \\
America, the hunter Indian;  \\
Moors, Malays, Lascars, the Tartar and Chinese,  \\
And Negro ladies in white muslin gowns.  \\
At leisure let us view from day to day,	  \\
As they present themselves, the spectacles  \\
Within doors: troops of wild beasts, birds and beasts  \\
Of every nature from all climes convened,  \\
And, next to these, those mimic sights that ape  \\
The absolute presence of reality,	  \\
Expressing as in mirror sea and land,  \\
And what earth is, and what she hath to shew---  \\
I do not here allude to subtlest craft,  \\
By means refined attaining purest ends,  \\
But imitations fondly made in plain	  \\
Confession of man's weakness and his loves.  \\
Whether the painter---fashioning a work  \\
To Nature's circumambient scenery,  \\
And with his greedy pencil taking in  \\
A whole horizon on all sides---with power	  \\
Like that of angels or commissioned spirits,  \\
Plant us upon some lofty pinnacle  \\
Or in a ship on waters, with a world  \\
Of life and lifelike mockery to east,  \\
To west, beneath, behind us, and before,	  \\
Or more mechanic artist represent  \\
By scale exact, in model, wood or clay,  \\
From shading colours also borrowing help,  \\
Some miniature of famous spots and things,  \\
Domestic, or the boast of foreign realms:	  \\
The Firth of Forth, and Edinburgh, throned  \\
On crags, fit empress of that mountain land;  \\
St Peter's Church; or, more aspiring aim,  \\
In microscopic vision, Rome itself;  \\
Or else, perhaps, some rural haunt, the Falls	  \\
Of Tivoli, and dim Frescati's bowers,  \\
And high upon the steep that mouldering fane,  \\
The Temple of the Sibyl---every tree  \\
Through all the landscape, tuft, stone, scratch minute,  \\
And every cottage, lurking in the rocks---	  \\
All that the traveller sees when he is there.  \\
And to these exhibitions mute and still  \\
Others of wider scope, where living men,  \\
Music, and shifting pantomimic scenes,	  \\
Together joined their multifarious aid  \\
To heighten the allurement. Need I fear  \\
To mention by its name, as in degree  \\
Lowest of these, and humblest in attempt---  \\
Yet richly graced with honours of its own---	  \\
Half-rural Sadler's Wells? Though at that time  \\
Intolerant, as is the way of youth  \\
Unless itself be pleased, I more than once  \\
Here took my seat, and, maugre frequent fits  \\
Of irksomeness, with ample recompense	  \\
Saw singes, rope-dancers, giants and dwarfs,  \\
Clowns, conjurors, posture-masters, harlequins,  \\
Amid the uproar of the rabblement,  \\
Perform their feats. Nor was it mean delight  \\
To watch crude Nature work in untaught minds,	  \\
To note the laws and progress of belief---  \\
Though obstinate on this way, yet on that  \\
How willingly we travel, and how far!---  \\
To have, for instance, brought upon the scene  \\
The champion, Jack the Giant-killer; lo,	  \\
He dons his coat of darkness, on the stage  \\
Walks, and atchieves his wonders, from the eye  \\
Of living mortal safe as is the moon  \\
'Hid in her vacant interlunar cave'.  \\
Delusion bold (and faith must needs be coy)	  \\
How is it wrought?---his garb is black, the word  \\
INVISIBLE flames forth upon his chest.  \\
Nor was it unamusing here to view  \\
Those samples, as of the ancient comedy  \\
And Thespian times, dramas of living men	  \\
And recent things yet warm with life: a sea-fight,  \\
Shipwreck, or some domestic incident  \\
The fame of which is scattered through the land,  \\
Such as this daring brotherhood of late  \\
Set forth---too holy theme for such a place,	  \\
And doubtless treated with irreverence,  \\
Albeit with their very best of skill---  \\
I mean, O distant friend, a story drawn  \\
From our own ground, the Maid of Buttermere,  \\
And how the spoiler came, 'a bold bad man'	  \\
To God unfaithful, children, wife, and home,  \\
And wooed the artless daughter of the hills,  \\
And wedded her, in cruel mockery  \\
Of love and marriage bonds. O friend, I speak  \\
With tender recollection of that time	  \\
When first we saw the maiden, then a name  \\
By us unheard of---in her cottage-inn  \\
Were welcomed, and attended on by her,  \\
Both stricken with one feeling of delight,  \\
An admiration of her modest mien	  \\
And carriage, marked by unexampled grace.  \\
Not unfamiliarly we since that time  \\
Have seen her, her discretion have observed,  \\
Her just opinions, female modesty,  \\
Her patience, and retiredness of mind	  \\
Unspoiled by commendation and excess  \\
Of public notice. This memorial verse  \\
Comes from the poet's heart, and is her due;  \\
For we were nursed---as almost might be said---  \\
On the same mountains, children at one time,	  \\
Must haply often on the self-same day  \\
Have from our several dwellings gone abroad  \\
To gather daffodils on Coker's stream.  \\
These last words uttered, to my argument  \\
I was returning, when---with sundry forms	  \\
Mingled, that in the way which I must tread  \\
Before me stand---thy image rose again,  \\
Mary of Buttermere! She lives in peace  \\
Upon the spot where she as born and reared;  \\
Without contamination does she live	  \\
In quietness, without anxiety.  \\
Beside the mountain chapel sleeps in earth  \\
Her new-born infant, fearless as a lamb  \\
That thither comes from some unsheltered place  \\
To rest beneath the little rock-like pile	  \\
When storms are blowing. Happy are they both,  \\
Mother and child! These feelings, in themselves  \\
Trite, do yet scarcely seem so when I think  \\
Of those ingenuous moments of our youth  \\
Ere yet by use we have learnt to slight the crimes  \\
And sorrows of the world. Those days are now  \\
My theme, and, 'mid the numerous scenes which they  \\
Have left behind them, foremost I am crossed  \\
Here by remembrance of two figures: one  \\
A rosy babe, who for a twelvemonth's space  \\
Perhaps had been of age to deal about  \\
Articulate prattle, child as beautiful  \\
As ever sate upon a mother's knee;  \\
The other was the parent of that babe---  \\
But on the mother's cheek the tints were false,   \\
A painted bloom. 'Twas at a theatre  \\
That I beheld this pair; the boy had been  \\
The pride and pleasure of all lookers-on  \\
In whatsoever place, but seemed in this  \\
A sort of alien scattered from the clouds.   \\
Of lusty vigour, more than infantine,  \\
He was in limbs, in face a cottage rose  \\
Just three part blown---a cottage-child, but ne'er  \\
Saw I by cottage or elsewhere a babe  \\
By Nature's gifts so honored. Upon a board,   \\
Whence an attendant of the theatre  \\
Served out refreshments, had this child been placed,  \\
And there he sate environed with a ring  \\
Of chance spectators, chiefly dissolute men  \\
And shameless women---treated and caressed---   \\
Ate, drank, and with the fruit and glasses played,  \\
While oaths, indecent speech, and ribaldry  \\
Were rife about him as are songs of birds  \\
In springtime after showers. The mother, too,  \\
Was present, but of her I know no more   \\
Than hath been said, and scarcely at this time  \\
Do I remember her; but I behold  \\
The lovely boy as I beheld him then,  \\
Among the wretched and the falsely gay,  \\
Like one of those who walked with hair unsinged   \\
Amid the fiery furnace. He hath since  \\
Appeared to me ofttimes as if embalmed  \\
By Nature---through some special privilege  \\
Stopped at the growth he had---destined to live,  \\
To be, to have been, come, and go, a child	  \\
And nothing more, no partner in the years  \\
That bear us forward to distress and guilt,  \\
Pain and abasement; beauty in such excess  \\
Adorned him in that miserable place.  \\
So have I thought of him a thousand times---	  \\
And seldom otherwise---but he perhaps,  \\
Mary, may now have lived till he could look  \\
With envy on thy nameless babe that sleeps  \\
Beside the mountain chapel undisturbed.  \\
It was but little more than three short years	  \\
Before the season which I speak of now  \\
When first, a traveller from our pastoral hills,  \\
Southward two hundred miles I had advanced,  \\
And for the first time in my life did hear  \\
The voice of woman utter blasphemy---	  \\
Saw woman as she is to open shame  \\
Abandoned, and the pride of public vice.  \\
Full surely from the bottom of my heart  \\
I shuddered; but the pain was almost lost,  \\
Absorbed and buried in the immensity	  \\
Of the effect: a barrier seemed at once  \\
Thrown in, that from humanity divorced  \\
The human form, splitting the race of man  \\
In twain, yet leaving the same outward shape.  \\
Distress of mind ensued upon this sight,	  \\
And ardent meditation---afterwards  \\
A milder sadness on such spectacles  \\
Attended: thought, commiseration, grief,  \\
For the individual and the overthrow  \\
Of her soul's beauty---farther at that time	  \\
Than this I was but seldom led; in truth  \\
The sorrow of the passion stopped me here.  \\
I quit this painful theme, enough is said  \\
To shew what thoughts must often have been mine  \\
At theatres, which then were my delight---	  \\
A yearning made more strong by obstacles  \\
Which slender funds imposed. Life then was new,  \\
The senses easily pleased; the lustres, lights,  \\
The carving and the gilding, paint and glare,  \\
And all the mean upholstery of the place,	  \\
Wanted not animation in my sight,  \\
Far less the living figures on the stage,  \\
Solemn or gay---whether some beauteous dame  \\
Advanced in radiance through a deep recess  \\
Of thick-entangled forest, like the moon	  \\
Opening the clouds; or sovereign king, announced  \\
With flourishing trumpets, came in full-blown state  \\
Of the world's greatness, winding round with train  \\
Of courtiers, banners, and a length of guards;  \\
Or captive led in abject weeds, and jingling	  \\
His slender manacles; or romping girl  \\
Bounced, leapt, and pawed the air; or mumbling sire,  \\
A scarecrow pattern of old age, patched up  \\
Of all the tatters of infirmity,  \\
All loosely put together, hobbled in	  \\
Stumping upon a cane, with which he smites  \\
From time to time the solid boards and makes them  \\
Prat somewhat loudly of the whereabout  \\
Of one so overloaded with his years.  \\
But what of this?---the laugh, the grin, grimace,	  \\
And all the antics and buffoonery,  \\
The least of them not lost, were all received  \\
With charitable pleasure. Through the night,  \\
Between the show, and many-headed mass  \\
Of the spectators, and each little nook	  \\
That had its fray or brawl, how eagerly  \\
And with what flashes, as it were, the mind  \\
Turned this way, that way---sportive and alert  \\
And watchful, as a kitten when at play,  \\
While winds are blowing round her, among grass	  \\
And rustling leaves. Enchanting age and sweet---  \\
Romantic almost, looked at through a space,  \\
How small, of intervening years! For then,  \\
Though surely no mean progress had been made  \\
In meditations holy and sublime,	  \\
Yet something of a girlish childlike gloss  \\
Of novelty survived for scenes like these---  \\
Pleasure that had been handed down from times  \\
When at a country playhouse, having caught  \\
In summer through the fractured wall a glimpse	  \\
Of daylight, at the thought of where I was  \\
I gladdened more than if I had beheld  \\
Before me some bright cavern of romance,  \\
Or than we do when on our beds we lie  \\
At night, in warmth, when rains are beating hard.	  \\
The matter which detains me now will seem  \\
To many neither dignified enough  \\
Nor arduous, and is doubtless in itself  \\
Humble and low---yet not to be despised  \\
By those who have observed the curious props	  \\
By which the perishable hours of life  \\
Rest on each other, and the world of thought  \\
Exists and is sustained. More lofty themes,  \\
Such as at least do wear a prouder face,  \\
Might here be spoken of; but when I think	  \\
Of these I feel the imaginative power  \\
Languish within me. Even then it slept,  \\
When, wrought upon by tragic sufferings,  \\
The heart was full---amid my sobs and tears  \\
It slept, even in the season of my youth.	  \\
For though I was most passionately moved,  \\
And yielded to the changes of the scene  \\
With most obsequious feeling, yet all this  \\
Passed not beyond the suburbs of the mind.  \\
If aught there were of real grandeur here	  \\
'Twas only then when gross realities,  \\
The incarnation of the spirits that moved  \\
Amid the poet's beauteous world---called forth  \\
With that distinctness which a contrast gives,  \\
Or opposition---made me recognise	  \\
As by a glimpse, the things which I had shaped  \\
And yet not shaped, had seen and scarcely seen,  \\
Had felt, and thought of in my solitude.  \\
Pass we from entertainments that are such  \\
Professedly, to others titled higher,	  \\
Yet, in the estimate of youth at least,  \\
More near akin to these than names imply---  \\
I mean the brawls of lawyers in their courts  \\
Before the ermined judge, or that great stage  \\
Where senators, tongue-favored men, perform,	  \\
Admired and envied. Oh, the beating heart,  \\
When one among the prime of these rose up,  \\
One of whose name from childhood we had heard  \\
Familiarly, a household term, like those---  \\
The Bedfords, Glocesters, Salisburys of old---	  \\
Which the fifth Harry talks of. Silence, hush,  \\
This is no trifler, no short-flighted wit,  \\
No stammerer of a minute, painfully  \\
Delivered. No, the orator hath yoked  \\
The hours, like young Aurora, to his car---	  \\
O presence of delight, can patience e'er  \\
Grow weary of attending on a track  \\
That kindles with such glory? Marvellous,  \\
The enchantment spreads and rises---all are rapt  \\
Astonished---like a hero in romance	  \\
He winds away his never-ending horn:  \\
Words follow words, sense seems to follow sense---  \\
What memory and what logic!---till the strain  \\
Transcendent, superhuman as it is,  \\
Grows tedious even in a young man's ear.	  \\
These are grave follies; other public shows  \\
The capital city teems with of a kind  \\
More light---and where but in the holy church?  \\
There have I seen a comely bachelor,  \\
fresh from a toilette of two hours, ascend   \\
The pulpit, with seraphic glance look up,  \\
and in a tone elaborately low  \\
Beginning, lead his voice through many a maze  \\
A minuet course, and, winding up his mouth  \\
From time to time into an orifice   \\
Most delicate, a lurking eyelet, small  \\
And only not invisible, again  \\
Open it out, diffusing thence a smile  \\
Of rapt irradiation exquisite.  \\
Meanwhile the Evangelists, Isaiah, Job,   \\
Moses, and he who penned the other day  \\
The Death of Abel, Shakespear, Doctor Young,  \\
And Ossian---doubt not, 'tis the naked truth---  \\
Summoned from streamy Morven, each and all  \\
Must in their turn lend ornament and flowers   \\
To entwine the crook of eloquence with which  \\
This pretty shepherd, pride of all the plains,  \\
Leads up and down his captivated flock.  \\
I glance but at a few conspicuous marks,  \\
Leaving ten thousand others that do each---   \\
In hall or court, conventicle, or shop,  \\
In public room or private, park or street---  \\
With fondness reared on his own pedestal,  \\
Look out for admiration. Folly, vice,  \\
Extravagance in gesture, mien and dress,   \\
And all the strife of singularity---  \\
Lies to the ear, and lies to every sense---  \\
Of these and of the living shapes they wear  \\
There is no end. Such candidates for regard,  \\
Although well pleased to be where they were found,   \\
I did not hunt after or greatly prize,  \\
Nor made unto myself a secret boast  \\
Of reading them with quick and curious eye,  \\
But as a common produce---things that are  \\
Today, tomorrow will be---took of them   \\
Such willing note as, on some errand bound  \\
Of pleasure or of love, some traveller might,  \\
Among a thousand other images,  \\
Of sea-shells that bestud the sandy beach,  \\
Or daisies swarming through the fields in June.	  \\
But foolishness, and madness in parade,  \\
Though most at home in this their dear domain,  \\
Are scattered everywhere, no rarities,  \\
Even to the rudest novice of the schools.  \\
O friend, one feeling was there which belonged	  \\
To this great city by exclusive right:  \\
How often in the overflowing streets  \\
Have I gone forwards with the crowd, and said  \\
Unto myself, 'The face of every one  \\
That passes by me is a mystery.'	  \\
Thus have I looked, nor ceased to look, oppressed  \\
By thoughts of what, and whither, when and how,  \\
Until the shapes before my eyes became  \\
A second-sight procession, such as glides  \\
Over still montains, or appears in dreams,	  \\
And all the ballast of familiar life---  \\
The present, and the past, hope, fear, all stays,  \\
All laws of acting, thinking, speaking man---  \\
Went from me, neither knowing me, nor known.  \\
And once, far travelled in such mood, beyond	  \\
The reach of common indications, lost  \\
Amid the moving pageant, 'twas my chance  \\
Abruptly to be smitten with the view  \\
Of a blind beggar, who, with upright face,  \\
Stood propped against a wall, upon his chest	  \\
Wearing a written paper, to explain  \\
The story of the man, and who he was.  \\
My mind did at this spectacle turn round  \\
As with the might of waters, and it seemed  \\
To me that in this label was a type	  \\
Or emblem of the utmost that we know  \\
Both of ourselves and of the universe,  \\
And on the shape of this unmoving man,  \\
His fixe`d face and sightless eyes, I looked,  \\
As if admonished from another world.	  \\
Though reared upon the base of outward things,  \\
These chiefly are such structures as the mind  \\
Builds for itself. Scenes different there are---  \\
Full-formed---which take, with small internal help,  \\
Possession of the faculties: the peace	  \\
Of night, for instance, the solemnity  \\
Of Nature's intermediate hours of rest  \\
When the great tide of human life stands still,  \\
The business of the day to come unborn,  \\
Of that gone by locked up as in the grave;	  \\
The calmness, beauty, of the spectacle,  \\
Sky, stillness, moonshine, empty streets, and sounds  \\
Unfrequent as in desarts; at late hours  \\
Of winter evenings when unwholesome rains  \\
Are falling hard, with people yet astir,	  \\
The feeble salutation from the voice  \\
Of some unhappy woman now and then  \\
Heard as we pass, when no one looks about,  \\
Nothing is listened to. But these I fear  \\
Are falsely catalogued things that are, are not,	  \\
Even as we give them welcome, or assist---  \\
Are prompt, or are remiss. What say you then  \\
To times when half the city shall break out  \\
Full of one passion---vengeance, rage, or fear---  \\
To executions, to a street on fire,	  \\
Mobs, riots, or rejoicings? From those sights  \\
Take one, an annual festival, the fair  \\
Holden where martyrs suffered in past time,  \\
And named of St. Bartholomew, there see  \\
A work that's finished to our hands, that lays,	  \\
If any spectacle on earth can do,  \\
The whole creative powers of man asleep.  \\
For once the Muse's help will we implore,  \\
And she shall lodge us---wafted on her wings  \\
Above the press and danger of the crowd---	  \\
Upon some showman's platform. What a hell  \\
For eyes and ears, what anarchy and din  \\
Barbarian and infernal---'tis a dream  \\
Monstrous in colour, motion, shape, sight, sound.  \\
Below, the open space, through every nook	  \\
Of the wide area, twinkles, is alive  \\
With heads; the midway region and above  \\
Is thronged with staring pictures and huge scrolls,  \\
Dumb proclamations of the prodigies;  \\
And chattering monkeys dangling from their poles,	  \\
And children whirling in their roundabouts;  \\
With those that stretch the neck, and strain the eyes,  \\
And crack the voice in rivalship, the crowd  \\
Inviting; with buffoons against buffoons  \\
Grimacing, writhing, screaming; him who grinds	  \\
The hurdy-gurdy, at the fiddle weaves,  \\
Rattles the salt-box, thumps the kettle-drum,  \\
And him who at the trumpet puffs his cheeks,  \\
The silver-collared negro with his timbrel,  \\
Equestrians, tumblers, women, girls, and boys,	  \\
Blue-breeched, pink-vested, and with towering plumes.  \\
All moveables of wonder from all parts  \\
Are here, albinos, painted Indians, dwarfs,  \\
The horse of knowledge, and the learned pig,  \\
The stone-eater, the man that swallows fire,	  \\
Giants, ventriloquists, the invisible girl,  \\
The bust that speaks and moves its goggling eyes,  \\
The waxwork, clockwork, all the marvellous  \\
craft Of modern Merlins, wild beasts, puppet-shows,  \\
All out-o'-th'-way, far-fetched, perverted things,	  \\
All freaks of Nature, all Promethean thoughts  \\
Of man---his dulness, madness, and other feats,  \\
All jumbled up together to make up  \\
This parliament of monsters. Tents and booths  \\
Meanwhile---as if the whole were one vast mill---	  \\
Are vomiting, receiving, on all sides,  \\
Men, women, three-years' children, babes in arms.  \\
O, blank confusion, and a type not false  \\
Of what the mighty city is itself  \\
To all, except a straggler here and there---	  \\
To the whole swarm of its inhabitants---  \\
An undistinguishable world to men,  \\
The slaves unrespited of low pursuits,  \\
Living amid the same perpetual flow  \\
Of trivial objects, melted and reduced	  \\
To one identity by differences  \\
That have no law, no meaning, and no end---  \\
Oppression under which even highest minds  \\
Must labour, whence the strongest are not free.  \\
But though the picture weary out the eye,	  \\
By nature an unmanageable sight,  \\
It is not wholly so to him who looks  \\
In steadiness, who hath among least things  \\
An under-sense of greatest, sees the parts  \\
As parts, but with a feeling of the whole.	  \\
This, of all acquisitions first, awaits  \\
On sundry and most widely different modes  \\
Of education---nor with least delight  \\
On that through which I passed. Attention comes,  \\
And comprehensiveness and memory,	  \\
From early converse with the works of God  \\
Among all regions, chiefly where appear  \\
Most obviously simplicity and power.  \\
By influence habitual to the mind  \\
The mountain's outline and its steady form	  \\
Gives a pure grandeur, and its presence shapes  \\
The measure and the prospect of the soul  \\
To majesty: such virtue have the forms  \\
Perennial of the ancient hills---nor less  \\
The changeful language of their countenances	  \\
Gives movement of the thoughts, and multitude,  \\
With order and relation. This (if still,  \\
As hitherto, with freedom I may speak,  \\
And the same perfect openness of mind,  \\
Not violating any just restraint,	  \\
As I would hope, of real modesty),  \\
This did I feel in that vast receptacle.  \\
The spirit of Nature was upon me here,  \\
The soul of beauty and enduring life  \\
Was present as a habit, and diffused---	  \\
Through meagre lines and colours, and the press  \\
Of self-destroying, transitory things---  \\
Composure and ennobling harmony.  \\
\end{verse}

%%%%%%%%%%%%%%%%%%%%%%%%%%%%%%%%%%%%%%%%%%%%%%%%%%% \\
\chapter*[Book Eighth]{Book Eighth \\ Retrospect: Love of Nature Leading to Love of Mankind}
\addcontentsline{toc}{chapter}{Book Eighth Retrospect: Love of Nature Leading to Love of Mankind}

\begin{verse}
WHAT sounds are those, Helvellyn, which are heard  \\
Up to thy summit, through the depth of air  \\
Ascending as if distance had the power  \\
To make the sounds more audible? What crowd  \\
Is yon, assembled in the gay green field?	  \\
Crowd seems it, solitary hill, to thee,  \\
Though but a little family of men---  \\
Twice twenty---with their children and their wives,  \\
And here and there a stranger interspersed.  \\
It is a summer festival, a fair,	  \\
Such as---on this side now, and now on that,  \\
Repeated through his tributary vales---  \\
Helvellyn, in the silence of his rest  \\
Sees annually, if storms be not abroad  \\
And mists have left him an unshrouded head.	  \\
Delightful day it is for all who dwell  \\
In this secluded glen, and eagerly  \\
They give it welcome. Long ere heat of noon,  \\
Behold the cattle are driven down; the sheep  \\
That have for traffic been culled out are penned	  \\
In cotes that stand together on the plain  \\
Ranged side by side; the chaffering is begun;  \\
The heifer lows uneasy at the voice  \\
Of a new master; bleat the flocks aloud.  \\
Booths are there none: a stall or two is here,	  \\
A lame man, or a blind (the one to beg,  \\
The other to make music); hither too  \\
From far, with basket slung upon her arm  \\
Of hawker's wares---books, pictures, combs, and pins---  \\
Some aged woman finds her way again,	  \\
Year after year a punctual visitant;  \\
The showman with his freight upon his back,  \\
And once perchance in lapse of many years,  \\
Prouder itinerant---mountebank, or he  \\
Whose wonders in a covered wain lie hid.	  \\
But one is here, the loveliest of them all,  \\
Some sweet lass of the valley, looking out  \\
For gains---and who that sees her would not buy?  \\
Fruits of her father's orchard, apples, pears  \\
(On that day only to such office stooping),	  \\
She carries in her basket, and walks round  \\
Among the crowd, half pleased with, half ashamed  \\
Of her new calling, blushing restlessly.  \\
The children now are rich, the old man now  \\
Is generous, so gaiety prevails	  \\
Which all partake of, young and old.  \\
Immense  \\
Is the recess, the circumambient world  \\
Magnificent, by which they are embraced.  \\
They move about upon the soft green field;	  \\
How little they, they and their doings, seem,  \\
Their herds and flocks about them, they themselves,  \\
And all which they can further or obstruct---  \\
Through utter weakness pitiably dear,  \\
As tender infants are---and yet how great,	  \\
For all things serve them: them the morning light  \\
Loves as it glistens on the silent rocks,  \\
And them the silent rocks, which now from high  \\
Look down upon them, the reposing clouds,  \\
The lurking brooks from their invisible haunts,	  \\
And old Helvellyn, conscious of the stir,  \\
And the blue sky that roofs their calm abode.  \\
With deep devotion, Nature, did I feel  \\
In that great city what I owed to thee:  \\
High thoughts of God and man, and love of man,	  \\
Triumphant over all those loathsome sights  \\
Of wretchedness and vice, a watchful eye,  \\
Which, with the outside of our human life  \\
Not satisfied, must read the inner mind.  \\
For I already had been taught to love	  \\
My fellow-beings, to such habits trained  \\
Among the woods and mountains, where I found  \\
In thee a gracious guide to lead me forth  \\
Beyond the bosom of my family,  \\
My friends and youthful playmates.  \\
'Twas thy power	  \\
That raised the first complacency in me,  \\
And noticeable kindliness of heart,  \\
Love human to the creature in himself  \\
As he appeared, a stranger in my path,  \\
Before my eyes a brother of this world---	  \\
Thou first didst with those motions of delight  \\
Inspire me. I remember, far from home  \\
Once having strayed while yet a very child,  \\
I saw a sight---and with what joy and love!  \\
It was a day of exhalations spread	  \\
Upon the mountains, mists and steam-like fogs  \\
Redounding everywhere, not vehement,  \\
But calm and mild, gentle and beautiful,  \\
With gleams of sunshine on the eyelet spots  \\
And loopholes of the hills, wherever seen,	  \\
Hidden by quiet process, and as soon  \\
Unfolded, to be huddled up again---  \\
Along a narrow valley and profound  \\
I journeyed, when aloft above my head,  \\
Emerging from the silvery vapours, lo,	  \\
A shepherd and his dog, in open day.  \\
Girt round with mists they stood, and looked about  \\
From that enclosure small, inhabitants  \\
Of an ae��rial island floating on,  \\
As seemed, with that abode in which they were,	  \\
A little pendant area of grey rocks,  \\
By the soft wind breathed forward. With delight  \\
As bland almost, one evening I beheld---  \\
And at as early age (the spectacle  \\
Is common, but by me was then first seen)---	  \\
A shepherd in the bottom of a vale,  \\
Towards the centre standing, who with voice,  \\
And hand waved to and fro as need required,  \\
Gave signal to his dog, thus teaching him  \\
To chace along the mazes of steep crags	  \\
The flock he could not see. And so the brute---  \\
Dear creature---with a man's intelligence,  \\
Advancing, or retreating on his steps,  \\
Through every pervious strait, to right or left,  \\
Thridded a way unbaffled, while the flock	  \\
Fled upwards from the terror of his bark  \\
Through rocks and seams of turf with liquid gold  \\
Irradiate---that deep farewell light by which  \\
The setting sun proclaims the love he bears  \\
To mountain regions.	  \\
Beauteous the domain  \\
Where to the sense of beauty first my heart  \\
Was opened---tract more exquisitely fair  \\
Than in that paradise of ten thousand trees,  \\
Or Gehol's famous gardens, in a clime	  \\
Chosenfrom widest empire, for delight Of the  \\
Tartarian dynasty composed  \\
Beyond that mighty wall, not fabulous  \\
(China's stupendous mound!) by patient skill  \\
Of myriads, and boon Nature's lavish help:	  \\
Scene linked to scene, and ever-growing change,  \\
Soft, grand, or gay, with palaces and domes  \\
Of pleasure spangled over, shady dells  \\
For eastern monasteries, sunny mounds  \\
With temples crested, bridges, gondolas,	  \\
Rocks, dens and groves of foliage, taught to melt  \\
Into each other their obsequious hues---  \\
Going and gone again, in subtile chace,  \\
Too fine to be pursued---or standing forth  \\
In no discordant opposition, strong	  \\
And gorgeous as the colours side by side  \\
Bedded among the plumes of tropic birds;  \\
And mountains over all, embracing all,  \\
And all the landscape endlessly enriched  \\
With waters running, falling, or asleep.	  \\
But lovelier far than this the paradise  \\
Where I was reared, in Nature's primitive gifts  \\
Favored no less, and more to every sense  \\
Delicious, seeing that the sun and sky,  \\
The elements, and seasons in their change,	  \\
Do find their dearest fellow-labourer there  \\
The heart of man---a district on all sides  \\
The fragrance breathing of humanity,  \\
Man free, man working for himself, with choice  \\
Of time, and place, and object; by his wants,	  \\
His comforts, native occupations, cares,  \\
Conducted on to individual ends  \\
Or social, and still followed by a train,  \\
Unwooed, unthought-of even: simplicity,  \\
And beauty, and inevitable grace.	  \\
Yea, doubtless, at any age when but a glimpse  \\
Of those resplendent gardens, with their frame  \\
Imperial, and elaborate ornaments,  \\
Would to a child be transport over-great,  \\
When but a half-hour's roam through such a place	  \\
Would leave behind a dance of images  \\
That shall break in upon his sleep for weeks,  \\
Even then the common haunts of the green earth  \\
With the ordinary human interests  \\
Which they embosom---all without regard	  \\
As both may seem---are fastening on the heart  \\
Insensibly, each with the other's help,  \\
So that we love, not knowing that we love,  \\
And feel, not knowing whence our feeling comes.  \\
Such league have these two principles of joy   \\
In our affections. I have singled out  \\
Some moments, the earliest that I could, in which  \\
Their several currents, blended into one---  \\
Weak yet, and gathering imperceptibly---  \\
Flowed in by gushes. My first human love,   \\
As hath been mentioned, did incline to those  \\
Whose occupations and concerns were most  \\
Illustrated by Nature, and adorned,  \\
And shepherds were the men who pleased me first:  \\
Not such as, in Arcadian fastnesses   \\
Sequestered, handed down among themselves,  \\
So ancient poets sing, the golden age;  \\
Nor such---a second race, allied to these---  \\
As Shakespeare in the wood of Arden placed,  \\
Where Phoebe sighed for the false Ganymede,   \\
Or there where Florizel and Perdita  \\
Together dance, Queen of the feast and King;  \\
Nor such as Spenser fabled. True it is  \\
That I had heard, what he perhaps had seen,  \\
Of maids at sunrise bringing in from far   \\
Their May-bush, and along the streets in flocks  \\
Parading, with a song of taunting rhymes  \\
Aimed at the laggards slumbering within doors---  \\
Had also heard, from those who yet remembered,  \\
Tales of the maypole dance, and flowers that decked   \\
The posts and the kirk-pillars, and of youths,  \\
That each one with his maid at break of day,  \\
By annual custom, issued forth in troops  \\
To drink the waters of some favorite well,  \\
And hang it round with garlands. This, alas,   \\
Was but a dream: the times had scattered all  \\
These lighter graces, and the rural ways  \\
And manners which it was my chance to see  \\
In childhood were severe and unadorned,  \\
The unluxuriant produce of a life   \\
Intent on little but substantial needs,  \\
Yet beautiful---and beauty that was felt.  \\
But images of danger and distress  \\
And suffering, these took deepest hold of me,  \\
Man suffering among awful powers and forms:	  \\
Of this I heard and saw enough to make  \\
The imagination restless---nor was free  \\
Myself from frequent perils. Nor were tales  \\
Wanting, the tragedies of former times,  \\
Or hazards and escapes, which in my walks	  \\
I carried with me among crags and woods  \\
And mountains; and of these may here be told  \\
One as recorded by my household dame.  \\
'At the first falling of autumnal snows  \\
A shepherd and his son one day went forth',	  \\
Thus did the matron's tale begin, 'to seek  \\
A straggler of their flock. They both had ranged  \\
Upon this service the preceding day  \\
All over their own pastures and beyond,  \\
And now, at sunrise sallying out again,	  \\
Renewed their search, begun where from  \\
Dove Crag--- Ill home for bird so gentle---they looked down  \\
On Deepdale Head, and Brothers Water (named  \\
From those two brothers that were drowned therein)  \\
Thence, northward, having passed by Arthur's Seat,	  \\
To Fairfield's highest summit. On the right  \\
Leaving St Sunday's Pike, to Grisedale Tarn  \\
They shot, and over that cloud-loving hill,  \\
Seat Sandal---a fond lover of the clouds---  \\
Thence up Helvellyn, a superior mount	  \\
With prospect underneath of Striding Edge  \\
And Grisedale's houseless vale, along the brink  \\
Of Russet Cove, and those two other coves,  \\
Huge skeletons of crags, which from the trunk  \\
Of old Helvellyn spread their arms abroad	  \\
And make a stormy harbour for the winds.  \\
Far went those shepherds in their devious quest,  \\
From mountain ridges peeping as they passed  \\
Down into every glen; at length the boy  \\
Said, "Father, with your leave I will go back,	  \\
And range the ground which we have searched before."  \\
So speaking, southward down the hill the lad  \\
Sprang like a gust of wind, crying aloud,  \\
"I know where I shall find him." 'For take note',  \\
Said here my grey-haired dame, 'that though the storm	  \\
Drive one of these poor creatures miles and miles,  \\
If he can crawl he will return again  \\
To his own hills, the spots where when a lamb  \\
He learnt to pasture at his mother's side.  \\
After so long a labour suddenly	  \\
Bethinking him of this, the boy  \\
Pursued his way towards a brook whose course  \\
Was through that unfenced tract of mountain ground  \\
Which to his father's little farm belonged,  \\
The home and ancient birthright of their flock.	  \\
Down the deep channel of the stream he went,  \\
Prying through every nook. Meanwhile the rain  \\
Began to fall upon the mountain tops,  \\
Thick storm and heavy which for three hours' space  \\
Abated not, and all that time the boy	  \\
Was busy in his search, until at length  \\
He spied the sheep upon a plot of grass,  \\
An island in the brook. It was a place  \\
Remote and deep, piled round with rocks, where foot  \\
Of man or beast was seldom used to tread;	  \\
But now, when everywhere the summer grass  \\
Had failed, this one adventurer, hunger-pressed,  \\
Had left his fellows, and made his way alone  \\
To the green plot of pasture in the brook.  \\
Before the boy knew well what he had seen,	  \\
He leapt upon the island with proud heart  \\
And with a prophet's joy. Immediately  \\
The sheep sprang forward to the further shore  \\
And was borne headlong by the roaring flood---  \\
At this the boy looked round him, and his heart	  \\
Fainted with fear. Thrice did he turn his face  \\
To either brink, nor could he summon up  \\
The courage that was needful to leap back  \\
Cross the tempestuous torrent: so he stood,  \\
A prisoner on the island, not without	  \\
More than one thought of death and his last hour.  \\
Meanwhile the father had returned alone  \\
To his own house; and now at the approach  \\
Of evening he went forth to meet his son,  \\
Conjecturing vainly for what cause the boy	  \\
Had stayed so long. The shepherd took his way  \\
Up his own mountain grounds, where, as he walked  \\
Along the steep that overhung the brook  \\
He seemed to hear a voice, which was again  \\
Repeated, like the whistling of a kite.	  \\
At this, now knowing why, as oftentimes  \\
Long afterwards he has been heard to say,  \\
Down to the brook he went, and tracked its course  \\
Upwards among the o'erhanging rocks---nor thus  \\
Had he gone far, ere he espied the boy,	  \\
Where on that little plot of ground he stood  \\
Right in the middle of the roaring stream,  \\
Now stronger every moment and more fierce.  \\
The sight was such as no one could have seen  \\
Without distress and fear. The shepherd heard	  \\
The outcry of his son, he stretched his staff  \\
Towards him, bade him leap---which word scarce said,  \\
The boy was safe within his father's arms.'  \\
Smooth life had flock and shepherd in old time,  \\
Long springs and tepid winters on the banks	  \\
Of delicate Galesus---and no less  \\
Those scattered along Adria's myrtle shores---  \\
Smooth life the herdman and his snow-white herd,  \\
To triumphs and to sacrificial rites  \\
Devoted, on the inviolable stream	  \\
Of rich Clitumnus; and the goatherd lived  \\
As sweetly underneath the pleasant brows  \\
Of cool Lucretilis, where the pipe was heard  \\
Of Pan, the invisible God, thrilling the rocks  \\
With tutelary music, from all harm	  \\
The fold protecting. I myself, mature  \\
In manhood then, have seen a pastoral tract  \\
Like one of these, where fancy might run wild,  \\
Though under skies less generous and serene;  \\
Yet there, as for herself, had Nature framed	  \\
A pleasure-ground, diffused a fair expanse  \\
Of level pasture, islanded with groves  \\
And banked with woody risings---but the plain  \\
Endless, here opening widely out, and there  \\
Shut up in lesser lakes or beds of lawn	  \\
And intricate recesses, creek or bay  \\
Sheltered within a shelter, where at large  \\
The shepherd strays, a rolling hut his home:  \\
Thither he comes with springtime, there abides  \\
All summer, and at sunrise ye may hear	  \\
His flute or flagelet resounding far.  \\
There's not a nook or hold of that vast space,  \\
Nor strait where passage is, but it shall have  \\
In turn its visitant, telling there his hours  \\
In unlaborious pleasure, with no task	  \\
More toilsome than to carve a beechen bowl  \\
For spring or fountain, which the traveller finds  \\
When through the region he pursues at will  \\
His devious course.  \\
A glimpse of such sweet life	  \\
I saw when, from the melancholy walls  \\
Of Goslar, once imperial, I renewed  \\
My daily walk along that chearful plain,  \\
Which, reaching to her gates, spreads east and west  \\
And northwards, from beneath the mountainous verge	  \\
Of the Hercynian forest. Yet hail to you,  \\
Your rocks and precipices, ye that seize  \\
The heart with firmer grasp, your snows and streams  \\
Ungovernable, and your terrifying winds,  \\
That howled so dismally when I have been	  \\
Companionless among your solitudes!  \\
There, 'tis the shepherd's task the winter long  \\
To wait upon the storms: of their approach  \\
Sagacious, from the height he drives his flock  \\
Down into sheltering coves, and feeds them there	  \\
Through the hard time, long as the storm is 'locked'  \\
(So do they phrase it), bearing from the stalls  \\
A toilsome burthen up the craggy ways  \\
To strew it on the snow. And when the spring  \\
Looks out, and all the mountains dance with lambs,	  \\
He through the enclosures won from the steep waste,  \\
And through the lower heights hath gone his rounds;  \\
And when the flock with warmer weather climbs  \\
Higher and higher, him his office leads  \\
To range among them through the hills dispersed,	  \\
And watch their goings, whatsoever track  \\
Each wanderer chuses for itself---a work  \\
That lasts the summer through. He quits his home  \\
At dayspring, and no sooner doth the sun  \\
Begin to strike him with a fire-like heat,	  \\
Than he lies down upon some shining place,  \\
And breakfasts with his dog. When he hath stayed---  \\
As for the most he doth---beyond this time,  \\
He springs up with a bound, and then away!  \\
Ascending fast with his long pole in hand,	  \\
Or winding in and out among the crags.  \\
What need to follow him through what he does  \\
Or sees in his day's march? He feels himself  \\
In those vast regions where his service is  \\
A freeman, wedded to his life of hope	  \\
And hazard, and hard labour interchanged  \\
With that majestic indolence so dear  \\
To native man.  \\
A rambling schoolboy, thus  \\
Have I beheld him; without knowing why,	  \\
Have felt his presence in his own domain  \\
As of a lord and master, or a power,  \\
Or genius, under Nature, under God,  \\
Presiding---and severest solitude  \\
Seemed more commanding oft when he was there.	  \\
Seeking the raven's nest and suddenly  \\
Surprized with vapours, or on rainy days  \\
When I have angled up the lonely brooks,  \\
Mine eyes have glanced upon him, few steps off,  \\
In size a giant, stalking through the fog,	  \\
His sheep like Greenland bears. At other times,  \\
When round some shady promontory turning,  \\
His form hath flashed upon me glorified  \\
By the deep radiance of the setting sun;  \\
Or him have I descried in distant sky,	  \\
A solitary object and sublime,  \\
Above all height, like an ae��rial cross,  \\
As it is stationed on some spiry rock  \\
Of the Chartreuse, for worship. Thus was man  \\
Ennobled outwardly before mine eyes,	  \\
And thus my heart at first was introduced  \\
To an unconscious love and reverence  \\
Of human nature; hence the human form  \\
To me was like an index of delight,  \\
Of grace and honour, power and worthiness.	  \\
Meanwhile, this creature---spiritual almost  \\
As those of books, but more exalted far,  \\
Far more of an imaginative form---  \\
Was not a Corin of the groves, who lives  \\
For his own fancies, or to dance by the hour	  \\
In coronal, with Phyllis in the midst,  \\
But, for the purpose of kind, a man  \\
With the most common---husband, father---learned,  \\
Could teach, admonish, suffered with the rest  \\
From vice and folly, wretchedness and fear.	  \\
Of this I little saw, cared less for it,  \\
But something must have felt.  \\
Call ye these appearances  \\
Which I beheld of shepherds in my youth,  \\
This sanctity of Nature given to man,	  \\
A shadow, a delusion?---ye who are fed  \\
By the dead letter, not the spirit of things,  \\
Whose truth is not a motion or a shape  \\
Instinct with vital functions, but a block  \\
Or waxen image which yourselves have made,	  \\
And ye adore. But blesse`d be the God  \\
Of Nature and of man that this was so,  \\
That men did at the first present themselves  \\
Before my untaught eyes thus purified,  \\
Removed, and at a distance that was fit.	  \\
And so we all of us in some degree  \\
Are led to knowledge, whencesoever led,  \\
And howsoever---were it otherwise,  \\
And we found evil fast as we find good  \\
In our first years, or think that it is found,	  \\
How could the innocent heart bear up and live?  \\
But doubly fortunate my lot: not here  \\
Alone, that something of a better life  \\
Perhaps was round me than it is the privilege  \\
Of most to move in, but that first I looked	  \\
At man through objects that were great and fair,  \\
First communed with him by their help. And thus  \\
Was founded a sure safeguard and defence  \\
Against the weight of meanness, selfish cares,  \\
Coarse manners, vulgar passions, that beat in	  \\
On all sides from the ordinary world  \\
In which we traffic. Starting from this point,  \\
I had my face towards the truth, began  \\
With an advantage, furnished with that kind  \\
Of prepossession without which the soul	  \\
Receives no knowledge that can bring forth good---  \\
No genuine insight ever comes to her---  \\
Happy in this, that I with Nature walked,  \\
Not having a too early intercourse  \\
With the deformities of crowded life,	  \\
And those ensuing laughters and contempts  \\
Self-pleasing, which if we would wish to think  \\
With admiration and respect of man  \\
Will not permit us, but pursue the mind  \\
That to devotion willingly would be raised,	  \\
Into the temple of the temple's heart.  \\
Yet do not deem, my friend, though thus I speak  \\
Of man as having taken in my mind  \\
A place thus early which might almost seem  \\
Preeminent, that this was really so.	  \\
Nature herself was at this unripe time  \\
But secondary to my own pursuits  \\
And animal activities, and all  \\
Their trivial pleasures. And long afterwards  \\
When those had died away, and Nature did	  \\
For her own sake become my joy, even then,  \\
And upwards through late youth until not less  \\
Than three-and-twenty summers had been told,  \\
Was man in my affections and regards  \\
Subordinate to her, her awful forms	  \\
And viewless agencies---a passion, she,  \\
A rapture often, and immediate joy  \\
Ever at hand; he distant, but a grace  \\
Occasional, and accidental thought,  \\
His hour being not yet come. Far less had then	  \\
The inferior creatures, beast or bird, attuned  \\
My spirit to that gentleness of love,  \\
Won from me those minute obeisances  \\
Of tenderness which I may number now  \\
With my first blessings. Nevertheless, on these	  \\
The light of beauty did not fall in vain,  \\
Or grandeur circumfuse them to no end.  \\
Why should I speak of tillers of the soil?---  \\
The ploughman and his team; or men and boys  \\
In festive summer busy with the rake,	  \\
Old men and ruddy maids, and little ones  \\
All out together, and in sun and shade  \\
Dispersed among the hay-grounds alder-fringed;  \\
The quarryman, far heard, that blasts the rock;  \\
The fishermen in pairs, the one to row,	  \\
And one to drop the net, plying their trade  \\
''Mid tossing lakes and tumbling boats' and winds  \\
Whistling; the miner, melancholy man,  \\
That works by taper-light, while all the hills  \\
Are shining with the glory of the day.	  \\
But when that first poetic faculty  \\
Of plain imagination and severe---  \\
No longer a mute influence of the soul,  \\
An element of the nature's inner self---  \\
Began to have some promptings to put on	  \\
A visible shape, and to the works of art,  \\
The notions and the images of books,  \\
Did knowingly conform itself (by these  \\
Enflamed, and proud of that her new delight),  \\
There came among these shapes of human life	  \\
A wilfulness of fancy and conceit  \\
Which gave them new importance to the mind---  \\
And Nature and her objects beautified  \\
These fictions, as, in some sort, in their turn  \\
They banished her. From touch of this new power	  \\
Nothing was safe: the elder-tree that grew  \\
Beside the well-known charnel-house had then  \\
A dismal look, the yew-tree had its ghost  \\
That took its station there for ornament.  \\
Then common death was none, common mishap,	  \\
But matter for this humour everywhere,  \\
The tragic super-tragic, else left short.  \\
Then, if a widow staggering with the blow  \\
Of her distress was known to have made her way  \\
To the cold grave in which her husband slept,	  \\
One night, or haply more than one---through pain  \\
Or half-insensate impotence of mind---  \\
The fact was caught at greedily, and there  \\
She was a visitant the whole year through,  \\
Wetting the turf with never-ending tears,	  \\
And all the storms of heaven must beat on her.  \\
Through wild obliquities could I pursue  \\
Among all objects of the fields and groves  \\
These cravings: when the foxglove, one by one,  \\
Upwards through every stage of its tall stem	  \\
Had shed its bells, and stood by the wayside  \\
Dismantled, with a single one perhaps  \\
Left at the ladder's top, with which the plant  \\
Appeared to stoop, as slender blades of grass  \\
Tipped with a bead of rain or dew, behold,	  \\
If such a sight were seen, would fancy bring  \\
Some vagrant thither with her babes and seat her  \\
Upon the turf beneath the stately flower,  \\
Drooping in sympathy and making so  \\
A melancholy crest above the head	  \\
Of the lorn creature, while her little ones,  \\
All unconcerned with her unhappy plight,  \\
Were sporting with the purple cups that lay  \\
Scattered upon the ground. There was a copse,  \\
An upright bank of wood and woody rock	  \\
That opposite our rural dwelling stood,  \\
In which a sparkling patch of diamond light  \\
Was in bright weather duly to be seen  \\
On summer afternoons, within the wood  \\
At the same place. 'Twas doubtless nothing more	  \\
Than a black rock, which, wet with constant springs,  \\
Glistered far seen from out its lurking-place  \\
As soon as ever the declining sun  \\
Had smitten it. Beside our cottage hearth  \\
Sitting with open door, a hundred times	  \\
Upon this lustre have I gazed, that seemed  \\
To have some meaning which I could not find---  \\
And now it was a burnished shield, I fancied,  \\
Suspended over a knight's tomb, who lay  \\
Inglorious, buried in the dusky wood;	  \\
An entrance now into some magic cave,  \\
Or palace for a fairy of the rock.  \\
Nor would I, though not certain whence the cause  \\
Of the effulgence, thither have repaired  \\
Without a precious bribe, and day by day	  \\
And month by month I saw the spectacle,  \\
Nor ever once have visited the spot  \\
Unto this hour. Thus sometimes were the shapes  \\
Of wilful fancy grafted upon feelings  \\
Of the imagination, and they rose	  \\
In worth accordingly.  \\
My present theme  \\
Is to retrace the way that led me on  \\
Through Nature to the love of human-kind;  \\
Nor could I with such object overlook	  \\
The influence of this power which turned itself  \\
Instinctively to human passions, things  \\
Least understood---,of this adulterate power,  \\
For so it may be called, and without wrong,  \\
When with that first compared. Yet in the midst	  \\
Of these vagaries, with an eye so rich  \\
As mine was---through the chance, on me not wasted,  \\
Of having been brought up in such a grand  \\
And lovely region---I had forms distinct  \\
To steady me. These thoughts did oft revolve	  \\
About some centre palpable, which at once  \\
Incited them to motion, and controlled,  \\
And whatsoever shape the fit might take,  \\
And whencesoever it might come, I still  \\
At all times had a real solid world	  \\
Of images about me, did not pine  \\
As one in cities bred might do---as thou,  \\
Beloved friend, hast told me that thou didst,  \\
Great spirit as thou art---in endless dreams  \\
Of sickness, disjoining, joining things,	  \\
Without the light of knowledge. Where the harm  \\
If when the woodman languished with disease  \\
From sleeping night by night among the woods  \\
Within his sod-built cabin, Indian-wise,  \\
I called the pangs of disappointed love	  \\
And all the long etcetera of such thought  \\
To help him to his grave?---meanwhile the man,  \\
If not already from the woods retired  \\
To die at home, was haply, as I knew,  \\
Pining alone among the gentle airs,	  \\
Birds, running streams, and hills so beautiful  \\
On golden evenings, while the charcoal-pile  \\
Breathed up its smoke, an image of his ghost  \\
Or spirit that was soon to take its flight.  \\
There came a time of greater dignity,	  \\
Which had been gradually prepared, and now  \\
Rushed in as if on wings---the time in which  \\
The pulse of being everywhere was felt,  \\
When all the several frames of things, like stars  \\
Through every magnitude distinguishable,	  \\
Were half confounded in each other's blaze,  \\
One galaxy of life and joy. Then rose  \\
Man, inwardly contemplated, and present  \\
In my own being, to a loftier height---  \\
As of all visible natures crown, and first	  \\
In capability of feeling what  \\
Was to be felt, in being rapt away  \\
By the divine effect of power and love---  \\
As, more than any thing we know, instinct  \\
With godhead, and by reason and by will	  \\
Acknowledging dependency sublime.  \\
Erelong, transported hence as in a dream, I  \\
found myself begirt with temporal shapes  \\
Of vice and folly thrust upon my view,  \\
Objects of sport and ridicule and scorn,	  \\
Manners and characters discriminate,  \\
And little busy passions that eclipsed,  \\
As well they might, the impersonated thought,  \\
The idea or abstraction of the kind.  \\
An idler among academic bowers,	  \\
Such was my new condition---as at large  \\
Hath been set forth---yet here the vulgar light  \\
Of present, actual, superficial life,  \\
Gleaming through colouring of other times,  \\
Old usages and local privilege,	  \\
Thereby was softened, almost solemnized,  \\
And rendered apt and pleasing to the view.  \\
This notwithstanding, being brought more near  \\
As I was now to guilt and wretchedness,  \\
I trembled, thought of human life at times	  \\
With an indefinite terror and dismay,  \\
Such as the storms and angry elements  \\
Had bred in me; but gloomier far, a dim  \\
Analogy to uproar and misrule,  \\
Disquiet, danger, and obscurity.	  \\
It might be told (but wherefore speak of things  \\
Common to all?) that, seeing, I essayed  \\
To give relief, began to deem myself  \\
A moral agent, judging between good  \\
And evil not as for the mind's delight	  \\
But for her safety, one who was to act---  \\
As sometimes to the best of my weak means  \\
I did, by human sympathy impelled,  \\
And through dislike and most offensive pain  \\
Was to the truth conducted---of this faith	  \\
Never forsaken, that by acting well,  \\
And understanding, I should learn to love  \\
The end of life and every thing we know.  \\
Preceptress stern, that didst instruct me next,  \\
London, to thee I willingly return.	  \\
Erewhile my verse played only with the flowers  \\
Enwrought upon the mantle, satisfied  \\
With this amusement, and a simple look  \\
Of childlike inquisition now and then  \\
Cast upwards on thine eye to puzzle out	  \\
Some inner meanings which might harbour there.  \\
Yet did I not give way to this light mood  \\
Wholly beguiled, as one incapable  \\
Of higher things, and ignorant that high things  \\
Were round me. Never shall I forget the hour,	  \\
The moment rather say, when, having thridded  \\
The labyrinth of suburban villages,  \\
At length I did unto myself first seem  \\
To enter the great city. On the roof  \\
Of an itinerant vehicle I sate,	  \\
With vulgar men about me, vulgar forms  \\
Of houses, pavement, streets, of men and things,  \\
Mean shapes on every side; but, at the time,  \\
When to myself it fairly might be said  \\
(The very moment that I seemed to know)	  \\
'The threshold now is overpast', great God!  \\
That aught external to the living mind  \\
Should have such mighty sway, yet so it was:  \\
A weight of ages did at once descend  \\
Upon my heart---no thought embodied, no	  \\
Distinct remembrances, but weight and power,  \\
Power growing with the weight. Alas,  \\
I feel That I am trifling. 'Twas a moment's pause:  \\
All that took place within me came and went  \\
As in a moment, and I only now	  \\
Remember that it was a thing divine.  \\
As when a traveller hath from open day  \\
With torches passed into some vault of earth,  \\
The grotto of Antiparos, or the den  \\
Of Yordas among Craven's mountain tracts,	  \\
He looks and sees the cavern spread and grow,  \\
Widening itself on all sides, sees, or thinks  \\
He sees, erelong, the roof above his head,  \\
Which instantly unsettles and recedes---  \\
Substance and shadow, light and darkness, all	  \\
Commingled, making up a canopy  \\
Of shapes, and forms, and tendencies to shape,  \\
That shift and vanish, change and interchange  \\
Like spectres---ferment quiet and sublime,  \\
Which, after a short space, works less and less	  \\
Till, every effort, every motion gone,  \\
The scene before him lies in perfect view  \\
Exposed, and lifeless as a written book.  \\
But let him pause awhile and look again,  \\
And a new quickening shall succeed, at first	  \\
Beginning timidly, then creeping fast  \\
Through all which he beholds: the senseless mass,  \\
In its projections, wrinkles, cavities,  \\
Through all its surface, with all colours streaming,  \\
Like a magician's airy pageant, parts,	  \\
Unites, embodying everywhere some pressure  \\
Or image, recognised or new, some type  \\
Or picture of the world---forests and lakes,  \\
Ships, rivers, towers, the warrior clad in mail,  \\
The prancing steed, the pilgrim with his staff,	  \\
A mitred bishop and the throne`d king---  \\
A spectacle to which there is no end.  \\
No otherwise had I at first been moved---  \\
With such a swell of feeling, followed soon  \\
By a blank sense of greatness passed away---	  \\
And afterwards continued to be moved,  \\
In presence of that vast metropolis,  \\
The fountain of my country's destiny  \\
And of the destiny of earth itself,  \\
That great emporium, chronicle at once	  \\
And burial-place of passions, and their home  \\
Imperial, and chief living residence.  \\
With strong sensations teeming as it did  \\
Of past and present, such a place must needs  \\
Have pleased me in those times. I sought not then	  \\
Knowledge, but craved for power---and power I found  \\
In all things. Nothing had a circumscribed  \\
And narrow influence; but all objects, being  \\
Themselves capacious, also found in me  \\
Capaciousness and amplitude of mind---	  \\
Such is the strength and glory of our youth.  \\
The human nature unto which I felt  \\
That I belonged, and which I loved and reverenced,  \\
Was not a punctual presence, but a spirit  \\
Living in time and space, and far diffused.	  \\
In this my joy, in this my dignity  \\
Consisted: the external universe,  \\
By striking upon what is found within,  \\
Had given me this conception, with the help  \\
Of books and what they picture and record.	  \\
'Tis true the history of my native land,  \\
With those of Greece compared and popular Rome---  \\
Events not lovely nor magnanimous,  \\
But harsh and unaffecting in themselves;  \\
And in our high-wrought modern narratives	  \\
Stript of their humanizing soul, the life  \\
Of manners and familiar incidents---  \\
Had never much delighted me. And less  \\
Than other minds I had been used to owe  \\
The pleasure which I found in place or thing	  \\
To extrinsic transitory accidents,  \\
To records or traditions; but a sense  \\
Of what had been here done, and suffered here  \\
Through ages, and was doing, suffering, still,  \\
Weighed with me, could support the test of thought---	  \\
Was like the enduring majesty and power  \\
Of independent nature. And not seldom  \\
Even individual remembrances,  \\
By working on the shapes before my eyes,  \\
Became like vital functions of the soul;	  \\
And out of what had been, what was, the place  \\
Was thronged with impregnations, like those wilds  \\
In which my early feelings had been nursed,  \\
And naked valleys full of caverns, rocks,  \\
And audible seclusions, dashing lakes,	  \\
Echoes and waterfalls, and pointed crags  \\
That into music touch the passing wind.  \\
Thus here imagination also found  \\
An element that pleased her, tried her strength  \\
Among new objects, simplified, arranged,	  \\
Impregnated my knowledge, made it live---  \\
And the result was elevating thoughts  \\
Of human nature. Neither guilt nor vice,  \\
Debasement of the body or the mind,  \\
Nor all the misery forced upon my sight,	  \\
Which was not lightly passed, but often scanned  \\
Most feelingly, could overthrow my trust  \\
In what we may become, induce belief  \\
that I was ignorant, had been falsely taught,  \\
A solitary, who with vain conceits	  \\
Had been inspired, and walked about in dreams.  \\
When from that rueful prospect, overcast  \\
And in eclipse, my meditations turned,  \\
Lo, every thing that was indeed divine  \\
Retained its purity inviolate	  \\
And unencroached upon, nay, seemed brighter far  \\
For this deep shade in counterview, the gloom  \\
Of opposition, such as shewed itself  \\
To the eyes of Adam, yet in Paradise  \\
Though fallen from bliss, when in the East he saw	  \\
Darkness ere day's mid course, and morning light  \\
More orient in the western cloud, that drew  \\
'O'er the blue firmament a radiant white,  \\
Descending slow with something heavenly fraught.'  \\
Add also, that among the multitudes	  \\
Of that great city oftentimes was seen  \\
Affectingly set forth, more than elsewhere  \\
Is possible, the unity of man,  \\
One spirit over ignorance and vice  \\
Predominant, in good and evil hearts	  \\
One sense for moral judgments, as one eye  \\
For the sun's light. When strongly breathed upon  \\
By this sensation---whencesoe'er it comes,  \\
Of union or communion---doth the soul  \\
Rejoice as in her highest joy; for there,	  \\
There chiefly, hath she feeling whence she is,  \\
And passing through all Nature rests with God.  \\
And is not, too, that vast abiding-place  \\
Of human creatures, turn where'er we may,  \\
Profusely sown with individual sights	  \\
Of courage, and integrity, and truth,  \\
And tenderness, which, here set off by foil,  \\
Appears more touching? In the tender scenes  \\
Chiefly was my delight, and one of these  \\
Never will be forgotten. 'Twas a man,	  \\
Whom I saw sitting in an open square  \\
Close to the iron paling that fenced in  \\
The spacious grass-plot: on the corner-stone  \\
Of the low wall in which the pales were fixed  \\
Sate this one man, and with a sickly babe	  \\
Upon his knee, whom he had thither brought  \\
For sunshine, and to breathe the fresher air.  \\
Of those who passed, and me who looked at him,  \\
He took no note; but in his brawny arms  \\
(The artificer was to the elbow bare,	  \\
And from his work this moment had been stolen)  \\
He held the child, and, bending over it  \\
As if he were afraid both of the sun  \\
And of the air which he had come to seek,  \\
He eyed it with unutterable love.	  \\
Thus from a very early age, O friend,  \\
My thoughts had been attracted more and more  \\
By slow gradations towards human-kind,  \\
And to the good and ill of human life.  \\
Nature had led me on, and now I seemed	  \\
To travel independent of her help,  \\
As if I had forgotten her---but no,  \\
My fellow-beings still were unto me  \\
Far less than she was: though the scale of love  \\
Were filling fast, 'twas light as yet compared	  \\
With that in which her mighty objects lay. \\
\end{verse}

%%%%%%%%%%%%%%%%%%%%%%%%%%%%%%%%%%%%%%%%%%%%%%%% \\
\chapter*[Book Ninth]{Book Ninth \\ Residence in France}
\addcontentsline{toc}{chapter}{Book Ninth Residence in France}

\begin{verse}
AS oftentimes a river, it might seem,  \\
Yielding in part to old remembrances,  \\
Part swayed by fear to tread an onward road  \\
That leads direct to the devouring sea,  \\
Turns and will measure back his course---far back,	  \\
Towards the very regions which he crossed  \\
In his first outset---so have we long time  \\
Made motions retrograde, in like pursuit  \\
Detained. But now we start afresh: I feel  \\
An impulse to precipitate my verse.	  \\
Fair greetings to this shapeless eagerness,  \\
Whene'er it comes, needful in work so long,  \\
Trice needful to the argument which now  \\
Awaits us---oh, how much unlike the past---  \\
One which though bright the promise, will be found	  \\
Ere far we shall advance, ungenial, hard  \\
To treat of, and forbidding in itself.  \\
Free as a colt at pasture on the hills  \\
I ranged at large through the metropolis  \\
Month after month. Obscurely did I live,	  \\
Not courting the society of men,  \\
By literature, or elegance, or rank,  \\
Distinguished---in the midst of things, it seemed,  \\
Looking as from a distance on the world  \\
That moved about me. Yet insensibly	  \\
False preconceptions were corrected thus,  \\
And errors of the fancy rectified  \\
(Alike with reference to men and things),  \\
And sometimes from each quarter were poured in  \\
Novel imaginations and profound.	  \\
A year thus spent, this field, with small regret---  \\
Save only for the bookstalls in the streets  \\
(Wild produce, hedgerow fruit, on all sides hung  \\
To lure the sauntering traveller from his track)---  \\
I quitted, and betook myself to France,	  \\
Let thither chiefly by a personal wish  \\
To speak the language more familiarly,  \\
With which intent I chose for my abode  \\
A city on the borders of the Loire.  \\
Through Paris lay my readiest path, and there	  \\
I sojourned a few days, and visited  \\
In haste each spot of old and recent fame---  \\
The latter chiefly---from the field of Mars  \\
Down to the suburbs of St. Anthony,  \\
And from Mont Martyr southward to the Dome	  \\
Of Genevie`ve. In both her clamorous halls,  \\
The National Synod and the Jacobins,  \\
I saw the revolutionary power  \\
Toss like a ship at anchor, rocked by storms,  \\
The Arcades I traversed in the Palace huge	  \\
Of Orleans, coasted round and round the line  \\
Of tavern, brothel, gaming-house, and shop,  \\
Great rendezvous of worst and best, the walk  \\
Of all who had a purpose, or had not;  \\
I stared and listened with a stranger's ears,	  \\
To hawkers and haranguers, hubbub wild,  \\
And hissing factionists with ardent eyes,  \\
In knots, or pairs, or single, ant-like swarms  \\
Of builders and subverters, every face  \\
That hope or apprehension could put on---	  \\
Joy, anger, and vexation, in the midst  \\
Of gaiety and dissolute idleness.  \\
Where silent zephyrs sported with the dust  \\
Of the Bastile I sate in the open sun  \\
And from the rubbish gathered up a stone,	  \\
And pocketed the relick in the guise  \\
Of an enthusiast; yet, in honest truth,  \\
Though not without some strong incumbencies,  \\
And glad---could living man be otherwise?---  \\
I looked for something which I could not find,	  \\
Affecting more emotion than I felt.  \\
For 'tis most certain that the utmost force  \\
Of all these various objects which may shew  \\
The temper of my mind as then it was  \\
Seemed less to recompense the traveller's pains,	  \\
Less moved me, gave me less delight, than did  \\
A single picture merely, hunted out  \\
Among other sights, the Magdalene of le Brun,  \\
A beauty exquisitely wrought---fair face  \\
And rueful, with its ever-flowing tears.	  \\
But hence to my more permanent residence  \\
I hasten: there, by novelties in speech,  \\
Domestic manners, customs, gestures, looks,  \\
And all the attire of ordinary life,  \\
Attention was at first engrossed; and thus	  \\
Amused and satisfied, I scarcely felt  \\
The shock of these concussions, unconcerned,  \\
Tranquil almost, and careless as a flower  \\
Glassed in a greenhouse, or a parlour-shrub,  \\
When every bush and tree the country through,	  \\
Is shaking to the roots---indifference this  \\
Which may seem strange, but I was unprepared  \\
With needful knowledge, had abruptly passed  \\
Into a theatre of which the stage  \\
Was busy with an action far advanced.	  \\
Like others I had read, and eagerly  \\
Sometimes, the master pamphlets of the day,  \\
Nor wanted such half-insight as grew wild  \\
Upon that meagre soil, helped out by talk  \\
And public news; but having never chanced	  \\
To see a regular chronicle which might shew---  \\
If any such indeed existed then---  \\
Whence the main organs of the public power  \\
Had sprung, their transmigrations, when and how  \\
Accomplished (giving thus unto events	  \\
A form and body), all things were to me  \\
Loose and disjointed, and the affections left  \\
Without a vital interest. At that time,  \\
Moreover, the first storm was overblown,  \\
And the strong hand of outward violence	  \\
Locked up in quiet. For myself---I fear  \\
Now in connection with so great a theme  \\
To speak, as I must be compelled to do,  \\
Of one so unimportant---a short time  \\
I loitered, and frequented night by night	  \\
Routs, card-tables, the formal haunts of men  \\
Whom in the city privilege of birth  \\
Sequestered from the rest, societies  \\
Where, through punctilios of elegance  \\
And deeper causes, all discourse, alike	  \\
Of good and evil, in the time, was shunned  \\
With studious care. But 'twas not long ere this  \\
Proved tedious, and I gradually withdrew  \\
Into a noisier world, and thus did soon  \\
Become a patriot---and my heart was all	  \\
Given to the people, and my love was theirs.  \\
A knot of military officers  \\
That to a regiment appertained which then  \\
Was stationed in the city were the chief  \\
Of my associates; some of these wore swords	  \\
Which had been seasoned in the wars, and all  \\
Were men well-born, at least laid claim to such  \\
Distinction, as the chivalry of France.  \\
In age and temper differing, they had yet  \\
One spirit ruling in them all---alike	  \\
(Save only one, hereafter to be named)  \\
Were bent upon undoing what was done.  \\
This was their rest, and only hope; therewith  \\
No fear had they of bad becoming worse,  \\
For worst to them was come---nor would have stirred,	  \\
Or deemed it worth a moment's while to stir,  \\
In any thing, save only as the act  \\
Looked thitherward. One, reckoning by years,  \\
Was in the prime of manhood, and erewhile  \\
He had sate lord in many tender hearts,	  \\
Though heedless of such honours now, and changed:  \\
His temper was quite mastered by the times,  \\
And they had blighted him, had eat away  \\
The beauty of his person, doing wrong  \\
Alike to body and to mind. His port,	  \\
Which once had been erect and open, now  \\
Was stooping and contracted, and a face  \\
By nature lovely in itself, expressed,  \\
As much as any that was ever seen,  \\
A ravage out of season. made by thoughts	  \\
Unhealthy and vexatious. At the hour,  \\
The most important of each day, in which  \\
The public news was read, the fever came,  \\
A punctual visitant, to shake this man,  \\
Disarmed his voice and fanned his yellow cheek	  \\
Into a thousand colours. While he read,  \\
Or mused, his sword was haunted by his touch  \\
Continually, like an uneasy place  \\
In his own body. 'Twas in truth an hour  \\
Of universal ferment---mildest men	  \\
Were agitated, and commotions, strife  \\
Of passion and opinion, filled the walls  \\
Of peaceful houses with unquiet sounds.  \\
The soil of common life was at that time  \\
Too hot to tread upon. Oft said I then,	  \\
And not then only, 'What a mockery this  \\
Of history, the past and that to come!  \\
Now do I feel how I have been deceived,  \\
Reading of nations and their works in faith---  \\
Faith given to vanity and emptiness---	  \\
Oh, laughter for the page that would reflect  \\
To future times the face of what now is!'  \\
The land all swarmed with passion, like a plain  \\
Devoured by locusts---Carra, Gorsas---add  \\
A hundred other names, forgotten now,	  \\
Nor to be heard of more; yet were they powers,  \\
Like earthquakes, shocks repeated day by day,  \\
And felt through every nook of town and field.  \\
The men already spoken of as chief  \\
Of my associates were prepared for flight	  \\
To augment the band of emigrants in arms  \\
Upon the borders of the Rhine, and leagued  \\
With foreign foes mustered for instant war.  \\
This was their undisguised intent, and they  \\
Were waiting with the whole of their desires	  \\
The moment to depart. An Englishman,  \\
Born in a land the name of which appeared  \\
To licence some unruliness of mind,  \\
A stranger, with youth 's further privilege,  \\
And that indulgence which a half-learned speech	  \\
Wins from the courteous, I---who had been else  \\
Shunned and not tolerated---freely lived  \\
With these defenders of the crown, and talked,  \\
And heard their notions; nor did they disdain  \\
The wish to bring me over to their cause.	  \\
But though untaught by thinking or by books  \\
To reason well of polity or law,  \\
And nice distinctions---then on every tongue---  \\
Of natural rights and civil, and to acts  \\
Of nations, and their passing interests	  \\
(I speak comparing these with other things)  \\
Almost indifferent, even the historian's tale  \\
Prizing but little otherwise than I prized  \\
Tales of poets---as it made my heart  \\
Beat high and filled my fancy with fair forms,	  \\
Old heroes and their sufferings and their deeds---  \\
Yet in the regal sceptre, and the pomp  \\
Of orders and degrees, I nothing found  \\
Then, or had ever even in crudest youth,  \\
That dazzled me, but rather what my soul	  \\
Mourned for, or loathed, beholding that the best  \\
Ruled not, and feeling that they ought to rule.  \\
For, born in a poor district, and which yet  \\
Retaineth more of ancient homeliness,  \\
Manners erect, and frank simplicity,	  \\
Than any other nook of English land,  \\
It was my fortune scarcely to have seen  \\
Through the whole tenor of my schoolday time  \\
The face of one, who, whether boy or man,  \\
Was vested with attention or respect	  \\
Through claims of wealth or blood. Nor was it least  \\
Of many debts which afterwards I owed  \\
To Cambridge and an academic life,  \\
That something there was holden up to view  \\
Of a republic, where all stood thus far	  \\
Upon equal ground, that they were brothers all  \\
In honour, as of one community---  \\
Scholars and gentlemen---where, furthermore,  \\
Distinction lay open to all that came,  \\
And wealth and titles were in less esteem	  \\
Than talents and successful industry.  \\
Add unto this, subservience from the first  \\
To God and Nature's single sovereignty  \\
(Familiar presences of awful power),  \\
And fellowship with venerable books	  \\
To sanction the proud workings of the soul,  \\
And mountain liberty. It could not be  \\
But that one tutored thus, who had been formed  \\
To thought and moral feeling in the way  \\
This story hath described, should look with awe	  \\
Upon the faculties of man, receive  \\
Gladly the highest promises, and hail  \\
As best the government of equal rights  \\
And individual worth. And hence, O friend,  \\
If at the first great outbreak I rejoiced	  \\
Less than might well befit my youth, the cause  \\
In part lay here, that unto me the events  \\
seemed nothing out of nature's certain course---  \\
A gift that rather was come late than soon.  \\
No wonder then if advocates like these	  \\
Whom I have mentioned, at this riper day  \\
Were impotent to make my hopes put on  \\
The shape of theirs, my understanding bend  \\
In honour to their honour. Zeal which yet  \\
Had slumbered, now in opposition burst	  \\
Forth like a Polar summer. Every word  \\
They uttered was a dart by counter-winds  \\
Blown back upon themselves; their reason seemed  \\
Confusion-stricken by a higher power  \\
Than human understanding, their discourse	  \\
Maimed, spiritless---and, in their weakness strong,  \\
I triumphed.  \\
Meantime day by day the roads,  \\
While I consorted with these royalists,	  \\
Were crowded with the bravest youth of France  \\
And all the promptest of her spirits, linked  \\
In gallant soldiership, and posting on  \\
To meet the war upon her frontier-bounds.  \\
Yet at this very moment do tears start	  \\
Into mine eyes---I do not say I weep,  \\
I wept not then, but tears have dimmed my sight---  \\
In memory of the farewells of that time,  \\
Domestic severings, female fortitude  \\
At dearest separation, patriot love	  \\
And self-devotion, and terrestrial hope  \\
Encouraged with a martyr's confidence.  \\
Even files of strangers merely, seen but once  \\
And for a moment, men from far, with sound  \\
Of music, martial tunes, and banners spread,	  \\
Entering the city, here and there a face  \\
Or person singled out among the rest  \\
Yet still a stranger, and beloved as such---  \\
Even by these passing spectacles my heart  \\
Was oftentimes uplifted, and they seemed	  \\
Like arguments from Heaven that 'twas a cause  \\
Good, and which no one could stand up against  \\
Who was not lost, abandoned, selfish, proud,  \\
Mean, miserable, wilfully depraved,  \\
Hater perverse of equity and truth.	  \\
Among that band of officers was one,  \\
Already hinted at, of other mold---  \\
A patriot, thence rejected by the rest,  \\
And with an oriental loathing spurned  \\
As of a different cast. A meeker man	  \\
Than this lived never, or a more benign---  \\
Meek, though enthusiastic to the height  \\
Of highest expectation. Injuries  \\
Made him more gracious, and his nature then  \\
Did breathe its sweetness out most sensibly,	  \\
As aromatic flowers on Alpine turf  \\
When foot hath crushed them. He through the events  \\
Of that great change wandered in perfect faith,  \\
As through a book, an old romance, or tale  \\
Of Fairy, or some dream of actions wrought	  \\
Behind the summer clouds. By birth he ranked  \\
With the most noble, but unto the poor  \\
Among mankind he was in service bound  \\
As by some tie invisible, oaths professed  \\
To a religious order. Man he loved	  \\
As man, and to the mean and the obscure,  \\
And all the homely in their homely works,  \\
Transferred a courtesy which had no air  \\
Of condescension, but did rather seem  \\
A passion and a gallantry, like that	  \\
Which he, a soldier, in his idler day  \\
Had payed to woman. Somewhat vain he was,  \\
Or seemed so---yet it was not vanity,  \\
But fondness, and a kind of radiant joy  \\
That covered him about when he was bent	  \\
On works of love or freedom, or revolved  \\
Complacently the progress of a cause  \\
Whereof he was a part---yet this was meek  \\
And placid, and took nothing from the man  \\
That was delightful. Oft in solitude	  \\
With him did I discourse about the end  \\
Of civil government, and its wisest forms,  \\
Of ancient prejudice and chartered rights,  \\
Allegiance, faith, and laws by time matured,  \\
Custom and habit, novelty and change,	  \\
Of self-respect, and virtue in the few  \\
For patrimonial honour set apart,  \\
And ignorance in the labouring multitude.  \\
For he, an upright man and tolerant,  \\
Balanced these contemplations in his mind,	  \\
And I, who at that time was scarcely dipped  \\
Into the turmoil, had a sounder judgement  \\
Than afterwards, carried about me yet  \\
With less alloy to its integrity  \\
The experience of past ages, as through help	  \\
Of books and common life it finds its way  \\
To youthful minds, by objects over near  \\
Not pressed upon, nor dazzled or misled  \\
By struggling with the crowd for present ends.  \\
But though not deaf and obstinate to find	  \\
Error without apology on the side  \\
Of those who were against us, more delight  \\
We took, and let this freely be confessed,  \\
In painting to ourselves the miseries  \\
Of royal courts, and that voluptuous life	  \\
Unfeeling where the man who is of soul  \\
The meanest thrives the most, where dignity,  \\
True personal dignity, abideth not---  \\
A light and cruel world, cut off from all  \\
The natural inlets of just sentiment,	  \\
From lowly sympathy, and chastening truth,  \\
When good and evil never have the name,  \\
That which they ought to have, but wrong prevails,  \\
And vice at home. We added dearest themes,  \\
Man and his noble nature, as it is	  \\
The gift of God and lies in his own power,  \\
His blind desires and steady faculties  \\
Capable of clear truth, the one to break  \\
Bondage, the other to build liberty  \\
On firm foundations, making social life,	  \\
Through knowledge spreading and imperishable,  \\
As just in regulation, and as pure,  \\
As individual in the wise and good.  \\
We summoned up the honorable deeds  \\
Of ancient story, thought of each bright spot	  \\
That could be found in all recorded time,  \\
Of truth preserved and error passed away,  \\
Of single spirits that catch the flame from heaven,  \\
And how the multitude of men will feed  \\
And fan each other---thought of sects, how keen	  \\
They are to put the appropriate nature on,  \\
Triumphant over every obstacle  \\
Of custom, language, country, love and hate,  \\
And what they do and suffer for their creed,  \\
How far they travel, and how long endure---	  \\
How quickly mighty nations have been formed  \\
From least beginnings, how, together locked  \\
By new opinions, scattered tribes have made  \\
One body, spreading wide as clouds in heaven.  \\
To aspirations then of our own minds	  \\
Did we appeal; and, finally, beheld  \\
A living confirmation of the whole  \\
Before us in a people risen up  \\
Fresh as the morning star. Elate we looked  \\
Upon their virtues, saw in rudest men	  \\
Self-sacrifice the firmest, generous love  \\
And continence of mind, and sense of right  \\
Uppermost in the midst of fiercest strife.  \\
Oh, sweet it is in academic groves---   \\
Or such retirement, friend, as we have known  \\
Among the mountains by our Rotha's stream,  \\
Greta, or Derwent, or some nameless rill---  \\
To ruminate, with interchange of talk,  \\
On rational liberty and hope in man,   \\
Justice and peace. But far more sweet such toil  \\
(Toil, say I, for it leads to thoughts abstruse)  \\
If Nature then be standing on the brink  \\
Of some great trial, and we hear the voice  \\
Of one devoted, one whom circumstance   \\
Hath called upon to embody his deep sense  \\
In action, give it outwardly a shape,  \\
And that of benediction to the world.  \\
Then doubt is not, and truth is more than truth---  \\
A hope it is and a desire, a creed   \\
Of zeal by an authority divine  \\
Sanctioned, of danger, difficulty, or death.  \\
Such conversation under Attic shades  \\
Did Dion hold with Plato, ripened thus  \\
For a deliverer's glorious task, and such   \\
He, on that ministry already bound,  \\
Held with Eudemus and Timonides,  \\
Surrounded by adventurers in arms,  \\
When those two vessels with their daring freight  \\
For the Sicilian tyrant's overthrow   \\
Sailed from Zacynthus---philosophic war  \\
Led by philosophers. With harder fate,  \\
Though like ambition, such was he, O friend,  \\
Of whom I speak. So Beaupuis---let the name  \\
Stand near the worthiest of antiquity---   \\
Fashioned his life, and many a long discourse  \\
With like persuasion honored we maintained,  \\
He on his part accoutred for the worst.  \\
He perished fighting, in supreme command,  \\
Upon the borders of the unhappy Loire,   \\
For liberty, against deluded men,  \\
His fellow countrymen; and yet most blessed  \\
In this, that he the fate of later times  \\
Lived not to see, nor what we now behold  \\
Who have as ardent hearts as he had then.	  \\
Along that very Loire, with festivals  \\
Resounding at all hours, and innocent yet  \\
Of civil slaughter, was our frequent walk,  \\
Or in wide forests of the neighbourhood,  \\
High woods and over-arched, with open space	  \\
On every side, and footing many a mile,  \\
Inwoven roots, and moss smooth as the sea---  \\
A solemn region. Often in such place  \\
From earnest dialogues I slipped in thought,  \\
And let remembrance steal to other times	  \\
When hermits, from their sheds and caves forth strayed,  \\
Walked by themselves, so met in shades like these,  \\
And if a devious traveller was heard  \\
Approaching from a distance, as might chance,  \\
With speed and echoes loud of trampling hoofs	  \\
From the hard floor reverberated, then  \\
It was Angelica thundering through the woods  \\
Upon her palfrey, or that gentler maid  \\
Erminia, fugitive as fair as she.  \\
Sometimes I saw methought a pair of knights	  \\
Joust underneath the trees, that as in storm  \\
Did rock above their heads, anon the din  \\
Of boisterous merriment and music's roar,  \\
With sudden proclamation, burst from haunt  \\
Of satyrs in some viewless glade, with dance	  \\
Rejoicing o'er a female in the midst,  \\
A mortal beauty, their unhappy thrall.  \\
The width of those huge forests, unto me  \\
A novel scene, did often in this way  \\
Master my fancy while I wandered on	  \\
With that revered companion. And sometimes  \\
When to a convent in a meadow green  \\
By a brook-side we came---a roofless pile,  \\
And not by reverential touch of time  \\
Dismantled, but by violence abrupt---	  \\
In spite of those heart-bracing colloquies,  \\
In spite of real fervour, and of that  \\
Less genuine and wrought up within myself,  \\
I could not but bewail a wrong so harsh,  \\
And for the matin-bell---to sound no more---	  \\
Grieved, and the evening taper, and the cross  \\
High on the topmost pinnacle, a sign  \\
Admonitory to the traveller,  \\
First seen above the woods.  \\
And when my friend	  \\
Pointed upon occasion to the site  \\
Of Romarentin, home of ancient kings,  \\
To the imperial edifice of Blois,  \\
Or to that rural castle, name now slipped  \\
From my remembrance, where a lady lodged	  \\
By the first Francis wooed, and bound to him  \\
In chains of mutual passion---from the tower,  \\
As a tradition of the country tells,  \\
Practised to commune with her royal knight  \\
By cressets and love-beacons, intercourse	  \\
'Twixt her high-seated residence and his  \\
Far off at Chambord on the plain beneath---  \\
Even here, though less than with the peaceful house  \\
Religious, 'mid these frequent monuments  \\
Of kings, their vices and their better deeds,	  \\
Imagination, potent to enflame  \\
At times with virtuous wrath and noble scorn,  \\
Did also often mitigate the force  \\
Of civic prejudice, the bigotry,  \\
So call it, of a youthful patriot's mind,	  \\
And on these spots with many gleams I looked  \\
Of chivalrous delight. Yet not the less,  \\
Hatred of absolute rule, where will of one  \\
Is law for all, and of that barren pride  \\
In those who by immunities unjust	  \\
Betwixt the sovereign and the people stand,  \\
His helpers and not theirs, laid stronger hold  \\
Daily upon me---mixed with pity too,  \\
And love, for where hope is, there love will be  \\
For the abject multitude. And when we chanced	  \\
One day to meet a hunger-bitten girl  \\
Who crept along fitting her languid self  \\
Unto a heifer's motion---by a cord  \\
Tied to her arm, and picking thus from the lane  \\
Its sustenance, while the girl with her two hands	  \\
Was busy knitting in a heartless mood  \\
Of solitude---and at the sight my friend  \\
In agitation said, ''Tis against that  \\
Which we are fighting', I with him believed  \\
Devoutly that a spirit was abroad	  \\
Which could not be withstood, that poverty,  \\
At least like this, would in a little time  \\
Be found no more, that we should see the earth  \\
Unthwarted in her wish to recompense  \\
The industrious, and the lowly child of toil,	  \\
All institutes for ever blotted out  \\
That legalized exclusion, empty pomp  \\
Abolished, sensual state and cruel power,  \\
Whether by edict of the one or few---  \\
And finally, as sum and crown of all,	  \\
Should see the people having a strong hand  \\
In making their own laws, whence better days  \\
To all mankind. But, these things set apart,  \\
Was not the single confidence enough  \\
To animate the mind that ever turned	  \\
A thought to human welfare?---that henceforth  \\
Captivity by mandate without law  \\
Should cease, and open accusation lead  \\
To sentence in the hearing of the world,  \\
And open punishment, if not the air	  \\
Be free to breathe in, and the heart of man  \\
Dread nothing. Having touched this argument  \\
I shall not, as my purpose was, take note  \\
Of other matters which detained us oft  \\
In thought or conversation---public acts,	  \\
And public persons, and the emotions wrought  \\
Within our minds by the ever-varying wind  \\
Of record and report which day by day  \\
Swept over us---but I will here instead  \\
Draw from obscurity a tragic tale,	  \\
Not in its spirit singular, indeed,  \\
But haply worth memorial, as I heard  \\
The events related by my patriot friend  \\
And others who had borne a part therein.  \\
Oh, happy time of youthful lovers---thus	  \\
My story may begin---oh, balmy time  \\
In which a love-knot on a lady's brow  \\
Is fairer than the fairest star in heaven!  \\
To such inheritance of blessedness  \\
Young Vaudracour was brought by years that had	  \\
A little overstepped his stripling prime.  \\
A town of small repute in the heart of France  \\
Was the youth's birthplace; there he vowed his love  \\
To Julia, a bright maid from parents sprung  \\
Not mean in their condition, but with rights	  \\
Unhonoured of nobility---and hence  \\
The father of the young man, who had place  \\
Among that order, spurned the very thought  \\
Of such alliance. From their cradles up,  \\
With but a step between their several homes,	  \\
Th pair had thriven together year by year,  \\
Friends, playmates, twins in pleasure, after strife  \\
And petty quarrels had grown fond again,  \\
Each other's advocate, each other's help,  \\
Nor ever happy if they were apart.	  \\
A basis this for deep and solid love,  \\
And endless constancy, and placid truth---  \\
But whatsoever of such treasures might,  \\
Beneath the outside of their youth, have lain  \\
Reserved for mellower years, his present mind	  \\
Was under fascination---he beheld  \\
A vision, and he loved the thing he saw.  \\
Arabian fiction never filled the world  \\
With half the wonders that were wrought for him:  \\
Earth lived in one great presence of the spring,	  \\
Life turned the meanest of her implements  \\
Before his eyes to price above all gold,  \\
The house she dwelt in was a sainted shrine,  \\
Her chamber-window did surpass in glory  \\
The portals of the east, all paradise	  \\
Could by the simple opening of a door  \\
Let itself in upon him---pathways, walks,  \\
Swarmed with enchantment, till his spirits sunk  \\
Beneath the burthen, overblessed for life.  \\
This state was theirs, till---whether through effect	  \\
Of some delirious hour, or that the youth,  \\
Seeing so many bars betwixt himself  \\
And the dear haven where he wished to be  \\
In honorable wedlock with his love,  \\
Without a certain knowledge of his own	  \\
Was inwardly prepared to turn aside  \\
From law and custom and entrust himself  \\
To Nature for a happy end of all,  \\
And thus abated of that pure reserve  \\
Congenial to his loyal heart, with which	  \\
It would have pleased him to attend the steps  \\
Of maiden so divinely beautiful,  \\
I know not---but reluctantly must add  \\
That Julia, yet without the name of wife,  \\
Carried about her for a secret grief	  \\
The promise of a mother.  \\
To conceal  \\
The threatened shame the parents of the maid  \\
Found means to hurry her away, by night  \\
And unforewarned, that in a distant town	  \\
She might remain shrouded in privacy  \\
Until the babe was born. When morning came  \\
The lover, thus bereft, stung with his loss  \\
And all uncertain whither he should turn,  \\
Chafed like a wild beast in the toils. At length,	  \\
Following as his suspicions led, he found---  \\
O joy!---sure traces of the fugitives,  \\
Pursued them to the town where they had stopped,  \\
And lastly to the very house itself  \\
Which had been chosen for the maid's retreat.	  \\
The sequel may be easily divined:  \\
Walks backwards, forwards, morning, noon, and night  \\
(When decency and caution would allow), And  \\
Julia, who, whenever to herself  \\
She happened to be left a moment's space,	  \\
Was busy at her casement as a swallow  \\
About its nest, erelong did thus espy  \\
Her lover; thence a stolen interview  \\
By night accomplished, with a ladder's help.  \\
I pass the raptures of the pair, such theme	  \\
Hath by a hundred poets been set forth  \\
In more delightful verse than skill of mine  \\
Could fashion---chiefly by that darling bard  \\
Who told of Juliet and her Romeo,	  \\
And of the lark's note heard before its time,  \\
And of the streaks that laced the evening clouds  \\
In the unrelenting east. 'Tis mine to tread  \\
The humbler province of plain history,  \\
And, without choice of circumstance, submissively	  \\
Relate what I have heard. The lovers came  \\
To this resolve---with which they parted, pleased  \\
And confident---that Vaudracour should hie  \\
Back to his father's house, and there employ  \\
Means aptest to obtain a sum of gold,	  \\
A final portion even, if that might be;  \\
Which done, together they could then take flight  \\
To some remote and solitary place  \\
Where they might live with no one to behold  \\
Their happiness, or to disturb their love.	  \\
Immediately, and with this mission charged,  \\
Home to his father's house did he return,  \\
And there remained a time without hint given  \\
Of his design. But if a word were dropped  \\
Touching the matter of his passion, still,	  \\
In hearing of his father, Vaudracour  \\
Persisted openly that nothing less  \\
Than death should make him yield up hope to be  \\
A blesse`d husband of the maid he loved.  \\
Incensed at such obduracy, and slight	  \\
Of exhortations and remonstrances,  \\
The father threw out threats that by a mandate  \\
Bearing the private signet of the state  \\
He should be baffled of his mad intent---  \\
And that should cure him. From this time the youth	  \\
Conceived a terror, and by night or day  \\
Stirred nowhere without arms. Soon afterwards  \\
His parents to their country seat withdrew  \\
Upon some feigned occasion, and the son  \\
Was left with one attendant in the house.	  \\
Retiring to his chamber for the night,  \\
While he was entering at the door, attempts  \\
Were made to seize him by three arme`d men,  \\
The instruments of ruffian power. The youth  \\
In the first impulse of his rage laid one	  \\
Dead at his feet, and to the second gave  \\
A perilous wound---which done, at sight  \\
Of the dead man, he peacefully resigned  \\
His person to the law, was lodged in prison,  \\
And wore the fetters of a criminal.	  \\
Through three weeks' space, by means which love devised,  \\
The maid in her seclusion had received  \\
Tidings of Vaudracour, and how he sped  \\
Upon his enterprize. Thereafter came  \\
A silence; half a circle did the moon	  \\
Complete, and then a whole, and still the same  \\
Silence; a thousand thousand fears and hopes  \\
Stirred in her mind---thoughts waking, thoughts of sleep,  \\
Entangled in each other---and at last  \\
Self-slaughter seemed her only resting-place:	  \\
So did she fare in her uncertainty.  \\
At length, by interference of a friend,  \\
One who had sway at court, the youth regained  \\
His liberty, on promise to sit down  \\
Quietly in his father's house, nor take	  \\
One step to reunite himself with her  \\
Of whom his parents disapproved---hard law,  \\
To which he gave consent only because  \\
His freedom else could nowise by procured.  \\
Back to his father's house he went, remained	  \\
Eight days, and then his resolution failed---  \\
He fled to Julia, and the words with which  \\
He greeted her were these: '  \\
All right is gone, Gone from me.  \\
Thou no longer now art mine,  \\
I thine. A murderer, Julia, cannot love	  \\
An innocent woman. I behold thy face,  \\
I see thee, and my misery is complete.'  \\
She could not give him answer; afterwards  \\
She coupled with his father's name some words  \\
Of vehement indignation, but the youth	  \\
Checked her, nor would he hear of this, for thought  \\
Unfilial, or unkind, had never once  \\
Found harbour in his breast. The lovers, thus  \\
United once again, together lived  \\
For a few days, which were to Vaudracour	  \\
Days of dejection, sorrow and remorse  \\
For that ill deed of violence which his hand  \\
Had hastily committed---for the youth  \\
Was of a loyal spirit, a conscience nice,  \\
And over tender for the trial which	  \\
His fate had called him to. The father's mind  \\
Meanwhile remained unchanged, and Vaudracour  \\
Learned that a mandate had been newly issued  \\
To arrest him on the spot. Oh pain it was  \\
To part!---he could not, and he lingered still	  \\
To the last moment of his time, and then,  \\
At dead of night, with snow upon the ground,  \\
He left the city, and in villages,  \\
The most sequestered of the neighbourhood,  \\
Lay hidden for the space of several days,	  \\
Until, the horseman bringing back report  \\
That he was nowhere to be found, the search Was ended.  \\
Back returned the ill-fated youth,  \\
And from the house where Julia lodged---to which  \\
He now found open ingress, having gained	  \\
The affection of the family, who loved him  \\
Both for his own, and for the maiden's sake---  \\
One night retiring, he was seized.  \\
But here  \\
A portion of the tale may well be left	  \\
In silence, though my memory could add  \\
Much how the youth, and in short space of time,  \\
Was traversed from without---much, too, of thoughts  \\
By which he was employed in solitude  \\
Under privation and restraint, and what	  \\
Through dark and shapeless fear of things to come,  \\
And what through strong compunction for the past,  \\
He suffered, breaking down in heart and mind.  \\
Such grace, if grace it were, had been vouchsafed---  \\
Or such effect had through the father's want	  \\
Of power, or through his negligence, ensued---  \\
That Vaudracour was suffered to remain,  \\
Though under guard and without liberty,  \\
In the same city with the unhappy maid  \\
From whom he was divided. So they fared,	  \\
Objects of general concern, till, moved  \\
With pity for their wrongs, the magistrate  \\
(The same who had placed the youth in custody)  \\
By application to the minister  \\
Obtained his liberty upon condition	  \\
That to his father's house he should return.  \\
He left his prison almost on the eve  \\
Of Julia's travail. She had likewise been,  \\
As from the time, indeed, when she had first  \\
Been brought for secresy to this abode,	  \\
Though treated with consoling tenderness,  \\
Herself a prisoner---a dejected one,  \\
Filled with a lover's and a woman's fears---  \\
And whensoe'er the mistress of the house  \\
Entered the room for the last time at night,	  \\
And Julia with a low and plaintive voice  \\
Said, 'You are coming then to lock me up',  \\
The housewife when these words---always the same---  \\
Were by her captive languidly pronounced,  \\
Could never hear them uttered without tears.	  \\
A day or two before her childbed time  \\
Was Vaudracour restored to her, and, soon  \\
As he might be permitted to return  \\
Into her chamber after the child's birth,  \\
The master of the family begged that all	  \\
The household might be summoned, doubting not  \\
But that they might receive impressions then  \\
Friendly to human kindness. Vaudracour  \\
(This heard I from one present at the time)  \\
Held up the new-born infant in his arms	  \\
And kissed, and blessed, and covered it with tears,  \\
Uttering a prayer that he might never be  \\
As wretched as his father. Then he gave  \\
The child to her who bare it, and she too  \\
Repeated the same prayer---took it again,	  \\
And, muttering something faintly afterwards,  \\
He gave the infant to the standers-by,  \\
And wept in silence upon Julia's neck.  \\
Two months did he continue in the house,  \\
And often yielded up himself to plans	  \\
Of future happiness. 'You shall return,  \\
Julia', said he, 'and to your father's house  \\
Go with your child; you have been wretched, yet  \\
It is a town where both of us were born---  \\
None will reproach you, for our loves are known.	  \\
With ornaments the prettiest you shall dress  \\
Your boy, as soon as he can run about,  \\
And when he thus is at his play my father  \\
Will see him from the window, and the child  \\
Will by his beauty move his grandsire's heart,	  \\
So that it shall be softened, and our loves  \\
End happily, as they began.' These gleams  \\
Appeared but seldom; oftener he was seen  \\
Propping a pale and melancholy face  \\
Upon the mother's bosom, resting thus	  \\
His head upon one breast, while from the other  \\
The babe was drawing in its quiet food.  \\
At other times, when he in silence long A  \\
nd fixedly had looked upon her face,  \\
He would exclaim, 'Julia, how much thine eyes	  \\
Have cost me! During daytime, when the child  \\
Lay in its cradle, by its side he sate,  \\
Not quitting it an instant. The whole town  \\
In his unmerited misfortunes now  \\
Took part, and if he either at the door	  \\
Or window for a moment with his child  \\
Appeared, immediately the street was thronged;  \\
While others, frequently, without reserve,  \\
Passed and repassed before the house to steal  \\
A look at him. Oft at this time he wrote	  \\
Requesting, since he knew that the consent  \\
Of Julia's parents never could be gained  \\
To a clandestine marriage, that his father  \\
Would from the birthright of an eldest son  \\
Exclude him, giving but, when this was done,	  \\
A sanction to his nuptials. Vain request,  \\
To which no answer was returned.  \\
And now  \\
From her own home the mother of his love  \\
Arrived to apprise the daughter of her fixed	  \\
and last resolve, that, since all hope to move  \\
The old man's heart proved vain, she must retire  \\
Into a convent and be there immured.  \\
Julia was thunderstricken by these words,  \\
And she insisted on a mother's rights	  \\
To take her child along with her---a grant  \\
Impossible, as she at last perceived.  \\
The persons of the house no sooner heard  \\
Of this decision upon Julia's fate  \\
Than everyone was overwhelmed with grief,	  \\
Nor could they frame a manner soft enough  \\
To impart the tidings to the youth. But great  \\
Was their astonishment when they beheld him  \\
Receive the news in calm despondency,  \\
Composed and silent, without outward sign	  \\
Of even the least emotion. Seeing this,  \\
When Julia scattered some upbraiding words  \\
Upon his slackness, he thereto returned  \\
No answer, only took the mother's hand  \\
(Who loved him scarcely less than her own child)	  \\
And kissed it, without seeming to be pressed  \\
By any pain that 'twas the hand of one  \\
Whose errand was to part him from his love  \\
For ever. In the city he remained  \\
A season after Julia had retired	  \\
And in the convent taken up her home,  \\
To the end that he might place his infant babe  \\
With a fit nurse; which done, beneath the roof  \\
Where now his little one was lodged he passed  \\
The day entire, and scarcely could at length	  \\
Tear himself from the cradle to return  \\
Home to his father's house---in which he dwelt  \\
Awhile, and then came back that he might see  \\
Whether the babe had gained sufficient strength  \\
To bear removal. He quitted this same town	  \\
For the last time, attendant by the side  \\
Of a close chair, a litter or sedan,  \\
In which the child was carried. To a hill  \\
Which rose at a league's distance from the town  \\
The family of the house where he had lodged	  \\
Attended him, and parted from him there,  \\
Watching below until he disappeared  \\
On the hill-top. His eyes he scarcely took  \\
Through all that journey from the chair in which  \\
The babe was carried, and at every inn	  \\
Or place at which they halted or reposed  \\
Laid him upon his knees, nor would permit  \\
The hands of any but himself to dress  \\
The infant, or undress. By one of those  \\
Who bore the chair these facts, at his return,	  \\
Were told, and in relating them he wept.  \\
This was the manner in which Vaudracour  \\
Departed with his infant, and thus reached  \\
His father's house, where to the innocent child  \\
Admittance was denied. The young man spake	  \\
No words of indignation or reproof,  \\
But of his father begged, a last request,  \\
That a retreat might be assigned to him---  \\
A house where in the country he might dwell  \\
With such allowance as his wants required---	  \\
And the more lonely that the mansion was  \\
'Twould be more welcome. To a lodge that stood  \\
Deep in a forest, with leave given, at the age  \\
Of four and twenty summers he retired,  \\
And thither took with him his infant babe	  \\
And one domestic for their common needs,  \\
An aged woman. It consoled him here  \\
To attend upon the orphan and perform  \\
The office of a nurse to his young child,  \\
Which, after a short time, by some mistake	  \\
Or indiscretion of the father, died.  \\
The tale I follow to its recess  \\
Of suffering or of peace, I know not which---  \\
Theirs be the blame who caused the woe, not mine.  \\
From that time forth he never uttered word	  \\
To any living. An inhabitant  \\
Of that same town in which the pair had left  \\
So lively a remembrance of their griefs,  \\
By chance of business coming within reach  \\
Of his retirement, to the spot repaired	  \\
With the intent to visit him; he reached  \\
The house and only found the matron there,  \\
Who told him that his pains were thrown away,  \\
For that her master never uttered word  \\
To living soul---not even to her. Behold,	  \\
While they were speaking Vaudracour approached,  \\
But, seeing some one there, just as his hand  \\
Was stretched towards the garden-gate, he shrunk  \\
And like a shadow glided out of view.  \\
Shocked at his savage outside, from the place	  \\
The visitor retired.  \\
Thus lived the youth,  \\
Cut off from all intelligence with man,  \\
And shunning even the light of common day.  \\
Nor could the voice of freedom, which through France	  \\
Soon afterwards resounded, public hope,  \\
Or personal memory of his own deep wrongs,  \\
Rouze him, but in those solitary shades  \\
His days he wasted, an imbecile mind. \\
\end{verse}

%%%%%%%%%%%%%%%%%%%%%%%%%%%%%%%%%%%%%%%%%%%%%%%%% \\
\chapter*[Book Tenth]{Book Tenth \\ Residence in France and French Revolution}
\addcontentsline{toc}{chapter}{Book Tenth Residence in France and French Revolution}

\begin{verse}
IT was a beautiful and silent day  \\
That overspread the countenance of earth,  \\
Then fading, with unusual quietness,  \\
When from the Loire I parted, and through scenes  \\
Of vineyard, orchard, meadow-ground and tilth,	  \\
Calm waters, gleams of sun, and breathless trees,  \\
Towards the fierce metropolis turned my steps  \\
Their homeward way to England. From his throne  \\
The King had fallen; the congregated host---  \\
Dire cloud, upon the front of which was written	  \\
The tender mercies of the dismal wind  \\
That bore it---on the plains of Liberty  \\
Had burst innocuously. Say more, the swarm  \\
That came elate and jocund, like a band  \\
Of eastern hunters, to enfold in ring	  \\
Narrowing itself by moments, and reduce  \\
To the last punctual spot of their despair,  \\
A race of victims---so they seemed---themselves  \\
Had shrunk from sight of their own task, and fled  \\
In terror. Desolation and dismay	  \\
Remained for them whose fancies had grown rank  \\
With evil expectations: confidence  \\
And perfect triumph to the better cause.  \\
The state, as if to stamp the final seal  \\
On her security, and to the world	  \\
Shew what she was, a high and fearless soul---  \\
Or rather in a spirit of thanks to those  \\
Who had stirred up her slackening faculties  \\
To a new transition---had assumed with joy  \\
The body and the venerable name	  \\
Of a republic. Lamentable crimes,  \\
'Tis true, had gone before this hour---the work  \\
Of massacre, in which the senseless sword  \\
Was prayed to as a judge---but these were past,  \\
Earth free from them for ever (as was thought),	  \\
Ephemeral monsters, to be seen but once,  \\
Things that could only shew themselves and die.  \\
This was the time in which, enflamed with hope,  \\
To Paris I returned. Again I ranged,  \\
More eagerly than I had done before,	  \\
Through the wide city, and in progress passed  \\
The prison where the unhappy monarch lay,  \\
Associate with his children and his wife  \\
In bondage, and the palace, lately stormed  \\
With roar of cannon and a numerous host.	  \\
I crossed---a black and empty area then---  \\
The square of the Carousel, a few weeks back  \\
Heaped up with dead and dying, upon these  \\
And other sights looking as doth a man  \\
Upon a volume whose contents he knows	  \\
Are memorable but from him locked up,  \\
Being written in a tongue he cannot read,  \\
So that he questions the mute leaves with pain,  \\
And half upbraids their silence. But that night  \\
When on my bed I lay, I was most moved	  \\
And felt most deeply in what world I was;  \\
My room was high and lonely, near the roof  \\
Of a large mansion or hotel, a spot  \\
That would have pleased me in more quiet times---  \\
Nor was it wholly without pleasure then.	  \\
With unextinguished taper I kept watch,  \\
Reading at intervals. The fear gone by  \\
Pressed on me almost like a fear to come.  \\
I thought of those September massacres,  \\
Divided from me by a little month,	  \\
And felt and touched them, a substantial dread  \\
(The rest was conjured up from tragic fictions,  \\
And mournful calendars of true history,  \\
Remembrances and dim admonishments):  \\
'The horse is taught his manage, and the wind	  \\
Of heaven wheels round and treads in his own steps;  \\
Year follows year, the tide returns again,  \\
Day follows day, all things have second birth;  \\
The earthquake is not satisfied at once'---  \\
And in such way I wrought upon myself,	  \\
Until I seemed to hear a voice that cried  \\
To the whole city, 'Sleep no more!' To this  \\
Add comments of a calmer mind---from which  \\
I could not gather full security---  \\
But at the best it seemed a place of fear,	  \\
Unfit for the repose of night,  \\
Defenceless as a wood where tigers roam.  \\
Betimes next morning to the Palace-walk  \\
Of Orleans I repaired, and entering there  \\
Was greeted, among divers other notes,	  \\
By voices of the hawkers in the crowd  \\
Brawling, Denunciation of the crimes  \\
Of Maximilian Robespierre. The speech  \\
Which in their hands they carried was the same  \\
Which had been recently pronounced---the day	  \\
When Robespierre, well known for what mark  \\
Some words of indirect reproof had been  \\
Intended, rose in hardihood, and dared  \\
The man who had ill surmise of him  \\
To bring his charge in openness. Whereat,	  \\
When a dead pause ensued and no one stirred,  \\
In silence of all present, from his seat  \\
Louvet walked singly through the avenue  \\
And took his station in the Tribune, saying,  \\
'I, Robespierre, accuse thee!' 'Tis well known	  \\
What was the issue of that charge, and how  \\
Louvet was left alone without support  \\
Of his irresolute friends, but these are things  \\
Of which I speak only as they were storm  \\
Or sunshine to my individual mind,	  \\
No further. Let me than relate that now---  \\
In some sort seeing with my proper eyes  \\
That liberty, and life, and death, would soon  \\
To the remotest corners of the land  \\
Lie in the arbitrement of those who ruled	  \\
The capital city; what was struggled for,  \\
And by what combatants victory must be won;  \\
The indecision on their part whose aim  \\
Seemed best, and the straightforward path of those  \\
Who in attack or in defence alike	  \\
Were strong through their impiety---greatly  \\
I Was agitated. Yea, I could almost  \\
Have prayed that throughout earth upon all souls  \\
Worthy of liberty, upon every soul  \\
Matured to live in plainness and in truth,	  \\
The gift of tongues might fall, and men arrive  \\
From the four quarters of the winds to do  \\
For France what without help she could not do,  \\
A work of honour---think not that to this  \\
I added, work of safety: from such thought,	  \\
And the least fear about the end of things,  \\
I was as far as angels are from guilt.  \\
Yet did I grieve, nor only grieved, but thought  \\
Of opposition and of remedies:  \\
An insignificant stranger and obscure,	  \\
Mean as I was, and little graced with powers  \\
Of eloquence even in my native speech,  \\
And all unfit for tumult and intrigue,  \\
Yet would I willingly have taken up  \\
A service at this time for cause so great,	  \\
However dangerous. Inly I revolved  \\
How much the destiny of man had still  \\
Hung upon single persons; that there was,  \\
Transcendent to all local patrimony,  \\
One nature as there is one sun in heaven;	  \\
That objects, even as they are great, thereby  \\
Do come within the reach of humblest eyes;  \\
That man was only weak through his mistrust  \\
And want of hope, where evidence divine  \\
Proclaimed to him that hope should be most sure;	  \\
That, with desires heroic and firm sense,  \\
A spirit thoroughly faithful to itself,  \\
Unquenchable, unsleeping, undismayed,  \\
Was as an instinct among men, a stream  \\
That gathered up each petty straggling rill	  \\
And vein of water, glad to be rolled on  \\
In safe obedience; that a mind whose rest  \\
Was where it ought to be, in self-restraint,  \\
In circumspection and simplicity,  \\
Fell rarely in entire discomfiture	  \\
Below its aim, or met with from without  \\
A treachery that defeated it or foiled.  \\
On the other side, I called to mind those truths  \\
Which are the commonplaces of the schools,  \\
A theme for boys, too trite even to be felt,	  \\
Yet with revelation's liveliness  \\
In all their comprehensive bearings known  \\
And visible to philosophers of old,  \\
Men who, to business of the world untrained,  \\
Lived in the shade; and to Harmodius known,	  \\
And his compeer Aristogiton; known  \\
To Brutus---that tyrannic power is weak,  \\
Hath neither gratitude, nor faith nor love,  \\
Nor the support of good or evil men,  \\
To trust in; that the godhead which is ours	  \\
Can never utterly be charmed or stilled;  \\
That nothing hath a natural right to last  \\
But equity and reason; that all else  \\
Meets foes irreconcilable, and at best  \\
Doth live but by variety of disease.	  \\
Well might my wishes be intense, my thoughts  \\
Strong and perturbed, not doubting at that time---  \\
Creed which ten shameful years have not annulled---  \\
But that the virtue of one paramount mind	  \\
Would have abashed those impious crests, have quelled  \\
Outrage and bloody power, and in despite  \\
Of what the people were through ignorance  \\
And immaturity, and in the teeth  \\
Of desperate opposition from without,	  \\
Have cleared a passage for just government,  \\
And left a solid birthright to the state,  \\
Redeemed according to example given  \\
By ancient lawgivers. In this frame of mind  \\
Reluctantly to England I returned,	  \\
Compelled by nothing less than absolute want  \\
Of funds for my support; else, well assured  \\
That I both was and must be of small worth,  \\
No better than an alien in the land,  \\
I doubtless should have made a common cause	  \\
With some who perished, haply perished too---  \\
A poor mistaken and bewildered offering,  \\
Should to the breast of Nature have gone back,  \\
With all my resolutions, all my hopes,  \\
A poet only to myself, to men	  \\
Useless, and even, belove`d friend, a soul  \\
To thee unknown.  \\
When to my native land,  \\
After a whole year's absence, I returned,  \\
I found the air yet busy with the stir	  \\
Of a contention which had been raised up  \\
Against the traffickers in Negro blood,  \\
An effort which, though baffled, nevertheless  \\
Had called back old forgotten principles  \\
Dismissed from service, had diffused some truths,	  \\
And more of virtuous feeling, through the heart  \\
Of the English people. And no few of those,  \\
So numerous---little less in verity  \\
Than a whole nation crying with one voice---  \\
Who had been crossed in this their just intent	  \\
And righteous hope, thereby were well prepared  \\
To let that journey sleep awhile, and join  \\
Whatever other caravan appeared  \\
To travel forward towards Liberty  \\
With more success. For me that strife had ne'er	  \\
Fastened on my affections, nor did now  \\
Its unsuccessful issue much excite  \\
My sorrow, having laid this faith to heart,  \\
That if France prospered good men would not long  \\
Pay fruitless worship to humanity,	  \\
And this most rotten branch of human shame  \\
(Object, as seemed, of superfluous pains)  \\
Would fall together with its parent tree.  \\
Such was my then belief---that there was one,  \\
And only one, solicitude for all.	  \\
And now the strength of Britain was put forth  \\
In league with the confederated host;  \\
Not in my single self alone I found,  \\
But in the minds of all ingenuous youth,  \\
Change and subversion from this hour. No shock	  \\
Given to my moral nature had I known  \\
Down to that very moment---neither lapse  \\
Nor turn of sentiment---that might be named  \\
A revolution, save at this one time:  \\
All else was progress on the self-same path	  \\
On which with a diversity of pace  \\
I had been travelling; this, a stride at once  \\
Into another region. True it is,  \\
'Twas not concealed with what ungracious eyes  \\
Our native rulers from the very first	  \\
Had looked upon regenerated France;  \\
Nor had I doubted that this day would come---  \\
But in such contemplation I had thought  \\
Of general interests only, beyond this  \\
Had never once foretasted the event.	  \\
Now had I other business, for I felt  \\
The ravage of this most unnatural strife  \\
In my own heart; there lay it like a weight,  \\
At enmity with all the tenderest springs  \\
Of my enjoyments. I, who with the breeze	  \\
Had played, a green leaf on the blessed tree  \\
Of my beloved country---nor had wished  \\
For happier fortune than to wither there---  \\
Now from my pleasant station was cut off,  \\
And tossed about in whirlwinds. I rejoiced,	  \\
Yes, afterwards, truth painful to record,  \\
Exulted in the triumph of my soul  \\
When Englishmen by thousands were o'erthrown,  \\
Left without glory on the field, or driven,  \\
Brave hearts, to shameful flight. It was a grief---	  \\
Grief call it not, 'twas any thing but that---  \\
A conflict of sensations without name,  \\
Of which he only who may love the sight  \\
Of a village steeple as I do can judge,  \\
When in the congregation, bending all	  \\
To their great Father, prayers were offered up  \\
Or praises for our country's victories,  \\
And, 'mid the simple worshippers perchance  \\
I only, like an uninvited guest  \\
Whom no one owned, sate silent---shall I add,	  \\
Fed on the day of vengeance yet to come!  \\
Oh, much have they to account for, who could tear  \\
By violence at one decisive rent  \\
From the best youth in England their dear pride,  \\
Their joy, in England. This, too, at a time	  \\
In which worst losses easily might wear  \\
The best of names; when patriotic love  \\
Did of itself in modesty give way  \\
Like the precursor when the deity  \\
Is come, whose harbinger he is---a time	  \\
In which apostacy from ancient faith  \\
Seemed but conversion to a higher creed;  \\
Withal a season dangerous and wild---  \\
A time in which Experience would have plucked  \\
Flowers out of any hedge to make thereof	  \\
A chaplet, in contempt of his grey locks.  \\
Ere yet the fleet of Britain had gone forth  \\
On this unworthy service, whereunto  \\
The unhappy counsel of a few weak men  \\
Had doomed it, I beheld the vessels lie---	  \\
A brood of gallant creatures---on the deep  \\
I saw them in their rest, a sojourner  \\
Through a whole month of calm and glassy days  \\
In that delightful island which protects  \\
Their place of convocation. There I heard	  \\
Each evening, walking by the still sea-shore,  \\
A monitory sound which never failed---  \\
The sunset cannon. When the orb went down  \\
In the tranquillity of Nature, came  \\
That voice---ill requiem---seldom heard by me	  \\
Without a spirit overcast, a deep  \\
Imagination, thought of woes to come,  \\
And sorrow for mankind, and pain of heart.  \\
In France, the men who for their desperate ends  \\
Had plucked up mercy by the roots were glad	  \\
Of this new enemy. Tyrants, strong before  \\
In devilish pleas, were ten times stronger now,  \\
And thus beset with foes on every side,  \\
The goaded land waxed mad; the crimes of few  \\
Spread into madness of the many; blasts	  \\
From hell came sanctified like airs from heaven.  \\
The sternness of the just, the faith of those  \\
Who doubted not that Providence had times  \\
Of anger and of vengeance, theirs who throned  \\
The human understanding paramount	  \\
And made of that their god, the hopes of those  \\
Who were content to barter short-lived pangs  \\
For a paradise of ages, the blind rage  \\
Of insolent tempers, the light vanity  \\
Of intermeddlers, steady purposes	  \\
Of the suspicious, slips of the indiscreet,  \\
And all the accidents of life, were pressed  \\
Into one service, busy with one work.  \\
The Senate was heart-stricken, not a voice  \\
Uplifted, none to oppose or mitigate.	  \\
Domestic carnage now filled all the year  \\
With feast-days: the old man from the chimney-nook,  \\
The maiden from the bosom of her love,  \\
The mother from the cradle of her babe,  \\
The warrior from the field---all perished, all---	  \\
Friends, enemies, of all parties, ages, ranks,  \\
Head after head, and never heads enough  \\
For those who bade them fall. They found their joy,  \\
They made it, ever thirsty, as a child---  \\
If light desires of innocent little ones	  \\
May with such heinous appetites be matched---  \\
Having a toy, a windmill, though the air  \\
Do of itself blow fresh and makes the vane  \\
Spin in his eyesight, he is not content,  \\
But with the plaything at arm's length he sets	  \\
His front against the blast, and runs amain  \\
To make it whirl the faster.  \\
In the depth  \\
Of these enormities, even thinking minds  \\
Forgot at seasons whence they had their being---	  \\
Forgot that such a sound was ever heard  \\
As Liberty upon earth---yet all beneath  \\
Her innocent authority was wrought,  \\
Nor could have been, without her blesse`d name.  \\
The illustrious wife of Roland, in the hour	  \\
Of her composure, felt that agony  \\
And gave it vent in her last words. O friend,  \\
It was a lamentable time for man,  \\
Whether a hope had e'er been his or not;  \\
A woeful time for them whose hopes did still	  \\
Outlast the shock; most woeful for those few---  \\
They had the deepest feeling of the grief---  \\
Who still were flattered, and had trust in man.  \\
Meanwhile the invaders fared as they deserved:  \\
The herculean Commonwealth had put forth her arms,	  \\
And throttled with an infant godhead's might  \\
The snakes about her cradle---that was well,  \\
And as it should be, yet no cure for those  \\
Whose souls were sick with pain of what would be  \\
Hereafter brought in charge against mankind.	  \\
Most melancholy at that time, O friend,  \\
Were my day-thoughts, my dreams were miserable;  \\
Through months, through years, long after the last beat  \\
Of those atrocities (I speak bare truth,  \\
As if to thee alone in private talk)	  \\
I scarcely had one night of quiet sleep,  \\
Such ghastly visions had I of despair,  \\
And tyranny, and implements of death,  \\
And long orations which in dreams I pleaded  \\
Before unjust tribunals, with a voice	  \\
Labouring, a brain confounded, and a sense  \\
Of treachery and desertion in the place  \\
The holiest that I knew of---my own soul.  \\
When I began at first, in early youth,  \\
To yield myself to Nature---when that strong	  \\
And holy passion overcame me first---  \\
Neither day nor night, evening or morn,  \\
Were free from the oppression, but, great God,  \\
Who send'st thyself into this breathing world  \\
Through Nature and through every kind of life,	  \\
And mak'st man what he is, creature divine,  \\
In single or in social eminence,  \\
Above all these raised infinite ascents  \\
When reason, which enables him to be,  \\
Is not sequestered---what a change is here!	  \\
How different ritual for this after-worship,  \\
What countenance to promote this second love!  \\
That first was service but to things which lie  \\
At rest, within the bosom of thy will:  \\
Therefore to serve was high beatitude;	  \\
The tumult was a gladness, and the fear  \\
Ennobling, venerable; sleep secure,  \\
And waking thoughts more rich than happiest dreams.  \\
But as the ancient prophets were enflamed,  \\
Nor wanted consolations of their own	  \\
And majesty of mind, when they denounced  \\
On towns and cities, wallowing in the abyss  \\
Of their offences, punishment to come;  \\
Or saw like other men with bodily eyes  \\
Before them in some desolated place	  \\
The consummation of the wrath of Heaven;  \\
So did some portion of that spirit fall  \\
On me to uphold me through those evil times,  \\
And in their rage and dog-day heat I found  \\
Something to glory in, as just and fit,	  \\
And in the order of sublimest laws.  \\
And even if that were not, amid the awe  \\
Of unintelligible chastisement  \\
I felt a kind of sympathy with power---  \\
Motions raised up within me, nevertheless,	  \\
Which had relationship to highest things.  \\
Wild blasts of music thus did find their  \\
way Into the midst of terrible events,  \\
So that worst tempests might be listened to:  \\
Then was the truth received into my heart	  \\
That under heaviest sorrow earth can bring,  \\
Griefs bitterest of ourselves or of our kind,  \\
If from the affliction somewhere do not grow  \\
Honour which could not else have been---a faith,  \\
An elevation, and a sanctity---	  \\
If new strength be not given, or old restored,  \\
The blame is ours, not Nature's.  \\
When a taunt Was taken up by scoffers in their pride,  \\
Saying, 'Behold the harvest which we reap  \\
From popular government and equality',	  \\
I saw that it was neither these nor aught  \\
Of wild belief engrafted on their names  \\
By false philosophy, that caused the woe,  \\
But that it was a reservoir of guilt  \\
And ignorance, filled up from age to age,	  \\
That could no longer hold its loathsome charge,  \\
But burst and spread in deluge through the land.  \\
And as the desert hath green spots, the sea  \\
Small islands in the midst of stormy waves,  \\
So that disastrous period did not want	  \\
Such sprinklings of all human excellence  \\
As were a joy to hear of. Yet---nor less  \\
For those bright spots, those fair examples given  \\
Of fortitude, and energy, and love,  \\
And human nature faithful to itself	  \\
Under worst trials---was I impelled to think  \\
Of the glad time when first I traversed France,  \\
A youthful pilgrim; above all remembered  \\
That day when through an arch that spanned the street,  \\
A rainbow made of garish ornaments	  \\
(Triumphal pomp for Liberty confirmed)  \\
We walked, a pair of weary travellers,  \\
Along the town of Arras---place from which  \\
Issued that Robespierre, who afterwards  \\
Wielded the sceptre of the atheist crew.	  \\
When the calamity spread far and wide,  \\
And this same city, which had even appeared  \\
To outrun the rest in exultation, groaned  \\
Under the vengeance of her cruel son,  \\
As Lear reproached the winds, I could almost	  \\
Have quarrelled with that blameless spectacle  \\
For being yet an image in my mind  \\
To mock me under such a strange reverse.  \\
O friend, few happier moments have been mine  \\
Through my whole life than that when first I heard	  \\
That this foul tribe of Moloch was o'erthrown,  \\
And their chief regent levelled with the dust.  \\
The day was one which haply may deserve  \\
A separate chronicle. Having gone abroad  \\
From a small village where I tarried then,	  \\
To the same far-secluded privacy  \\
I was returning. Over the smooth sands  \\
Of Leven's ample aestuary lay  \\
My journey, and beneath a genial sun,  \\
With distant prospect among gleams of sky	  \\
And clouds, and intermingled mountain-tops,  \\
In one inseparable glory clad---  \\
Creatures of one ethereal substance, met  \\
In consistory, like a diadem  \\
Or crown of burning seraphs, as they sit	  \\
In the empyrean. Underneath this show Lay, as  \\
I knew, the nest of pastoral vales  \\
Among whose happy fields I had grown up  \\
From childhood. On the fulgent spectacle,  \\
Which neither changed, nor stirred, nor passed away,	  \\
I gazed, and with a fancy more alive  \\
On this account---that I had chanced to find  \\
That morning, ranging through the churchyard graves  \\
Of Cartmell's rural town, the place in which  \\
An honored teacher of my youth was laid.	  \\
While we were schoolboys he had died among us,  \\
And was born hither, as I knew, to rest  \\
With his own family. A plain stone, inscribed  \\
With name, date, office, pointed out the spot,  \\
To which a slip of verses was subjoined---	  \\
By his desire, as afterwards I learned---  \\
A fragment from the Elegy of Gray.  \\
A week, or little less, before his death  \\
He had said to me, 'My head will soon lie low';  \\
And when I saw the turf that covered him,	  \\
After the lapse of full eight years, those words,  \\
With sound of voice, and countenance of the man,  \\
Came back upon me, so that some few tears  \\
Fell from me in my own despite. And now,  \\
Thus travelling smoothly o'er the level sands,   \\
I thought with pleasure of the verses graven  \\
Upon his tombstone, saying to myself,  \\
'He loved the poets, and if now alive  \\
Would have loved me, as one not destitute  \\
Of promise, nor belying the kind hope   \\
Which he had formed when I at his command  \\
Began to spin, at first, my toilsome songs.'  \\
Without me and within as I advanced  \\
All that I saw, or felt, or communed with,  \\
Was gentleness and peace. Upon a small   \\
And rocky island near, a fragment stood---  \\
Itself like a sea rock---of what had been  \\
A Romish chapel, where in ancient times  \\
Masses were said at the hour which suited those  \\
Who crossed the sands with ebb of morning tide.   \\
Not far from this still ruin all the plain  \\
Was spotted with a variegated crowd  \\
Of coaches, wains, and travellers, horse and foot,  \\
Wading, beneath the conduct of their guide,  \\
In loose procession through the shallow stream   \\
Of inland water; the great sea meanwhile  \\
Was at safe distance, far retired. I paused,  \\
Unwilling to proceed, the scene appeared  \\
So gay and cheerful---when a traveller  \\
Chancing to pass, I carelessly inquired   \\
If any news were stirring, he replied  \\
In the familiar language of the day  \\
That, Robespierre was dead. Nor was a doubt,  \\
On further question, left within my mind  \\
But that the tidings were substantial truth---   \\
That he and his supporters all were fallen.  \\
Great was my glee of spirit, great my joy  \\
In vengeance, and eternal justice, thus  \\
Made manifest. 'Come now, ye golden times',  \\
Said I, forth-breathing on those open sands	  \\
A hymn of triumph, 'as the morning comes  \\
Out of the bosom of the night, come ye.  \\
Thus far our trust is verified: behold,  \\
They who with clumsy desperation brought  \\
Rivers of blood, and preached that nothing else	  \\
Could cleanse the Augean stable, by the might  \\
Of their own helper have been swept away.  \\
Their madness is declared and visible;  \\
Elsewhere will safety now be sought, and earth  \\
March firmly towards righteousness and peace.'	  \\
Then schemes I framed more calmly, when and how  \\
The madding factions might be tranquillized,  \\
And---though through hardships manifold and long---  \\
The mighty renovation would proceed.  \\
Thus, interrupted by uneasy bursts	  \\
Of exultation, I pursued my way  \\
Along that very shore which I had skimmed  \\
In former times, when, spurring from the Vale  \\
Of Nightshade, and St. Mary's mouldering fane,  \\
And the stone abbot, after circuit made	  \\
In wantonness of heart, a joyous crew  \\
Of schoolboys, hastening to their distant home,  \\
Along the margin of the moonlight sea,  \\
We beat with thundering hoofs the level sand.  \\
From this time forth in France, as is well known,	  \\
Authority put on a milder face,  \\
Yet every thing was wanting that might give  \\
Courage to those who looked for good by light  \\
Of rational experience---good I mean  \\
At hand, and in the spirit of past aims.	  \\
The same belief I nevertheless retained:  \\
The language of the Senate, and the acts  \\
And public measures of the Government,  \\
Though both of heartless omen, had not power  \\
To daunt me. In the people was my trust,	  \\
And in the virtues which mine eyes had seen,  \\
And to the ultimate repose of things  \\
I looked with unabated confidence.  \\
I knew that wound external could not take  \\
Life from the young Republic, that new foes	  \\
Would only follow in the path of shame  \\
Their brethren, and her triumphs be in the end  \\
Great, universal, irresistible.  \\
This faith, which was an object in my mind  \\
Of passionate intuition, had effect	  \\
Not small in dazzling me; for thus, through zeal,  \\
Such victory I confounded in my thoughts  \\
With one far higher and more difficult:  \\
Triumphs of unambitious peace at home,  \\
And noiseless fortitude. Beholding still	  \\
Resistance strong as heretofore, I thought  \\
That what was in degree the same was likewise  \\
The same in quality, that as the worse  \\
Of the two spirits then at strife remained  \\
Untired, the better surely would preserve	  \\
The heart that first had rouzed him---never dreamt  \\
That transmigration could be undergone,  \\
A fall of being suffered, and of hope,  \\
By creature that appeared to have received  \\
Entire conviction what a great ascent	  \\
Had been accomplished, what high faculties  \\
It had been called to. Youth maintains, I knew,  \\
In all conditions of society  \\
Communion more direct and intimate  \\
With Nature, and the inner strength she has---	  \\
And hence, ofttimes, no less with reason too---  \\
Than age, or manhood even. To Nature then,  \\
Power had reverted: habit, custom, law,  \\
Had left an interregnum's open space  \\
For her to stir about in, uncontrolled.	  \\
The warmest judgments, and the most untaught,  \\
Found in events which every day brought forth  \\
Enough to sanction them---and far, far more  \\
To shake the authority of canons drawn  \\
From ordinary practice. I could see	  \\
How Babel-like the employment was of those  \\
Who, by the recent deluge stupefied,  \\
With their whole souls went culling from the day  \\
Its petty promises to build a tower  \\
For their own safety---laughed at gravest heads,	  \\
Who, watching in their hate of France for signs  \\
Of her disasters, if the stream of rumour  \\
Brought with it one green branch, conceited thence  \\
That not a single tree was left alive  \\
In all her forests. How could I believe	  \\
That wisdom could in any shape come near  \\
Men clinging to delusions so insane?  \\
And thus, experience proving that no few  \\
Of my opinions had been just, I took  \\
Like credit to myself where less was due,	  \\
And thought that other notions were as sound---  \\
Yea, could not but be right---because I saw  \\
That foolish men opposed them.  \\
To a strain  \\
More animated I might here give way,	  \\
And tell, since juvenile errors are my theme,  \\
What in those days through Britain was performed  \\
To turn all judgements out of their right course;  \\
But this is passion over near ourselves,  \\
Reality too close and too intense,	  \\
And mingled up with something, in my mind,  \\
Of scorn and condemnation personal  \\
That would profane the sanctity of verse.  \\
Our shepherds (this say merely) at that time  \\
Thirsted to make the guardian crook of law	  \\
A tool of murder. They who ruled the state,  \\
Though with such awful proof before their eyes  \\
That he who would sow death, reaps death, or worse,  \\
And can reap nothing better, childlike longed  \\
To imitate---not wise enough to avoid.	  \\
Giants in their impiety alone,  \\
But in their weapons and their warfare base  \\
As vermin working out of reach, they leagued  \\
Their strength perfidiously to undermine  \\
Justice, and make an end of liberty.	  \\
But from these bitter truths I must return  \\
To my own history. It hath been told  \\
That I was led to take an eager part  \\
In arguments of civil polity  \\
Abruptly, and indeed before my time:	  \\
I had approached, like other youth, the shield  \\
Of human nature from the golden side,  \\
And would have fought even to the death to attest  \\
The quality of the metal which I saw.  \\
What there is best in individual man,	  \\
Of wise in passion and sublime in power,  \\
What there is strong and pure in household love,  \\
Benevolent in small societies,  \\
And great in large ones also, when called forth  \\
By great occasions---these were things of which	  \\
I something knew; yet even these themselves,  \\
Felt deeply, were not thoroughly understood  \\
By reason. Nay, far from it; they were yet,  \\
As cause was given me afterwards to learn,  \\
Not proof against the injuries of the day---	  \\
Lodged only at the sanctuary's door,  \\
Not safe within its bosom. Thus prepared,  \\
And with such general insight into evil,  \\
And of the bounds which sever it from good,  \\
As books and common intercourse with life	  \\
Must needs have given (to the noviciate mind,  \\
When the world travels in a beaten road,  \\
Guide faithful as is needed), I began  \\
To think with fervour upon management  \\
Of nations---what it is and ought to be,	  \\
And how their worth depended on their laws,  \\
And on the constitution of the state.  \\
O pleasant exercise of hope and joy,  \\
For great were the auxiliars which then stood  \\
Upon our side, we who were strong in love.   \\
Bliss was it in that dawn to be alive,  \\
But to be young was very heaven! O times,  \\
In which the meagre, stale, forbidding ways  \\
Of custom, law, and statute took at once  \\
The attraction of a country in romance---   \\
When Reason seemed the most to assert her rights  \\
When most intent on making of herself  \\
A prime enchanter to assist the work  \\
Which then was going forwards in her name.  \\
Not favored spots alone, but the whole earth,   \\
The beauty wore of promise, that which sets  \\
(To take an image which was felt, no doubt,  \\
Among the bowers of Paradise itself)  \\
The budding rose above the rose full-blown.  \\
What temper at the prospect did not wake   \\
To happiness unthought of? The inert  \\
Were rouzed, and lively natures rapt away.  \\
They who had fed their childhood upon dreams---  \\
The playfellows of fancy, who had made  \\
All powers of swiftness, subtlety, and strength   \\
Their ministers, used to stir in lordly wise  \\
Among the grandest objects of the sense,  \\
And deal with whatsoever they found there  \\
As if they had within some lurking right  \\
To wield it---they too, who, of gentle mood,   \\
Had watched all gentle motions, and to these  \\
Had fitted their own thoughts (schemers more mild,  \\
And in the region of their peaceful selves),  \\
Did now find helpers to their hearts' desire  \\
And stuff at hand plastic as they could wish,   \\
Were called upon to exercise their skill  \\
Not in Utopia---subterraneous fields,  \\
Or some secreted island, heaven knows where---  \\
But in the very world which is the world  \\
Of all of us, the place in which, in the end,	  \\
We find our happiness, or not at all.  \\
Why should I not confess that earth was then  \\
To me what an inheritance new-fallen  \\
Seems, when the first time visited, to one	  \\
Who thither comes to find in it his home?  \\
He walks about and looks upon the place  \\
With cordial transport---moulds it and remoulds---  \\
And is half pleased with things that are amiss,  \\
'Twill be such joy to see them disappear.	  \\
An active partisan, I thus convoked  \\
From every object pleasant circumstance  \\
To suit my ends. I moved among mankind  \\
With genial feelings still predominant,  \\
When erring, erring on the better side,	  \\
And in the kinder spirit---placable,  \\
Indulgent ofttimes to the worst desires,  \\
As, on one side, not uninformed that men  \\
See as it hath been taught them, and that time  \\
Gives rights to error; on the other hand	  \\
That throwing off oppression must be work  \\
As well of licence as of liberty;  \\
And above all (for this was more than all),  \\
Not caring if the wind did now and then  \\
Blow keen upon an eminence that gave	  \\
Prospect so large into futurity---  \\
In brief, a child of Nature, as at first,  \\
Diffusing only those affections wider  \\
That from the cradle had grown up with me,  \\
And losing, in no other way than light	  \\
Is lost in light, the weak in the more strong.  \\
In the main outline, such it might be said,  \\
Was my condition, till with open war  \\
Britain opposed the liberties of France.  \\
This threw me first out of the pale of love,	  \\
Soured and corrupted upwards to the source,  \\
My sentiments; was not, as hitherto,  \\
A swallowing up of lesser things in great,  \\
But change of them into their opposites,  \\
And thus a way was opened for mistakes	  \\
And false conclusions of the intellect,  \\
As gross in their degree, and in their kind  \\
Far, far more dangerous. What had been a pride  \\
Was now a shame, my likings and my loves  \\
Ran in new channels, leaving old ones dry;	  \\
And thus a blow, which in maturer age  \\
Would but have touched the judgement, struck more deep  \\
Into sensations near the heart. Meantime,  \\
As from the first, wild theories were afloat,  \\
Unto the subtleties of which at least,	  \\
I had but lent a careless ear---assured  \\
Of this, that time would soon set all things right,  \\
Prove that the multitude had been oppressed,  \\
And would be so no more. But when events  \\
Brought less encouragement, and unto these	  \\
The immediate proof of principles no more  \\
Could be entrusted---while the events themselves,  \\
Worn out in greatness, and in novelty,  \\
Less occupied the mind, and sentiments  \\
Could through my understanding's natural growth	  \\
No longer justify themselves through faith  \\
Of inward consciousness, and hope that laid  \\
Its hand upon its object---evidence  \\
Safer, of universal application, such  \\
As could not be impeached, was sought elsewhere.	  \\
And now, become oppressors in their turn,  \\
Frenchmen had changed a war of self-defence  \\
For one of conquest, losing sight of all  \\
Which they had struggled for, and mounted up,  \\
Openly in the view of earth and heaven,	  \\
The scale of Liberty. I read her doom,  \\
Vexed inly somewhat, it is true, and sore,  \\
But not dismayed, nor taking to the shame  \\
Of a false prophet. But, rouzed up, I stuck  \\
More Firmly to old tenets, and, to prove	  \\
Their temper, strained them more; and thus, in heat  \\
Of contest, did opinions every day  \\
Grow into consequence, till round my mind  \\
They clung as if they were the life of it.  \\
This was the time when, all things tending fast	  \\
To depravation, the philosophy  \\
That promised to abstract the hopes of man  \\
Out of his feelings, to be fixed thenceforth  \\
For ever in a purer element,  \\
Found ready welcome. Tempting region that	  \\
For zeal to enter and refresh herself,  \\
Where passions had the privilege to work,  \\
And never hear the sound of their own names---  \\
But, speaking more in charity, the dream  \\
Was flattering to the young ingenuous mind	  \\
Pleased with extremes, and not the least with that  \\
Which makes the human reason's naked self  \\
The object of its fervour. What delight!---  \\
How glorious!---in self-knowledge and self-rule  \\
To look through all the frailties of the world,	  \\
And, with a resolute mastery shaking off  \\
The accidents of nature, time, and place,  \\
That make up the weak being of the past,  \\
Build social freedom on its only basis:  \\
The freedom of the individual mind,	  \\
Which, to the blind restraint of general laws  \\
Superior, magisterially adopts  \\
One guide---the light of circumstances, flashed  \\
Upon an independent intellect.  \\
For howsoe'er unsettled, never once	  \\
Had I thought ill of human-kind, or been  \\
Indifferent to its welfare, but, enflamed  \\
With thirst of a secure intelligence,  \\
And sick of other passion, I pursued  \\
A higher nature---wished that man should start	  \\
Out of the worm-like state in which he is,  \\
And spread abroad the wings of Liberty,  \\
Lord of himself, in undisturbed delight.  \\
A noble aspiration!---yet I feel  \\
The aspiration---but with other thoughts	  \\
And happier: for I was perplexed and sought  \\
To accomplish the transition by such means  \\
As did not lie in nature, sacrificed  \\
The exactness of a comprehensive mind  \\
To scrupulous and microscopic views	  \\
That furnished out materials for a work  \\
Of false imagination, placed beyond  \\
The limits of experience and of truth.  \\
Enough, no doubt, the advocates themselves  \\
Of ancient institutions had performed	  \\
To bring disgrace upon their very names;  \\
Disgrace of which custom, and written law,  \\
And sundry moral sentiments, as props  \\
And emanations of these institutes,  \\
Too justly bore a part. A veil had been	  \\
Uplifted. Why deceive ourselves?---'twas so,  \\
'Twas even so---and sorrow for the man  \\
Who either had no eyes wherewith to see,  \\
Or seeing hath forgotten. Let this pass,  \\
Suffice it that a shock had then been given	  \\
To old opinions, and the minds of all men  \\
Had felt it---that my mind was both let loose,  \\
Let loose and goaded. After what hath been  \\
Already said of patriotic love,  \\
And hinted at in other sentiments,	  \\
We need not linger long upon this theme,  \\
This only may be said, that from the first  \\
Having two natures in me (joy the one,  \\
The other melancholy), and withal  \\
A happy man, and therefore bold to look	  \\
On painful things---slow, somewhat, too, and stern  \\
In temperament---I took the knife in hand,  \\
And, stopping not at parts less sensitive,  \\
Endeavoured with my best of skill to probe  \\
The living body of society	  \\
Even to the heart. I pushed without remorse  \\
My speculations forward, yea, set foot  \\
On Nature's holiest places.  \\
Time may come  \\
When some dramatic story may afford	  \\
Shapes livelier to convey to thee, my friend,  \\
What then I learned---or think I learned---of truth,  \\
And the errors into which I was betrayed  \\
By present objects, and by reasonings false  \\
From the beginning, inasmuch as drawn	  \\
Out of a heart which had been turned aside  \\
From Nature by external accidents,  \\
And which was thus confounded more and more,  \\
Misguiding and misguided. Thus I fared,  \\
Dragging all passions, notions, shapes of faith,	  \\
Like culprits of the bar, suspiciously  \\
Calling the mind to establish in plain day  \\
Her titles and her honours, now believing,  \\
Now disbelieving, endlessly perplexed  \\
With impulse, motive, right and wrong, the ground	  \\
Of moral obligation---what the rule,  \\
And what the sanction---till, demanding proof,  \\
And seeking it in every thing, I lost  \\
All feeling of conviction, and, in fine,  \\
Sick, wearied out with contrarieties,	  \\
Yielded up moral questions in despair,  \\
And for my future studies, as the sole  \\
Employment of the inquiring faculty,  \\
Turned towards mathematics, and their clear  \\
And solid evidence.	  \\
Ah, then it was  \\
That thou, most precious friend, about this time  \\
First known to me, didst lend a living help  \\
To regulate my soul. And then it was  \\
That the belove`d woman in whose sight	  \\
Those days were passed---now speaking in a voice  \\
Of sudden admonition like a brook  \\
That does but cross a lonely road; and now  \\
Seen, heard and felt, and caught at every turn,  \\
Companion never lost through many a league---	  \\
Maintained for me a saving intercourse  \\
With my true self (for, though impaired, and changed  \\
Much, as it seemed, I was no further changed  \\
Than as a clouded, not a waning moon);  \\
She, in the midst of all, preserved me still	  \\
A poet, made me seek beneath that name  \\
My office upon earth, and nowhere else.  \\
And lastly, Nature's self, by human love  \\
Assisted, through the weary labyrinth  \\
Conducted me again to open day,	  \\
Revived the feelings of my earlier life,  \\
Gave me that strength and knowledge full of peace,  \\
Enlarged, and never more to be disturbed,  \\
Which through the steps of our degeneracy,  \\
All degradation of this age, hath still	  \\
Upheld me, and upholds me at this day  \\
In the catastrophe (for so they dream,  \\
And nothing less), when, finally to close  \\
And rivet up the gains of France, a Pope  \\
Is summoned in to crown an Emperor---	  \\
This last opprobrium, when we see the dog  \\
Returning to his vomit, when the sun  \\
That rose in splendour, was alive, and moved  \\
In exultation among living clouds,  \\
Hath put his function and his glory off,	  \\
And, turned into a gewgaw, a machine, sets like an opera phantom.  \\
Thus, O friend,  \\
Through times of honour, and through times of shame,  \\
Have I descended, tracing faithfully	  \\
The workings of a youthful mind, beneath  \\
The breath of great events---its hopes no less  \\
Than universal, and its boundless love---  \\
A story destined for thy ear, who now,  \\
Among the basest and the lowest fallen	  \\
Of all the race of men, dost make abode  \\
Where Etna looketh down on Syracuse,  \\
The city of Timoleon. Living God,  \\
How are the mighty prostrated!---they first,  \\
They first of all that breathe, should have awaked	  \\
When the great voice was heard out of the tombs  \\
Of ancient heroes. If for France I have grieved,  \\
Who in the judgement of no few hath been  \\
A trifler only, in her proudest day---  \\
Have been distressed to think of what she once	  \\
Promised, now is---a far more sober cause  \\
Thine eyes must see of sorrow in a land  \\
Strewed with the wreck of loftiest years, a land  \\
Glorious indeed, substantially renowned  \\
Of simple virtue once, and manly praise,	  \\
Now without one memorial hope, not even  \\
A hope to be deferred---for that would serve  \\
To chear the heart in such entire decay.  \\
But indignation works where hope is not,  \\
And thou, O friend, wilt be refreshed. There is	  \\
One great society alone on earth:  \\
The noble living and the noble dead.  \\
Thy consolation shall be there, and time  \\
And Nature shall before thee spread in store  \\
Imperishable thoughts, the place itself	  \\
Be conscious of they presence, and the dull  \\
Sirocco air of its degeneracy  \\
Turn as thou mov'st into a healthful breeze  \\
To cherish and invigorate thy frame.  \\
Thine be those motions strong and sanative,	  \\
A ladder for thy spirit to reascend  \\
To health and joy and pure contentedness:  \\
To me the grief confined that thou art gone  \\
From this last spot of earth where Freedom now  \\
Stands single in her only sanctuary---	  \\
A lonely wanderer art gone, by pain  \\
Compelled and sickness, at this latter day,  \\
This heavy time of change for all mankind.  \\
I feel for thee, must utter what I feel;  \\
The sympathies, erewhile in part discharged,	  \\
Gather afresh, and will have vent again.  \\
My own delights do scarcely seem to me  \\
My own delights: the lordly Alps themselves,  \\
Those rosy peaks from which the morning looks  \\
Abroad on many nations, are not now	  \\
Since thy migration and departure, friend,  \\
The gladsome image in my memory  \\
Which they were used to be. To kindred scenes,  \\
On errand---at a time how different---  \\
Thou tak'st thy way, carrying a heart more ripe	  \\
For all divine enjoyment, with the soul  \\
Which Nature gives to poets, now by thought  \\
Matured, and in the summer of its strength.  \\
Oh, wrap him in your shades, ye giant woods,  \\
On Etna's side, and thou, O flowery vale	  \\
Of Enna, is there not some nook of thine  \\
From the first playtime of the infant earth  \\
Kept sacred to restorative delight?  \\
Child of the mountains, among shepherds reared,  \\
Even from my earliest schoolday time, I loved	  \\
To dream of Sicily; and now a sweet  \\
And gladsome promise wafted from that land  \\
Comes o'er my heart.  \\
There's not a single name  \\
Of note belonging to that honored isle,  \\
Philosopher or bard, Empedocles,	  \\
Or Archimedes---deep and tranquil soul---  \\
That is not like a comfort to my grief.  \\
And, O Theocritus, so far have some  \\
Prevailed among the powers of heaven and earth  \\
By force of graces which were theirs, that they   \\
Have had, as thou reportest, miracles  \\
Wrought for them in old time: yea, not unmoved,  \\
When thinking on my own belove`d friend,  \\
I hear thee tell how bees with honey fed  \\
Divine Comates, by his tyrant lord   \\
Within a chest imprisoned impiously---  \\
How with their honey from the fields they came  \\
And fed him there, alive, from month to month,  \\
Because the goatherd, blesse`d man, had lips  \\
Wet with the Muse's nectar.   \\
Thus I soothe  \\
The pensive moments by this calm fireside,  \\
And find a thousand fancied images  \\
That chear the thoughts of those I love, and mine.  \\
Our prayers have been accepted: thou wilt stand   \\
Not as an exile but a visitant  \\
On Etna's top; by pastoral Arethuse---  \\
Or if that fountain be indeed no more,  \\
Then near some other spring which by the name  \\
Thou gratulatest, willingly deceived---   \\
Shalt linger as a gladsome votary,  \\
And not a captive pining for his home. \\
\end{verse}

%%%%%%%%%%%%%%%%%%%%%%%%%%%%%%%%%%%%%%%%%%%%% \\
\chapter*[Book Eleventh]{Book Eleventh \\ Imagination, How Impaired and Restored}
\addcontentsline{toc}{chapter}{Book Eleventh Imagination, How Impaired and Restored}

\begin{verse}
LONG time hath man's unhappiness and guilt  \\
Detained us: with what dismal sights beset  \\
For the outward view, and inwardly oppressed  \\
With sorrow, disappointment, vexing thoughts,  \\
Confusion of the judgement, zeal decayed---	  \\
And lastly, utter loss of hope itself  \\
And things to hope for. Not with these began  \\
Our song, and not with these our song must end.  \\
Ye motions of delight, that through the fields  \\
Stir gently, breezes and soft airs that breathe	  \\
The breath of paradise, and find your way  \\
To the recesses of the soul; ye brooks  \\
Muttering along the stones, a busy noise  \\
By day, a quiet one in silent night;  \\
And you, ye groves, whose ministry it is	  \\
To interpose the covert of your shades,  \\
Even as a sleep, betwixt the heart of man  \\
And the uneasy world---'twixt man himself,  \\
Not seldom, and his own unquiet heart---  \\
Oh, that I had a music and a voice	  \\
Harmonious as your own, that I might tell  \\
What ye have done for me. The morning shines,  \\
Nor heedeth man's perverseness; spring returns---  \\
I saw the spring return, when I was dead  \\
To deeper hope, yet had I joy for her	  \\
And welcomed her benevolence, rejoiced  \\
In common with the children of her love,  \\
Plants, insects, beasts in field, and birds in bower.  \\
So neither were complacency, nor peace,  \\
Nor tender yearnings, wanting for my good	  \\
Through those distracted times: in  \\
Nature still Glorying, I found a counterpoise to her,  \\
Which, when the spirit of evil was at height,  \\
Maintained for me a secret happiness.  \\
Her I resorted to, and loved so much	  \\
I seemed to love as much as heretofore---  \\
And yet this passion, fervent as it was,  \\
Had suffered change; how could there fail to be  \\
Some change, if merely hence, that years of life  \\
Were going on, and with them loss or gain	  \\
Inevitable, sure alternative?  \\
This history, my friend, hath chiefly told  \\
Of intellectual power from stage to stage  \\
Advancing hand in hand with love and joy,  \\
And of imagination teaching truth	  \\
Until that natural graciousness of mind  \\
Gave way to over-pressure of the times  \\
And their disastrous issues. What availed,  \\
When spells forbade the voyager to land,  \\
The fragrance which did ever and anon	  \\
Give notice of the shore, from arbours breathed  \\
Of blesse`d sentiment and fearless love?  \\
What did such sweet remembrances avail---  \\
Perfidious then, as seemed---what served they then?  \\
My business was upon the barren seas,	  \\
My errand was to sail to other coasts.  \\
Shall I avow that I had hope to see  \\
(I mean that future times would surely see)  \\
The man to come parted as by a gulph  \\
From him who had been?---that I could no more	  \\
Trust the elevation which had made me one  \\
With the great family that here and there  \\
Is scattered through the abyss of ages past,  \\
Sage, patriot, lover, hero; for it seemed  \\
That their best virtues were not free from taint	  \\
Of something false and weak, which could not stand  \\
The open eye of reason. Then I said,  \\
'Go to the poets, they will speak to thee  \\
More perfectly of purer creatures---yet  \\
If reason be nobility in man,	  \\
Can aught be more ignoble than the man  \\
Whom they describe, would fasten if they may  \\
Upon our love by sympathies of truth?'  \\
Thus strangely did I war against myself;  \\
A bigot to a new idolatry,	  \\
Did like a monk who hath forsworn the world  \\
Zealously labour to cut off my heart  \\
From all the sources of her former strength;  \\
And, as by simple waving of a wand,  \\
The wizard instantaneously dissolves	  \\
Palace or grove, even so did I unsoul  \\
As readily by syllogistic words  \\
(Some charm of logic, ever within reach)  \\
Those mysteries of passion which have made,  \\
And shall continue evermore to make---	  \\
In spite of all that reason hath performed,  \\
And shall perform, to exalt and to refine---  \\
One brotherhood of all the human race,  \\
Through all the habitations of past years,  \\
And those to come: and hence an emptiness	  \\
Fell on the historian's page, and even on that  \\
Of poets, pregnant with more absolute truth.  \\
The works of both withered in my esteem,  \\
Their sentence was, I thought, pronounced---their rights  \\
Seemed mortal, and their empire passed away.	  \\
What then remained in such eclipse, what light  \\
To guide or chear? The laws of things which lie  \\
Beyond the reach of human will or power,  \\
The life of Nature, by the God of love  \\
Inspired---celestial presence ever pure---	  \\
These left, the soul of youth must needs be rich  \\
Whatever else be lost; and these were mine,  \\
Not a deaf echo merely of the thought  \\
(Bewildered recollections, solitary),  \\
But living sounds. Yet in despite of this---	  \\
This feeling, which howe'er impaired or damped,  \\
Yet having been once born can never die---  \\
'Tis true that earth with all her appanage  \\
Of elements and organs, storm and sunshine,  \\
With its pure forms and colours, pomp of clouds,	  \\
Rivers, and mountains, objects among which  \\
It might be thought that no dislike or blame,  \\
No sense of weakness or infirmity  \\
Or aught amiss, could possibly have come,  \\
Yea, even the visible universe was scanned	  \\
With something of a kindred spirit, fell  \\
Beneath the domination of a taste Less elevated,  \\
which did in my mind  \\
With its more noble influence interfere,  \\
Its animation and its deeper sway.	  \\
There comes (if need be now to speak of this  \\
After such long detail of our mistakes),  \\
There comes a time when reason---not the grand  \\
And simple reason, but that humbler power  \\
Which carries on its no inglorious work	  \\
By logic and minute analysis---  \\
Is of all idols that which pleases most  \\
The growing mind. A trifler would he be  \\
Who on the obvious benefits should dwell  \\
That rise out of this process; but to speak	  \\
Of all the narrow estimates of things  \\
Which hence originate were a worthy theme  \\
For philosophic verse. Suffice it here  \\
To hint that danger cannot but attend  \\
Upon a function rather proud to be	  \\
The enemy of falsehood, than the friend  \\
Of truth---to sit in judgement than to feel.  \\
Oh soul of Nature, excellent and fair,  \\
That didst rejoice with me, with whom I too  \\
Rejoiced, through early youth, before the winds	  \\
And powerful waters, and in lights and shades  \\
That marched and countermarched about the hills  \\
In glorious apparition, now all eye  \\
And now all ear, but ever with the heart  \\
Employed, and the majestic intellect!	  \\
O soul of Nature, that dost overflow  \\
With passion and with life, what feeble men  \\
Walk on this earth, how feeble have I been  \\
When thou wert in thy strength! Nor this through stroke  \\
Of human suffering, such as justifies	  \\
Remissness and inaptitude of mind,  \\
But through presumption, even in pleasure pleased  \\
Unworthily, disliking here, and there  \\
Liking, by rules of mimic art transferred  \\
To things above all art. But more---for this,	  \\
Although a strong infection of the age,  \\
Was never much my habit---giving way  \\
To a comparison of scene with scene,  \\
Bent overmuch on superficial things,  \\
Pampering myself with meagre novelties	  \\
Of colour and proportion, to the moods  \\
Of nature, and the spirit of the place,  \\
Less sensible. Nor only did the love  \\
Of sitting thus in judgment interrupt  \\
My deeper feelings, but another cause,	  \\
More subtle and less easily explained,  \\
That almost seems inherent in the creature,  \\
Sensuous and intellectual as he is,  \\
A twofold frame of body and of mind:  \\
The state to which I now allude was one	  \\
In which the eye was master of the heart,  \\
When that which is in every stage of life  \\
The most despotic of our senses gained  \\
Such strength in me as often held my mind  \\
In absolute dominion. Gladly here,	  \\
Entering upon abstruser argument, Would  \\
I endeavour to unfold the means Which  \\
Nature studiously employs to thwart  \\
This tyranny, summons all the senses each  \\
To counteract the other and themselves,	  \\
And makes them all, and the objects with which all  \\
Are conversant, subservient in their turn  \\
To the great ends of liberty and power.  \\
But this is matter for another song;  \\
Here only let me add that my delights,	  \\
Such as they were, were sought insatiably.  \\
Though 'twas a transport of the outward sense,  \\
Not of the mind---vivid but not profound---  \\
Yet was I often greedy in the chace,  \\
And roamed from hill to hill, from rock to rock,	  \\
Still craving combinations of new forms,  \\
New pleasure, wider empire for the sight,  \\
Proud of its own endowments, and rejoiced  \\
To lay the inner faculties asleep.  \\
Amid the turns and counter-turns, the strife	  \\
And various trials of our complex being  \\
As we grow up, such thraldom of that sense  \\
Seems hard to shun; and yet I knew a maid,  \\
Who, young as I was then, conversed with things  \\
In higher style. From appetites like these	  \\
She, gentle visitant, as well she might,  \\
Was wholly free. Far less did critic rules  \\
Or barren intermeddling subtleties  \\
Perplex her mind, but, wise as women are  \\
When genial circumstance hath favored them,	  \\
She welcomed what was given, and craved no more.  \\
Whatever scene was present to her eyes,  \\
That was the best, to that she was attuned  \\
Through her humility and lowliness,  \\
And through a perfect happiness of soul	  \\
Whose variegated feelings were in this  \\
Sisters, that they were each some new delight.  \\
For she was Nature's inmate: her the birds  \\
And every flower she met with, could they but  \\
Have known her, would have loved. Methought such charm	  \\
Of sweetness did her presence breathe around  \\
That all the trees, and all the silent hills,  \\
And every thing she looked on, should have had  \\
An intimation how she bore herself  \\
Towards them and to all creatures. God delights	  \\
In such a being, for her common thoughts  \\
Are piety, her life is blessedness.  \\
Even like this maid, before I was called forth  \\
From the retirement of my native hills	  \\
I loved whate'er I saw, nor lightly loved,  \\
But fervently---did never dream of aught  \\
More grand, more fair, more exquisitely framed,  \\
Than those few nooks to which my happy feet  \\
Were limited. I had not at that time	  \\
Lived long enough, nor in the least survived  \\
The first diviner influence of this world  \\
As it appears to unaccustomed eyes.  \\
I worshipped then among the depths of things  \\
As my soul bade me; could I then take part	  \\
In aught but admiration, or be pleased  \\
With any thing but humbleness and love?  \\
I felt, and nothing else; I did not judge,  \\
I never thought of judging, with the gift  \\
Of all this glory filled and satisfied---	  \\
And afterwards, when through the gorgeous Alps  \\
Roaming, I carried with me the same heart.  \\
In truth, this degradation---howsoe'er  \\
Induced, effect in whatsoe'er degree  \\
Of custom that prepares such wantonness	  \\
As makes the greatest things give way to least,  \\
Or any other cause that hath been named,  \\
Or, lastly, aggravated by the times,  \\
Which with their passionate sounds might often make  \\
The milder minstrelsies of rural scenes	  \\
Inaudible---was transient. I had felt  \\
Too forcibly, too early in my life,  \\
Visitings of imaginative power  \\
For this to last: I shook the habit off  \\
Entirely and for ever, and again	  \\
In Nature's presence stood, as I stand now,  \\
A sensitive, and a creative soul.  \\
There are in our existence spots of time,  \\
Which with distinct preeminence retain  \\
A renovating virtue, whence, depressed	  \\
By false opinion and contentious thought,  \\
Or aught of heavier or more deadly weight  \\
In trivial occupations and the round  \\
Of ordinary intercourse, our minds  \\
Are nourished and invisibly repaired---	  \\
A virtue, by which pleasure is enhanced,  \\
That penetrates, enables us to mount  \\
When high, more high, and lifts us up when fallen.  \\
This efficacious spirit chiefly lurks  \\
Among those passages of life in which	  \\
We have had deepest feeling that the mind  \\
Is lord and master, and that outward sense  \\
Is but the obedient servant of her will.  \\
Such moments, worthy of all gratitude,  \\
Are scattered everywhere, taking their date	  \\
From our first childhood---in our childhood even  \\
Perhaps are most conspicuous. Life with me,  \\
As far as memory can look back, is full  \\
Of this beneficent influence.  \\
At a time	  \\
When scarcely (I was then not six years old)  \\
My hand could hold a bridle, with proud hopes  \\
I mounted, and we rode towards the hills:  \\
We were a pair of horsemen---honest James  \\
Was with me, my encourager and guide.	  \\
We had not travelled long ere some mischance  \\
Disjoined me from my comrade, and, through fear  \\
Dismounting, down the rough and stony moor  \\
I led my horse, and stumbling on, at length  \\
Came to a bottom where in former times	  \\
A murderer had been hung in iron chains.  \\
The gibbet-mast was mouldered down, the bones  \\
And iron case was gone, but on the turf  \\
Hard by, soon after that fell deed was wrought,  \\
Some unknown hand had carved the murderer's name.	  \\
The monumental writing was engraven  \\
In times long past, and still from year to year  \\
By superstition of the neighbourhood  \\
The grass is cleared away; and to this hour  \\
The letters are all fresh and visible.	  \\
Faltering, and ignorant where I was, at length  \\
I chanced to espy those characters inscribed  \\
On the green sod: forthwith I left the spot,  \\
And, reascending the bare common, saw  \\
A naked pool that lay beneath the hills,	  \\
The beacon on the summit, and more near,  \\
A girl who bore a pitcher on her head  \\
And seemed with difficult steps to force her way  \\
Against the blowing wind. It was, in truth,  \\
An ordinary sight, but I should need	  \\
Colours and words that are unknown to man  \\
To paint the visionary dreariness  \\
Which, while I looked all round for my lost guide,  \\
Did at that time invest the naked pool,  \\
The beacon on the lonely eminence,	  \\
The woman, and her garments vexed and tossed  \\
By the strong wind. When, in blesse`d season,  \\
With those two dear ones---to my heart so dear---  \\
When, in the blesse`d time of early love,  \\
Long afterwards I roamed about	  \\
In daily presence of this very scene,  \\
Upon the naked pool and dreary crags,  \\
And on the melancholy beacon, fell  \\
The spirit of pleasure and youth's golden gleam---  \\
And think ye not with radiance more divine	  \\
From these remembrances, and from the power  \\
They left behind? So feeling comes in aid  \\
Of feeling, and diversity of strength  \\
Attends us, if but once we have been strong.  \\
Oh mystery of man, from what a depth	  \\
Proceed thy honours! I am lost, but see  \\
In simple childhood something of the base  \\
On which thy greatness stands---but this I feel,  \\
That from thyself it is that thou must give,  \\
Else never canst receive. The days gone by	  \\
Come back upon me from the dawn almost  \\
Of life; the hiding-places of my power  \\
Seem open, I approach, and then they close;  \\
I see by glimpses now, when age comes on  \\
May scarcely see at all; and I would give	  \\
While yet we may, as far as words can give,  \\
A substance and a life to what I feel:  \\
I would enshrine the spirit of the past  \\
For future restoration. Yet another  \\
Of these to me affecting incidents,	  \\
With which we will conclude.  \\
One Christmas-time, The day before the holidays began,  \\
Feverish, and tired, and restless, I went forth  \\
Into the fields, impatient for the sight	  \\
Of those two horses which should bear us home,  \\
My brothers and myself. There was a crag,  \\
An eminence, which from the meeting-point  \\
Of two highways ascending overlooked  \\
At least a long half-mile of those two roads,	  \\
By each of which the expected steeds might come---  \\
The choice uncertain. Thither I repaired  \\
Up to the highest summit. 'Twas a day  \\
Stormy, and rough, and wild, and on the grass  \\
I sate half sheltered by a naked wall.	  \\
Upon my right hand was a single sheep,  \\
A whistling hawthorn on my left, and there,  \\
With those companions at my side, I watched, S  \\
training my eyes intensely as the mist  \\
Gave intermitting prospect of the wood	  \\
And plain beneath. Ere I to school returned  \\
That dreary time, ere I had been ten days  \\
A dweller in my father's house, he died,  \\
And I and my two brothers, orphans then,  \\
Followed his body to the grave. The event,	  \\
With all the sorrow which it brought, appeared  \\
A chastisement; and when I called to mind  \\
That day so lately past, when from the crag  \\
I looked in such anxiety of hope,  \\
With trite reflections of morality,	  \\
Yet in the deepest passion, I bowed low  \\
To God who thus corrected my desires.  \\
And afterwards the wind and sleety rain,  \\
And all the business of the elements,  \\
The single sheep, and the one blasted tree,	  \\
And the bleak music of that old stone wall,  \\
The noise of wood and water, and the mist  \\
Which on the line of each of those two roads  \\
Advanced in such indisputable shapes---  \\
All these were spectacles and sounds to which	  \\
I often would repair, and thence would drink  \\
As at a fountain. And I do not doubt  \\
That in this later time, when storm and rain  \\
Beat on my roof at midnight, or by day  \\
When I am in the woods, unknown to me	  \\
The workings of my spirit thence are brought.  \\
Thou wilt not languish here, O friend, for whom  \\
I travel in these dim uncertain ways---  \\
Thou wilt assist me, as a pilgrim gone  \\
In quest of highest truth. Behold me then	  \\
Once more in Nature's presence, thus restored,  \\
Or otherwise, and strengthened once again  \\
(With memory left of what had been escaped)  \\
To habits of devoutest sympathy. \\
\end{verse}

%%%%%%%%%%%%%%%%%%%%%%%%%%%%%%%%%%%%%%%%%%%%% \\
\chapter*[Book Twelfth]{Book Twelfth \\ Same Subject (Continued)}
\addcontentsline{toc}{chapter}{Book Twelfth Same Subject (Continued)}

\begin{verse}
FROM Nature doth emotion come, and moods  \\
Of calmness equally are Nature's gift:  \\
This is her glory---these two attributes  \\
Are sister horns that constitute her strength;  \\
This twofold influence is the sun and shower	  \\
Of all her bounties, both in origin  \\
And end alike benignant. Hence it is  \\
That genius, which exists by interchange  \\
Of peace and excitation, finds in her  \\
His best and purest friend---from her receives	  \\
That energy by which he seeks the truth,  \\
Is rouzed, aspires, grasps, struggles, wishes, craves  \\
From her that happy stillness of the mind  \\
Which fits him to receive it when unsought.  \\
Such benefit may souls of humblest frame	  \\
Partake of, each in their degree; 'tis mine  \\
To speak of what myself have known and felt---  \\
Sweet task, for words find easy way, inspired  \\
By gratitude and confidence in truth.  \\
Long time in search of knowledge desperate,	  \\
I was benighted heart and mind, but now  \\
On all sides day began to reappear,  \\
And it was proved indeed that not in vain  \\
I had been taught to reverence a power  \\
That is the very quality and shape	  \\
And image of right reason, that matures  \\
Her processes by steady laws, gives birth  \\
To no impatient or fallacious hopes,  \\
No heat of passion or excessive zeal,  \\
No vain conceits, provokes to no quick turns	  \\
Of self-applauding intellect, but lifts  \\
The being into magnanimity,  \\
Holds up before the mind, intoxicate  \\
With present objects and the busy dance  \\
Of things that pass away, a temperate shew	  \\
Of objects that endure---and by this course  \\
Disposes her, when over-fondly set  \\
On leaving her incumbrances behind,  \\
To seek in man, and in the frame of life  \\
Social and individual, what there is	  \\
Desirable, affecting, good or fair,  \\
Of kindred permanence, the gifts divine  \\
And universal, the pervading grace  \\
That hath been, is, and shall be. Above all  \\
Did Nature bring again this wiser mood,	  \\
More deeply reestablished in my soul,  \\
Which, seeing little worthy or sublime  \\
In what we blazon with the pompous names  \\
Of power and action, early tutored me  \\
To look with feelings of fraternal love	  \\
Upon those unassuming things that hold  \\
A silent station in this beauteous world.  \\
Thus moderated, thus composed, I found  \\
Once more in man an object of delight,  \\
Of pure imagination, and of love;	  \\
And, as the horizon of my mind enlarged,  \\
Again I took the intellectual eye  \\
For my instructor, studious more to see  \\
Great truths, than touch and handle little ones.  \\
Knowledge was given accordingly: my trust	  \\
Was firmer in the feelings which had stood  \\
The test of such a trial, clearer far  \\
My sense of what was excellent and right,  \\
The promise of the present time retired  \\
Into its true proportion; sanguine schemes,	  \\
Ambitious virtues, pleased me less; I sought  \\
For good in the familiar face of life,  \\
And built thereon my hopes of good to come.  \\
With settling judgements now of what would last,  \\
And what would disappear; prepared to find	  \\
Ambition, folly, madness, in the men  \\
Who thrust themselves upon this passive world  \\
As rulers of the world---to see in these  \\
Even when the public welfare is their aim  \\
Plans without thought, or bottomed on false thought	  \\
And false philosophy; having brought to test  \\
Of solid life and true result the books  \\
Of modern statists, and thereby perceived  \\
The utter hollowness of what we name  \\
The wealth of nations, where alone that wealth	  \\
Is lodged, and how encreased; and having gained  \\
A more judicious knowledge of what makes  \\
The dignity of individual man---  \\
Of man, no composition of the thought,  \\
Abstraction, shadow, image, but the man	  \\
Of whom we read, the man whom we behold  \\
With our own eyes---I could not but inquire,  \\
Not with less interest than heretofore,  \\
But greater, though in spirit more subdued,  \\
Why is this glorious creature to be found	  \\
One only in ten thousand? What one is,  \\
Why may not many be? What bars are thrown  \\
By Nature in the way of such a hope?  \\
Our animal wants and the necessities  \\
Which they impose, are these the obstacles?---	  \\
If not, then others vanish into air.  \\
Such meditations bred an anxious wish  \\
To ascertain how much of real worth,  \\
And genuine knowledge, and true power of mind,  \\
Did at this day exist in those who lived	  \\
By bodily labour, labour far exceeding  \\
Their due proportion, under all the weight  \\
Of that injustice which upon ourselves  \\
By composition of society  \\
Ourselves entail. To frame such estimate	  \\
I chiefly looked (what need to look beyond?)  \\
Among the natural abodes of men,  \\
Fields with their rural works---recalled to mind  \\
My earliest notices, with these compared  \\
The observations of my later youth	  \\
Continued downwards to that very day.  \\
For time had never been in which the throes  \\
And mighty hopes of nations, and the stir  \\
And tumult of the world, to me could yield---  \\
How far soe'er transported and possessed---	  \\
Full measure of content, but still I craved  \\
An intermixture of distinct regards  \\
And truths of individual sympathy  \\
Nearer ourselves. Such often might be gleaned  \\
From that great city---else it must have been	  \\
A heart-depressing wilderness indeed,  \\
Full soon to me a wearisome abode---  \\
But much was wanting; therefore did I turn  \\
To you, ye pathways and ye lonely roads,  \\
Sought you enriched with every thing I prized,	  \\
With human kindness and with Nature's joy.  \\
Oh, next to one dear state of bliss, vouchsafed  \\
Alas to few in this untoward world,  \\
The bliss of walking daily in life's prime  \\
Through field or forest with the maid we love	  \\
While yet our hearts are young, while yet we breathe  \\
Nothing but happiness, living in some place,  \\
Deep vale, or anywhere the home of both,  \\
From which it would be misery to stir---  \\
Oh, next to such enjoyment of our youth,	  \\
In my esteem next to such dear delight,  \\
Was that of wandering on from day to day  \\
Where I could meditate in peace, and find  \\
The knowledge which I love, and teach the sound  \\
Of poet's music to strange fields and groves,	  \\
Converse with men, where if we meet a face  \\
We almost meet a friend, on naked moors  \\
With long, long ways before, by cottage bench,  \\
Or well-spring where the weary traveller rests.  \\
I love a public road: few sights there are	  \\
That please me more---such object hath had power  \\
O'er my imagination since the dawn  \\
Of childhood, when its disappearing line  \\
Seen daily afar off, on one bare steep  \\
Beyond the limits which my feet had trod,	  \\
Was like a guide into eternity,  \\
At least to things unknown and without bound.  \\
Even something of the grandeur which invests  \\
The mariner who sails the roaring sea  \\
Through storm and darkness, early in my mind	  \\
Surrounded too the wanderers of the earth---  \\
Grandeur as much, and loveliness far more.  \\
Awed have I been by strolling bedlamites;  \\
From many other uncouth vagrants, passed  \\
In fear, have walked with quicker step---but why	  \\
Take note of this? When I began to inquire,  \\
To watch and question those I met, and held  \\
Familiar talk with them, the lonely roads  \\
Were schools to me in which I daily read  \\
With most delight the passions of mankind,	  \\
There saw into the depth of human souls---  \\
Souls that appear to have no depth at all  \\
To vulgar eyes. And now, convinced at heart  \\
How little that to which alone we give  \\
The name of education hath to do	  \\
With real feeling and just sense, how vain  \\
A correspondence with the talking world  \\
Proves to the most---and called to make good search  \\
If man's estate, by doom of Nature yoked  \\
With toil, is therefore yoked with ignorance,	  \\
If virtue be indeed so hard to rear,  \\
And intellectual strength so rare a boon---  \\
I prized such walks still more; for there I found  \\
Hope to my hope, and to my pleasure peace  \\
And steadiness, and healing and repose	  \\
To every angry passion. There I heard,  \\
From mouths of lowly men and of obscure,  \\
A tale of honour---sounds in unison  \\
With loftiest promises of good and fair.  \\
There are who think that strong affections, love	  \\
Known by whatever name, is falsely deemed  \\
A gift (to use a term which they would use)  \\
Of vulgar Nature---that its growth requires  \\
Retirement, leisure, language purified  \\
By manners thoughtful and elaborate---	  \\
That whoso feels such passion in excess  \\
Must live within the very light and air  \\
Of elegances that are made by man.  \\
True it is, where oppression worse than death  \\
Salutes the being at his birth, where grace	  \\
Of culture hath been utterly unknown,  \\
And labour in excess and poverty  \\
From day to day pre-occupy the ground  \\
Of the affections, and to Nature's self  \\
Oppose a deeper nature---there indeed	  \\
Love cannot be; nor does it easily thrive  \\
In cities, where the human heart is sick,  \\
And the eye feeds it not, and cannot feed:  \\
Thus far, no further, is that inference good.  \\
Yes, in those wanderings deeply did I feel	  \\
How we mislead each other, above all  \\
How books mislead us---looking for their fame  \\
To judgements of the wealthy few, who see  \\
By artificial lights---how they debase  \\
The many for the pleasure of those few,	  \\
Effeminately level down the truth  \\
To certain general notions for the sake  \\
Of being understood at once, or else  \\
Through want of better knowledge in the men  \\
Who frame them, flattering thus our self-conceit	  \\
With pictures that ambitiously set forth  \\
The differences, the outside marks by which  \\
Society has parted man from man,  \\
Neglectful of the universal heart.  \\
Here calling up to mind what then I saw	  \\
A youthful traveller, and see daily now  \\
Before me in my rural neighbourhood---  \\
Here might I pause, and bend in reverence  \\
To Nature, and the power of human minds,  \\
To men as they are men within themselves.	  \\
How oft high service is performed within  \\
When all the external man is rude in shew,  \\
Not like a temple rich with pomp and gold,  \\
But a mere mountain-chapel such as shields  \\
Its simple worshippers from sun and shower.	  \\
'Of these,' said I, 'shall be my song. Of these,  \\
If future years mature me for the task,  \\
Will I record the praises, making verse  \\
Deal boldly with substantial things---in truth  \\
And sanctity of passion speak of these,	  \\
That justice may be done, obeisance paid  \\
Where it is due. Thus haply shall I teach,  \\
Inspire, through unadulterated ears  \\
Pour rapture, tenderness, and hope, my theme  \\
No other than the very heart of man	  \\
As found among the best of those who live  \\
Not unexalted by religious faith,  \\
Not uninformed by books (good books, though few), In  \\
Nature's presence---thence may I select  \\
Sorrow that is not sorrow but delight,	  \\
And miserable love that is not pain  \\
To hear of, for the glory that redounds  \\
Therefrom to human-kind and what we are.  \\
Be mine to follow with no timid step  \\
Where knowledge leads me: it shall be my pride	  \\
That I have dared to tread this holy ground,  \\
Speaking no dream but things oracular,  \\
Matter not lightly to be heard by those  \\
Who to the letter of the outward promise  \\
Do read the invisible soul, by men adroit	  \\
In speech and for communion with the world  \\
Accomplished, minds whose faculties are then  \\
Most active when they are most eloquent,  \\
And elevated most when most admired.  \\
Men may be found of other mold than these,	  \\
Who are their own upholders, to themselves  \\
Encouragement, and energy, and will,  \\
Expressing liveliest thoughts in lively words  \\
As native passion dictates. Others, too,  \\
There are among the walks of homely life	  \\
Still higher, men for contemplation framed,  \\
Shy, and unpractised in the strife of phrase,  \\
Meek men, whose very souls perhaps would sink  \\
Beneath them, summoned to such intercourse:  \\
Theirs is the language of the heavens, the power,	  \\
The thought, the image, and the silent joy;  \\
Words are but under-agents in their souls---  \\
When they are grasping with their greatest strength  \\
They do not breathe among them. This I speak  \\
In gratitude to God, who feeds our hearts	  \\
For his own service, knoweth, loveth us,  \\
When we are unregarded by the world.'  \\
Also about this time did I receive  \\
Convictions still more strong than heretofore  \\
Not only that the inner frame is good,	  \\
And graciously composed, but that, no less,  \\
Nature through all conditions hath a power  \\
To consecrate---if we have eyes to see---  \\
The outside of her creatures, and to breathe  \\
Grandeur upon the very humblest face	  \\
Of human life. I felt that the array  \\
Of outward circumstance and visible form  \\
Is to the pleasure of the human mind  \\
What passion makes it; that meanwhile the forms  \\
Of Nature have a passion in themselves	  \\
That intermingles with those works of man  \\
To which she summons him, although the works  \\
Be mean, having nothing lofty of their own;  \\
And that the genius of the poet hence  \\
May boldly take his way among mankind	  \\
Wherever Nature leads---that he hath stood By  \\
Nature's side among the men of old,  \\
And so shall stand for ever. Dearest friend,  \\
Forgive me if I say that I, who long  \\
Had harboured reverentially a thought	  \\
That poets, even as prophets, each with each  \\
Connected in a mighty scheme of truth,  \\
Have each for his peculiar dower a sense  \\
By which he is enabled to perceive  \\
Something unseen before---forgive me, friend,	  \\
If I, the meanest of this band, had hope  \\
That unto me had also been vouchsafed  \\
An influx, that in some sort I possessed  \\
A privilege, and that a work of mine,  \\
Proceeding from the depth of untaught things,	  \\
Enduring and creative, might become  \\
A power like one of Nature's.  \\
To such a mood,  \\
Once above all---a traveller at that time  \\
Upon the plain of Sarum---was I raised:	  \\
There on the pastoral downs without a track  \\
To guide me, or along the bare white roads  \\
Lengthening in solitude their dreary line,  \\
While through those vestiges of ancient times  \\
I ranged, and by the solitude o'ercome,	  \\
I had a reverie and saw the past,  \\
Saw multitudes of men, and here and there  \\
A single Briton in his wolf-skin vest,  \\
With shield and stone-ax, stride across the wold;  \\
The voice of spears was heard, the rattling spear	  \\
Shaken by arms of mighty bone, in strength  \\
Long mouldered, of barbaric majesty.  \\
I called upon the darkness, and it took---  \\
A midnight darkness seemed to come and take---  \\
All objects from my sight; and lo, again	  \\
The desart visible by dismal flames!  \\
It is the sacrificial altar, fed  \\
With living men---how deep the groans!---the voice  \\
Of those in the gigantic wicker thrills  \\
Throughout the region far and near, pervades	  \\
The monumental hillocks, and the pomp  \\
Is for both worlds, the living and the dead.  \\
At other moments, for through that wide waste  \\
Three summer days I roamed, when 'twas my chance  \\
To have before me on the downy plain	  \\
Lines, circles, mounts, a mystery of shapes  \\
Such as in many quarters yet survive,  \\
With intricate profusion figuring o'er  \\
The untilled ground (the work, as some divine,  \\
Of infant science, imitative forms	  \\
By which the Druids covertly expressed  \\
Their knowledge of the heavens, and imaged forth  \\
The constellations), I was gently charmed,  \\
Albeit with an antiquarian's dream,  \\
And saw the bearded teachers, with white wands	  \\
Uplifted, pointing to the starry sky,  \\
Alternately, and plain below, while breath  \\
Of music seemed to guide them, and the waste  \\
Was cleared with stillness and a pleasant sound.  \\
This for the past, and things that may be viewed,	  \\
Or fancied, in the obscurities of time.  \\
Nor is it, friend, unknown to thee; at least---  \\
Thyself delighted---thou for my delight  \\
Hast said, perusing some imperfect verse  \\
Which in that lonesome journey was composed,	  \\
That also I must then have exercised  \\
Upon the vulgar forms of present things  \\
And actual world of our familiar days,  \\
A higher power---have caught from them a tone,  \\
An image, and a character, by books	  \\
Not hitherto reflected. Call we this  \\
But a persuasion taken up by thee  \\
In friendship, yet the mind is to herself  \\
Witness and judge, and I remember well  \\
That in life's everyday appearances	  \\
I seemed about this period to have sight  \\
Of a new world---a world, too, that was fit  \\
To be transmitted and made visible  \\
To other eyes, as having for its base  \\
That whence our dignity originates,	  \\
That which both gives it being, and maintains  \\
A balance, an ennobling interchange  \\
Of action from within and from without:  \\
The excellence, pure spirit, and best power,  \\
Both of the object seen, and eye that sees.	 \\
\end{verse}

%%%%%%%%%%%%%%%%%%%%%%%%%%%%%%%%%%%%%%%%%%%%%%%%% \\
\chapter*[Book Thirteenth]{Book Thirteenth \\ Conclusion}
\addcontentsline{toc}{chapter}{Book Thirteenth Conclusion}

\begin{verse}
IN one of these excursions, travelling then  \\
Through Wales on foot and with a youthful friend,  \\
I left Bethkelet's huts at couching-time,  \\
And westward took my way to see the sun  \\
Rise from the top of Snowdon. Having reached	  \\
The cottage at the mountain's foot,  \\
we there Rouzed up the shepherd who by ancient right  \\
Of office is the stranger's usual guide,  \\
And after short refreshment sallied forth.  \\
It was a summer's night, a close warm night,	  \\
Wan, dull, and glaring, with a dripping mist  \\
Low-hung and thick that covered all the sky,  \\
Half threatening storm and rain; but on we went  \\
Unchecked, being full of heart and having faith  \\
In our tried pilot. Little could we see,	  \\
Hemmed round on every side with fog and damp,  \\
And, after ordinary travellers' chat  \\
With our conductor, silently we sunk  \\
Each into commerce with his private thoughts.  \\
Thus did we breast the ascent, and by myself	  \\
Was nothing either seen or heard the while  \\
Which took me from my musings, save that once  \\
The shepherd's cur did to his own great joy  \\
Unearth a hedgehog in the mountain-crags,  \\
Round which he made a barking turbulent.	  \\
This small adventure---for even such it seemed  \\
In that wild place and at the dead of night---  \\
Being over and forgotten, on we wound  \\
In silence as before. With forehead bent  \\
Earthward, as if in opposition set	  \\
Against an enemy, I panted up  \\
With eager pace, and no less eager thoughts,  \\
Thus might we wear perhaps an hour away,  \\
Ascending at loose distance each from each,  \\
And I, as chanced, the foremost of the band---	  \\
When at my feet the ground appeared to brighten,  \\
And with a step or two seemed brighter still;  \\
Nor had I time to ask the cause of this,  \\
For instantly a light upon the turf  \\
Fell like a flash. I looked about, and lo,	  \\
The moon stood naked in the heavens at height  \\
Immense above my head, and on the shore  \\
I found myself of a huge sea of mist,  \\
Which meek and silent rested at my feet.  \\
A hundred hills their dusky backs upheaved	  \\
All over this still ocean, and beyond,  \\
Far, far beyond, the vapours shot themselves  \\
In headlands, tongues, and promontory shapes,  \\
Into the sea, the real sea, that seemed  \\
To dwindle and give up its majesty,	  \\
Usurped upon as far as sight could reach.  \\
Meanwhile, the moon looked down upon this shew  \\
In single glory, and we stood, the mist  \\
Touching our very feet; and from the shore  \\
At distance not the third part of a mile	  \\
Was a blue chasm, a fracture in the vapour,  \\
A deep and gloomy breathing-place, through which  \\
Mounted the roar of waters, torrents, steams  \\
Innumerable, roaring with one voice.  \\
The universal spectacle throughout	  \\
Was shaped for admiration and delight,  \\
Grand in itself alone, but in that breach  \\
Through which the homeless voice of waters rose,  \\
That dark deep thoroughfare, had Nature lodged  \\
The soul, the imagination of the whole.	  \\
A meditation rose in me that night  \\
Upon the lonely mountain when the scene  \\
Had passed away, and it appeared to me  \\
The perfect image of a mighty mind,  \\
Of one that feeds upon infinity,	  \\
That is exalted by an under-presence,  \\
The sense of God, or whatsoe'er is dim  \\
Or vast in its own being---above all,  \\
One function of such mind had Nature there  \\
Exhibited by putting forth, and that	  \\
With circumstance most awful and sublime:  \\
That domination which she oftentimes  \\
Exerts upon the outward face of things,  \\
So moulds them, and endues, abstracts, combines,  \\
Or by abrupt and unhabitual influence	  \\
Doth make one object so impress itself  \\
Upon all others, and pervades them so,  \\
That even the grossest minds must see and hear,  \\
And cannot chuse but feel. The power which these  \\
Acknowledge when thus moved, which Nature thus	  \\
Thrusts forth upon the senses, is the express  \\
Resemblance---in the fullness of its strength  \\
Made visible---a genuine counterpart  \\
And brother of the glorious faculty  \\
Which higher minds bear with them as their own.	  \\
This is the very spirit in which they deal  \\
With all the objects of the universe:  \\
They from their native selves can send abroad  \\
Like transformation, for themselves create  \\
A like existence, and, when'er it is	  \\
Created for them, catch it by an instinct.  \\
Them the enduring and the transient both  \\
Serve to exalt.  \\
They build up greatest things  \\
From least suggestions, ever on the watch,  \\
Willing to work and to be wrought upon.	  \\
They need not extraordinary calls  \\
To rouze them---in a world of life they live,  \\
By sensible impressions not enthralled,  \\
But quickened, rouzed, and made thereby more fit  \\
To hold communion with the invisible world.	  \\
Such minds are truly from the Deity,  \\
For they are powers; and hence the highest bliss  \\
That can be known is theirs---the consciousness  \\
Of whom they are, habitually infused  \\
Through every image, and through every thought,	  \\
And all impressions; hence religion, faith,  \\
And endless occupation for the soul,  \\
Whether discursive or intuitive;  \\
Hence sovereignty within and peace at will,  \\
Emotion which best foresight need not fear,	  \\
Most worthy then of trust when most intense;  \\
Hence chearfulness in every act of life;  \\
Hence truth in moral judgements; and delight  \\
That fails not, in the external universe.  \\
Oh, who is he that hath his whole life long	  \\
Preserved, enlarged, this freedom in himself?---  \\
For this alone is genuine liberty,  \\
Witness, ye solitudes, where I received  \\
My earliest visitations (careless then  \\
Of what was given me), and where now I roam,	  \\
A meditative, oft a suffering man,  \\
And yet I trust with undiminished powers;  \\
Witness---whatever falls my better mind,  \\
Revolving with the accidents of life,  \\
May have sustained---that, howsoe'er misled,	  \\
I never in the quest of right and wrong  \\
Did tamper with myself from private aims;  \\
Nor was in any of my hopes the dupe  \\
Of selfish passions; nor did wilfully  \\
Yield ever to mean cares and low pursuits;	  \\
But rather did with jealousy shrink back  \\
From every combination that might aid  \\
The tendency, too potent in itself,  \\
Of habit to enslave the mind---I mean  \\
Oppress it by the laws of vulgar sense,	  \\
And substitute a universe of death,  \\
The falsest of all worlds, in place of that  \\
Which is divine and true. To fear and love  \\
(To love as first and chief, for there fear ends)  \\
Be this ascribed, to early intercourse	  \\
In presence of sublime and lovely forms  \\
With the adverse principles of pain and joy---  \\
Evil as one is rashly named by those  \\
Who know not what they say. From love, for here  \\
Do we begin and end, all grandeur comes,	  \\
All truth and beauty---from pervading love---  \\
That gone, we are as dust. Behold the fields  \\
In balmy springtime, full of rising flowers  \\
And happy creatures; see that pair, the lamb  \\
And the lamb's mother, and their tender ways	  \\
Shall touch thee to the heart; in some green bower  \\
Rest, and be not alone, but have thou there  \\
The one who is thy choice of all the world---  \\
There linger, lulled, and lost, and rapt away---  \\
Be happy to thy fill; thou call'st this love,	  \\
And so it is, but there is higher love  \\
Than this, a love that comes into the heart  \\
With awe and a diffusive sentiment.  \\
Thy love is human merely: this proceeds  \\
More from the brooding soul, and is divine.	  \\
This love more intellectual cannot be  \\
Without imagination, which in truth  \\
Is but another name for absolute strength  \\
And clearest insight, amplitude of mind,  \\
And reason in her most exalted mood.	  \\
This faculty hath been the moving soul  \\
Of our long labour: we have traced the stream  \\
From darkness, and the very place of birth  \\
In its blind cavern, whence is faintly heard  \\
The sound of waters; followed it to light	  \\
And open day, accompanied its course  \\
Among the ways of Nature, afterwards  \\
Lost sight of it bewildered and engulphed,  \\
Then given it greeting as it rose once more  \\
With strength, reflecting in its solemn breast	  \\
The works of man, and face of human life;  \\
And lastly, from its progress have we drawn  \\
The feeling of life endless, the one thought  \\
By which we live, infinity and God.  \\
Imagination having been our theme,	  \\
So also hath that intellectual love,  \\
For they are each in each, and cannot stand  \\
Dividually. Here must thou be, O man,  \\
Strength to thyself---no helper hast thou here---  \\
Here keepest thou thy individual state:	  \\
No other can divide with thee this work,  \\
No secondary hand can intervene  \\
To fashion this ability. 'Tis thine,  \\
The prime and vital principle is thine  \\
In the recesses of thy nature, far	  \\
From any reach of outward fellowship,  \\
Else 'tis not thine at all. But joy to him,  \\
O, joy to him who here hath sown---hath laid  \\
Here the foundations of his future years---  \\
For all that friendship, all that love can do,	  \\
All that a darling countenance can look  \\
Or dear voice utter, to complete the man,  \\
Perfect him, made imperfect in himself,  \\
All shall be his. And he whose soul hath risen  \\
Up to the height of feeling intellect	  \\
Shall want no humbler tenderness, his heart  \\
Be tender as a nursing mother's heart;  \\
Of female softness shall his life be full,  \\
Of little loves and delicate desires,  \\
Mild interests and gentlest sympathies.	  \\
Child of my parents, sister of my soul,  \\
Elsewhere have strains of gratitude been breathed  \\
To thee for all the early tenderness  \\
Which I from thee imbibed. And true it is  \\
That later seasons owned to thee no less;	  \\
For, spite of thy sweet influence and the touch  \\
Of other kindred hands that opened out  \\
The springs of tender thought in infancy,  \\
And spite of all which singly I had watched  \\
Of elegance, and each minuter charm	  \\
In Nature or in life, still to the last---  \\
Even to the very going-out of youth,  \\
The period which our story now hath reached---  \\
I too exclusively esteemed that love,  \\
And sought that beauty, which as Milton sings	  \\
Hath terror in it. Thou didst soften down  \\
This over-sternness; but for thee, sweet friend,  \\
My soul, too reckless of mild grace, had been  \\
Far longer what by Nature it was framed---  \\
Longer retained its countenance severe---	  \\
A rock with torrents roaring, with the clouds  \\
Familiar, and a favorite of the stars;  \\
But thou didst plant its crevices with flowers,  \\
Hang it with shrubs that twinkle in the breeze,  \\
And teach the little birds to build their nests	  \\
And warble in its chambers. At a time When  \\
Nature, destined to remain so long  \\
Foremost in my affections, had fallen back  \\
Into a second place, well pleased to be  \\
A handmaid to a nobler than herself---	  \\
When every day brought with it some new sense  \\
Of exquisite regard for common things,  \\
And all the earth was budding with these gifts  \\
Of more refined humanity---thy breath,  \\
Dear sister, was a kind of gentler spring	  \\
That went before my steps.  \\
With such a theme  \\
Coleridge---with this my argument---of thee  \\
Shall I be silent? O most loving soul,  \\
Placed on this earth to love and understand,	  \\
And from thy presence shed the light of love,  \\
Shall I be mute ere thou be spoken of?  \\
Thy gentle spirit to my heart of hearts  \\
Did also find its way; and thus the life  \\
Of all things and the mighty unity	  \\
In all which we behold, and feel, and are,  \\
Admitted more habitually a mild  \\
Interposition, closelier gathering thoughts  \\
Of man and his concerns, such as become  \\
A human creature, be he who he may,	  \\
Poet, or destined to an humbler name;  \\
And so the deep enthusiastic joy,  \\
The rapture of the hallelujah sent  \\
From all that breathes and is, was chastened, stemmed,  \\
And balanced, by a reason which indeed	  \\
Is reason, duty, and pathetic truth---  \\
And God and man divided, as they ought,  \\
Between them the great system of the world,  \\
Where man is sphered, and which God animates.  \\
And now, O friend, this history is brought	  \\
To its appointed close: the discipline  \\
And consummation of the poet's mind  \\
In every thing that stood most prominent  \\
Have faithfully been pictured. We have reached  \\
The time, which was our object from the first,	  \\
When we may (not presumptuously, I hope)  \\
Suppose my powers so far confirmed, and such  \\
My knowledge, as to make me capable  \\
Of building up a work that should endure.  \\
Yet much hath been omitted, as need was---	  \\
Of books how much! and even of the other wealth  \\
Which is collected among woods and fields,  \\
Far more. For Nature's secondary grace,  \\
That outward illustration which is hers,  \\
Hath hitherto been barely touched upon:	  \\
The charm more superficial, and yet sweet,  \\
Which from her works finds way, contemplated  \\
As they hold forth a genuine counterpart  \\
And softening mirror of the moral world.  \\
Yes, having tracked the main essential power---	  \\
Imagination---up her way sublime,  \\
In turn might fancy also be pursued  \\
Through all her transmigrations, till she too  \\
Was purified, had learned to ply her craft  \\
By judgement steadied. Then might we return,	  \\
And in the rivers and the groves behold  \\
Another face, might hear them from all sides  \\
Calling upon the more instructed mind  \\
To link their images---with subtle skill  \\
Sometimes, and by elaborate research---	  \\
With forms and definite appearances  \\
Of human life, presenting them sometimes  \\
To the involuntary sympathy  \\
Of our internal being, satisfied  \\
And soothed with a conception of delight	  \\
Where meditation cannot come, which thought  \\
Could never heighten. Above all, how much  \\
Still nearer to ourselves is overlooked  \\
In human nature and that marvellous world  \\
As studied first in my own heart, and then	  \\
In life, among the passions of mankind  \\
And qualities commixed and modified  \\
By the infinite varieties and shades  \\
Of individual character. Herein  \\
It was for me (this justice bids me say)	  \\
No useless preparation to have been  \\
The pupil of a public school, and forced  \\
In hardy independence to stand up  \\
Among conflicting passions and the shock  \\
Of various tempers, to endure and note	  \\
What was not understood, though known to be---  \\
Among the mysteries of love and hate,  \\
Honour and shame, looking to right and left,  \\
Unchecked by innocence too delicate,  \\
And moral notions too intolerant,	  \\
Sympathies too contracted. Hence, when called  \\
To take a station among men, the step Was easier,  \\
the transition more secure,  \\
More profitable also; for the mind  \\
Learns from such timely exercise to keep	  \\
In wholesome separation the two natures---  \\
The one that feels, the other that observes.  \\
Let one word more of personal circumstance---  \\
Not needless, as it seems---be added here.  \\
Since I withdrew unwillingly from France,	  \\
The story hath demanded less regard  \\
To time and place; and where I lived and how,  \\
Hath been no longer scrupulously marked.  \\
Three years, until a permanent abode  \\
Received me with that sister of my heart	  \\
Who ought by rights the dearest to have been  \\
Conspicuous through this biographic verse---  \\
Star seldom utterly concealed from view---  \\
I led an undomestic wanderer's life.  \\
In London chiefly was my home, and thence	  \\
Excursively, as personal friendships, chance  \\
Or inclination led, or slender means  \\
Gave leave, I roamed about from place to place,  \\
Tarrying in pleasant nooks, wherever found,  \\
Through England or through Wales. A youth---he bore	  \\
The name of Calvert; it shall live,  \\
if words Of mine can give it life---without respect  \\
To prejudice or custom, having hope  \\
That I had some endowments by which good  \\
Might be promoted, in his last decay	  \\
From his own family withdrawing part  \\
Of no redundant patrimony, did  \\
By a bequest sufficient for my needs  \\
Enable me to pause for choice, and walk  \\
At large and unrestrained, nor damped too soon	  \\
By mortal cares. Himself no poet, yet  \\
Far less a common spirit of the world,  \\
He deemed that my pursuits and labors lay  \\
Apart from all that leads to wealth, or even  \\
Perhaps to necessary maintenance,	  \\
Without some hazard to the finer sense,  \\
He cleared a passage for me, and the stream  \\
Flowed in the bent of Nature.  \\
Having now  \\
Told what best merits mention, further pains	  \\
Our present labour seems not to require,  \\
And I have other tasks. Call back to mind  \\
The mood in which this poem was begun,  \\
O friend---the termination of my course  \\
Is nearer now, much nearer, yet even then	  \\
In that distraction and intense desire  \\
I said unto the life which I had lived,  \\
'Where art thou? Hear I not a voice from thee  \\
Which 'tis reproach to hear?' Anon I rose  \\
As if on wings, and saw beneath me stretched	  \\
Vast prospect of the world which I had been,  \\
And was; and hence this song, which like a lark  \\
I have protracted, in the unwearied heavens  \\
Singing, and often with more plaintive voice  \\
Attempered to the sorrows of the earth---	  \\
Yet centring all in love, and in the end  \\
All gratulant if rightly understood.  \\
Whether to me shall be allotted life,  \\
And with life power to accomplish aught of worth  \\
Sufficient to excuse me in men's sight	  \\
For having given this record of myself,  \\
Is all uncertain, but, belove`d friend,  \\
When looking back thou seest, in clearer view  \\
Than any sweetest sight of yesterday,  \\
That summer when on Quantock's grassy hills	  \\
Far ranging, and among the sylvan coombs,  \\
Thou in delicious words, with happy heart,  \\
Didst speak the vision of that ancient man,  \\
The bright-eyed Mariner, and rueful woes  \\
Didst utter of the Lady Christabel;	  \\
And I, associate in such labour, walked  \\
Murmuring of him, who---joyous hap---was found,  \\
After the perils of his moonlight ride,  \\
Near the loud waterfall, or her who sate  \\
In misery near the miserable thorn;	  \\
When thou dost to that summer turn thy thoughts,  \\
And hast before thee all which then we were,  \\
To thee, in memory of that happiness,  \\
It will be known---by thee at least, my friend,  \\
Felt---that the history of a poet's mind	  \\
Is labour not unworthy of regard:  \\
To thee the work shall justify itself.  \\
The last and later portions of this gift  \\
Which I for thee design have been prepared  \\
In times which have from those wherein we first	  \\
Together wandered in wild poesy  \\
Differed thus far, that they have been, my friend,  \\
Times of much sorrow, of a private grief  \\
Keen and enduring, which the frame of mind  \\
That in this meditative history	  \\
Hath been described, more deeply makes me feel,  \\
Yet likewise hath enabled me to bear  \\
More firmly; and a comfort now, a hope,  \\
One of the dearest which this life can give,  \\
Is mine: that thou art near, and wilt be soon	  \\
Restored to us in renovated health---  \\
When, after the first mingling of our tears,  \\
'Mong other consolations, we may find  \\
Some pleasure from this offering of my love.  \\
Oh, yet a few short years of useful life,	  \\
And all will be complete---thy race be run,  \\
Thy monument of glory will be raised.  \\
Then, though too weak to tread the ways of truth,  \\
This age fall back to old idolatry,  \\
Though men return to servitude as fast	  \\
As the tide ebbs, to ignominy and shame  \\
By nations sink together, we shall still  \\
Find solace in the knowledge which we have,  \\
Blessed with true happiness if we may be  \\
United helpers forward of a day	  \\
Of firmer trust, joint labourers in the work---  \\
Should Providence such grace to us vouchsafe---  \\
Of their redemption, surely yet to come.  \\
Prophets of Nature, we to them will speak  \\
A lasting inspiration, sanctified	  \\
By reason and by truth; what we have loved  \\
Others will love, and we may teach them how:  \\
Instruct them how the mind of man becomes  \\
A thousand times more beautiful than the earth  \\
On which he dwells, above this frame of things	  \\
(Which, 'mid all revolutions in the hopes  \\
And fears of men, doth still remain unchanged)  \\
In beauty exalted, as it is itself  \\
Of substance and of fabric more divine. \\
\end{verse}


\appendix
\pagestyle{app}
\chapter*{The Two-Book Prelude (1798--99)}
\addcontentsline{toc}{chapter}{Appendix: The Two-Book Prelude (1798--99)}

\section{\hfil Book 1 \hfil}
%\section*{\hfil Book 1 \hfil}

\begin{verse} 
\hspace{5cm} Was it for this \\
That one, the fairest of all rivers, loved \\
To blend his murmurs with my Nurse's song, \\
And from his alder shades, and rocky falls, \\
And from his fords and shallows, sent a voice \\
That flowed along my dreams? For this didst thou \\
O Derwent, traveling over the green plains \\
Near my "sweet birth-place," didst thou beauteous Stream \\
Make ceaseless music through the night and day, \\
Which with its steady cadence tempering     \\
Our human waywardness, composed my thoughts \\
To more than infant softness, giving me, \\
Among the fretful dwellings of mankind, \\
A knowledge, a dim earnest of the calm \\
Which Nature breathes among the fields and groves? \\
Beloved Derwent! Fairest of all Streams! \\
Was it for this that I, a four year's child, \\
A naked Boy, among thy silent pools \\
Made one long bathing of a summer's day? \\
Basked in the sun, or plunged into thy stream's     \\
Alternate, all a summer's day, or coursed \\
Over the sandy fields, and dashed the flowers \\
Of yellow grunsel, or whom crag and hill, \\
The woods and distant Skiddaw's lofty height \\
Were bronzed with a deep radiance, stood alone, \\
A naked Savage in the thunder shower? \\
And afterwards, 'twas in a later day \\
Though early, when upon the mountain-slope \\
The frost and breath of frosty wind had snapped \\
The last autumnal crocus, 'twas my joy     \\
To wander half the night among the cliffs \\
And the smooth hollows, where the woodcocks ran \\
Along the moonlight turf. In thought and wish, \\
That time, my shoulder all with springes hung, \\
I was a fell destroyer. Gentle Powers! \\
Who give us happiness and call it peace! \\
When scudding on from snare to snare I plied \\
My anxious visitation, hurrying on, \\
Still hurrying hurrying onward, how my heart \\
Panted; among the scattered yew-trees, and the crags     \\
The looked upon me, how my bosom beat \\
With expectation. Sometimes strong desire, \\
Resistless, overpowered me, and the bird \\
Which was the captive of another's toils \\
Became my prey; and when the deed was done \\
I heard among the solitary hills \\
Low breathings coming after me, and sounds \\
Of undistinguishable motion, steps \\
Almost as silent as the turf they trod, \\
Nor less, in spring-time, when on southern banks     \\
The shining sun had from his knot of leaves \\
Decoyed the primrose-flower, and when the vales \\
And woods were warm, was I a rover then \\
In the high places, on the longsome peaks, \\
Among the mountains and the winds. Though mean \\
And though inglorious were my views, then end \\
Was ignoble. Oh, when I have hung \\
Above the raven's nest, by knots of grass, \\
Or half-inch fissures in the slipp'ry rock, \\
But ill sustained, and almost, as it seemed,     \\
Suspended by the blast which blew amain, \\
Shouldering the naked crag, oh at that time, \\
While on the perilous ridge I hung alone, \\
With what strange utterance did the loud dry wind \\
Blow through my ears! The sky seemed not a sky \\
Of earth, and with what motion moved the clouds! \\
The mind of man is fashioned and built up \\
Even as strain of music: I believe \\
That there are spirits, which, when they would form \\
A favored being, from his very dawn     \\
Of infancy do open out the clouds \\
As at the touch of lightning, seeking him \\
With gentle visitation; quiet Powers! \\
Retired and seldom recognized, yet kind, \\
And to the very meanest not unknown; \\
With me, though rarely, in my early days \\
They communed: others too there are who use, \\
Yet haply aiming at the self-same end, \\
Severer interventions, ministry \\
More palpable, and of their school was I.     \\
They guided me: one evening, led by them, \\
I went alone into a Shepherd's boat, \\
A skiff that to a willow-tree was tied \\
Within a rocky cave, its usual home; \\
The moon was up, the lake was shining clear \\
Among the hoary mountains: from the shore \\
I pushed, and struck the oars, and struck again \\
In cadence, and my little Boat moved on \\
Just like a man who walks with stately step \\
Though bent on speed. It was an act of stealth     \\
And troubled pleasure; not without the voice \\
Of mountain-echoes did my boat move on, \\
Leaving behind her still on either side \\
Small circles glittering idly in the moon \\
Until they melted all into one track \\
Of sparkling light. A rocky steep uprose \\
Above the cavern of the willow tree, \\
And now, as suited one who proudly rowed \\
With his best skill, I fixed a steady view \\
Upon the top of that same craggy ridge,     \\
The bound of the horizon, for behind \\
Was nothing --- but the stars and the grey sky. \\
She was an elfin pinnace; twenty times \\
I dipped my oars into the silent lake. \\
And, as I rose upon the stroke, my Boat \\
Went heaving through the water, like a swan --- \\
When from behind that rocky steep, till then \\
The bound of the horizon, a huge Cliff, \\
As if voluntary power instinct, \\
Upreared its head: I struck, and struck again,     \\
And, growing still in statue, the huge cliff \\
Rose up between me and the starts, and still \\
With measured motion, like a living thing, \\
Strode after me. With trembling hands I turned, \\
And through the silent water stole my way \\
Back to the cavern of the willow-tree. \\
There, in her mooring-place I left my bark, \\
And through the meadows homeward went with grave \\
And serious thoughts; and after I had seen \\
That spectacle, for many days my brain     \\
Worked with a dim and undetermined sense \\
Of unknown modes of being; in my thoughts \\
There was darkness, call it solitude \\
Or blank desertion; no familiar objects \\
Of hourly objects, images of trees, \\
Of sea or sky, no colours of green fields; \\
But huge and mighty forms that do not live \\
Like living men, moved slowly through my mind \\
By day, and were the trouble of my dreams. \\
Ah! Not in vain ye Beings of the hills!     \\
And ye that walk the woods and open heaths \\
By moon or star-light, thus from my first dawn \\
Of childhood did ye love to intertwine \\
The passions that build up our human soul, \\
Not with the mean and vulgar works of man, \\
But with high objects, with eternal things, \\
With life and nature, purifying thus \\
The elements of feeling and of thought, \\
And sanctifying by such discipline \\
Both pain and fear, until we recognize     \\
A grandeur in the beatings of the heart. \\
Nor was this fellowship vouchsafed to me \\
With stinted kindness. In November days, \\
When vapours, rolling down the valleys, made \\
A lonely scene more lonesome, among woods \\
At noon, and 'mid the calm of summer nights \\
When by the margin of the trembling lake \\
Beneath the gloomy hills I homeward went \\
In solitude, such intercourse was mine. \\
And in the frosty season when the sun     \\
Was set, and, visible for many a mile, \\
The cottage windows through the twilight blazed, \\
I heeded not the summons: clear and loud \\
The village clock tolled six; I wheeled about \\
Proud and exulting like an untired horse \\
That cares not for its home. All shod with steel \\
We hissed along the polished ice, in games \\
Confederate, imitative of the chase \\
And woodland pleasures, the resounding horn, \\
The pack loud bellowing, and the hunted hare.     \\
So through the darkness and the cold we flew, \\
And not a voice was idle: with the din, \\
Meanwhile, the precipices rang aloud, \\
The leafless trees and every icy crag \\
Tinkled like iron, while the distant hills \\
Into the tumult sent an alien sound \\
Of melancholy not unnoticed while the stars, \\
Eastward, were sparkling clear, and in the west \\
The orange sky of evening died away. \\
Not seldom from the uproar I retired     \\
Into a silent bay, or sportively \\
Glanced sideway leaving the tumultuous throng \\
To cut across the shadow of a star \\
That gleamed upon the ice: and oftentimes \\
When we had given our bodies to the wind \\
And all the shadowy banks on either side \\
Came sweeping through the darkness, spinning still \\
The rapid line of motion, then at once \\
Have I, reclining back upon my heels, \\
Stopped short; yet still the solitary cliffs     \\
Wheeled by me, even as if the earth had rolled \\
With visible motion her diurnal round; \\
Behind me did they stretch in solemn train \\
Feebler and feebler, and I stood and watched \\
Till all was tranquil as a summer sea. \\
Ye Powers of earth! Ye Genii of the springs! \\
And ye that have your voices in the clouds \\
And ye that are Familiars of the lakes \\
And of the standing pools, I may not think \\
A vulgar hope was yours when ye employed     \\
Such ministry, when ye through many a year \\
Thus by the agency of boyish sports \\
On caves and trees, upon the woods and hills, \\
Impressed upon all forms the characters \\
Of danger and desire, and thus did make \\
The surface of the universal earth \\
With meanings of delight, of hope and fear, \\
Work like a sea. \\
     Not uselessly employed \\
I might pursue this theme through every change     \\
Of exercise and sport to which the year \\
Did summon us in its delightful round. \\
We were a noisy crew: the sun in heaven \\
Beheld not vales more beautiful than ours \\
Nor saw a race in happiness and joy \\
More worthy of the fields where they were sown. \\
I would record with no reluctant voice \\
Our home amusements by the warm peat fire \\
At evening, when with pencil, and with slate \\
In square divisions parcelled out, and all     \\
With crosses and with cyphers scribbled o'er, \\
We schemed and puzzled, head opposed to head \\
In strife too humble to be named in verse, \\
Or round the naked table, snow-white deal, \\
Cherry or maple, sat in close array \\
And to the combat --- Lu or Whist --- led on \\
A thick-ribbed army, not as in the world \\
Discarded and ungratefully thrown by \\
Even for the very service they had wrought, \\
But husbanded through many a long campaign.     \\
Oh with what echoes on the board they fell --- \\
Ironic diamonds, hearts of sable hue, \\
Queens gleaming through their splendour's last decay, \\
Knaves wrapt in one assimilating gloom, \\
And Kings indignant at the shame incurr'd \\
By royal visages. Meanwhile abroad \\
The heavy rain was falling, or the frost \\
Raged bitterly with keen and silent tooth, \\
And interrupting the impassioned game \\
Oft from the neighbouring lake the splitting ice     \\
While it sank down towards the water sent \\
Among the meadows and the hills its long \\
And frequent yellings, imitative some \\
Of wolves that howl along the Bothnic main. \\
Nor with less willing heart would I rehearse \\
The woods of autumn and their hidden bowers \\
With milk-white clusters hung; the rod and line. \\
True symbol of the foolishness of hope, \\
Which with its strong enchantment led me on     \\
By rocks and pools where never summer-star \\
Impressed its shadow, to forlorn cascades \\
Among the windings of the mountain-brooks; \\
The kite, in sultry calms from some high hill \\
Sent up, ascending thence till it was lost \\
Among the fleecy clouds, in gusty days \\
Launched from the lower grounds, and suddenly \\
Dash'd headlong---and rejected by the storm. \\
All these and more with rival claims demand \\
Grateful acknowledgment. It were a song     \\
Venial, and such as if I rightly judge \\
I might protract unblamed; but I perceive \\
That much is overlooked, and we should ill \\
Attain our object if from delicate fears \\
Of breaking in upon the unity \\
Of this my argument I should omit \\
To speak of such effects as cannot here \\
Be regularly classed, yet tend no less \\
To the same point, the growth of mental power \\
And love of Nature's works.     \\
Ere I had seen \\
Eight summers (and 'twas in the very week \\
When I was first transplanted to thy vale, \\
Beloved Hawkshead! when thy paths, thy shores \\
And brooks were like a dream of novelty \\
To my half-infant mind) I chanced to cross \\
One of those open fields which, shaped like ears, \\
Make green peninsulas on Esthwaite's lake, \\
Twilight was coming on, yet through the gloom \\
I saw distinctly on the opposite shore \\
Beneath a tree and close by the lake side     \\
A heap of garments, as if left by one \\
Who there was bathing: half an hour I watched \\
And no one owned them: meanwhile the calm lake \\
Grew dark with all the shadows on its breast, \\
And now and then a leaping fish disturbed \\
The breathless stillness. The succeeding day \\
There came a company, and in their boat \\
Sounded with iron hooks, and with long poles. \\
At length the dead man' mid that beauteous scene \\
Of trees, and hills, and water, bolt upright     \\
Rose with his ghastly face. I might advert \\
To numerous accidents in flood or field, \\
Quarry or moor, or 'mid the winter snows, \\
Distresses and disasters, tragic facts \\
Of rural history that impressed my mind \\
With images, to which in following years \\
Far other feelings were attached, with forms \\
That yet exist with independent life \\
And, like their archetypes, know no decay. \\
There are in our existence spots of time     \\
Which with distinct pre-eminence retain \\
A fructifying virtue, whence, depressed \\
By trivial occupations and the round \\
Of ordinary intercourse, our minds \\
(Especially the imaginative power) \\
Are nourished, and invisibly repaired. \\
Such moments chiefly seem to have their date \\
In our first childhood, I remember well \\
('Tis of an early season that I speak, \\
The twilight of rememberable life)     \\
While I was yet an urchin, one who scarce \\
Could hold a bridle, with ambitious hopes \\
I mounted, and we rode towards the hills; \\
We were a pair of horsemen: Honest James \\
Was with me, my encourager and guide. \\
We had not travelled long ere some mischance \\
Disjoined me from my comrade, and through fear \\
Dismounting, down the rough and stony moor \\
I led my horse and, stumbling on, at length \\
Came to a bottom where in former times     \\
A man, the murderer of his wife, was hung \\
In irons; mouldered was the gibbet mast, \\
The bones were gone, the iron and the wood, \\
Only a long green ridge of turf remained \\
Whose shape was like a grave. I left the spot, \\
And, reascending the bare slope, I saw \\
A naked pool that lay beneath the hills, \\
The beacon on the summit, and more near \\
A girl who bore a pitcher on her head \\
And seemed with difficult steps to force her way     \\
Against the blowing wind. It was in truth \\
An ordinary sight but I should need \\
Colours and words that are unknown to man \\
To paint the visionary dreariness \\
Which, while I looked all round for my lost guide, \\
Did, at that time, invest the naked pool, \\
The beacon on the lonely eminence, \\
The woman and her garments vexed and tossed \\
By the strong wind. Nor less I recollect \\
(Long after, though my childhood had not ceased)     \\
Another scene which left a kindred power \\
Implanted in my mind. \\
One Christmas time, \\
The day before the holidays began, \\
Feverish, and tired and restless, I went forth \\
Into the fields, impatient for the sight \\
Of those three horses which should bear us home, \\
My Brothers and myself. There was a crag, \\
An eminence which from the meeting point \\
Of two highways ascending overlooked     \\
At least a long half-mile of those two roads, \\
By each of which the expected steeds might come, \\
The choice uncertain. Thither I repaired \\
Up to the highest summit; 'twas a day \\
Stormy, and rough, and wild, and on the grass \\
I sat, half-sheltered by a naked wall; \\
Upon my right hand was a single sheep, \\
A whistling hawthorn on my left, and there, \\
Those two companions at my side, I watched \\
With eyes intensely straining as the mist     \\
Gave intermitting prospects of the wood \\
And plain beneath. Ere I to school returned \\
That dreary time, ere I had been ten days \\
A dweller in my Father's house, he died, \\
And I and my two Brothers, orphans then, \\
Followed his body to the grave. The event \\
With all the sorrow which it brought appeared \\
A chastisement, and when I called to mind \\
That day so lately passed when from the crag \\
I looked in such anxiety of hope,     \\
With trite reflections of morality \\
Yet with the deepest passion I bowed low \\
To God, who thus corrected my desires; \\
And afterwards the wind, and sleety rain, \\
And all the business of the elements, \\
The single sheep, and the one blasted tree, \\
And the bleak music of that old stone wall, \\
The noise of wood and water, and the mist \\
Which on the line of each of those two roads \\
Advanced in such indisputable shapes,     \\
All these were spectacles and sounds to which \\
I often would repair, and thence would drink \\
As at a fountain, and I do not doubt \\
That in this later time when storm and rain \\
Beat on my roof at midnight, or by day \\
When I am in the woods, unknown to me \\
The workings of my spirit thence are brought. \\
Nor sedulous to trace diligent \\
How Nature by collateral interest indirect \\
And by extrinsic passion peopled first     \\
My mind with forms, or beautiful or grand, \\
And made me love them, may I well forget \\
How other pleasures have been mine, and joys \\
Of subtler origin, how I have felt \\
Not seldom, even in that tempestuous time, \\
Those hallowed and pure motions of the sense \\
Which seem in their simplicity to own \\
An intellectual charm, that calm delight \\
Which, if I err not, surely must belong \\
To those first-born affinities that fit     \\
Our new existence to existing things \\
And in our dawn of being constitute \\
The bond of union betwixt life and joy. \\
Yes, I remember when the changeful earth \\
And twice five seasons on my mind had stamped \\
The faces of the moving year, even then, \\
A Child, I held unconscious intercourse \\
With the eternal Beauty, drinking in \\
A pure organic pleasure from the lines \\
Of curling mist or from the level plain     \\
Of waters coloured by the steady clouds. \\
The sands of Westmoreland, the creeks and bays \\
Of Cumbria's 2 rocky limits, they can tell \\
How when the sea threw off his evening shade \\
And to the Shepherd's hutt beneath the crags \\
Did send sweet notice of the rising moon, \\
How I have stood to images like these \\
A stranger, linking with the spectacle \\
No body of associated forms \\
And bringing with me no peculiar sense     \\
Of quietness or peace, yet I have stood \\
Even while my eye has moved o'er three long leagues \\
Of shining water, gathering as it seemed, \\
Through the wide surface of that field of light \\
New pleasure, like a bee among the flowers. \\
Thus often in those fits of vulgar joy \\
Which through all seasons on a child's pursuits \\
Are prompt attendants, 'mid that giddy bliss \\
Which like a tempest works along the blood \\
And is forgotten, even then I felt     \\
Gleams like the flashing of a shield; the earth \\
And common face of Nature spake to me \\
Rememberable things: sometimes, 'tis true, \\
By quaint associations, yet not vain \\
Nor profitless if haply they impressed \\
Collateral objects and appearances, \\
Albeit lifeless then, and doomed to sleep \\
Until maturer seasons called them forth \\
To impregnate and to elevate the mind. \\
And if the vulgar joy by its own weight     \\
Wearied itself out of memory, \\
The scenes which were witness of that joy \\
Remained, in their substantial lineaments \\
Depicted on the brain, and to the eye \\
Were visible, a daily sight: and thus \\
By the impressive agency of fear, \\
By pleasure and repeated happiness, \\
So frequently repeated, and by force \\
Of obscure feelings representative \\
Of joys that were forgotten, these same scenes     \\
So beauteous and majestic in themselves, \\
Though yet the day was distant, did at length \\
Become habitually dear, and all \\
Their hues and forms were by invisible links \\
Allied to the affections. \\
I began \\
My story early, feeling, as I fear, \\
The weakness of a human love for days \\
Disowned by memory, ere the birth of spring \\
Planting my snow-drops among winter snows.     \\
Nor will it seem to thee, my Friend, so prompt \\
In sympathy, that I have lengthened out \\
With fond and feeble tongue a tedious tale. \\
Meanwhile my hope has been that I might fetch \\
Reproaches from my former years, whose power \\
May spur me on, in manhood now mature, \\
To honourable toil. Yet, should it be \\
That this is but an impotent desire, \\
That I by such inquiry am not taught \\
To understand myself, nor thou to know     \\
With better knowledge how the heart was framed \\
Of him thou lovest, need I dread from thee \\
Harsh judgements if I am so loath to quit \\
Those recollected hours that have the charm \\
Of visionary things, and lovely forms \\
And sweet sensations that throw back our life \\
And make our infancy a visible scene \\
On which that sun is shining? \\
\end{verse} 

%%%%%%%%%%%%%%%%%%%%%%%%%%%%%%%%%%%%%%%%%
%\section*{\hfil Book 2 \hfil}
\section{\hfil Book 2 \hfil}

\begin{verse} 
Thus far my Friend, have we retraced the way \\
Through which I traveled when I first began \\
To love the woods and fields: the passion yet \\
Was in its birth, sustained as might befall \\
By nourishment that came unsought, for still \\
From week to week, from month to month, we lived \\
A round of tumult: duly were our games \\
Prolonged in summer till the day-light failed; \\
No chair remained before the doors, the bench \\
And the threshold steps were empty, fast asleep     \\
The labourer and the old man who had sat \\
A later lingerer, yet the revelry \\
Continued and the loud uproar: at last \\
When all the ground was dark, and the huge clouds \\
Were edged with twinkling stars, to bed we went \\
With weary joints and with a beating mind. \\
Ah! is there one who ever has been young \\
And needs a monitory voice to tame \\
The pride of virtue and of intellect, \\
And is there one, the wisest and the best     \\
Of all mankind, who does not sometimes wish \\
For things which cannot be, who would not give, \\
If so he might, to duty and to truth \\
The eagerness of infantine desire? \\
A tranquillizing spirit presses now \\
On my corporeal frame, so wide appears \\
The vacancy between me and those days \\
Which yet have such self-presence in my heart \\
That sometimes when I think of them I seem \\
Two consciousnesses, conscious of myself     \\
And of some other being. A grey stone \\
Of native rock, left midway in the square \\
Of our small market-village, was the home \\
And centre of these joys, and when, returned \\
After long absence, thither I repaired, \\
I found that it was split and gone to build \\
A smart assembly-room that perked and flared \\
With wash and rough-cast, elbowing the ground \\
Which had been ours. But let the fiddle scream \\
And be ye happy! yet I know, my friends,     \\
That more than one of you will think with me \\
Of those soft starry nights and that old dame \\
From whom the stone was named, who there had sat \\
And watched her table with its huckster's wares, \\
Assiduous, for the length of sixty years. \\
We ran a boisterous race, the year span round \\
With giddy motion. But the time approached \\
That brought with it a regular desire \\
For calmer pleasures, when the beauteous scenes \\
Of nature were collaterally attached     \\
To every scheme of holiday delilght \\
And every boyish sport, less grateful else \\
And languidly pursued. \\
When summer came \\
It was the pastime of our afternoons \\
To beat along the plain of Windermere \\
With rival oars; and the selected bourn \\
Was now an island musical with birds \\
That sang for ever, now a sister isle \\
Beneath the oak's umbrageous covert sown     \\
With lilies of the valley like a field, \\
And now a third small island where remained \\
An old stone table and one mouldered cave, \\
A hermit's history. In such a race, \\
So ended, disappointment could be none, \\
Uneasiness, or pain, or jealousy; \\
We rested in the shade all pleased alike, \\
Conquered and conqueror. Thus our selfishness \\
Was mellowed down, and thus the pride of strength \\
And the vain-glory of superior skill     \\
Were interfused with objects which subdued \\
And tempered them, and gradually produced \\
A quiet independence of the heart. \\
And to my Friend who knows me I may add, \\
Unapprehensive of reproof that hence \\
Ensued a diffidence and modesty, \\
And I was taught to feel, perhaps too much, \\
The self-sufficing power of solitude. \\
No delicate viands sapped our bodily strength; \\
More than we wished we knew the blessing then     \\
Of vigorous hunger, for our daily meals \\
Were frugal, Sabine fare! and then exclude \\
A little weekly stipend, and we lived \\
Through three divisions of the quartered year \\
In penniless poverty. But now to school \\
Returned from the half-yearly holidays, \\
We came with purses more profusely filled, \\
Allowance which abundantly sufficed \\
To gratify the palate with repasts \\
More costly than the Dame of whom I spake,     \\
That ancient woman, and her board supplied, \\
Hence inroads into distant vales, and long \\
Excursions far away among the hills; \\
Hence rustic dinners on the cool green ground \\
Or in the woods or by a river-side \\
Or fountain, festive banquets that provoked \\
The languid action of a natural scene \\
By pleasure of corporeal appetite. \\
Nor is my aim neglected if I tell \\
How twice in the long length of those half-years     \\
We from our funds perhaps with bolder hand \\
Drew largely, anxious for one day at least \\
To feel the motion of the galloping steed; \\
And with the good old Innkeeper in truth \\
I needs must say that sometimes we have used \\
Sly subterfuge, for the intended bound \\
Of the day's journey was too distant far \\
For any cautious man, a Structure famed \\
Beyond its neighborhood, the antique walls \\
Of a large Abbey with its fractured arch,     \\
Belfry, and images, and living trees, \\
A holy scene! Along the smooth green turf \\
Our horses grazed: in more than inland peace \\
Left by the winds that overpass the vale \\
In that sequestered ruin trees and towers \\
Both silent, and both motionless alike, \\
Hear all day long the murmuring sea that beats \\
Incessantly upon a craggy shore. \\
Our steeds remounted, and the summons given, \\
With whip and spur we by the Chantry flew     \\
In uncouth race, and left the cross-legged Knight \\
And the stone Abbot, and that single wren \\
Which one day sang so sweetly in the nave \\
Of the old church that, though from recent showers \\
The earth was comfortless, and touched by faint \\
Internal breezes from the roofless walls \\
The shuddering ivy dripped large drops, yet still \\
So sweetly 'mid the gloom the invisible bird \\
Sang to itself that there I could have made \\
My dwelling-place, and lived for ever there     \\
To hear such music. Through the walls we flew \\
And down the valley, and, a circuit made \\
In wantonness of heart, through rough and smooth \\
We scampered homeward. O ye rocks and streams \\
And that still spirit of the evening air, \\
Even in this joyous time I sometimes felt \\
Your presence, when with slackened step we breathed \\
Along the sides of the steep hills, or when, \\
Lightened by gleams of moonlight from the sea, \\
We beat the thundering hoofs the level sand.     \\
There was a row of ancient trees, since fallen, \\
That on the margin of a jutting land \\
Stood near the lake of Coniston and made \\
With its long boughs above the water stretched \\
A gloom through which a boat might sail along \\
As in a cloister. An old Hall was near, \\
Grotesque and beautiful, its gavel end \\
And huge round chimneys to the top o'ergrown \\
With fields of ivy. Thither we repaired, \\
'Twas even a custom with us, to the shore     \\
And to that cool piazza. They who dwelt \\
In the neglected mansion-house supplied \\
Fresh butter, tea-kettle, and earthen-ware, \\
And chafing-dish with smoking coals, and so \\
Beneath the trees we sat in our small boat \\
And in the covert eat our delicate meal \\
Upon the calm smooth lake. It was a joy \\
Worthy the heart of one who is full grown \\
To rest beneath those horizontal boughs \\
And mark the radiance of the setting sun,     \\
Himself unseen, reposing on the top \\
Of the high eastern hills. And there I said, \\
That beauteous sight before me, there I said \\
(Then first beginning in my thoughts to mark \\
That sense of dim similitude which links \\
Our moral feelings with external forms) \\
That in whatever region I should close \\
My mortal life I would remember you, \\
Fair scenes! that dying I would think on you, \\
My soul would send a longing look to you:     \\
Even as that setting sun while all the vale \\
Could nowhere catch one faint memorial gleam \\
Yet with the last remains of his last light \\
Still lingered, and a farewell luster threw \\
On the dear mountain-tops where first he rose. \\
'Twas then my fourteenth summer, and these words \\
Were uttered in casual access \\
Of sentiment, a momentary trance \\
That far outran the habit of my mind. \\
Upon the east     \\
Above the crescent of a pleasant bay, \\
There was an Inn, no homely-featured shed, \\
Brother of the surrounding cottages, \\
But 'twas a splendid place, the door beset \\
With chaises, grooms, and liveries, and within \\
Decanters, glasses, and the blood-red wine. \\
In ancient times, or ere the Hall was built \\
On the large island, had the dwelling been \\
More worthy of a poet's love, a hut \\
Proud of its one bright fire and sycamore shade.     \\
But though the rhymes were gone which once inscribed \\
The threshold, and large golden characters \\
On the blue-frosted sign-board had usurped \\
The place of the old Lion in contempt \\
And mockery of the rustic painter's hand, \\
Yet to this hour the spot to me is dear \\
With all its foolish pomp. The garden lay \\
Upon a slope surmounted by the plain \\
Of a small bowling-green; beneath us stood \\
A grove, with gleams of water through the trees     \\
And over the tree-tops; nor did we want \\
Refreshment, strawberries and mellow cream, \\
And there through half an afternoon we played \\
On the smooth platform, and the shouts we sent \\
Made all the mountains ring. But ere the fall \\
Of night, when in our pinnace we returned \\
Over the dusky lake, and to the beach \\
Of some small island steered our course with one, \\
The minstrel of our troop, and left him there \\
And rowed off gently while he blew his flute     \\
Alone upon the rock �C oh then the calm \\
And dead still water lay upon my mind \\
Even with a weight of pleasure, and the sky, \\
Never before so beautiful, sank down \\
Into my heart and held me like a dream. \\
Thus day by day my sympathies increased \\
And thus the common range of visible things \\
Grew dear to me: already I began \\
To love the sun, a Boy I loved the sun \\
Not, as I since have loved him, as a pledge     \\
And surety of my earthly life, a light \\
Which while I view I feel I am alive, \\
But for this cause, that I had seen him lay \\
His beauty on the morning hills, had seen \\
The western mountain touch his setting orb \\
In many a thoughtless hour, when from excess \\
Of happiness my blood appeared to flow \\
With its own pleasure and I breathed with joy. \\
And from like feelings, humble though intense, \\
To patriotic and domestic love     \\
Analogous, the moon to me was dear, \\
For I would dream away my purposes \\
Standing to look upon her while she hung \\
Midway between the hills as if she knew \\
No other region but belonged to thee, \\
Yea, appertained by a peculiar right \\
To thee and thy grey huts, my native vale. \\
Those incidental which were first attached \\
My heart to rural objects day by day \\
Grew weaker, and I hasten on to tell     \\
How nature, intervenient till this time \\
And secondary, now at length was sought \\
For her own sake. But who shall parcel out \\
His intellect by geometric rules, \\
Split like a province into round and square; \\
Who knows the individual hour in which \\
His habits were first sown, even as a seed; \\
Who that shall point as with a wand and say, \\
This portion of the river of my mind \\
Came from yon fountain? Thou, my Friend, art one     \\
More deeply read in thy own thoughts, no slave \\
Of that false secondary power by which \\
In weakness we create distinctions, then \\
Believe our puny boundaries are things \\
Which we perceive and not which we have made. \\
To thee, unblended by these outward shows, \\
The unity of all has been revealed \\
And thou wilt doubt with me, less aptly skilled \\
Than many are to class the cabinet \\
Of their sensations and in voluble phrase     \\
Run through the history and birth of each \\
As of a single independent thing. \\
Hard task to analyse a soul in which \\
Not only general habits and desires \\
But each most obvious and particular thoughts, \\
Not in a mystical and idle sense \\
But in the words of reason deeply weighed, \\
Hath no beginning, \\
Blessed be the infant Babe \\
(For with my best conjectures I would trace     \\
The progress of our being) blest the Babe \\
Nursed in his Mother's arms, the Babe who sleeps \\
Upon his Mother's breast, who when his soul \\
Claims manifest kindred with an earthly soul \\
Doth gather passion from his Mother's eye! \\
Such feelings pass into his torpid life \\
Like an awakening breeze, and hence his mind \\
Even in the first trial of its powers \\
Is prompt and watchful, eager to combine \\
In one appearance all the elements     \\
And parts of the same object, else detached \\
And loath to coalesce. Thus day by day \\
Subjected to the discipline of love \\
His organs and recipient faculties \\
Are quickened, are more vigorous, his mind spreads \\
Tenacious of the forms which it receives. \\
In one beloved presence, nay and more, \\
And those sensations which have been derived \\
From this beloved presence, there exists \\
A virtue which irradiates and exalts     \\
All objects through all intercourse of sense. \\
No outcast he, bewildered and depressed: \\
Along his infant veins are interfused \\
The gravitation and the filial bond \\
Of nature that connect him with the world. \\
Emphatically such a being lives \\
An inmate of this active universe; \\
From nature largely he receives, nor so \\
Is satisfied but largely gives again, \\
For feeling has to him imparted strength,     \\
And powerful in all sentiments of grief, \\
Of exultation, fear and joy, his mind, \\
Even as an agent of the one great mind, \\
Creates, creator and receiver both, \\
Working but in alliance with the works \\
Which it beholds. Such verily is the first \\
Poetic spirit of our human life, \\
By uniform control of after years \\
In most abated and suppressed, in some \\
Through every change of growth or of decay     \\
Preeminent till death. \\
From early days, \\
Beginning not long after that first time \\
In which, a Babe, by intercourse of touch \\
I held mute dialogues with my Mother's heart, \\
I have endeavoured to display the means \\
Whereby this infant sensibility, \\
Great birth-right of our being, was in me \\
Augmented and sustained. Yet is a path     \\
More difficult before me, and I fear \\
That in its broken windings we shall need \\
The Chamois sinews and the Eagle's wing: \\
For now a trouble came into my mind \\
From obscure causes. I was left alone \\
Seeking this visible world, nor knowing why: \\
The props of my affections were removed \\
And yet the buildings stood as if sustained \\
By its own spirit. All that I beheld \\
Was dear to me, and from this cause it came     \\
That now to Nature's finer influxes \\
My mind lay open, to that more exact \\
And intimate communion which our hearts \\
Maintain with the minuter properties \\
Of objects which already are beloved, \\
And of those only. Many are the joys \\
Of youth, but oh! What happiness to live \\
When every hour brings palpable access \\
Of knowledge, when all knowledge is delight, \\
And sorrow is not there. The seasons come     \\
And every season brought a countless store \\
Of modes and temporary qualities \\
Which but for this most watchful power of love \\
Had been neglected, left a register \\
Of permanent relations, else unknown: \\
Hence life, and change, and beauty, solitude \\
More active even than "best society," \\
Society made sweet as solitude \\
By silent inobtrusive sympathies \\
And gentle agitations of the mind     \\
From manifold distinctions, difference \\
Perceived in things where to the common eye \\
No difference is: and hence from the same source \\
Sublimer joy; for I would walk alone \\
In storm and tempest or in starlight nights \\
Beneath the quiet heavens, and at that time \\
Would feel whate'er there is of power in sound \\
To breathe an elevated mood by form \\
Or image unprofaned: and I would stand \\
Beneath some rock listening to sounds that are     \\
The ghostly language of the ancient earth \\
Or make their dim abode in distant winds. \\
Thence did I drink the visionary power. \\
I deem not profitless these fleeting moods \\
Of shadowy exaltation, not for this, \\
That they are kindred to our purer mind \\
And intellectual life, but that the soul \\
Remembering how she felt, but what she felt \\
Remembering not, retains an obscure sense \\
Of possible sublimity to which     \\
With growing faculties she doth aspire, \\
With faculties still growing, feeling still \\
That whatsoever point they gain, they still \\
Have something to pursue \\
And not alone \\
In grandeur and in tumult, but no less \\
In tranquil scenes, that universal power \\
And fitness in the latent qualities \\
And essences of things, by which the mind \\
Is moved with feelings of delight, to me     \\
Came strengthened with the superadded soul, \\
A virtue not its own. My morning walks \\
Were early; oft before the hours of school \\
I traveled round our little lake, five miles \\
Of pleasant wandering, happy time more dear \\
For this, that one was by my side, a Friend \\
Then passionately loved; with heart how full \\
Will he peruse these lines, this page, perhaps \\
A blank to other men, for many years \\
Have since flowed in between us, and, our minds     \\
Both silent to each other, at this time \\
We live as if those hours had never been. \\
Nor seldom did I lift our cottage latch \\
Far earlier, and before the vernal thrust \\
Was audible, among the hills I sat \\
Alone upon some jutting eminence \\
At the first hour of morning when the vale \\
Lay quiet in an utter solitude. \\
How shall I trace the history, where seek \\
The origin of what I then have felt?     \\
Oft in those moments such a holy calm \\
Did overspread my soul that I forgot \\
The agency of sight, and what I saw \\
Appeared like something in myself---a dream, \\
A prospect in my mind. 'Twere long to tell \\
What spring and autumn, what the winter-snows \\
And what the summer-shade, what day and night, \\
The evening and the morning, what my dreams \\
And what my waking thoughts supplied, to nurse \\
That spirit of religious love in which     \\
I walked with nature. But let this at least \\
Be not forgotten, that I still retained \\
My first creative sensibility, \\
That by the regular action of the world \\
My soul was unsubdued. A plastic power \\
Abode with me, a forming hand, at times \\
Rebellious, acting in a devious mood, \\
A local spirit of its own, at war \\
With general tendency, but for the most \\
Subservient strictly to the external things     \\
With which it communed. An auxiliary light \\
Came from my mind which on the setting sun \\
Bestowed new splendor, the melodious birds, \\
The gentle breezes, fountains that ran on \\
Murmuring so sweetly in themselves, obeyed \\
A like dominion, and the midnight storm \\
Grew darker in the presence of my eye. \\
Hence my obeisance, my devotion hence, \\
And hence my transport. \\
Nor should this perchance     \\
Pass unrecorded, that I still had loved \\
The exercise and produce of a toil \\
Than analytic industry to me \\
More pleasing, and whose character, I deem, \\
Is more poetic, as resembling more \\
Creative agency: I mean to speak \\
Of that interminable building reared \\
By observation of affinities \\
In objects where no brotherhood exists \\
To common minds. My seventeenth year was come,     \\
And whether from this habit rooted now \\
So deeply in my mind, or from excess \\
Of the great social principle of life \\
Coercing all things into sympathy, \\
To unorganic natures I transferred \\
My own enjoyments, or, the power of truth \\
Coming in revelation, I conversed \\
With things that really are. I at this time \\
Saw Blessings Spread around me like a sea. \\
Thus did my days pass on, and now at length     \\
From Nature and her overflowing soul \\
I had received so much that all my thoughts \\
Were steeped in feelings; I was only then \\
Contented when with bliss ineffable \\
I felt the sentiment of being spread \\
O'er all that moves, and all that seemeth still, \\
O'er all that, lost beyond the reach of thought \\
And human knowledge, to the human eye \\
Invisible, yet liveth to the heart,     \\
O'er all that leaps, and runs, and shouts and sings \\
Or beats the gladsome air, o'er all that glides \\
Beneath the wave, yea, in the wave itself \\
And might depth of waters: wonder not \\
If such my transports were, for in all things \\
I saw one life and felt that it was joy. \\
One song they sang, and it was audible, \\
Most audible ten when the fleshy ear, \\
O'ercome by grosser prelude of that strain, \\
Forgot its functions, and slept undisturbed.     \\
If this be error, and another faith \\
Find easier access to the pious mind, \\
Yet were I grossly destitute of all \\
Those human sentiments which make this earth \\
So dear, if I should fail with grateful voice \\
To speak of you, ye mountains! and ye lakes \\
And sounding cataracts! ye mists and winds \\
That dwell among the hills where I was born. \\
If, in my youth, I have been pure in heart, \\
If, mingling with the world, I am content     \\
With my own modest pleasures, and have lied \\
With God and Nature communing, removed \\
From little enmities and low desires, \\
The gift is yours: if in these times of fear, \\
This melancholy waste of hopes o'erthrown, \\
If, 'mid indifference and apathy \\
And wicked exultation, when good men \\
On every side fall off we know not how \\
To selfishness disguised in gentle names \\
Of peace, and quiet, and domestic love, \\
Yet mingled, not unwillingly, with sneers     \\
On visionary minds, if in this time \\
Of dereliction and dismay I yet \\
Despair not of our nature, but retain \\
A more than Roman confidence, a faith \\
That fails not, in all sorrow my support, \\
The blessing of my life, the gift is yours \\
Ye Mountains! thine, O Nature! Thou hast fed \\
My lofty speculations, and in thee \\
For this uneasy heart of ours I find \\
A never-failing principle of joy     \\
And purest passion. \\
Thou, my Friend, wast reared \\
In the great city mid far other scenes, \\
But we, by different roads, at length have gained \\
The self-same bourne. And from this cause to thee \\
I speak unapprehensive of contempt, \\
The insinuated scoff of coward tongues, \\
And all that silent language which so oft \\
In conversation betwixt man and man \\
Blots from the human countenance all trace     \\
Of beauty and of love. For thou hast sought \\
The truth in solitude, and thou art one, \\
The most intense of Nature's worshippers, \\
In many things my brother, chiefly here \\
In this my deep devotion. \\
Fare thee well! \\
Health and the quiet of a healthful mind \\
Attend thee! seeking oft the haunts of men \\
But yet more often living with thyself \\
And for thyself, so haply shall thy days     \\
Be many and a blessing to mankind. \\
\end{verse} 


%%%%%%%%%%%%%%%%%%%%
\backmatter
%\printindex
\cleardoublepage
\pagestyle{empty}
\null\vfil

\begin{adjustwidth}{1in}{1in}
\begin{center}
{\Large\textsf{Colophon}}
\end{center}
\begin{center}
  {This document was typeset using the xeTeX typesetting system
  and the memoir class. The body text is set to 14pt on a
  33pc measure with Adobe Minion Pro. The text was taken from 
  (\url{https://sms.cam.ac.uk/collection/1170406}) and
  (\url{http://jacklynch.net/Texts/prelude.html}).
  }
\end{center}
\end{adjustwidth}
%\addcontentsline{toc}{chapter}{Colophon}  % add Colophon to ToC

\vfil	

\end{document}
